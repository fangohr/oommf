\section{The 2D Micromagnetic Solver}\label{sec:mmsolve}%
\index{simulation~2D}

%begin{latexonly}
\newcounter{msoldsecnumdepth}
\setcounter{msoldsecnumdepth}{\value{secnumdepth}}
\setcounter{secnumdepth}{4}
%end{latexonly}


The OOMMF 2D micromagnetic computation engine, mmSolve, is capable of
solving problems defined on a two-dimensional grid of square cells with
three-dimensional spins.  This solver is older, less flexible and less
extensible than the \hyperrefhtml{Oxs}{Oxs (Sec.~}{)}{sec:oxs} solver.
Users are encouraged to migrate to Oxs where possible.

There are two interfaces provided to mmSolve, the interactive
\hyperrefhtml{\app{mmSolve2D}}{\app{mmSolve2D} (Sec.~}{)}{sec:mmsolve2d}
interface and the command line driven
\hyperrefhtml{\app{batchsolve}}{\app{batchsolve} (Sec.~}{)}{sec:batchsolve}
interface which can be used in conjunction with the
\hyperrefhtml{OOMMF Batch System}{OOMMF Batch System (Sec.~}{)}{sec:obs}.

Problem definition for mmSolve is accomplished using input files in the
\hyperrefhtml{\MIF~1.1 format}{\MIF~1.1 format (Sec.~}{)}{sec:mif1format}.
Please note that this format is incompatible with the newer \MIF~2.x
format used by the Oxs solver.  However, the command line utility
\hyperrefhtml{\app{mifconvert}}{\app{mifconvert}
(Sec.~}{)}{sec:mifconvert} can be used to aid conversion from the
\MIF~1.1 format to \MIF~2.1.

mmSolve will also accept files in the \hyperrefhtml{\MIF~1.2
format}{\MIF~1.2 format (Sec.~}{)}{sec:mif12format} format, provided the
\texttt{CellSize} record meets the restriction that the $x$- and
$y$-dimensions are the same, and that the $z$-dimension equals the part
thickness.

Note on \Tk\ dependence: If a problem is loaded that uses a
{\hyperrefhtml{bitmap mask file}{bitmap mask file
(Sec.~}{)}{sec:partgeometry}}\index{file!mask}\index{file!bitmap}, and
if that mask file is not in the PPM P3 (text) format, then
\app{mmSolve2D} will launch {\hyperrefhtml{any2ppm}{\app{any2ppm}
(Sec.~}{)}{sec:any2ppm}}\index{application!any2ppm} to convert it into
the PPM P3 format.  Since \app{any2ppm}
requires\index{requirement!Tk}\index{requirement!display} \Tk, at the
time the mask file is read a valid display must be available.  See the
\app{any2ppm} documentation for details.

\subsection{The 2D Micromagnetic Interactive Solver: mmSolve2D}%
\label{sec:mmsolve2d}\index{simulation~2D}\index{application!mmSolve2D}

\begin{center}
\includepic{mmsolve2d-ss}{mmSolve2D Screen Shot}
\end{center}

\sssechead{Overview}
The application \app{mmSolve2D} is a micromagnetic computation engine
capable of solving problems defined on two-dimensional square grids of
three-dimensional spins.  Within the \hyperrefhtml{\OOMMF\
architecture}{\OOMMF\ architecture (see Sec.~}{)}{sec:arch},
\app{mmSolve2D} is both a server and a client application.
\app{mmSolve2D} is a client of 
problem description server applications, data table display and storage
applications, and vector field display and storage applications.  
\app{mmSolve2D} is the server of a solver control service for which the
only client is \hyperrefhtml{\app{mmLaunch}}{\app{mmLaunch}
(Sec.~}{)}{sec:mmlaunch}\index{application!mmLaunch}.  It is through
this service that \app{mmLaunch} provides a user interface window (shown
above) on behalf of \app{mmSolve2D}.

\sssechead{Launching}
\app{mmSolve2D} may be started either by selecting the
\btn{mmSolve2D} button on \htmlref{mmLaunch}{sec:mmlaunch}, or from the
command line via
\begin{verbatim}
tclsh oommf.tcl mmSolve2D [standard options] [-restart <0|1>]
\end{verbatim}
\begin{description}
\item[\optkey{-restart \boa 0\pipe 1\bca\index{simulation~2D!restarting}}]
  Affects the behavior of the solver
  when a new problem is loaded.  Default value is 0.  When launched with
  \cd{-restart 1}, the solver will look for \fn{\textit{basename}.log}
  and \fn{\textit{basename*}.omf} files to restart a previous run from
  the last saved state (where \fn{\textit{basename}} is the ``Base
  Output Filename'' specified in the input {\latex{\MIF~1.1 problem
  specification file
  (Sec.~\ref{sec:mif1format})).}}{\html{\htmlref{\MIF~1.1}{sec:mif1outspec}
  problem specification file).}}  If these files cannot be found, then a
  warning is issued and the solver falls back to the default behavior
  (\cd{-restart 0}) of starting the problem from scratch.  The specified
  \cd{-restart} setting holds for \textbf{all} problems fed to the
  solver, not just the first.  (There is currently no interactive way to
  change the value of this switch.)
\end{description}

Since \app{mmSolve2D}\index{mmLaunch~user~interface} does not present
any user interface window of its own, it depends on
\app{mmLaunch}\index{application!mmLaunch} to provide an interface on
its behalf.  The entry for an instance of \app{mmSolve2D} in the
\btn{Threads}\index{threads} column of any running copy of
\app{mmLaunch} has a checkbutton next to it.  This button toggles the
presence of a user interface window through which the user may control
that instance of \app{mmSolve2D}.  The user interface window is divided
into panels, providing user interfaces to the
\btn{Inputs}, \btn{Outputs}, and \btn{Controls} of \app{mmSolve2D}.

\sssechead{Inputs}
The top panel of the user interface window may be opened and closed
by toggling the \btn{Inputs} checkbutton.  When open, the
\btn{Inputs} panel reveals two subpanels.  The left subpanel 
contains a list of the inputs required by \app{mmSolve2D}.  There is
only one item in the list: \btn{ProblemDescription}.  When
\btn{ProblemDescription} is selected, the right subpanel (labeled
\btn{Source Threads}\index{threads}) displays a list of applications
that can supply a problem description.  The user selects from among the
listed applications the one from which \app{mmSolve2D} should request a
problem description.

\sssechead{Outputs}
When \app{mmSolve2D} has outputs available to be controlled, a
\btn{Scheduled Outputs} checkbutton appears in the user interface
window.  Toggling the \btn{Scheduled Outputs} checkbutton causes a
bottom panel to open and close in the user interface window.  When open,
the \btn{Scheduled Outputs} panel contains three subpanels.  The
\btn{Outputs} subpanel is filled with a list of the types of output
\app{mmSolve2D} can generate while solving the loaded problem.  The
three elements in this list are \btn{TotalField}, for the output of a
vector field\index{file!vector~field} representing the total effective
field, \btn{Magnetization}, for the output of a vector field
representing the current magnetization state of the grid of spins, and
\btn{DataTable}, for the output of a table of data
values\index{file!data~table} describing other quantities of interest
calculated by \app{mmSolve2D}.

Upon selecting one of the output types from the \btn{Outputs} subpanel,
a list of applications appears in the
\btn{Destination Threads}\index{threads} subpanel which provide a
display and/or storage service for the type of output selected.  The
user may select from this list those applications to which the selected
type of output should be sent.

For each application selected, a final interface is displayed in the
\btn{Schedule}\index{output~schedule} subpanel.  Through this interface
the user may set the schedule according to which the selected type of
data is sent to the selected application for display or storage.  The
schedule is described relative to events in \app{mmSolve2D}.  An
\btn{Iteration}~event occurs at every step in the solution of the ODE.
A \btn{ControlPoint}\index{simulation~2D!control~point} event occurs
whenever the solver determines that a control point specification is
met.  (Control point specs are discussed in the {\htmlref{Experiment
parameters}{sec:expparams}} paragraph in the {\hyperrefhtml{\MIF~1.1
documentation}{\MIF~1.1 documentation (Sec.~}{)}{sec:mif1format}}, and
are triggered by solver equilibrium, simulation time, and iteration
count conditions.)  An {\btn{Interactive}} event occurs for a particular
output type whenever the corresponding ``Interactive Outputs'' button is
clicked in the {\btn{Runtime Control}} panel.  The \btn{Interactive}
schedule gives the user the ability to interactively force data to be
delivered to selected display and storage applications.  For the
\btn{Iteration} and \btn{ControlPoint} events, the granularity of the
output delivery schedule is under user control.  For example, the user
may elect to send vector field data describing the current magnetization
state to an \htmlref{\app{mmDisp}}{sec:mmdisp}\index{application!mmDisp}
instance for display every 25 iterations of the ODE, rather than every
iteration.

The quantities included in \btn{DataTable} output produced by
\app{mmSolve2D} include:
\begin{itemize}
\item {\bf Iteration:}\index{iteration} The iteration count of the ODE
      solver.
\item {\bf Field Updates:}\index{field!update~count} The number of times the
      ODE solver has calculated the effective field.
\item {\bf Sim Time (ns):}\index{simulation~2D!time} The elapsed simulated
      time.
\item {\bf Time Step (ns):}\index{time~step} The interval of simulated
      time spanned by the last step taken in the ODE solver.
\item {\bf Step Size:}\index{step~size} The magnitude of the last step
        taken by the ODE solver as a normalized value.  (This is
        currently the time step in seconds, multiplied by the
        gyromagnetic ratio times the damping coefficient times $M_s$.)
\item {\bf Bx, By, Bz (mT):}\index{field!applied} The $x$, $y$, and $z$
        components of the \hyperrefhtml{nominal applied field}{nominal
        applied field (see Sec.~}{)}{sec:expparams}.
\item {\bf B (mT):} The magnitude of the nominal applied field (always non-negative).
\item {\bf \pipe m x h\pipe:}\index{simulation~2D!mxh}  The maximum of the
        point-wise quantity
        \html{$|\vM\times\vH_{eff}|/M_s^2$}\latex{$\|\vM\times\vH_{\rm
        eff}\|/M_s^2$} over all the spins.  This ``torque'' value is
        used to test convergence to an equilibrium state (and raise
        control point --torque events).
\item {\bf Mx/Ms, My/Ms, Mz/Ms:}\index{magnetization} The $x$,
        $y$, and $z$ components of the average magnetization of the
        magnetically active elements of the simulated part.
\item {\bf Total Energy (J/m${}^3$):}\index{energy!total} The total
        average energy density for the magnetically active elements of
        the simulated part.
\item {\bf Exchange Energy (J/m${}^3$):}\index{energy!exchange} The
        component of the average energy density for the magnetically
        active elements of the simulated part due to exchange
        interactions.
\item {\bf Anisotropy Energy (J/m${}^3$):}\index{energy!anisotropy} The
        component of the average energy density for the magnetically
        active elements of the simulated part due to crystalline and
        surface anisotropies.
\item {\bf Demag Energy (J/m${}^3$):}\index{energy!demag} The component
        of the average energy density for the magnetically active
        elements of the simulated part due to self-demagnetizing fields.
\item {\bf Zeeman Energy (J/m${}^3$):}\index{energy!Zeeman}  The
        component of average energy density for the magnetically active
        elements of the simulated part due to interaction with the
        applied field.
\item {\bf Max Angle:}\index{max~angle}  The maximum angle (in degrees)
        between the magnetization orientation of any pair of neighboring
        spins in the grid.  (The neighborhood of a spin is the same as
        that defined by the exchange energy calculation.)
\end{itemize}
In addition, the solver automatically keeps a log file\index{file!log}
that records the input problem specification and miscellaneous runtime
information.  The name of this log file is \fn{\textit{basename}.log},
where \fn{\textit{basename}} is the ``Base Output Filename'' specified
in the input problem specification.  If this file already exists, then
new entries are appended to the end of the file.

\sssechead{Controls}
The middle section of the user interface window contains a series of
buttons providing user control over the
solver\index{simulation~2D!interactive~control}.  After a problem
description server application has been selected, the \btn{LoadProblem}
button triggers a fetch of a problem description from the selected
server.  The \btn{LoadProblem} button may be selected at any time to
(re-)load a problem description from the currently selected server.
After loading a new problem the solver goes automatically into a paused
state.  (If no problem description server is selected when the
\btn{LoadProblem} button is invoked, nothing will happen.)  The
\btn{Reset} button operates similarly, except that the current problem
specifications are used.

Once a problem is loaded, the solver can be put into any of three
states: run, relax and pause.  Selecting \btn{Relax} puts the solver
into the ``relax'' state, where it runs until a control point is
reached, after which the solver pauses.  If the \btn{Relax} button is
reselected after reaching a control point, then the solver will simply
re-pause immediately.  The \btn{Field+} or \btn{Field--} button must be
invoked to change the applied field state. (Field state schedules are
discussed below.)  The \btn{Run} selection differs in that when a
control point is reached, the solver automatically steps the nominal
applied field to the next value, and continues.  In ``run'' mode the
solver will continue to process until there are no more applied field
states in the problem description.  At any time the \btn{Pause} button
may be selected to pause the solver.  The solver will stay in this state
until the user reselects either \btn{Run} or \btn{Relax}.  The current
state of the solver is indicated in the \btn{Status} line in the center
panel of the user interface window.

The problem description (\MIF~1.x format)\index{file!MIF~1.x}
specifies a fixed \hyperrefhtml{applied field schedule}{applied field
schedule (see Sec.~}{)}{sec:expparams}.  This schedule defines an
ordered list of applied fields, which the solver in ``run'' mode steps
through in sequence.  The \btn{Field--} and \btn{Field+} buttons allow
the user to interactively adjust the applied field sequence.  Each click
on the
\btn{Field+} button advances forward one step through the specified
schedule, while \btn{Field--} reverses that process.  In general, the
step direction is {\em not} related to the magnitude of the applied
field.  Also note that hitting these buttons does not generate a
``ControlPoint'' event.  In particular, if you are manually accelerating
the progress of the solver through a hysteresis loop, and want to send
non-ControlPoint data to a display or archive widget before advancing
the field, then you must use the appropriate ``Interactive Output''
button.

The second row of buttons in the interaction control panel,
\btn{TotalField}, \btn{Magnetization} and \btn{DataTable}, allow the
user to view the current state of the solver at any time.  These buttons
cause the solver to send out data of the corresponding type to all
applications for which the ``Interactive'' schedule button for that 
data type has been selected, as discussed in the Outputs section above.

At the far right of the solver controls is the \btn{Exit} button, which
terminates\index{simulation~2D!termination} \app{mmSolve2D}.  Simply
closing the user interface window does not terminate \app{mmSolve2D},
but only closes the user interface window.  To kill the solver the
\btn{Exit} button must be pressed.

\sssechead{Details}
Given a problem description, \app{mmSolve2D} integrates the
Landau-Lifshitz equation\index{ODE!Landau-Lifshitz}~\cite{gilbert55,landau35}\\
\begin{equation}
\htmlimage{antialias}
  \frac{d\vM}{dt} = -|\bar{\gamma}|\,\vM\times\vH_{\rm eff}
   - \frac{|\bar{\gamma}|\alpha}{M_s}\,
     \vM\times\left(\vM\times\vH_{\rm eff}\right),
\label{eq:llode}
\end{equation}
where
\latex{
\begin{eqnarray*}
  \vM       && \mbox{is the pointwise magnetization (A/m),} \\
  \vH_{\rm eff} && \mbox{is the pointwise effective field (A/m),} \\
  \bar{\gamma}  && \mbox{is the Landau-Lifshitz gyromagnetic ratio
    (m/(A$\cdot$s)),} \\
  \alpha    && \mbox{is the damping coefficient (dimensionless).}
\end{eqnarray*}
} % close latex
% Use BLOCKQUOTE until mmHelp supports tables.
\html{
\begin{quotation}
  $\vM$              is the pointwise magnetization (A/m),\\
  $\vH_{\mbox{eff}}$ is the pointwise effective field (A/m),\\
  \abovemath{\bar{\gamma}} is the Landau-Lifshitz gyromagnetic ratio
                        (m/(A\begin{rawhtml}&#183;\end{rawhtml}s)),\\
  $\alpha$           is the damping coefficient (dimensionless).
\end{quotation}
} % close html
\latex{(Compare to (\ref{eq:oxsllode}), page~\pageref{eq:oxsllode}.)}%
\html{(See also the discussion of the \htmlref{Landau-Lifshitz-Gilbert
equations}{eq:oxsllode} in the Oxs documentation.)}

% Note 1: Don't use transparent images, because mmHelp renders them
%   rather slowly.
% Note 2: makeimage (maybe +tabular?) breaks on newer systems, with
%   latex2html v1.71 and v1.68.  OTOH, v1.68 works on older systems,
%   so it is not clear what the problem is.  The error message is:
%          panic: end_shift at /usr/local/bin/latex2html line 11720.
%   This is with Perl v5.8.0.  So, we have to render the table
%   directly.  This is actually preferred with real HTML browsers,
%   but as of this writing (Dec-2004) mmHelp doesn't do tables.
The effective field\index{field!effective} is defined as
\html{\begin{center}}
%\begin{makeimage}
%\htmlimage{no_transparent}
\begin{displaymath}
  \vH_{\rm eff} = -\mu_0^{-1} \frac{\partial E}{\partial\vM}.
\end{displaymath}
%\end{makeimage}
\html{\end{center}}
The average energy density\index{energy!total} $E$ is a function of
$\vM$ specified by Brown's equations \cite{brown78}, including
anisotropy\index{energy!anisotropy},
exchange\index{energy!exchange}, self-magnetostatic
(demagnetization)\index{energy!demag} and applied
field\index{energy!Zeeman} (Zeeman) terms.

The micromagnetic problem is impressed upon a regular 2D
grid\index{grid} of squares, with 3D magnetization spins positioned at
the centers of the cells.  Note that the constraint that the grid be
composed of square elements takes priority over the requested size of
the grid.  The actual size of the grid used in the computation will be
the nearest integral multiple of the grid's cell size to the requested
size.  It is important when comparing the results from grids with
different cell sizes to account for the possible change in size of the
overall grid.

The anisotropy and applied field energy terms are calculated
assuming constant magnetization in each cell.  The exchange energy is
calculated using the eight-neighbor bilinear interpolation described in
\cite{donahue97}, with Neumann boundary conditions.  The more common
four-neighbor scheme is available as a compile-time option.  Details can
be found in the source-code file \fn{oommf/app/mmsolve/magelt.cc}.

The self-magnetostatic field is calculated as the convolution of the
magnetization against a kernel that describes the cell to cell
magnetostatic interaction.  The convolution is evaluated using fast
Fourier transform (FFT)\index{FFT} techniques.  Several kernels are
supported; these are selected as part of the problem description in
\hyperrefhtml{\MIF~1.x format}{\MIF~1.x format; for details see Sec.~}{:
Demag specification}{sec:mifdemagspec}.  Each kernel represents a different
interpretation of the discrete magnetization.  The recommended model is
\cd{ConstMag}, which assumes the magnetization is constant in each cell,
and computes the average demagnetization field through the cell using
formulae from \cite{newell93} and \cite{aharoni98}.

The Landau-Lifshitz ODE\index{ODE!Landau-Lifshitz} (\ref{eq:llode}) is
integrated using a second order
predictor-corrector\index{ODE!predictor-corrector} technique of the
Adams type.  The right side of (\ref{eq:llode}) at the current and
previous step is extrapolated forward in a linear fashion, and is
integrated across the new time interval to obtain a quadratic prediction
for $\vM$ at the next time step.  At each stage the spins are
renormalized to $M_s$ before evaluating the energy and effective
fields.  The right side of (\ref{eq:llode}) is evaluated at the
predicted $\vM$, which is then combined with the value at the current
step to produce a linear interpolation of $d\vM/dt$ across the new
interval.  This is then integrated to obtain the final estimate of $\vM$
at the new step.  The local (one step) error of this procedure should be
\html{$O(dt^3)$}\latex{$O(\Delta t^3)$}.

The step is accepted if the total energy of the system decreases, and
the maximum error between the predicted and final $\vM$ is smaller than
a nominal value.  If the step is rejected, then the step size is reduced
and the integration procedure is repeated.  If the step is accepted,
then the error between the predicted and final $\vM$ is used to adjust
the size of the next step.  No fixed ratio between the previous and
current time step is assumed.

A fourth order Runge-Kutta\index{ODE!Runge-Kutta} solver is used to
prime the predictor-corrector solver, and is used as a backup in case
the predictor-corrector fails to find a valid step.  The Runge-Kutta
solver is not selectable as the primary solver at runtime, but may be so
selected at compile time by defining the \cd{RUNGE\_KUTTA\_ODE} macro.
See the file \fn{oommf/app/mmsolve/grid.cc} for all details of the
integration procedure.

For a given applied field, the integration continues until a
\htmlref{control point}{sec:expparams}
\latex{(cf.\ Experiment parameters, Sec.~\ref{sec:expparams})} is
reached.  A control point\index{simulation~2D!control~point} event may
be raised by the ODE iteration count, elapsed simulation time, or by the
maximum value of {\html{$|\vM\times\vH_{\mbox{eff}}|/M_s^2$}}
{\latex{$\|\vM\times\vH_{\mbox{eff}}\|/M_s^2$}} dropping below a
specified control point --torque value (implying an equilibrium state
has been reached).

Depending on the problem size, \app{mmSolve2D} can require a good deal
of working memory.  The exact amount depends on a number of factors, but
a reasonable estimate is 5~MB + 1500~bytes per cell.  For example, a
1~\micrometer~$\times$~1~\micrometer\ part discretized with 5~nm cells will require
approximately 62~MB.

\sssechead{Known Bugs}
\app{mmSolve2D} requires the damping coefficient to be non-zero.
See the \hyperrefhtml{\MIF~1.1 documentation}{\MIF~1.1 documentation
(Sec.~}{)}{sec:mif1format} for details on specifying the damping
coefficient.

When multiple copies of
\app{mmLaunch}\index{mmLaunch~user~interface}\index{application!mmLaunch}
are used, each can have its own interface to a running copy of
\app{mmSolve2D}.  When the interface presented by one copy of
\app{mmLaunch} is used to set the output schedule in \app{mmSolve2D},
those settings are not reflected in the interfaces presented by other
copies of \app{mmLaunch}.  For example, although the first interface
sets a schedule that DataTable data is to be sent to an instance of
\app{mmGraph}\index{application!mmGraph} every third Iteration, there is
no indication of that schedule presented to the user in the second
interface window.  It is unusual to have more than one copy of
\app{mmLaunch} running simultaneously. However, this bug also appears
when one copy of \app{mmLaunch} is used to load a problem and start a
solver, and later a second copy of \app{mmLaunch} is used to monitor the
status of that running solver.

%\index{platform!Windows!network socket bug|(}
%A bug in the network traffic handling code\index{network~socket} of
%\Tcl\ on \Windows~9X systems can sometimes interfere
%with communications between the control interface of \app{mmSolve2D} and
%the actual computation engine.  If \app{mmSolve2D} is sending out data
%to two or more data display services every iteration, the resulting
%network traffic can ``crowd out'' the receipt of control messages from
%the control interface.  You may observe this as a long delay between the
%time you click the \btn{Pause} button and the time the solver stops
%iterating.  This bug first appeared in \Tcl\ release 8.0.3, and remained
%through \Tcl\ release 8.1.1.  It is fixed in
%\Tcl\ releases 8.2 and later, which we recommend for \OOMMF\ users on
%\Windows~9X systems.  Other platforms do not have this problem.
%\index{platform!Windows!network socket bug|)}

%%%%%%%%%%%%%%%%%%%%%%%%%%%%%%%%%%%%%%%%%%%%%%%%%%%%%%%%%%%%%%%%%%%%%%%%
\subsection{\OOMMF\ 2D Micromagnetic Solver Batch System}\label{sec:obs}
\index{simulation~2D}\index{application!OOMMF~Batch~System}

The \OOMMF\ Batch System (\OBS) provides a scriptable interface
to the same micromagnetic solver engine used by
\hyperrefhtml{\app{mmSolve2D}}{\app{mmSolve2D}
(Sec.~}{)}{sec:mmsolve2d}\index{application!mmSolve2D}, in the
form of three \Tcl\ applicatons
(\app{batchmaster}, \app{batchslave}, and
\app{batchsolve}) that provide support for complex job scheduling.
All \OBS\ script files are in the \OOMMF\ distribution directory
\fn{app/mmsolve/scripts}.

Unlike much of the \OOMMF\ package, the \OBS\ is meant to be
driven primarily from the command line or shell (batch) script.
\OBS\ applications are launched from the command line using the
\hyperrefhtml{bootstrap application}{bootstrap application
(Sec.~}{)}{sec:cll}.

\subsubsection{2D Micromagnetic Solver Batch Interface: batchsolve}%
\label{sec:batchsolve}

\sssechead{Overview}
The application
\app{batchsolve}\index{application!batchsolve} provides a simple
command line interface to the \OOMMF\ 2D micromagnetic solver engine.

\sssechead{Launching}
The application \app{batchsolve} is launched by the command line:
\begin{verbatim}
tclsh oommf.tcl batchsolve [standard options] 
   [-end_exit <0|1>] [-end_paused] [-interface <0|1>] \
   [-restart <0|1>] [-start_paused] [file]
\end{verbatim}
where
\begin{description}
\item[\optkey{-end\_exit \boa 0\pipe 1\bca}]
  Whether or not to explicitly call exit at bottom of \fn{batchsolve.tcl}.
  When launched from the command line, the default is to exit after
  solving the problem in \cd{file}.  When sourced into another script,
  like \fn{batchslave.tcl}, the default is to wait for the caller script
  to provide further instructions.
\item[\optkey{-interface \boa 0\pipe 1\bca}]
  Whether to register with the account service
  directory\index{account~service~directory} application, so
  that \hyperrefhtml{\app{mmLaunch}}{\app{mmLaunch}
  (Sec.~}{)}{sec:mmlaunch}\index{application!mmLaunch}, can provide
  an interactive interface.  Default = 1 (do register), which will
  automatically start account service directory and
  host service directory applications as necessary.
\item[\optkey{-start\_paused}]
  Pause solver after loading problem.
\item[\optkey{-end\_paused}]
  Pause solver and enter event loop at bottom of {\fn{batchsolve.tcl}}
  rather than just falling off the end (the effect of which will
  depend on whether or not \Tk\ is loaded).
\item[\optkey{-restart \boa 0\pipe 1\bca\index{simulation~2D!restarting}}]
  Determines solver behavior when a new problem is loaded.  If 1, then
  the solver will look for \fn{\textit{basename}.log}\index{file!log}
  and \fn{\textit{basename*}.omf}\index{file!magnetization} files to
  restart a previous run from the last
  saved state (where \fn{\textit{basename}} is the ``Base Output
  Filename'' specified in the input problem specification).  If these
  files cannot be found, then a warning is issued and the solver falls
  back to the default behavior (equivalent to \cd{-restart 0}) of
  starting the problem from scratch.  The specified \cd{-restart}
  setting holds for \textbf{all} problems fed to the solver, not just
  the first.
\item[\optkey{file}]
  Immediately load and run the specified \MIF~1.x
  file\index{file!MIF~1.x}.
\end{description}

The input file {\bf\fn{file}} should contain a
\hyperrefhtml{Micromagnetic Input Format}{Micromagnetic Input
Format (Sec.~}{)}{sec:mifformat} 1.x problem
description\index{file!MIF~1.x}, such as produced by
\hyperrefhtml{\app{mmProbEd}}{\app{mmProbEd}
(Sec.~}{)}{sec:mmprobed}\index{application!mmProbEd}.  The batch solver
searches several directories for this file, including the current
working directory\index{working~directory}, the \fn{data} and
\fn{scripts} subdirectories, and parallel directories relative to the
directories \fn{app/mmsolve} and
\fn{app/mmpe} in the \OOMMF\ distribution.  Refer to the
\cd{mif\_path} variable in \fn{batchsolve.tcl} for the complete list.

If \cd{-interface} is set to 1 (enabled), \app{batchsolve} registers
with the account service directory\index{account~service~directory}
application, and \app{mmLaunch} will be able to provide an interactive
interface.  Using this interface, \app{batchsolve} may be controlled in
a manner similar to \hyperrefhtml{\app{mmSolve2D}}{\app{mmSolve2D}
(Sec.~}{)}{sec:mmsolve2d}\index{application!mmSolve2D}.  The
interface\index{simulation~2D!interactive~control} allows you to pause,
un-pause, and terminate\index{simulation~2D!termination} the current
simulation, as well as to attach data display applications to monitor
the solver's progress.  If more interactive control is needed,
\app{mmSolve2D} should be used.

If \cd{-interface} is 0 (disabled), \app{batchsolve} does not register,
leaving it without an interface, unless it is sourced into another
script (e.g., \fn{batchslave.tcl}\index{application!batchslave}) that
arranges for an interface on the behalf of \app{batchsolve}.

Use the {\bf\verb|-start_paused|} switch to monitor the progress of
\app{batchsolve} from the very start of a simulation.  With this
switch the solver will be paused immediately after loading the
specified \MIF\ file, so you can bring up the interactive interface
and connect display applications before the simulation begins.  Start the
simulation by selecting the \btn{Run} command from the interactive
interface.  This option cannot be used if \cd{-interface} is disabled.

The {\bf\verb|-end_paused|} switch insures that the solver does
not automatically terminate after completing the specified
simulation.  This is not generally useful, but may find application
when \app{batchsolve} is called from inside a \Tcl -only wrapper
script.

Note on \Tk\ dependence: If a problem is loaded that uses a
{\hyperrefhtml{bitmap mask file}{bitmap mask file
(Sec.~}{)}{sec:partgeometry}}\index{file!mask}\index{file!bitmap}, and
if that mask file is not in the PPM P3 (text) format, then
\app{batchsolve} will launch {\hyperrefhtml{any2ppm}{\app{any2ppm}
(Sec.~}{)}{sec:any2ppm}}\index{application!any2ppm} to convert it into
the PPM P3 format.  Since \app{any2ppm}
requires\index{requirement!Tk}\index{requirement!display} \Tk, at the
time the mask file is read a valid display must be available.  See the
\app{any2ppm} documentation for details.

\sssechead{Output}
The output may be changed by a \Tcl\ \hyperrefhtml{wrapper
script}{wrapper script (see Sec.~}{)}{sec:batchsolvepi}, but the default
output behavior of \app{batchsolve} is to write tabular text data and
the magnetization state at the control point for each applied field
step.  The tabular data are appended to the file \fn{{\em
basename}.odt}\index{file!data~table}, where {\em basename} is the
``Base Output Filename'' specified in the input
\MIF~1.x file\index{file!MIF~1.x}.  See the routine \cd{GetTextData} in
\fn{batchsolve.tcl} for details, but at present the output consists of
the solver iteration count\index{iteration}, nominal applied field
\vB\index{field!applied}, reduced average
magnetization \vm\index{magnetization}, and total
energy\index{energy!total}.  This output is in the \ODT\ file format.

The magnetization data are written to a series of \OVF\ (\OOMMF\ Vector
Field) files\index{file!vector~field},
\fn{\textit{basename}.field\textit{nnnn}.omf}, where \fn{\textit{nnnn}}
starts at \fn{0000} and is incremented at each applied
field\index{field!applied} step.  (The ASCII text header inside each
file records the nominal applied field at that step.)  These files are
viewable using \hyperrefhtml{\app{mmDisp}}{\app{mmDisp}
(Sec.~}{)}{sec:mmdisp}\index{application!mmDisp}.

The solver also automatically appends the input problem specification
and miscellaneous runtime information to the log file
\fn{{\em basename}.log}\index{file!log}.


\sssechead{Programmer's interface}\label{sec:batchsolvepi}
In addition to directly launching \app{batchsolve} from the command
line, \fn{batchsolve.tcl}  may also be sourced into another \Tcl\ script
that provides additional control structures.  Within the scheduling
system of \OBS,
\fn{batchsolve.tcl} is sourced into \app{batchslave}, which provides
additional control structures that support scheduling control by
\app{batchmaster}.
There are several variables and routines
inside \fn{batchsolve.tcl} that may be accessed and redefined from such
a wrapper script to provide enhanced functionality.

%\paragraph{Global variables}\par
\sssechead{Global variables}
\begin{description}
\item[\cd{\bf mif}] A \Tcl\ handle to a global \cd{mms\_mif} 
  object holding the problem description defined by the input
  \MIF~1.x file\index{file!MIF~1.x}.
\item[\cd{\bf solver}] A \Tcl\ handle to the \cd{mms\_solver} object.
\item[\cd{\bf search\_path}] Directory search path used by the
 \latex{\cd{FindFile} proc (see below).}\html{\htmlref{FindFile
 proc}{pt:batchsolveFindFile}.}
\end{description}
Refer to the source code and sample scripts for details on manipulation
of these variables.

%\paragraph{Batchsolve procs}\label{sec:batchsolveprocs}\par
\sssechead{Batchsolve procs}\label{sec:batchsolveprocs}
The following \Tcl\ procedures are designed for external use and/or
redefinition:
\begin{description}
\item[\cd{\bf SolverTaskInit}]
   Called at the start of each task.
\item[\cd{\bf BatchTaskIterationCallback}]
   Called after each iteration in the simulation.
\item[\cd{\bf BatchTaskRelaxCallback}]
   Called at each control point reached in the simulation.
\item[\cd{\bf SolverTaskCleanup}]
   Called at the conclusion of each task.
\begin{rawhtml}
<A NAME="pt:batchsolveFindFile"></A>
\end{rawhtml}
\item[\cd{\bf FindFile}]
   Searches the directories specified by the global variable
   \cd{search\_path} for a specified file.  The default
   \cd{SolverTaskInit} proc uses this routine to locate the requested
   input \MIF\ file.
\end{description}
\cd{SolverTaskInit} and \cd{SolverTaskCleanup} accept an arbitrary
argument list (\cd{args}), which is copied over from the \cd{args}
argument to the \cd{BatchTaskRun} and \cd{BatchTaskLaunch} procs in
\fn{batchsolve.tcl}.  Typically one copies the default procs (as needed)
into a \htmlref{task script}{sec:batchschedex}, and makes appropriate
modifications.  You may (re-)define these procs either before or after
sourcing \fn{batchsolve.tcl}\index{application!batchsolve}.
\latex{See Sec.~\ref{sec:batchschedex} for example scripts.}

\subsubsection{2D Micromagnetic Solver Batch Scheduling System}

\sssechead{Overview}
The \OBS\ supports complex scheduling of multiple batch jobs
with two applications, \app{batchmaster} and \app{batchslave}.
The user launches \app{batchmaster} and provides it with
a task script\index{task~script}.  The task script is a
\Tcl\ script that describes the set of tasks for \app{batchmaster}
to accomplish.  The work is actually done by instances of
\app{batchslave} that are launched by \app{batchmaster}.
The task script may be
modeled after the included {\fn{simpletask.tcl}} or {\fn{multitask.tcl}}
\hyperrefhtml{sample scripts}{sample scripts
(Sec.~}{)}{sec:batchschedex}.

The \OBS\ has been designed to control multiple sequential and
concurrent micromagnetic simulations, but 
\app{batchmaster} and \app{batchslave} are completely general
and may be used to schedule other types of jobs as well.

\paragraph{Master Scheduling Control: batchmaster}\par
The application \app{batchmaster}\index{simulation~2D!scheduling} is
launched by the command line:
\begin{verbatim}
tclsh oommf.tcl batchmaster [standard options] task_script \
      [host [port]]
\end{verbatim}
\begin{description}
\item[{\tt\bf task\_script}]
  is the user defined task (job) definition \Tcl\ script,
\item[{\tt\bf host}]
  specifies the network address for the master to use (default is {\em
  localhost}),
\item[{\tt\bf port}]
  is the port address for the master (default is {\em 0}, which
  selects an arbitrary open port).
\end{description}

When \app{batchmaster}\index{application!batchmaster} is run, it
sources the task script.  \Tcl\ commands in the task script
should modify the global object \cd{\$TaskInfo}
to inform the master what tasks to perform and
optionally how to launch slaves to perform those tasks.  
The easiest way to create a task script is to modify one of the 
\hyperrefhtml{included example scripts}{example scripts in
Sec.~}{}{sec:batchschedex}.  More detailed instructions are in
\html{the }\hyperrefhtml{Batch task
scripts}{Sec.~}{}{sec:batchschedtask}\html{ section}.

After sourcing the task script, \app{batchmaster} launches all the
specified slaves, initializes each with a slave initialization script,
and then feeds tasks sequentially from the task list to the slaves.
When a slave completes a task it reports back to the master and is given
the next unclaimed task.  If there are no more tasks, the slave is shut
down.  When all the tasks are complete, the master prints a summary of
the tasks and exits.

When the task script requests the launching and controlling of jobs off
the local machine, with slaves running on remote machines, then the
command line argument \fn{host} {\bf must} be set to the local machine's
network name, and the \cd{\$TaskInfo} methods \cd{AppendSlave} and
\cd{ModifyHostList} will need to be called from inside the task script.
Furthermore, \OOMMF\ does not currently supply any methods for launching
jobs on remote machines, so a task script which requests the launching
of jobs on remote machines requires a working
\verb+ssh+\index{application!ssh} command or
equivalent\index{requirement!ssh}.
\hyperrefhtml{(Details.)}{See Sec.~}{ for
details.}{sec:batchschedtask}

\paragraph{Task Control: batchslave}\par
The application \app{batchslave} may be launched by the command line:
\begin{verbatim}
tclsh oommf.tcl batchslave [standard options] \
   host port id password [auxscript [arg ...]]
\end{verbatim}
\begin{description}
\item[{\tt\bf host, port}]
  Host and port at which to contact the master to serve.
\item[{\tt\bf id, password}]
  ID and password to send to the master for identification.
\item[{\tt\bf auxscript arg ...}]
  The name of an optional script to source (which actually performs the
  task the slave is assigned), and any arguments it needs.
\end{description}

In normal operation, the user does not launch
\app{batchslave}.  Instead, instances of \app{batchslave} are
launched by \app{batchmaster} as instructed by a task script.
Although \app{batchmaster} may launch any slaves requested
by its task script, by default it launches instances of
\app{batchslave}.

The function of \app{batchslave} is to make a connection to
a master program, source the \cd{auxscript} and pass it the
list of arguments \cd{aux\_arg ...}.  Then it receives commands
from the master, and evaluates them, making use of the
facilities provided by \cd{auxscript}.  Each command is typically a
long-running one, such as solving a complete micromagnetic problem.
When each command is complete, the \app{batchslave} reports back to 
its master program, asking for the next command.  When the master
program has no more commands \app{batchslave} terminates.

Inside \app{batchmaster}, each instance of \app{batchslave} is
launched by evaluating a \Tcl\ command.  This command is called
the spawn command, and it may be redefined by the task script
in order to completely control which slave applications are launched
and how they are launched.  When \app{batchslave} is to be launched,
the spawn command might be:
\begin{rawhtml}<BLOCKQUOTE>\end{rawhtml}
%begin<latexonly>
\begin{quote}
%end<latexonly>
\begin{verbatim}
exec tclsh oommf.tcl batchslave -tk 0 -- $server(host) $server(port) \
   $slaveid $passwd batchsolve.tcl -restart 1 &
\end{verbatim}
%begin<latexonly>
\end{quote}
%end<latexonly>
\begin{rawhtml}</BLOCKQUOTE>\end{rawhtml}
The \Tcl\ command \cd{exec} is used to launch subprocesses.  When
the last argument to \cd{exec} is \cd{\&}, the subprocess runs in
the background.  The rest of the spawn command should look familiar
as the command line syntax for launching \app{batchslave}.

The example spawn command above cannot be completely provided by
the task script, however, because parts of it are only known
by \app{batchmaster}.  Because of this, the task script should
define the spawn command using ``percent variables'' which are
substituted by \app{batchmaster}.  Continuing the example, the task
script provides the spawn command:
\begin{rawhtml}<BLOCKQUOTE>\end{rawhtml}
%begin<latexonly>
\begin{quote}
%end<latexonly>
\begin{verbatim}
exec %tclsh %oommf batchslave -tk 0 %connect_info \
   batchsolve.tcl -restart 1
\end{verbatim}
%begin<latexonly>
\end{quote}
%end<latexonly>
\begin{rawhtml}</BLOCKQUOTE>\end{rawhtml}
\app{batchmaster} replaces \cd{\%tclsh} with the path to \fn{tclsh},
and \cd{\%oommf} with the path to the \OOMMF\ bootstrap application.  
It also replaces \cd{\%connect\_info} with the five arguments from \verb+--+
through \cd{\$password} that provide \app{batchslave}
the hostname and port where \app{batchmaster} is waiting for
it to report to, and the ID and password it should pass back.
In this example, the task script instructs \app{batchslave} to source the
file \fn{batchsolve.tcl} and pass it the arguments \cd{-restart 1}.
Finally, \app{batchmaster} always appends the argument \cd{\&} to
the spawn command so that all slave applications are launched in the
background.

The communication protocol\index{communication~protocol} between
\app{batchmaster} and \app{batchslave} is evolving and is not
described here.  Check the source code for the latest details.

\paragraph{Batch Task Scripts}\label{sec:batchschedtask}\par
The application \app{batchmaster} 
creates an instance of a \cd{BatchTaskObj} object with
the name \cd{\$TaskInfo}.  The task script\index{task~script} uses
method calls to this object to set up tasks to be performed.  The only
required call is to the \cd{AppendTask} method, e.g.,
\begin{rawhtml}<BLOCKQUOTE>\end{rawhtml}
%begin<latexonly>
\begin{quote}
%end<latexonly>
\begin{verbatim}
$TaskInfo AppendTask A "BatchTaskRun taskA.mif"
\end{verbatim}
%begin<latexonly>
\end{quote}
%end<latexonly>
\begin{rawhtml}</BLOCKQUOTE>\end{rawhtml}
This method expects two arguments, a label for the task (here ``A'') and
a script to accomplish the task.  
The script will be passed across a
network socket\index{network~socket} from 
\app{batchmaster} to a slave application, and
then the script will be interpreted by the slave.  In particular, keep
in mind that the file system seen by the script will be that of the
machine on which the slave process is running.

This example uses the default \fn{batchsolve.tcl} procs to run the
simulation defined by the \fn{taskA.mif} \MIF~1.x
file\index{file!MIF~1.x}.  If you want to make changes to the \MIF\
problem specifications on the fly, you will need to modify the default
procs.  This is done by creating a slave initialization script, via the
call
\begin{rawhtml}<BLOCKQUOTE>\end{rawhtml}
%begin<latexonly>
\begin{quote}
%end<latexonly>
\begin{verbatim}
$TaskInfo SetSlaveInitScript { <insert script here> }
\end{verbatim}
%begin<latexonly>
\end{quote}
%end<latexonly>
\begin{rawhtml}</BLOCKQUOTE>\end{rawhtml}
The slave initialization script does global initializations, and also
usually redefines the \cd{SolverTaskInit} proc; optionally the
\cd{BatchTaskIterationCallback}, \cd{BatchTaskRelaxCallback} and
\cd{SolverTaskCleanup} procs may be redefined as well.  At the start of
each task \cd{SolverTaskInit} is called by \cd{BatchTaskRun} (in
\fn{batchsolve.tcl}), after each iteration
\cd{BatchTaskIterationCallback} is executed, at each control
point\index{simulation~2D!control~point} \cd{BatchTaskRelaxCallback} is
run, and at the end of each task \cd{SolverTaskCleanup} is called.
\cd{SolverTaskInit} and \cd{SolverTaskCleanup} are passed the arguments
that were passed to \cd{BatchTaskRun}.  A simple \cd{SolverTaskInit}
proc could be 
\begin{rawhtml}<BLOCKQUOTE>\end{rawhtml}
%begin<latexonly>
\begin{quote}
%end<latexonly>
\begin{verbatim}
proc SolverTaskInit { args } {
   global mif basename outtextfile
   set A [lindex $args 0]
   set outbasename "$basename-A$A"
   $mif SetA $A
   $mif SetOutBaseName $outbasename
   set outtextfile [open "$outbasename.odt" "a+"]
   puts $outtextfile [GetTextData header \
         "Run on $basename.mif, with A=[$mif GetA]"]
}
\end{verbatim}
%begin<latexonly>
\end{quote}
%end<latexonly>
\begin{rawhtml}</BLOCKQUOTE>\end{rawhtml}
This proc receives the exchange constant \cd{A}
for this task on the argument list, and makes use of the global
variables \cd{mif} and \cd{basename}.  (Both should be initialized in
the slave initialization script outside the \cd{SolverTaskInit} proc.)
It then stores the requested value of \cd{A} in the
\cd{mif}\index{file!MIF~1.x} object, sets up the base filename to use for
output, and opens a text file to which tabular
data\index{file!data~table} will be appended.  The handle to this text
file is stored in the global \cd{outtextfile}, which is closed by the
default \cd{SolverTaskCleanup} proc.  A corresponding task script could
be
\begin{rawhtml}<BLOCKQUOTE>\end{rawhtml}
%begin<latexonly>
\begin{quote}
%end<latexonly>
\begin{verbatim}
$TaskInfo AppendTask "A=13e-12 J/m" "BatchTaskRun 13e-12"
\end{verbatim}
%begin<latexonly>
\end{quote}
%end<latexonly>
\begin{rawhtml}</BLOCKQUOTE>\end{rawhtml}
which runs a simulation with \cd{A} set to
\latex{$13\times 10^{-12}$~J/m.}\html{13e-12 J/m.}
This example is taken from the \fn{multitask.tcl}
\hyperrefhtml{sample script}{script in Sec.~}{}{sec:batchschedex}.  (For
commands accepted by \cd{mif} objects, see the file \fn{mmsinit.cc}.
Another object than can be gainfully manipulated is \cd{solver}, which
is defined in \fn{solver.tcl}.)

If you want to run more than one task at a time, then the
\cd{\$TaskInfo} method \cd{AppendSlave} will have to be invoked.  This
takes the form
\begin{rawhtml}<BLOCKQUOTE>\end{rawhtml}
%begin<latexonly>
\begin{quote}
%end<latexonly>
\begin{verbatim}
$TaskInfo AppendSlave <spawn count> <spawn command>
\end{verbatim}
%begin<latexonly>
\end{quote}
%end<latexonly>
\begin{rawhtml}</BLOCKQUOTE>\end{rawhtml}
where \cd{<spawn command>} is the command to launch the slave
process, and \cd{<spawn count>} is the number of slaves to launch
with this command.  (Typically \cd{<spawn count>} should not be
larger than the number of processors on the target system.)  The default
value for this item (which gets overwritten with the first call to
\cd{\$TaskInfo AppendSlave}) is
\begin{rawhtml}<BLOCKQUOTE>\end{rawhtml}
%begin<latexonly>
\begin{quote}
%end<latexonly>
\begin{verbatim}
 1 {Oc_Application Exec batchslave -tk 0 %connect_info batchsolve.tcl}
\end{verbatim}
%begin<latexonly>
\end{quote}
%end<latexonly>
\begin{rawhtml}</BLOCKQUOTE>\end{rawhtml}
The \Tcl\ command \cd{Oc\_Application Exec} is supplied by \OOMMF\
and provides access to the same application-launching capability
that is used by the \OOMMF\ 
\hyperrefhtml{bootstrap application}{bootstrap application
(Sec.~}{)}{sec:cll}.  Using a \cd{<spawn command>} of
\cd{Oc\_Application Exec} instead of \cd{exec \%tclsh \%oommf}
saves the spawning of an additional process.
The default \cd{<spawn command>} 
launches the \app{batchslave}
application, with connection information provided by \app{batchmaster}, and
using the \cd{auxscript} \fn{batchsolve.tcl}.  

Before evaluating the \cd{<spawn command>}, \app{batchmaster}
applies several percent-style substitutions useful in slave
launch scripts: \%tclsh, \%oommf, \%connect\_info, \%oommf\_root, and
\%\%.  The first is the \Tcl\ shell to use, the second is an absolute
path to the \OOMMF\ bootstrap program on the master machine, the third
is connection information needed by the \fn{batchslave} application, the
fourth is the path to the \OOMMF\ root directory on the master machine,
and the last is interpreted as a single percent.
\app{batchmaster} automatically appends the argument
\cd{\&} to the
\cd{<spawn command>} so that the slave applications
are launched in the background.

To launch \app{batchslave} on a remote host, use \fn{ssh}\index{application!ssh}
in the spawn command, e.g.,
\begin{rawhtml}<BLOCKQUOTE>\end{rawhtml}
%begin<latexonly>
\begin{quote}
%end<latexonly>
\begin{verbatim}
$TaskInfo AppendSlave 1 {exec ssh foo tclsh oommf/oommf.tcl \
      batchslave -tk 0 %connect_info batchsolve.tcl}
\end{verbatim}
%begin<latexonly>
\end{quote}
%end<latexonly>
\begin{rawhtml}</BLOCKQUOTE>\end{rawhtml}
This example assumes \fn{tclsh} is in the execution path on the remote
machine \fn{foo}, and \OOMMF\ is installed off of your home directory.
In addition, you will have to add the machine \fn{foo} to the host
connect list with
\begin{rawhtml}<BLOCKQUOTE>\end{rawhtml}
%begin<latexonly>
\begin{quote}
%end<latexonly>
\begin{verbatim}
$TaskInfo ModifyHostList +foo
\end{verbatim}
%begin<latexonly>
\end{quote}
%end<latexonly>
\begin{rawhtml}</BLOCKQUOTE>\end{rawhtml}
and \fn{batchmaster} must be run with the network interface specified
as the server host (instead of the default \fn{localhost}), e.g.,
\begin{rawhtml}<BLOCKQUOTE>\end{rawhtml}
%begin<latexonly>
\begin{quote}
%end<latexonly>
\begin{verbatim}
tclsh oommf.tcl batchmaster multitask.tcl bar
\end{verbatim}
%begin<latexonly>
\end{quote}
%end<latexonly>
\begin{rawhtml}</BLOCKQUOTE>\end{rawhtml}
where \fn{bar} is the name of the local machine.

This may seem a bit complicated, but the examples in the
next section should make things clearer.

\paragraph{Sample task scripts}\label{sec:batchschedex}\par
The
\begin{rawhtml}
<A NAME="pt:batchschedsimpletask"></A>
\end{rawhtml}
\hyperrefhtml{first sample task script}{first sample task script
(Fig.~}{)}{fig:batchschedsimpletask} is a simple example that runs the
3 micromagnetic simulations described by the \MIF~1.x files
\fn{taskA.mif}, \fn{taskB.mif} and \fn{taskC.mif}\index{file!MIF~1.x}.  It
is launched with the command
\begin{rawhtml}<BLOCKQUOTE>\end{rawhtml}
%begin<latexonly>
\begin{quote}
%end<latexonly>
\begin{verbatim}
tclsh oommf.tcl batchmaster simpletask.tcl
\end{verbatim}
%begin<latexonly>
\end{quote}
%end<latexonly>
\begin{rawhtml}</BLOCKQUOTE>\end{rawhtml}
This example uses the default slave launch script, so a single slave is
launched on the current machine, and the 3 simulations will be run
sequentially.  Also, no slave initialization script is given, so the
default procs in \fn{batchsolve.tcl} are used.  Output will be magnetization
states\index{file!magnetization} and tabular data\index{file!data~table}
at each control point\index{simulation~2D!control~point}, stored in
files on the local machine with base names as specified in the \MIF\
files.

\begin{codelisting}{f}{fig:batchschedsimpletask}{Sample task script
\fn{simpletask.tcl}.}{pt:batchschedsimpletask}
\begin{verbatim}
# FILE: simpletask.tcl
#
# This is a sample batch task file.  Usage example:
#
#   tclsh oommf.tcl batchmaster simpletask.tcl
#
# Form task list
$TaskInfo AppendTask A "BatchTaskRun taskA.mif"
$TaskInfo AppendTask B "BatchTaskRun taskB.mif"
$TaskInfo AppendTask C "BatchTaskRun taskC.mif"
\end{verbatim}
\end{codelisting}

The\begin{rawhtml}
<A NAME="pt:batchoctrltask"></A>
\end{rawhtml}
\hyperrefhtml{second sample task script}{second sample task script
(Fig.~}{)}{fig:batchoctrltask} builds on the previous example by
defining \cd{BatchTaskIterationCallback} and
\cd{BatchTaskRelaxCallback} procedures in the slave init script.
The first set up to write tabular data every 10 iterations, while the
second writes tabular data on each control point event.  The data is
written to the output file specified by the \cd{Base Output Filename}
entry in the input \MIF\ files.  Note that there is no magnetization
vector field output in this example.  This task script is launched the
same way as the previous example:
\begin{rawhtml}<BLOCKQUOTE>\end{rawhtml}
%begin<latexonly>
\begin{quote}
%end<latexonly>
\begin{verbatim}
tclsh oommf.tcl batchmaster octrltask.tcl
\end{verbatim}
%begin<latexonly>
\end{quote}
%end<latexonly>
\begin{rawhtml}</BLOCKQUOTE>\end{rawhtml}

\begin{codelisting}{f}{fig:batchoctrltask}{Task script with
iteration output \fn{octrltask.tcl}.}{pt:batchoctrltask}
\begin{verbatim}
# FILE: octrltask.tcl
#
# This is a sample batch task file, with expanded output control.
# Usage example:
#
#        tclsh oommf.tcl batchmaster octrltask.tcl
#
# "Every" output selection count
set SKIP_COUNT 10

# Initialize solver. This is run at global scope
set init_script {
    # Text output routine
    proc MyTextOutput {} {
        global outtextfile
        puts $outtextfile [GetTextData data]
        flush $outtextfile
    }
    # Change control point output
    proc BatchTaskRelaxCallback {} {
        MyTextOutput
    }
    # Add output on iteration events
    proc BatchTaskIterationCallback {} {
        global solver
        set count [$solver GetODEStepCount]
        if { ($count % __SKIP_COUNT__) == 0 } { MyTextOutput }
    }
}

# Substitute $SKIP_COUNT in for __SKIP_COUNT__ in above "init_script"
regsub -all -- __SKIP_COUNT__ $init_script $SKIP_COUNT init_script
$TaskInfo SetSlaveInitScript $init_script

# Form task list
$TaskInfo AppendTask A "BatchTaskRun taskA.mif"
$TaskInfo AppendTask B "BatchTaskRun taskB.mif"
$TaskInfo AppendTask C "BatchTaskRun taskC.mif"
\end{verbatim}
\end{codelisting}

The
\begin{rawhtml}
<A NAME="pt:batchschedmultitask"></A>
\end{rawhtml}
 \hyperrefhtml{third task script}{third task script
(Fig.~}{)}{fig:batchschedmultitask} is a more complicated example
running concurrent processes\index{simulation~2D!scheduling} on two
machines.  This script should be run with the command
\begin{rawhtml}<BLOCKQUOTE>\end{rawhtml}
%begin<latexonly>
\begin{quote}
%end<latexonly>
\begin{verbatim}
tclsh oommf.tcl batchmaster multitask.tcl bar
\end{verbatim}
%begin<latexonly>
\end{quote}
%end<latexonly>
\begin{rawhtml}</BLOCKQUOTE>\end{rawhtml}
where \fn{bar} is the name of the local machine.

Near the top of the \fn{multitask.tcl} script several \Tcl\ variables
(\cd{RMT\_MACHINE} through \cd{A\_list}) are defined; these are used
farther down in the script.  The remote machine is specified as
\fn{foo}, which is used in the \cd{\$TaskInfo AppendSlave} and
\cd{\$TaskInfo ModifyHostList} commands.

There are two \cd{AppendSlave} commands, one to run two slaves on the
local machine, and one to run a single slave on the remote machine
(\fn{foo}).  The latter changes to a specified
working directory\index{working~directory}  before
launching the \fn{batchslave} application on the remote machine.  (For
this to work you must have \cd{ssh} configured properly\index{application!ssh}.)

Below this the slave initialization script is defined.  The \Tcl\
\cd{regsub} command is used to place the task script defined value of
\cd{BASEMIF} into the init script template.  The init script is run on
the slave when the slave is first brought up.  It first reads the base
\MIF\ file into a newly created \cd{mms\_mif} instance.  (The \MIF\ file
needs to be accessible by the slave process, irrespective of which
machine it is running on.)  Then replacement \cd{SolverTaskInit} and
\cd{SolverTaskCleanup} procs are defined.  The new \cd{SolverTaskInit}
interprets its first argument as a value for the exchange constant
\cd{A}.  Note that this is different from the default
\cd{SolverTaskInit} proc, which interprets its first argument as the
name of a \MIF~1.x file\index{file!MIF~1.x} to load.  With this task
script, a \MIF\ file is read once when the slave is brought up, and then
each task redefines only the value of \cd{A} for the simulation (and
corresponding changes to the output filenames and data table header).

Finally, the \Tcl\ loop structure
\begin{rawhtml}<BLOCKQUOTE>\end{rawhtml}
%begin<latexonly>
\begin{quote}
%end<latexonly>
\begin{verbatim}
foreach A $A_list {
    $TaskInfo AppendTask "A=$A" "BatchTaskRun $A"
}
\end{verbatim}
%begin<latexonly>
\end{quote}
%end<latexonly>
\begin{rawhtml}</BLOCKQUOTE>\end{rawhtml}
is used to build up a task list consisting of one task for each value
of \cd{A} in \cd{A\_list} (defined at the top of the task script).  For
example, the first value of \cd{A} is 10e-13, so the first task
will have the label \cd{A=10e-13} and the corresponding script is
\cd{BatchTaskRun 10e-13}.  The value 10e-13 is passed on by
\cd{BatchTaskRun} to the \cd{SolverTaskInit} proc, which has been
redefined to process this argument as the value for \cd{A}, as
described above.

There are 6 tasks in all, and 3 slave processes, so the first three
tasks will run concurrently in the 3 slaves.  As each slave finishes
it will be given the next task, until all the tasks are complete.

\begin{codelisting}{p}{fig:batchschedmultitask}{Advanced sample task
script \fn{multitask.tcl}.}{pt:batchschedmultitask}
\begin{verbatim}
# FILE: multitask.tcl
#
# This is a sample batch task file.  Usage example:
#
#   tclsh oommf.tcl batchmaster multitask.tcl hostname [port]
#
# Task script configuration
set RMT_MACHINE   foo 
set RMT_TCLSH      tclsh
set RMT_OOMMF      "/path/to/oommf/oommf.tcl"
set RMT_WORK_DIR   "/path/to/oommf/app/mmsolve/data"
set BASEMIF taskA
set A_list { 10e-13 10e-14 10e-15 10e-16 10e-17 10e-18 }

# Slave launch commands
$TaskInfo ModifyHostList +$RMT_MACHINE
$TaskInfo AppendSlave 2 "exec %tclsh %oommf batchslave -tk 0 \
        %connect_info batchsolve.tcl"
$TaskInfo AppendSlave 1 "exec ssh $RMT_MACHINE \
        cd $RMT_WORK_DIR \\\;\
        $RMT_TCLSH $RMT_OOMMF batchslave -tk 0 %connect_info batchsolve.tcl"

# Slave initialization script (with batchsolve.tcl proc
# redefinitions)
set init_script {
    # Initialize solver. This is run at global scope
    set basename __BASEMIF__      ;# Base mif filename (global)
    mms_mif New mif
    $mif Read [FindFile ${basename}.mif]
    # Redefine TaskInit and TaskCleanup proc's
    proc SolverTaskInit { args } {
        global mif outtextfile basename
        set A [lindex $args 0]
        set outbasename "$basename-A$A"
        $mif SetA $A
        $mif SetOutBaseName $outbasename
        set outtextfile [open "$outbasename.odt" "a+"]
        puts $outtextfile [GetTextData header \
                "Run on $basename.mif, with A=[$mif GetA]"]
        flush $outtextfile
    }
    proc SolverTaskCleanup { args } {
        global outtextfile
        close $outtextfile
    }
}
# Substitute $BASEMIF in for __BASEMIF__ in above script
regsub -all -- __BASEMIF__ $init_script $BASEMIF init_script
$TaskInfo SetSlaveInitScript $init_script

# Create task list
foreach A $A_list {
    $TaskInfo AppendTask "A=$A" "BatchTaskRun $A"
}
\end{verbatim}
\end{codelisting}

%begin{latexonly}
\setcounter{secnumdepth}{\value{msoldsecnumdepth}}
%end{latexonly}
