
% Leave a blank line at top to work around some flakiness in l2h
% on RedHat 5.2/AXP (otherwise the images.tex file can get some
% bad characters dumped into it).
%
% The following interlock is taken from Knuth's  ``The TeXbook'',
% p383 (Appendix D: Dirty Tricks).  It insures this file gets read
% at most once.  (NOTE: Unfortunately, latex2html doesn't do
% \if statements.)
%\ifx\oommfheadread\relax\endinput\else\let\oommfheadread=\relax\fi

% \documentclass[12pt]{article}  % Put this in file inputting this one

\usepackage{makeidx}
\usepackage{multirow}

%begin{latexonly}
% On the one hand, latexml 0.8.6 hiccups on color, but xcolor is OK.
% On the other hand, latex2html 2021 chokes on xcolor. There is no
% "else" equivalent in %begin{latexonly}, so wait until package html
% is loaded and do \html{\usepackage{color}}.
\usepackage{xcolor}
\usepackage{graphics}

\usepackage{ifthen}   % \ifthenelse construct

% Test if a command is defined, by Ulrich Diez in
% tex.stackexchange.com, 3-Nov-2020. Use like
% \checkfor{foo}{\foo is defined}{\foo is undefined}
%begin{latexonly}
\makeatletter
\DeclareRobustCommand\checkfor[1]{%
  \begingroup
  \expandafter\ifx\csname#1\endcsname\relax\expandafter\@firstoftwo\else\expandafter\@secondoftwo\fi
  {%
    \expandafter\endgroup\expandafter\ifx\csname#1\endcsname\relax\expandafter\@firstoftwo\else\expandafter\@secondoftwo\fi
  }{\endgroup\@firstoftwo}%
}%
\makeatother
%end{latexonly}

%begin{latexonly}
% Are we running pdftex with pdf output?
% (This logic is from Heiko Oberdiek's ifpdf package.)
\ifx\pdfoutput\undefined
  % not running PDFTeX
  \def\oommfpdf{0}
\else
  \ifx\pdfoutput\relax
    % not running PDFTeX
    \def\oommfpdf{0}
  \else
    % running PDFTeX, with...
    \ifnum\pdfoutput>0
      % ...PDF output
      \def\oommfpdf{1}
    \else
      % ...DVI output
      \def\oommfpdf{0}
    \fi
  \fi
\fi
%end{latexonly}


%begin{latexonly}
\IfFileExists{latexml.sty}{\usepackage{latexml}}{
  \newif\iflatexml % Assume no latexml.sty => no latexml
  % (\newif creates a command initialized to \iffalse.)
}
%end{latexonly}
% latex2html builds won't load latexml, so won't define \iflatexml.
% But all \if* commands should be wrapped in latexonly blocks so that
% latex2html doesn't see them, since latex2html doesn't process them
% anyway. You can check the tail bit of the latex2html run to see a
% list of commands it didn't recognize. If inputs are coded properly
% then \iflatexml should not be on that list.
%
% NOTE: latexml with unmodified html package bindings chunders
% on \begin/end{rawhtml} and \begin/end{htmlonly} environments if
% the \end part is indented. The html.sty.ltxml in LaTeXML 0.86 has this
% binding for the rawhtml environment (htmlonly is completely
% analogous):
%
%   DefConstructorI(T_CS("\\begin{rawhtml}"), undef, '', reversion => '',
%     afterDigest => [sub {
%         my ($stomach, $whatsit) = @_;
%         my $endmark = "\\end{rawhtml}";
%         my $nlines  = 0;
%         my ($line);
%         my $gullet = $stomach->getGullet;
%         $gullet->readRawLine;    # IGNORE 1st line (after the \begin{$name} !!!
%         while (defined($line = $gullet->readRawLine) && ($line ne $endmark)) {
%           $nlines++; }
%         NoteLog("Skip rawhtml ($nlines lines)"); }]);
%
% The problem is the comparison of $line against $endmark requires an
% exact full-line match. One workaround is to  strip off any leading
% whitespace before making the comparison, e.g.,
%
%     while (defined($line = $gullet->readRawLine)
%            && $line =~ s/^\s*// && ($line ne $endmark)) {
%
% Another more general approach is to use a regexp match, e.g.,
%
%   DefConstructorI(T_CS("\\begin\{htmlonly\}"), undef, '', reversion => '',
%     afterDigest => [sub {
%         my ($stomach, $whatsit) = @_;
%         my $endmark = '^\\s*\\\\end\{rawhtml\}\\s*$';
%         my $nlines  = 0;
%         my ($line);
%         my $gullet = $stomach->getGullet;
%         $gullet->readRawLine;    # IGNORE 1st line (after the \begin{$name} !!!
%         while (defined($line = $gullet->readRawLine) && ($line !~ m/$endmark/)) {
%           $nlines++; }
%         NoteLog("Skip htmlonly ($nlines lines)"); }]);
%
% which allows any amount of leading or trailing whitespace on the \end
% line.
%
% NOTE: LaTeXML will still choke on single-line constructs such as
%   \begin{rawhtml}</BLOCKQUOTE>\end{rawhtml}
% but this is arguably sleazy LaTeX anyway.
%
% To use this, copy the distributed html.sty.ltxml to a local directory,
% edit the file as shown above, and then use the --path option to
% latexml to include the local directory on the latexml search path.

% Is the build targeting PostScript output? The first definition
% of \psonly and \notpsonly holds in place if latex2html is running.
\newcommand{\psonly}[1]{}
\newcommand{\notpsonly}[1]{#1}
%begin{latexonly}
\newif\ifoommfps
\ifnum\oommfpdf=0
  \unless\iflatexml
    \oommfpstrue
  \fi
\fi
\ifoommfps
  \renewcommand{\psonly}[1]{#1}
  \renewcommand{\notpsonly}[1]{}
\fi
%end{latexonly}

%begin{latexonly}
%\ifnum\oommfpdf=0
% pdflatex command not in use
% The html package included below uses \pdfoutput to determine whether
% or not pdf-TeX is being used.  Unfortunately, the code in html.sty
% that determines this is broken, at least v1.39 2001/10/01 as shipped
% with Fedora Core 6 (FC6) when used with the latex in FC6.  This
% results in breakage of some commands defined in html.sty, including at
% least \htmlimage and \htmladdnormallink.  The breakage is such that
% the pdf-versions of these commands are wrongly defined in the case
% where latex or latex2html is running.  One workaround would be to
% redefine these commands after \usepackage{html}, but a more general
% fix would appear to be to just redefine \pdfoutput so that the logic
% in html.sty works.  This latter approach is done here, by unsetting
% \pdfoutput.  I've included this lengthy note because the problematic
% case is when \pdfoutput is defined to value 0.  Is there code
% someplace that functions differently if \pdfoutput is 0 than if it is
% undefined?  I don't know.  If it turns out that this breaks something,
% then one can try redefining \pdfoutput to 0 after \usepackage{html},
% or otherwise leaving \pdfoutput alone and just redefining \htmlimage
% etc. as needed.
% UPDATE 28-May-2022: This breaks \usepackage{html} on latex, or at
%  least on the latex install on my Mac, which is
%   pdfTeX 3.141592653-2.6-1.40.24 (TeX Live 2022/MacPorts 2022.62882_0)
%  YMMV.
%\let\pdfoutput\relax
%\fi
%end{latexonly}

% \ifnum\oommfpdf=0
% pdflatex command not in use
% \renewcommand{\htmlimage}[1]{} % pdf-mode detection code is broken
%    in some versions of html.sty, causing \htmlimage to be re-defined
%    as taking 2 arguments.  This causes some havoc.  Define it back to
%    be safe.
%    NOTE: This code superceded by \let\pdfoutput\relax code above.
%    The code in this current stanza is left in but commented out in
%    case it occurs that the more general fix above breaks stuff.
% \renewcommand{\htmladdnormallink}[2]{#1} % Ditto.
% \fi

\notpsonly{\usepackage[colorlinks=true]{hyperref}}
% Note 1: latex2html reads the .aux files created by latex and dvips,
% but the hyperref package writes material to .aux that latex2html
% hiccups on, with errors like
%
%   *** sub wrap_cmd_HyPL@Entry  failed: Illegal declaration of
%   subroutine main::wrap_cmd_HyPL at (eval 573) line 1.
%
% I don't know if this actually matters or not, but the ps output
% doesn't do anything with hyperref, so just don't load hyperref in
% this case.
%
% Note 2: The html package loads package hyperref if it isn't already
% loaded, and loads with default options.  So to change defaults you
% need to either load hyperref first with desired options, or else use
% the \hypersetup command afterwards.
%
% Note 3: At one time the hyperref package load command we used was
%   \usepackage[pdftex, colorlinks=true, citecolor=blue]{hyperref}
% But on 25-Apr-2022 we dropped the pdftex option because the hyperref
% docs claim the driver (e.g., dvips, pdftex) should be detected
% automatically.
%
% Note 4: The hyperref docs recommend loading hyperref after all other
% packages, because hyperref redefines a lot of commands from other
% packages. So keep this in mind and rearrange package ordering as
% needed.

\usepackage{html}

\html{
\usepackage{l2hbugs}
% latex2html 2021 chokes on xcolor (see xcolor note above)
\usepackage{color}
}

% LINK REFERENCE:
%
% Links to \label
% \htmlonlyref{text}{label}
%   The label argument is ignored in nonHTML mode. In HTML output the
%   text links to the label. In LaTeX2HTML this is a wrapper
%   around \htmlref{}{} from package html. In LaTeXML it's a wrapper
%   about \hyperref from package hyperref, with args reversed.
%
% \hyperrefhtml{htmltext}{nonhtmlpretext}{nonhtmlposttext}{label}
%   HTML output uses htmltext and label, nonHTML output uses
%   pre/posttext + label. For nonHTML this is a wrapper
%   abouht \hyperrefpage from package hyperref.
%
% \pagehyperref{htmltext}{nonhtmlpretext}{nonhtmlposttext}{label}
%   Same as \hyperrefhtml, except that for nonHTML output the page
%   number containing the label is given instead of the label value.
%
%
% Links to arbpts
% \arbtarget{targettext}{label}
% \arbtargetlink{linktext}{prepage}{postpage}{label}
%   The first command sets an anchor at the specified point, the second
%   provides a link in HTML and PDF output, and a page reference in PDF
%   and PS output.
%
% \pttarget[targettext]{label}
% \ptlink{linktext}{label}
%   The first command sets an anchor at the specified point, the second
%   provides a link in HTML and PDF output but no page reference. In PS
%   the linktext is printed alone.
%
% More details on \arb* and \pt* commands are provided further down in
% this file.
%
% Links to URLs
% \htmladdnormallink{text}{url}
%   The url argument is ignored in non-html mode. In html output mode
%   the text links to the url.
%
% \htmladdnormallinkfoot{text}{url}
%   Same as \htmladdnormallink, except that a footnote is added in
%   non-html mode to the url.

% Crutch for latexml issues. The \iflatexml construct is insufficient
% because latex2html doesn't process \if statements.
%begin{latexonly}
\iflatexml
 \newcommand{\latexmlonly}[1]{#1}
 \newcommand{\notlatexmlonly}[1]{}
\else
 \newcommand{\latexmlonly}[1]{}
 \newcommand{\notlatexmlonly}[1]{#1}
\fi
%end{latexonly}
\html{
 \newcommand{\latexmlonly}[1]{}
 \newcommand{\notlatexmlonly}[1]{#1}
}

% latexml defines command \iflatexml true if latexml is running,
% otherwise false. Some one can do
%
%  \iflatexml
%    % latexml version
%  \else
%    % plain latex version
%  \fi
%
% The \else clause is optional if you want just the latexml branch.
% If you want to reorder the branches, or just have the plain latex
% branch, use \unless:
%
% \unless\iflatexml
%    % plain latex version
%  \else
%    % latexml version
% \fi
%
% It would be nice to work latex2html into this processing scheme, but
% according to my notes latex2html doesn't do \if statements.  For
% reference, \newif\iffoo defines a new command \iffoo defined initially
% to be \iffalse, but also commands \footrue and \foofalse which
% redefine \iffoo to be \iftrue and \iffalse, respectively.
%
% The html.sty file provided with latex2html provides five conditional
% environments:
%
%   htmlonly, latexonly, rawhtml, imagesonly, makeimage
%
% and related \html{}, \latex{}, \latexhtml{}{} commands, plus a further
% eight commands:
%
%  htmlref                  61
%  htmladdnormallink        41
%  htmladdnormallinkfoot    26
%  hyperref                311
%  htmlimage                14
%  htmladdimage              0
%  htmlcite                  0
%  htmlrule                  2
%
% The numbers to the right is a count of the lines in .tex files under
% oommf/doc that currently include the command text (excluding this
% appearance but including mention in other comments).
%
% One can also exclude text from processing by latex2html by wrapping it
% inside
%
%  %begin{latexonly}
%  ...
%  %end{latexonly}
%
% Unlike the \begin/\end forms, these aren't processed by LaTeX and
% don't put the contents inside a group. Also, these are processed by
% latex2html directly and only, and so can be used prior to
% \usepackage{html}.
%
% See the LaTeXML document "Bindings" section to see a list of supported
% classes and packages. I note that "html.sty" is included in the
% package list...
%
% BTW, latexml chunders on this construct
%    \begin{htmlonly}
%    ...
%    \end{htmlonly} % My little note
%
% The stock latexml 0.8.6 binding for the htmlonly and rawhtml
% environments only captures the \end{...} statement if it begins in the
% first column and has no trailer. I slapped together a straightforward
% fix that allows whitespace (though not other material) on the \end
% line---see doc/common/xmlextras/html.sty.ltxml. The doc makerules
% files are set up such that this html binding file is loaded instead of
% the stock version, but I'll see if Bruce Miller will change the stock
% version in the next release.

%begin{latexonly}
\renewcommand\htmlimage[1]{}  % Otherwise breaks 'latex' (for .ps
% output) on macOS 25-Apr-2022. No idea why...

%%%%%%%%%%%%%%%%%%%%%%%%%%%%%%%%%% TOCLOFT %%%%%%%%%%%%%%%%%%%%%%%%%%%%%
% If there are more than nine subsections in a section (for example,
% in the ``Command Line Utilities'' section, then in the table of
% contents the subsection numbers run into the subsection titles.  One
% workaround is:
%
%   \usepackage{tocloft}
%   \setlength{\cftsubsecnumwidth}{2.7em}
%
% However this increases the spacing between subsection numbers and
% titles for all subsections in the toc.  A slightly less ugly
% alternative is
%
%begin{latexonly}
\iflatexml\else
\usepackage{tocloft}
\newlength{\oommftocsslen}
\setlength{\oommftocsslen}{0.5em} % need some extra space at end of number
\renewcommand{\cftsecpresnum}{\hfill} % note the double `l'
\renewcommand{\cftsecaftersnum}{\hspace*{\oommftocsslen}}
\addtolength{\cftsecnumwidth}{\oommftocsslen}
\fi
%end{latexonly}
% (Note: latex2html and latexml don't have a bindings for tocloft, so
% exclude this block from latex2html and latexml processing.)
%
% This typesets subsection numbers flushright.  With this the
% subsection numbers after .9 stick out to the left, but otherwise
% everything lines up.
%
% Another approach may be to hack the userguide.toc file directly. Or
% maybe ``Command Line Utilities'' just has too many sections and needs
% to be broken up.
%%%%%%%%%%%%%%%%%%%%%%%%%%%%%%%%%% TOCLOFT %%%%%%%%%%%%%%%%%%%%%%%%%%%%%

%begin{latexonly}
%\ifx\undefined\pdfpagewidth % pdflatex command not in use
\ifnum\oommfpdf=0
% pdflatex command not in use
\newcommand{\pdfonly}[1]{}
\newcommand{\ifnotpdf}[1]{#1}
\else                       % pdflatex command in use
\newcommand{\pdfonly}[1]{#1}
\newcommand{\ifnotpdf}[1]{}
\pdfcompresslevel=9
%\pdfpagesattr={/CropBox [60 290 480 720]}
%\pdfpagewidth=6.0in
%\pdfpageheight=5.5in
%\pdfcatalog{/PageMode /UseOutlines}
\pdfcatalog{            % Catalog dictionary of PDF output.
    /PageMode /UseOutlines
    /URI (https://math.nist.gov/oommf/)
}
% openaction goto page 1 {/Fit}
\fi

\iflatexml
\newcommand{\htmlonlyref}[2]{\hyperref[#2]{#1}}
\newcommand{\pagehyperref}[4]{\htmlonlyref{#1}{#4}}
% Note: In latexml 0.8.6, the "~" in \htmlref{\MIF~1.2}{sec:mif12format}
% gets transferred to HTML as a literal "~" instead of a nbsp. But
% \htmlonlyref{\MIF~1.2}{sec:mif12format} defined as above works OK.
\else
\newcommand{\htmlonlyref}[2]{#1}
\newcommand{\pagehyperref}[4]{\hyperrefpage{#1}{#2}{#3}{#4}}
\fi
%end{latexonly}

\html{
\newcommand{\pdfonly}[1]{}
\newcommand{\ifnotpdf}[1]{#1}
%\let\hyperrefhtml=\hyperref
\newcommand{\hyperrefhtml}[4]{\htmlref{#1}{#4}}
\newcommand{\htmlonlyref}[2]{\htmlref{#1}{#2}}
% Use \htmlonlyref for links to be available in HTML, but not
% in PDF.  In particular, this applies to \label{} commands not
% placed near counter updates, since latex2html drops an anchor
% tag at the right location, but pdflatex just drops the ball
% (well at least pdflatex Version 3.14159-13d (Web2C 7.3.1) does).
\newcommand{\pagehyperref}[4]{\htmlonlyref{#1}{#4}}
}

% Selection by HTML vs non-HTML output
%begin{latexonly}
\iflatexml
  \newcommand{\HTMLoutput}[1]{#1}
  \newcommand{\NONHTMLoutput}[1]{}
\else
  \newcommand{\HTMLoutput}[1]{}
  \newcommand{\NONHTMLoutput}[1]{#1}
\fi
%end{latexonly}
\html{
% NOTE: Do NOT use \newcommand inside \NONHTMLoutput; the \NONHTMLoutput
% wrapper will be ignored and latex2html will define the command as
% given, even if the command is already defined (i.e., should require
% a \renewcommand). If you absolutely positively have to do this, follow
% the \NONHTMLoutput command definition with a \HTMLoutput
% definition. Actually, it is probably more robust to define the command
% as desired for non-HTML output w/o \NONHTMLoutput, and then
% use \HTMLoutput{\renewcommand...}.
%   A working alternative is to use \HTMLoutput and \NONHTMLoutput
% inside \newcommand, rather than the other way around.
  \newcommand{\HTMLoutput}[1]{#1}
  \newcommand{\NONHTMLoutput}[1]{}
}

\setlength{\textwidth}{6.5in}
\setlength{\oddsidemargin}{0in}
\setlength{\textheight}{8.5in}

\begin{htmlonly}
\HTMLset{myTODAY}{\today}
\usepackage{localmods}
\end{htmlonly}

\newcommand{\myimage}[2]{\HTMLcode[#1 #2]{IMG}}

\htmlinfo*
\bodytext{BGCOLOR="#FFFFFF",text="#000000",LINK="#0000FF",
            VLINK="#4498F0",ALINK="00FFFF"}

%\htmladdtonavigation{\htmladdnormallink
%    {\myimage{ALT="OOMMF Home",BORDER="2"}{https://math.nist.gov/oommf/images/oommficon.gif}}{https://math.nist.gov/oommf}}

\htmladdtonavigation{\htmladdnormallink
    {\myimage{ALT="OOMMF Home",BORDER="2"}{oommficon.gif}}{https://math.nist.gov/oommf}}

\usepackage{alltt}  % Verbatim package supporting control sequences.
% Both LaTeXML and LaTeX2HTML have bindings for the alltt environment,
% but the LaTeX2HTML exhibits wonky behavior with alltt. When I first
% starting using alltt, I found that LaTeX2HTML support for the
% inline \textcolor{<color>}{<text>} command was broken inside alltt
% environments. However, it seemed that using \color from the color
% package (see notes above about color vs.  xcolor package conflicts in
% latex2html and latexml), instead of \textcolor worked OK, aside from
% ungrouped \color commands changing hyperref link color in latexml.  So
% I defined \colorit to be a synonym for \textcolor except in
% latex2html, where a grouped \color command was used instead.
%   But then later when I tried to build the documentation on a Rocky 8
% Linux box (9-Aug-2022), LaTeX2HTML threw an error on \color:
%
%  11/23:chapter:..."Debugging OOMMF" for Debugging_OOMMF.html
%  ;........Unescaped left brace in regex is illegal here in regex;
%    marked by <-- HERE in m/^color{ <-- HERE shellcmdcolor}/
%    at /usr/bin/latex2html line 4531.
%
% But \textcolor seems okay. We can probably do away with \colorit,
% but I'm leaving the infrastructure in place for reference.
% begin{latexonly}
\newcommand{\colorit}[2]{\textcolor{#1}{#2}}
%end{latexonly}
\html{
%\newcommand{\colorit}[2]{{\color{#1}{#2}}}
\newcommand{\colorit}[2]{\textcolor{#1}{#2}}
}
% Macros for coloring shell and program commands in alltt environments.
% The colors can be changed using \renewcommand. Constructs like
% "Program commands are colored \pgmcmd{\pgmcmdcolorname}\ for
% visibility" help text-to-color identification for colorblind
% individuals or grayscale printing. The colors are tweaked a little
% to make them either easier to read against a white background (cyan)
% or easier to distinguish from black in grayscale (red).
\definecolor{shellcmdcolor}{RGB}{0,187,238}
\newcommand{\shellcmdcolorname}{cyan}
\newcommand{\shellcmd}[1]{\colorit{shellcmdcolor}{#1}}
\definecolor{pgmcmdcolor}{RGB}{255,51,17}
\newcommand{\pgmcmdcolorname}{red}
\newcommand{\pgmcmd}[1]{\colorit{pgmcmdcolor}{#1}}

%begin{latexonly}
\iflatexml\else
\usepackage{upquote} % Use upright quotes in verbatim environments
\fi
%end{latexonly}
% Note: latexml docs say upquote package is supported, but if used I get
%    Warning:missing_file:upquote Can't find binding for package upquote
% latex2html also doesn't have a binding for upquote.

% I want upright quotes in verbatim environments, but upquote package
% seems to be not well supported. Might try \textquotesingle and
% \textquotedbl in alltt environment.
\usepackage[T1]{fontenc} % for \textquotedbl
\usepackage{textcomp}    % for \textquotesingle (not needed for post-2019 TeX)
\latex{\newcommand{\ssquote}{\textquotesingle}}  % Straight quotes
\latex{\newcommand{\sdquote}{\textquotedbl}}
\html{\newcommand{\ssquote}{\verb+'+}}
\html{\newcommand{\sdquote}{\verb+"+}}

% WARNING: --- Single quote madness ---
% Unicode defines the following single quote characters:
%
% Unicode    UTF-8    Description
% U+0027         27   Apostrophe, neutral (straight) quote
% U+0060         60   Grave accent (backtick)
% U+00B4      C2 B4   Accute accent (mirror image of backtick)
% U+2018   E2 80 98   Left single quotation mark (curved quote)
% U+2019   E2 80 99   Right single quotation mark (curved quote)
%
% If the upquote package is not used, then in verbatim (and alltt)
% environemnts LaTeX renders the keyboard backtick character as U+2018
% and and the keyboard apostrophe U+2019, i.e., as curved quotes. These
% don't look especially similar to their keyboard glyphs, which is
% problematic if you are using backticks to illustrate execution of an
% embedded command in a command shell (in Bourne shell derivatives
% '$(...)'  can be used, but csh only has backticks), or a code sequence
% using paired single quotes.
%    The upquote package handles this fairly well, rendering backticks
% as U+0060 and single quotes as U+0027 inside verbatim (and alltt)
% environments. LaTeX2HTML doesn't have a binding for upquote, but also
% follows this behavior.
%    LaTeXML, OTOH, follows the LaTeX convention and doesn't have a
% binding for upquote (at least as of LaTeXML 0.8.6). Even worse, the
% monospace font used on at least macOS 11.6.7 (Big Sur) has nearly
% indistinguishable glyphs for U+2018 and U+2019.
%    A workaround for the backtick that seems to work in alltt for
% LaTeX, LaTeX2HTML, and LaTeXML, both with and without the upquote
% package, is to request a grave accent on an empty string, as in the
% "\backtick" command defined below. This renders U+0060 in all three
% cases. The "\fronttick" command is defined for completeness, but it
% renders U+00B4, which doesn't have an obvious application to verbatim
% keyboard rendering. To get U+0027, use either the upquote
% package, \textquotesingle, or the "\uptick" command (which assumes tt
% font) defined below.
%
\newcommand{\backtick}{\`{}}
\newcommand{\fronttick}{\'{}}
%begin{latexonly}
\newcommand{\uptick}{\char'15}
%end{latexonly}
\html{\newcommand{\uptick}{{'}}}

% A solution to the ``set marker at arbitrary point on a page'' problem:
%
% Usage: \arbtarget{targettext}{label}
%        \arbtargetlink{linktext}{prepage}{postpage}{label}
%
%        \pttarget[targettext]{label}
%        \ptlink{linktext}{label}
%
% The first pair provide a link in html and pdf output, and a page
% reference in pdf and ps output. The second pair provide a link in html
% and pdf output, but no page reference; \ptlink is transparent in ps
% output, printing only the linktext.
%
% NB: Text like
%   \begin{rawhtml}<BLOCKQUOTE>\end{rawhtml}
% will break LaTeXML parsing and can prevent proper operation of these
% commands. Instead, move the \end{rawhtml} to a separate line. LaTeXML
% also doesn't like text following \end{rawhtml}, e.g.,
%   \end{rawhtml}\index{foobar}
% so break that into two lines too.
%
% NB: If you want to align a pttarget with the top of a document
% section, do not place \pttarget before sectioning command, i.e.,
%
%   \pttarget{label}\section{My Favorite Toys}     %% BAD
%
% because if \section causes a page break then the target anchor will be
% on the preceding page. Putting \pttarget after the sectioning command,
%
%   \section{My Favorite Toys}\pttarget{label}     %% LESS BAD
%
% is a bit better, but the link will tend to point to the line after the
% section header, and if the viewer scrolls to put the anchor at the top
% of the page then you won't see the section title.
%
% A better solution is to use the standard section labels and reference
% with \hyperref, e.g.,
%
%   \section{My Favorite Toys}\label{sec:mft}     %% BEST
%   ...
%   \hyperref[sec:mft]{(Description.)}  % hyperref pkg w/o html pkg
% or
%   \htmlref{(Description.)}{sec:mft}   % w/ html pkg
%
% The html package redefines \hyperref, so if html is loaded you need to
% use the second version if you want active links in the PDF output.
% Also note the change in argument ordering between \hyperref and
% \htmlref, and also the square brackets on the former.
%
% Unfortunately, the \section links are broken in some cases in LaTeXML
% (version 0.8.6). If document splitting is such that sec:mft is the top
% level section in it's .html file, any links to sec:mft from within
% that file aren't placed into the .html file. Since file splitting
% isn't decided until latexmlpost, AFAICT the .xml if fine, it is just a
% bug in the conversion from xml to html.
%
% On a related note, if you use \ptlink in a moving argument like
% \caption, you'll need to \protect it, e.g.,
%
%  \caption{Sheep in a snow storm\protect.\notpsonly{
%    \protect\ptlink{(Description.)}}}
%
% The text to a \section command is also a moving argument, but AFAICT
% \pttarget as used above does not require \protect.
%
%
% Background: In standard LaTeX and PDFLaTeX, the \label command drops a
% pin at the point of the most recent \refstepcounter, so we can define
% links to arbitrary points in the document by creating a special-use
% counter and stepping that at the point of interest. If the hypertex
% package is loaded then in the pdf version the link even points to the
% right position on the page, so this all works nicely.
%
% LaTeXML handles label references differently than (PDF)LaTeX. Rather
% than tying a target to the stepping of a counter, it expects a
% reference point to lie inside some enclosing environment, and links to
% (the start of?) that environment. So we can't use the counter-based
% target method with LaTeXML. Instead, we use the \hyper{target,link}
% commands from the \hyperref package.
%
% LaTeX2HTML Version 2021 does not have a binding for
% \hyper{target,link}, but it does handle \label the same way as
% (PDF)LaTeX, so we use the counter-based method for LaTeX2HTML.
%
% NB: If the label name has the form "html:foobar", then LaTeXML will
% complain about \arbtargetlink calls with the error:
%
%    Error:malformed:document Document fails RelaxNG validation (LaTeXML)
%
% IDK if source of the error is LaTeXML or the hyperref package; the
% docs for the latter indicate special handling for constructs like
%   \href{https://foo.org}{foo home}
%   \href{mailto:bar@foo.org}{bar@foo.org}
%   \href{run:/path/to/my/file.ext}{text displayed}
%
% Regardless, the safest course is to avoid colons in \arbtarget
% labels. By convention, maybe use the label format "PTfoo"?
%
%begin{latexonly}
\iflatexml % LaTeXML version
\newcommand{\arbtarget}[2]{\hypertarget{#2}{#1}}
\newcommand{\arbtargetlink}[4]{\hyperlink{#4}{#1}}
\newcommand{\pttarget}[2][]{\hypertarget{#2}{#1}}
\newcommand{\ptlink}[2]{\hyperlink{#2}{#1}}

\newcommand{\arbtargetlinkonly}[2]{\hyperlink{#2}{#1}}
\else % LaTeX and PDFLaTeX version
  \newcounter{arbtargetcounter}
  \newcommand{\arbtarget}[2]{#1\refstepcounter{arbtargetcounter}\label{#2}}
  \newcommand{\arbtargetlink}[4]{#2\pageref{#4}#3}
  \ifnum\oommfpdf=1 % PDF output (with links)
    \newcommand{\pttarget}[2][]{\hypertarget{#2}{#1}}
    \newcommand{\ptlink}[2]{\hyperlink{#2}{#1}}
  \else % PS output (no links)
    \newcommand{\pttarget}[2][]{#1}
    \newcommand{\ptlink}[2]{#1}
  \fi
\fi
%end{latexonly}
\html{ % LaTeX2HTML version
  \newcounter{arbtargetcounter}
  \newcommand{\arbtarget}[2]{\refstepcounter{arbtargetcounter}\label{#2}}
  \newcommand{\arbtargetlink}[4]{\htmlref{#1}{#4}}
  \newcommand{\pttarget}[2][]{#1\refstepcounter{arbtargetcounter}\label{#2}}
  \newcommand{\ptlink}[2]{\htmlref{#1}{#2}}
}

\newcommand{\MailLink}[2]{%
\HTMLoutput{\htmladdnormallink{#1}{mailto:#2}}%
\NONHTMLoutput{#1 (#2)}}

\newcommand{\blackhole}[1]{}
\newcommand{\Unix}{Unix}
\newcommand{\X}{X}
\newcommand{\Linux}{Linux}
\newcommand{\Windows}{Windows}
\newcommand{\MacOSX}{macOS}
\newcommand{\DOS}{DOS}
\newcommand{\Tcl}{Tcl}  % Tcl Developer Xchange = https://www.tcl-lang.org/
\newcommand{\C}{C}
\newcommand{\Cplusplus}{C++}
\newcommand{\Tk}{Tk}
\newcommand{\OOMMF}{OOMMF}
\newcommand{\MIF}{MIF}
\newcommand{\ODT}{ODT}
\newcommand{\OVF}{OVF}
\newcommand{\SVF}{SVF}
\newcommand{\VIO}{VIO}
\newcommand{\OBS}{OBS}
\newcommand{\eps}{Encapsulated PostScript}
\newcommand{\postscript}{PostScript}
\latex{\newcommand{\mumag}{$\mu$MAG}}
\html{\newcommand{\mumag}{muMAG}}
\newcommand{\micrometer}{\latex{$\mu$m}\html{\begin{rawhtml}&micro;m\end{rawhtml}}}
\newcommand{\munaught}{\latex{$\mu_0$}\html{\begin{rawhtml}&micro;<SUB>0</SUB>\end{rawhtml}}}
\newcommand{\SI}{SI}     % as in SI units
\newcommand{\ASCII}{ASCII}
\newcommand{\emdash}{\latex{---}\html{\begin{rawhtml}&mdash;\end{rawhtml}}}

% --- Font control reference --
% The first set below are cumulative. They come in both command
% (e.g., \textbf{...}) and declarative (e.g., {\bfseries ...}) flavors:
%
% \textrm{...}      {\rmfamily ...}    Roman (default)
% \textsf{...}      {\sffamily ...}    Sans serif
% \texttt{...}      {\ttfamily ...}    Typewriter (monospaced)
%
% \textup{...}      {\upshape ...}     Upright (default)
% \textit{...}      {\itshape ...}     Italics
% \textsl{...}      {\slshape ...}     Slanted
% \textsc{...}      {\scshape ...}     Small caps
%
% \textmd{...}      {\mdseries ...}    Medium weight (default)
% \textbf{...}      {\bfseries ...}    Boldface.
%
% \textnormal{...}  {\normalfont ...}  Main document font
%
% There is also \emph{...} for text to be emphasized. The effect depends
% on the currently active font; typically \emph will switch to italic,
% but will switch e.g. to roman if the active font is italic.
%
%
% The following, older font control method below are "unconditional,"
% meaning non-cumulative. These are all declarative:
%
% {\rm ...}  Roman
% {\sf ...}  Sans serif
% {\tt ...}  Typewriter (monospace, fixed-width)
%
% {\it ...}  Italics
% {\sl ...}  Slanted (oblique)
% {\sc ...}  Small caps
%
% {\bf ...}  Switch to bold face
%
% {\cal ...} Switch to calligraphic letters for math
%
% The \em command is the unconditional form of \emph.
%
% The following, non-cumulative commands are for math mode:
%
% \mathrm{...} Roman
% \mathsf{...} Sans serif
% \mathtt{...} Typewriter
%
% \mathit{...} Italics, aka \mit{...}
%
% \mathbf{...} Boldface
%
% \mathnormal{...} Normal, used inside another type style declaration
%
% \mathcal{...} Calligraphic letters
%
% There are additionally the commands \mathversion{bold}
% and \mathversion{normal} for switching between bold and normal fonts.
%
%
% NB: Latex2html 2021 doesn't handle nested face requests, e.g.,
% \texttt{\textrm{...}} properly --- it drops a <DIV> block that
% effects a newline. So separate \html{...} declarations may be
% needed to get the best achievable results.

\latex{\newcommand{\oxslabel}[1]{\textbf{\textrm{#1}}}}
\latex{\newcommand{\oxsval}[1]{\textit{\textrm{#1}}}}
\html{\newcommand{\oxslabel}[1]{\textbf{#1}}}
\html{\newcommand{\oxsval}[1]{\textit{#1}}}
% Use \oxslabel to refer to Oxs Specify block labels the first time in
% the running text. Use \oxsval for the value portion of label+value
% keys in both the TeX version of the Specify block, and in the running
% text. (BTW, latex2html gives a "cannot wrap" warning on \renewcommand,
% so just use separate \latex and \html versions.)


% Filenames and program code identifiers
\blackhole{
\definecolor{fn}{rgb}{0,0.5,0}
\definecolor{cd}{rgb}{0.5,0,0}
\definecolor{btn}{rgb}{0.5,0,0}
\newcommand{\fn}[1]{\latex{{\tt #1}}\html{\textcolor{fn}{#1}}}   % Files
\newcommand{\cd}[1]{\latex{{\tt #1}}\html{\textcolor{cd}{#1}}}   % Code
\newcommand{\btn}[1]{\latex{{\tt #1}}\html{\textcolor{btn}{#1}}} % Buttons
} % blackhole

%begin{latexonly}
\newcommand{\bftt}[1]{\textsf{\textbf{#1}}}
%% This is meant to be a bold tt, but there is no boldface cmtt (TeX
%% Typewriter font)!
\newcommand{\app}[1]{\textbf{#1}}    % Apps
\newcommand{\key}[1]{\texttt{#1}}    % Keys
\newcommand{\fn}[1]{\texttt{#1}}     % Files
\newcommand{\cd}[1]{\texttt{#1}}     % Code
\newcommand{\btn}[1]{\bftt{#1}}      % Buttons
\newcommand{\wndw}[1]{\textbf{#1}}   % Windows
%end{latexonly}

\begin{htmlonly}
\newcommand{\bftt}[1]{\texttt{\textbf{#1}}}
%% NOTE: There is no boldface cmtt (TeX), but HTML browsers
%%  may render differently?!
\newcommand{\app}[1]{\textbf{#1}}    % Apps
\newcommand{\key}[1]{\bftt{#1}} % Keys
\newcommand{\fn}[1]{\bftt{#1}}  % Files
\newcommand{\cd}[1]{\texttt{#1}}  % Code
\definecolor{btn}{rgb}{0.5,0,0}          % Buttons
\newcommand{\btn}[1]{{\textcolor{btn}{\textbf{#1}}}}
\newcommand{\wndw}[1]{{\bf #1}} % Windows
\end{htmlonly}

% Latex2html inserts unwanted whitespace after \rm, in structures like
%      \newcommand{\vB}{{\rm\bf B}}
% but the following seem to work:
\newcommand{\vB}{\textbf{B}}
\newcommand{\vH}{\textbf{H}}
\newcommand{\vM}{\textbf{M}}
\newcommand{\vm}{\textbf{m}}
\newcommand{\vh}{\textbf{h}}
\newcommand{\vx}{\textbf{x}}
\newcommand{\vu}{\textbf{u}}

\newcommand{\lb}{\texttt{\#}}  % "Pound" symbol
\newcommand{\pipe}{\latex{{\tt|}}\html{|}} % "Pipe" symbol
\newcommand{\bs}{\texttt{\char'134}} % Backslash, tt font
\newcommand{\fs}{\texttt{/}} % Forward slash, tt font

% MIF 2.x Specify block definitions.
\newcommand{\bi}{\hspace*{2em}}
% \bi is bullet indent.
\latex{\newcommand{\ocb}{\textrm{\{}}}
\latex{\newcommand{\ccb}{\textrm{\}}}}
% \ocb is open curly brace, \ccb is close curly brace.
% Latex2html 2021 sometimes mishandles typeface requests, so try
\html{\newcommand{\ocb}{\{}}
\html{\newcommand{\ccb}{\}}}

% O open and close angle brackets (aka less-than and greater-than
% symbols)
\newcommand{\oab}{\latex{{$<$}}\html{\texttt{<}}}
\newcommand{\cab}{\latex{{$>$}}\html{\texttt{>}}}

% Bold open and close angle brackets (aka less-than and greater-than
% symbols)
\newcommand{\boa}{\latex{{\boldmath$<$}}\html{\texttt{\textbf{<}}}}
\newcommand{\bca}{\latex{{\boldmath$>$}}\html{\texttt{\textbf{>}}}}

% ``Launching'' option keyword lists font selection
\newcommand{\optkey}[1]{\latex{\textbf{#1}}\html{\texttt{\textbf{#1}}}}

% Codelisting environment
% \newenvironment{codelisting}[4]{%
%  \def\codelistingtype{#1}     % f for float, p for ``in page''
%  \def\codelistinglabel{#2}    % \label tag
%  \def\codelistingcaption{#3}  % caption
%  \def\codelistingdescription{#4} % Link back to text
%  \def\codelistingdesctype{#5} % Link anchor type: ref or hyperlink
% Parameters #4 and #5 are ignored in postscript output, but for PDF and
% both HTML outputs these provide a link back to a point in the text
% describing the contents. That link can either be a counter-based ref
% accessed via \hyperref or else a \hypertarget type accessed via
% \hyperlink, as indicated by parameter #5.  (In the latex and latexml
% cases the conditional is handled using the ifthenelse command from the
% latex \ifthen package. For latex2html the condition is implemented by
% variable name interpolation.)
%
%begin{latexonly}
\iflatexml %%%%%%%%%%%%%%%%%%%%%%%%%%%%%%%%%%%%%%%%%%%%%%%%%%%%%%%%%%%%%%%%%%%%%%%%
\newenvironment{codelisting}[5]{%
 \def\codelistingtype{#1} % f for float, p for in page (ignored for html)
 \def\codelistinglabel{#2}       % \label tag
 \def\codelistingcaption{#3}     % caption
 \def\codelistingdescription{#4} % Link back to text
 \def\codelistingdesctype{#5}    % Link anchor type: ref or hyperlink
 \ifthenelse{\equal{\codelistingdesctype}{ref}}{
   \newcommand{\cldxyz}{\htmlonlyref{(description)}{\codelistingdescription}}
 }{
   \newcommand{\cldxyz}{\ptlink{(description)}{\codelistingdescription}}
    % Note space between caption and (description)!
 }
 \begin{figure}[h!]
   \centerline{\rule[1ex]{\textwidth}{0.5ex}}
   \caption{\codelistingcaption\label{\codelistinglabel}
     \protect\cldxyz} % Note space between caption and (description)!
}{
  \centerline{\rule[1ex]{\textwidth}{0.5ex}}
  \end{figure}
}
\else % !iflatexml %%%%%%%%%%%%%%%%%%%%%%%%%%%%%%%%%%%%%%%%%%%%%%%%%%%%%%%%%%%%%%%%
\newenvironment{codelisting}[5]{%
 \def\codelistingtype{#1}     % f for float, p for ``in page''
 \def\codelistinglabel{#2}    % \label tag
 \def\codelistingcaption{#3}  % caption
 \def\codelistingdescription{#4} % Link back to text
 \def\codelistingdesctype{#5}    % Link anchor type: ref or hyperlink
 \ifthenelse{\equal{\codelistingdesctype}{ref}}{
   \newcommand{\cldxyz}{\htmlref{(description)}{\codelistingdescription}}
 }{
   \newcommand{\cldxyz}{\ptlink{(description)}{\codelistingdescription}}
 }
 \if\codelistingtype f \begin{figure}
 \fi
}{
 \if\codelistingtype f
   \caption{\codelistingcaption\label{\codelistinglabel}\notpsonly{
       \protect\cldxyz}}\end{figure}
   % Note space between caption and (description)!
 \else
    \nopagebreak\parbox{\textwidth}{
    \begin{center}
    \refstepcounter{figure}
    Figure \thefigure: {\codelistingcaption\label{\codelistinglabel}\notpsonly{
    \cldxyz}}  % Note space between caption and (description)!
    \end{center}
   }\pagebreak[2]
 \fi
}

\fi % iflatexml %%%%%%%%%%%%%%%%%%%%%%%%%%%%%%%%%%%%%%%%%%%%%%%%%%%%%%%%%%%%%%%%%%%%
%end{latexonly}

\html{
\newenvironment{codelisting}[5]{%
  \addtocounter{figure}{1}\label{#2}
  \HTMLsetenv{codelistingcaption}{#3}
  \HTMLsetenv{textlink}{#4}
  \HTMLsetenv{desctype}{#5}   % Link anchor type: ref or hyperlink
  % A cunning plan for conditional processing:
  \HTMLsetenv{Xclref}{\htmlonlyref{(description)}{#4}}
  \HTMLsetenv{Xclhyperlink}{\ptlink{(description)}{#4}}
  \HTMLsetenv{backlink}{\HTMLget{Xcl#5}}
  \htmlrule
}{
  \begin{center}
  Figure \thefigure:
    \HTMLget{codelistingcaption}
    \HTMLget{backlink}
  \end{center}
  \htmlrule
}}

% List structure compatible with latex, pdflatex, latex2html, and
% latexml that can be used to create aligned text like
%
%           tclsh oommf.tcl oxspkg list
%    or
%           tclsh oommf.tcl oxspkg listfiles pkg
%    or
%           tclsh oommf.tcl oxspkg readme pkg
%
% with latex code
%
%    \begin{duplex}
%    \item \verb+tclsh oommf.tcl oxspkg list+
%    \item[\textbf{or}]\html{\\}
%    \item \verb+tclsh oommf.tcl oxspkg listfiles pkg [pkg ...]+
%    \item[\textbf{or}]\html{\\}
%    \item \verb+tclsh oommf.tcl oxspkg readme pkg [pkg ...]+
%    \end{duplex}
%
% Note the interspersed \item commands with labels and no text and text
% with no labels. If you don't include the \html{\\} in the no text
% case then latex2html puts the successive item on the same line.
%
\newenvironment{duplex}%
{\begin{list}%
     {\hspace{\notlatexmlonly{2em}\latexmlonly{4em}}} % labeling
     { \setlength{\leftmargin}{0.5em}
       \setlength{\listparindent}{0pt}
       \setlength{\parindent}{0pt}
       \setlength{\itemsep}{-0.5\baselineskip}
       \setlength{\labelwidth}{0pt}
     } % set spacing
}{\end{list}}


% Ersatz figure environment.  This is a standard figure environment in
% LaTex, but a dummy block in HTML.  This is useful because LaTeX2HTML
% passes figure environments to LaTex, and converts the resulting
% postscript to a graphics bitmap for inclusion.  Sometimes we don't
% want this, for example if the figure data is already in bitmap format.
% Also, we may want to throw in an ALT tag.
% SAMPLE USAGE:
%   \ofig{\includeimage{6in}{!}{oxsclass}{Oxs class diagram}}{OXS
%        top-level class diagram.}{fig:oxsclass}
% Note: The fourth argument to \includeimage is an ALT tag. Beware
%   that newlines in the ALT tag field cause breakage in LaTeX2HTML
%   handling that results in the ALT tag being dropped altogether.
%
%begin{latexonly}
\iflatexml
% \newcommand{\ofig}[3]%
% {\begin{center}
%  \addtocounter{figure}{1}\label{#3}
%  \textbf{Figure \thefigure: #2}\\
%  #1
% \end{center}}
%
%
\newcommand{\ofig}[3]{%
\begin{figure}
 \begin{center}
   #1\\
   \caption{#2\label{#3}}
 \end{center}
\end{figure}}
\else
\newcommand{\ofig}[3]{%
\begin{figure}
 \begin{center}
   #1\\
   \caption{#2\label{#3}}
 \end{center}
\end{figure}}
\fi
%end{latexonly}
\html{
\newcommand{\ofig}[3]%
{\begin{center}
 \addtocounter{figure}{1}\label{#3}
 \textbf{Figure \thefigure: #2}\\
 #1
\end{center}}
}
%% Is \refstepcounter{figure} needed in the \html def?


% Graphics inclusion.
%  Usage: \includepic{basename}{altstring}
%     A fixed scale parameter is used in the LaTeX code;
%   under HTML the graphic is brought directly in without any scaling.
%     Basename is the name of the graphic to include,
%   expanded as psfiles/basename.ps under latex, and
%   giffiles/basename.gif under html.
%     Altstring it a string to be passed to the ALT= tag
%   in HTML.  It is not used in the LaTeX code.
% Note: Previously the scaling for psfiles was set to "0.5", with the
%   note that that setting provided excellent onscreen rendering in
%   ghostview if scaling were set to 4.0 "pixel-based", although this
%   made the PostScript images slightly larger than in the PDF output.
%   To match the sizes the \scalebox setting needed to be 0.462.
%   However, PostScript for screenshots made in July 2021 used
%   different processing, based on the ImageMagick 'convert' tool. The
%   screenshots were collected using the gnome-screenshot command on
%   Linux,
%
%    gnome-screenshot -wbd 5 -e shadow -f mmdisp-ss.png
%
%   and then the pixel density was set like so:
%
%    convert -quality 97 -density 125 -units pixelsperinch mmdisp-ss.png
%
%   The density determines the scaling for the PDF output. The eps
%   files for the PostScript version of the userguide were created
%   with
%
%    convert mmdisp-ss.png pdf:- | pdftops -eps - mmdisp-ss.ps
%
%   With this processing chain the Postscript renders at the same size
%   as the PDF.

%begin{latexonly}
\iflatexml
% includepic for latexml
\newcommand{\includepic}[2]{%
\scalebox{1.0}{\includegraphics{pngfiles/#1.png}} }
\else
 \ifnum\oommfpdf=0
   % includepic for latex
   \newcommand{\includepic}[2]{%
   \scalebox{1.0}{\includegraphics{psfiles/#1.ps}} }
 \else
   % includepic for pdflatex
   \newcommand{\includepic}[2]{%
   \scalebox{1.0}{\includegraphics{pngfiles/#1.png}} }
 \fi
\fi
%end{latexonly}
\begin{htmlonly}
% includepic for latex2html
\newcommand{\includepic}[2]{%
\HTMLcode[../giffiles/#1.gif,ALT="#2"]{IMG}
}
\end{htmlonly}

% Alternate graphics inclusion
%  Usage: \includeimage{width}{height}{basename}{altstring}
%     Width and height are dimensions, e.g., 4in.  One of
%   these may be an exclamation mark '!', in which case
%   the corresponding dimension will be scaled as necessary
%   to keep the original aspect ratio.  Presently these two
%   parameters are used only in the LaTeX code; under HTML
%   (both LaTeX2HTML and LaTeXML) the graphic is brought
%   directly in without any scaling.
%     Basename is the name of the graphic to include,
%   expanded as psfiles/basename.ps under latex, and
%   giffiles/basename.gif or pngfiles/basename.png under html.
%     Altstring it a string to be passed to the ALT= tag
%   in HTML. It works with latex2html, but is ignored by
%   latex and pdflatex. It is also currently ignored by
%   latexml, although there has been some discussion on this
%   on the latexml github issues page, Feb-Dec 2021. Check
%   back later?
%
%   NB: The ALT tag is mostly read verbatim, and newlines cause breakage
%       resulting in the ALT tag being dropped altogether.  Whitespace
%       is retained, but LaTeX non-breaking spaces characters "~" are
%       converted to HTML "&nbsp;". This can be used to protect text
%       against automatic line splitting from text editors.
%
%begin{latexonly}
\iflatexml % latexml
\renewcommand{\includeimage}[4]{%
\includegraphics{pngfiles/#3.png}%
}
\else
\ifnum\oommfpdf=0 % latex
\newcommand{\includeimage}[4]{%
\resizebox{#1}{#2}{\includegraphics{psfiles/#3.ps}}%
}
\else % pdflatex
\newcommand{\includeimage}[4]{%
\resizebox{#1}{#2}{\includegraphics{pngfiles/#3.png}}%
}
\fi % end \oommfpdf=0
\fi % end \iflatexml
%end{latexonly}
\begin{htmlonly}
\newcommand{\includeimage}[4]{%
\HTMLcode[../giffiles/#3.gif,ALT="#4"]{IMG}
}
\end{htmlonly}

% Workaround for some apparently broken LaTeX2HTML Table of Contents
% controls.
% Also, a hackish way to stop LaTeXML file splitting at said sections.
% I think it may be possible to accomplish this via an appropriate XPATH
% option to --splitpath=, but I haven't been able to figure one out.
\latex{
\iflatexml
 \newcommand{\starsechead}[1]{\par\noindent{\Large\textbf{#1}}\\}
 \newcommand{\starssechead}[1]{\par\noindent{\large\textbf{#1}}\\}
 \newcommand{\starsssechead}[1]{\par\noindent{\large\textbf{#1}}\\}
\else
 \def\starsechead{\section*}
 \def\starssechead{\subsection*}
 \def\starsssechead{\subsubsection*}
\fi
}
\html{
\newcommand{\starsechead}[1]{\par\noindent{\Large\bf{#1}}\\}
\newcommand{\starssechead}[1]{\par\noindent{\large\bf{#1}}\\}
\newcommand{\starsssechead}[1]{\par\noindent{\large\bf{#1}}\\}
}


% If an inline formula has positive depth, then LaTeX2HTML handles
% vertical positioning of that formula by adding a vertical rule so
% that the depth and height are equal.  The resulting image is then
% marked in the HTML with the align=middle tag, which aligns the
% vertical center of the image with the current baseline.  This adds
% extra whitespace below the image, sometimes a lot, which can yield
% essentially an extra blank line in the viewed HTML.  The \abovemath
% command raises the math-mode formulae just enough so that the depth
% is zero, in which case the generated image is aligned in the HTML
% with the align=bottom tag.  This also looks bad, so it is a matter
% of choice which is the worse evil.  But it is probably an improvement
% in situations with the formula extends just a *little* below the
% baseline.  WRT the TeX output, this command just renders the formula
% in in-line math mode.
\newcommand{\nodepth}[1]{% Auxiliary command
$\mbox{\renewcommand{\arraystretch}{0}%
$\begin{array}[b]{@{}c@{}}#1\\\rule{1pt}{0pt}\end{array}$}$}
\newcommand{\abovemath}[1]{\latex{$#1$}\html{\nodepth{#1}}}

% Hyphenation
\hyphenation{DataTable}

% Index generation
\makeindex

\usepackage{l2hbugs}

% If there are more than nine subsections in a section (for example,
% in the ``Command Line Utilities'' section, then in the table of
% contents the subsection numbers run into the subsection titles.  One
% workaround is:
%
%   \usepackage{tocloft}
%   \setlength{\cftsubsecnumwidth}{2.7em}
%
% However this increases the spacing between subsection numbers and
% titles for all subsections in the toc.  A slightly less ugly
% alternative is
%
\usepackage{tocloft}
\newlength{\oommftocsslen}
\setlength{\oommftocsslen}{0.5em} % need some extra space at end of number
\renewcommand{\cftsubsecpresnum}{\hfill} % note the double `l'
\renewcommand{\cftsubsecaftersnum}{\hspace*{\oommftocsslen}}
\addtolength{\cftsubsecnumwidth}{\oommftocsslen}
%
% This typesets subsection numbers flushright.  With this the
% subsection numbers after .9 stick out to the left, but otherwise
% everything lines up.
%
% Another approach may be to hack the userguide.toc file directly.
% But probably the first thing to do is to change the \documentclass
% from article to something more fitting for a 250 page document.
% (Perhaps ``report''?)  And maybe ``Command Line Utilities'' just has
% too many sections and needs to be broken up.

%\HTMLset{toppage}{userguide.html}
%\htmladdtonavigation{\htmladdnormallink{\htmladdimg{../common/contents.gif}}{userguide.html}}

\begin{document}

\nocite{*}  % Include all entries from .bib file.  Putting this at the
	    % top retains the .bib file ordering.


\pagenumbering{roman}
\begin{titlepage}
\label{page:contents}
\par
\vspace*{\fill}
\begin{center}
\Large\bf
\OOMMF\\
User's Guide\\[2ex]
\large
{\today}
{}\\[2ex]
This manual documents release 2.0a3.\\[1ex]
WARNING: In this alpha release, the
documentation may not be up to date.

\end{center}
\vspace{10\baselineskip}
\begin{abstract}
This manual describes \OOMMF\ (Object Oriented Micromagnetic Framework),
a public domain micromagnetics program developed at the
\htmladdnormallink{National Institute of Standards and Technology}
{https://www.nist.gov/}.  The program is designed
to be portable, flexible, and extensible, with a user-friendly graphical
interface.  The code is written in C++ and Tcl/Tk. Target systems
include a wide range of \Unix, \Windows, and \MacOSX\ platforms.
\end{abstract}
\vspace*{\fill}
\par
\end{titlepage}

\begin{latexonly}
\tableofcontents
\end{latexonly}

% Index cross-references; if these are moved to the bottom of this file
% then two LaTeX passes are required to get them in the .idx file.
% See, however, ``seealso'' entries at below.
\index{bitmap~files|see{file,~bitmap}}
\index{Borland~C++|see{platform,~Windows,~Borland~C++}}
\index{bug~reports|see{reporting~bugs}}
\index{control~points|see{simulation,~control~point}}
\index{Cygwin|see{platform,~Windows,~Cygwin~environment}}
\index{decompress|see{compressed~files}}
\index{file!hosts|see{platform,~Windows,~hosts~file}}
\index{file!odt|see{file,~data~table}}
\index{file!omf|see{file,~magnetization}}
\index{file!obf|see{file,~vector~field}}
\index{file!ohf|see{file,~vector~field}}
\index{file!ovf|see{file,~vector~field}}
\index{file!restart|see{file,~checkpoint}}
\index{Internet|see{TCP/IP}}
\index{Landau-Lifshitz|see{ODE,~Landau-Lifshitz}}
\index{mask~file|see{file,~mask}}
\index{mesh|see{grid}}
\index{MIF|see{file,~mif}}
\index{movies|see{animations}}
\index{mxh|see{simulation,~mxh}}
\index{network~socket!bug|see{platform,~Windows,~network~socket~bug}}
\index{OBS|see{application,~OOMMF~Batch~System}}
\index{batch processing|see{application, Boxsi, launchhost, and OOMMF~Batch~System}}
\index{requirement!application~version|see{launch,~version~requirement}}
\index{self-magnetostatic|see{demagnetization}}
\index{threads,~parallel|see{parallelization}}
\index{torque|see{simulation,~mxh}}
\index{total~field|see{field,~effective}}
\index{Xvfb|see{application,~Xvfb}}

\newpage
\section*{Disclaimer}
\addcontentsline{toc}{section}{Disclaimer}

The research software described in this manual (``software'') is provided
by NIST as a public service. You may use, copy and distribute copies of
the software in any medium, provided that you keep intact this entire notice.
You may improve, modify and create derivative works of the software or any
portion of the software, and you may copy and distribute such modifications
or works. Modified works should carry a notice stating that you changed the
software and should note the date and nature of any such change. Please
explicitly acknowledge the National Institute of Standards and Technology
as the source of the software.

The software is expressly provided "AS IS." NIST MAKES NO WARRANTY OF ANY
KIND, EXPRESS, IMPLIED, IN FACT OR ARISING BY OPERATION OF LAW, INCLUDING,
WITHOUT LIMITATION, THE IMPLIED WARRANTY OF MERCHANTABILITY, FITNESS FOR A
PARTICULAR PURPOSE, NON-INFRINGEMENT AND DATA ACCURACY. NIST NEITHER
REPRESENTS NOR WARRANTS THAT THE OPERATION OF THE SOFTWARE WILL BE
UNINTERRUPTED OR ERROR-FREE, OR THAT ANY DEFECTS WILL BE CORRECTED. NIST
DOES NOT WARRANT OR MAKE ANY REPRESENTATIONS REGARDING THE USE OF THE
SOFTWARE OR THE RESULTS THEREOF, INCLUDING BUT NOT LIMITED TO THE
CORRECTNESS, ACCURACY, RELIABILITY, OR USEFULNESS OF THE SOFTWARE. 

You are solely responsible for determining the appropriateness of using
and distributing the software and you assume all risks associated with
its use, including but not limited to the risks and costs of program errors,
compliance with applicable laws, damage to or loss of data, programs or
equipment, and the unavailability or interruption of operation. This
software is not intended to be used in any situation where a failure
could cause risk of injury or damage to property.  The software was
developed by NIST employees. NIST employee contributions are not subject
to copyright protection within the United States. 

We would appreciate acknowledgement if the software is used.  When
referencing \OOMMF\ software, we recommend citing the NIST technical
report, M. J. Donahue and D. G. Porter, ``OOMMF User's Guide, Version
1.0,'' \textbf{NISTIR 6376}, National Institute of Standards and
Technology, Gaithersburg, MD (Sept 1999).

Commercial equipment and software referred to on these pages are
identified for informational purposes only, and does not imply
recommendation of or endorsement by the National Institute of Standards
and Technology, nor does it imply that the products so identified are
necessarily the best available for the purpose.

\newpage

\pagenumbering{arabic}

\chapter{Programming Overview of \OOMMF}\label{sec:overview}
The
\htmladdnormallinkfoot{\OOMMF}{https://math.nist.gov/oommf/} (Object
Oriented Micromagnetic Framework) project in the
\htmladdnormallinkfoot{Information Technology Laboratory}{https://www.nist.gov/itl/}
(ITL) at the
\htmladdnormallinkfoot{National Institute of Standards and
Technology}{https://www.nist.gov/} (NIST) is intended to develop a
portable, extensible public domain micromagnetic program and associated
tools.  This manual aims to document the programming interfaces to
\OOMMF\ at the source code level.  The main developers of this code are
\psonly{\htmladdnormallinkfoot{Mike Donahue}{https://math.nist.gov/\%7EMDonahue}}
\notpsonly{\htmladdnormallink{Mike Donahue}{https://math.nist.gov/\%7EMDonahue}}
and
\psonly{\htmladdnormallinkfoot{Don Porter}{https://math.nist.gov/\%7EDPorter}.}
\notpsonly{\htmladdnormallink{Don Porter}{https://math.nist.gov/\%7EDPorter}.}

The underlying numerical engine for \OOMMF\ is written in \Cplusplus,
which provides a reasonable compromise with respect to efficiency,
functionality, availability and portability.  The interface and glue
code is written primarily in \Tcl/\Tk, which hides most platform
specific issues. \Tcl\ and \Tk\ are available for free
\htmladdnormallinkfoot{download}{http://purl.org/tcl/home/software/tcltk/choose.html}
from the
\htmladdnormallinkfoot{Tcl Developer Xchange}{http://purl.org/tcl/home/}.

The code may actually be modified at 3 distinct levels.  At the top
level, individual programs interact via well-defined protocols across
network sockets\index{network~socket}.  One may connect these modules
together in various ways from the user interface, and new modules
speaking the same protocol can be transparently added.  The second level
of modification is at the \Tcl/\Tk\ script level.  Some modules allow
\Tcl/\Tk\ scripts to be imported and executed at run time, and the top
level scripts are relatively easy to modify or replace.  The lowest
level is the \Cplusplus\ source code.  The OOMMF extensible solver, OXS,
is designed with modification at this level in mind.

If you want to receive e-mail\index{e-mail}
notification\index{announcements} of updates to this project, register
your e-mail address with the ``{\mumag} Announcement'' mailing list:
% Note: For some reason, the braces above about \mumag discourage line
% breaking between $\mu$ and Mag in the latexml/browser display.
\begin{center}
\htmladdnormallink{\url{https://www.ctcms.nist.gov/~rdm/email-list.html}}{https://www.ctcms.nist.gov/\%7Erdm/email-list.html}.
\end{center}

The \OOMMF\ developers are always interested in your comments about
\OOMMF.  See the \hyperrefhtml{Credits}{Credits (Ch.~}{) }{sec:credits}
for instructions on how to contact them.


\section{Installation}\label{sec:install}
\index{installation}

\subsection{Requirements}\label{sec:install.requirements}
\OOMMF\ software is written in C++ and \Tcl.  It uses the \Tcl-based
\Tk\ Windowing Toolkit to create graphical user interfaces that are
portable to many varieties of \Unix\ as well as Microsoft 
\Windows.  

\Tcl\ and \Tk\ must be installed before installing \OOMMF.  
\Tcl\ and \Tk\ are available for free from the
\htmladdnormallinkfoot{Tcl Developer Xchange}{http://purl.org/tcl/home/}.
We recommend the latest stable 
versions of \Tcl\ and \Tk\ concurrent with this release of \OOMMF.
\OOMMF\ requires\index{requirement!Tcl/Tk}
at least \Tcl\ version 7.5 and \Tk\ version 4.1 
on \Unix\ platforms, and requires at least \Tcl\ version 8.0 and 
\Tk\ version 8.0 on Microsoft \Windows\ platforms.  \OOMMF\ software 
does not support any alpha or beta versions of \Tcl/\Tk, and 
each release of \OOMMF\ may not work with later releases of
\Tcl/\Tk.  Check the release dates of both \OOMMF\ and
\Tcl/\Tk\ to ensure compatibility.

A \Tcl/\Tk\ installation includes two shell programs.  The names of 
these programs may vary depending on the \Tcl/\Tk\ version and the 
type of platform.  The first shell program contains an interpreter 
for the base \Tcl\ language.  In the \OOMMF\ documentation we refer 
to this program as \fn{tclsh}\index{application!tclsh}.  
The second shell program contains 
an interpreter for the base \Tcl\ language extended by the 
\Tcl\ commands supplied by the \Tk\ toolkit.  In the 
\OOMMF\ documentation we refer to this program as 
\fn{wish}\index{application!wish}.  
Consult your \Tcl/\Tk\ documentation to determine 
the actual names of these programs on your platform (for example, 
\fn{tclsh83.exe} or \fn{wish8.0}).

\OOMMF\ applications communicate via TCP/IP\index{TCP/IP} network sockets.
This means that \OOMMF\ requires\index{requirement!TCP/IP}
support for networking, even 
on a stand-alone machine.  At a minimum, \OOMMF\ must be able to 
access the loopback interface so that the host can talk to 
itself using TCP/IP.

\index{application!Xvfb|(}
\OOMMF\ applications that use \Tk\ require a windowing system and
a valid display.  On Unix systems, this means that an X server must
be running.  If you need to run \OOMMF\ applications on a Unix system
without display hardware or software, you may need to start the 
application with command line option 
\hyperrefhtml{\texttt{\textbf{-tk 0}}}{\textsf{\textbf{-tk 0}} (see Sec.~}{)}{sec:cll}
or use the
\htmladdnormallinkfoot{Xvfb}{http://www.itworld.com/AppDev/1461/UIR000330xvfb/}
virtual frame buffer.
\index{application!Xvfb|)}

\index{requirement!disk~space|(}
The \OOMMF\ source distribution unpacks into a directory tree containing
about 800 files and directories, occupying approximately 10 MB of
storage.  The amount of disk space needed for compiling and linking
varies greatly between platforms; allow an additional 15 MB to 80 MB for
the build.  Removing intermediate object modules (cf.\ the \cd{pimake}
``objclean'' target, in \html{the} \hyperrefhtml{Reducing Disk Space
Usage}{Reducing Disk Space Usage, Sec.~}{,
below}{sec:install.reducedisk} \html{section}) reduces the final space
requirement for source + binary executables to between 15 MB and 25 MB.
The \OOMMF\ distribution containing \Windows\ executables unpacks into a
directory tree occupying about 15 MB of storage.  {\bf Note:} On a
non-compressed {\tt FAT16} file system on a large disk, \OOMMF\ may take
up much more disk space.  This is because on such systems, the minimum
size of any file is large, as much as 32 KB.  Since this is much larger
than many files in the \OOMMF\ distribution require, a great deal of
disk space is wasted.
\index{requirement!disk~space|)}

\index{compilers}\index{platforms}
To build \OOMMF\ software from source code, you will
need\index{requirement!C++~compiler} a C++ compiler capable of handling
C++ templates, C++ exceptions, and (for the \OOMMF\ eXtensible Solver)
the C++ Standard Template Library.  You will need other software
development utilities for your platform as well.  We do development and
test builds on the following platforms, although porting to others
should not be difficult:

% Note 1: Don't use transparent images, because mmHelp renders them
%   rather slowly.
% Note 2: makeimage (maybe +tabular?) breaks on newer systems, with
%   latex2html v1.71 and v1.68.  OTOH, v1.68 works on older systems,
%   so it is not clear what the problem is.  The error message is:
%          panic: end_shift at /usr/local/bin/latex2html line 11720.
%   This is with Perl v5.8.0.  So, we have to render the table
%   directly.  This is actually preferred with real HTML browsers,
%   but as of this writing (Dec-2004) mmHelp doesn't do tables.
\begin{center}
%\begin{makeimage}
%\htmlimage{no_transparent}
\begin{tabular}{|l|l|}\hline
Platform & Compilers \\ \hline
% AIX & VisualAge C++ (xlC), Gnu g++ \\
%Alpha/Linux & Compaq C++, Gnu g++ \\
%Alpha/Tru64 & Compaq C++ (cxx) \\
%Alpha/Windows NT & Microsoft Visual C++ \\
%HP-UX & aCC \\
Windows &
Microsoft Visual C++, Borland C++, \\
 & Intel C++, MinGW g++
%, Digital Mars dmc
\\
Linux/x86 & Gnu g++, Intel C++, Portland Group pgCC \\
Linux/Itanium & Intel C++, Gnu g++ \\
Mac OS X & Gnu g++ \\
MIPS/IRIX 6 (SGI) & MIPSpro C++, Gnu g++ \\
SPARC/Solaris & Sun Workshop C++, Gnu g++ \\ \hline
\end{tabular}
%\end{makeimage}
\end{center}

\par\noindent
System Notes:
\begin{itemize}
\item \textbf{Linux/x86}: Both 32-bit x86 and 64-bit x86\_64 supported.
\item \textbf{Windows:} Microsoft Visual C++ compiler 6.0 or later required.
%\item \textbf{HP-UX:} The older HP cfront compiler will not build the
%   Oxs (3D) solver.
\end{itemize}

\subsection{Basic Installation}

Follow the instructions in the following sections, in order,
to prepare \OOMMF\ software for use on your computer.

\subsubsection{Download}
\index{download}

The latest release of the \OOMMF\ software may be retrieved from the
\htmladdnormallinkfoot{\OOMMF\ download
page}{http://math.nist.gov/oommf/software.html}.  Each release is
available in two formats.  The first format is a gzipped tar file
containing an archive of all the \OOMMF\ source code.  The second format
is a \fn{.zip} compressed archive containing source code and
pre-compiled executables for Microsoft \Windows.  Each \Windows\ binary
distribution is compatible with only a particular sequence of releases
of \Tcl/\Tk.  For example, a Windows binary release for \Tcl/\Tk\ 8.3.x
is compatible with \Tcl/\Tk\ 8.3.0, 8.3.1, \ldots.  Other release
formats, e.g., pre-compiled executables for Microsoft
\Windows~NT running on a Compaq Alpha Systems RISC-based microprocessor
system, and/or compatible with older versions of \Tcl/\Tk, may be made
available upon request.

For the first format, unpack the distribution archive using gunzip and
tar:
\begin{verbatim}
gunzip -c oommf12a4_20040908.tar.gz | tar xvf -
\end{verbatim}

For the other format(s), you will need a utility program to unpack the
\fn{.zip} archive.  This program must preserve the directory structure
of the files in the archive, and it must be able to generate files with
names not limited to the old MSDOS 8.3 format.  Some very old
versions of the pkzip utility do not have these properties.  One utility
program which is known to work is
\htmladdnormallinkfoot{UnZip}{http://www.info-zip.org/pub/infozip/UnZip.html}.

Using your utility, unpack the \fn{.zip} archive, e.g.
\begin{verbatim}
unzip oommf12a4_20040908_84.zip
\end{verbatim}

For either distribution format, the unpacking sequence creates a
subdirectory \fn{oommf} which contains all the files and directories
of the \OOMMF\ distribution.  If a subdirectory named \fn{oommf}
already existed (say, from an earlier \OOMMF\ release), then
files in the new distribution overwrite those of the same name already
on the disk.  Some care may be needed in that circumstance to be
sure that the resulting mix of files from an old and a new 
\OOMMF\ distribution combine to create a working set of files.

\subsubsection{Check Your Platform Configuration}
\index{platform!configuration}

After downloading and unpacking the \OOMMF\ software distribution, all
the \OOMMF\ software is contained in a subdirectory named \fn{oommf}.
Start a command line interface (a shell on \Unix, or the MS-DOS Prompt 
on Microsoft \Windows), and change the 
working directory\index{working~directory} to the 
directory \fn{oommf}.  
Find the Tcl shell program installed as part of your Tcl/Tk 
installation.  In this manual we call the Tcl shell program
\fn{tclsh}, but the actual name of the executable depends
on the release of \Tcl/\Tk\ and your platform type.  Consult
your \Tcl/\Tk\ documentation.

In the root directory of the \OOMMF\ distribution is a file
named \fn{oommf.tcl}.  It is the 
\hyperrefhtml{bootstrap application}{bootstrap application (Sec.~}{)}{sec:cll}
which is used to launch all \OOMMF\ software.  With the command line
argument \cd{+platform}, 
it will print a summary of your
platform configuration when it is evaluated by \fn{tclsh}.
This summary describes your platform type, your C++ compiler,
and your \Tcl/\Tk\ installation.  As an example, 
here is the typical output on a Linux/x86\_64 system:
\begin{verbatim}
$ tclsh8.4 oommf.tcl +platform
oommf.tcl 1.2.0.4  info:
OOMMF release 1.2.0.4
Platform Name:          linux-x86_64
Tcl name for OS:        Linux 2.6.18-274.7.1.el5
C++ compiler:           /usr/bin/g++ 
 Version string:         g++ (GCC) 4.1.2 20080704 (Red Hat 4.1.2-51)
Shell details ---
 tclsh (running):       /usr/bin/tclsh8.4
                         --> Version 8.4.13, 64 bit, threaded
 tclsh (OOMMF):         /usr/bin/tclsh8.4
                         --> Version 8.4.13, 64 bit, threaded
 filtersh:              /usr/local/oommf/app/omfsh/linux-x86_64/filtersh
                         --> Version 8.4.13, 64 bit, threaded
 tclConfig.sh:          /usr/lib64/tclConfig.sh
                         --> Version 8.4.13
 wish (OOMMF):          /usr/bin/wish8.4
                         --> Version 8.4.13, Tk 8.4.13, 64 bit, threaded
 tkConfig.sh:           /usr/lib64/tkConfig.sh
                         --> Tk Version 8.4.13
OOMMF threads:          Yes
  Default thread count:   4
  NUMA support:           No
Temp file directory:    /tmp
\end{verbatim}

If \cd{oommf.tcl +platform} doesn't print a summary similar to the
above, it should instead print an error message describing why it can't.
For example, if your \Tcl\ installation is older than release 7.5, the
error message will report that fact.  Follow the instructions provided
and repeat until \cd{oommf.tcl +platform} successfully prints a summary
of the platform configuration information.

\index{platform!names|(}
The first line of the example summary reports that \OOMMF\ recognizes
the platform by the name \cd{linux-x86\_64}.  \OOMMF\ software recognizes many
of the more popular computing platforms, and assigns each a platform
name.  The platform name is used by \OOMMF\ in index and configuration
files and to name directories so that a single \OOMMF\ installation can
support multiple platform types.  If \cd{oommf.tcl +platform} reports
the platform name to be ``unknown'', then you will need to add some
configuration files to help \OOMMF\ assign a name to your platform type,
and associate with that name some of the key features of your computer.
See the section on
\hyperrefhtml{Managing \OOMMF\ platform names}
{``Managing \OOMMF\ platform names'' (Sec.~}{)}{sec:platformNames}
for further instructions.
\index{platform!names|)}

The second line reports the operating system version, which is mainly
useful to \OOMMF\ developers when fielding bug reports.  The third line
reports what C++ compiler will be used to build \OOMMF\ from its C++
source code.  If you downloaded an \OOMMF\ release with pre-compiled
binaries for your platform, you may ignore this line.  Otherwise, if
this line reports ``none selected'', or if it reports a compiler other
than the one you wish to use, then you will need to tell \OOMMF\ what
compiler to use.  To do that, you must edit the appropriate
configuration file for your platform.  Continuing the example above, one
would edit the file
\fn{config/platforms/linux-x86\_64.tcl}\index{configuration~values}.
Editing instructions are contained within the file.  On other platforms
the name \fn{linux-x86\_64} in \fn{config/platforms/linux-x86\_64.tcl}
should be replaced with the platform name \OOMMF\ reports for your
platform.  For example, on a 32-bit Windows machine using an x86
processor, the corresponding configuration file is
\fn{config/platforms/wintel.tcl}.

\index{installation!Tcl/Tk|(}
The next group of lines describe the \Tcl\ configuration \OOMMF\ finds
on your platform.  The first couple of lines, ``tclsh (running)'',
describe the \Tcl\ shell running the oommf.tcl script.  After that, the
``tclsh (\OOMMF)'' subgroup describes the \Tcl\ shell that \OOMMF\ will
launch when it needs to run vanilla \Tcl\ scripts.  If the \OOMMF\ binaries
have been built, then there will also be a \cd{filtersh} subgroup, which
describes the augmented \Tcl\ shell used to run many of the
\OOMMF\ support scripts.  All of these shells should report the same
version, bitness, and threading information.  If \OOMMF\ can't find
\cd{tclsh}, or if it finds the wrong one, you can correct this by
setting the environment variable
OOMMF\_TCLSH\index{environment~variables!OOMMF\_TCLSH} to the absolute
location of \cd{tclsh}.  (For information about setting environment
variables, see your operating system documentation.)

Following the \Tcl\ shell information, the \cd{tclConfig.sh} lines
report the name of the configuration file installed as part of \Tcl, if
any.  Conventional \Tcl\ installations on Unix systems and within the
Cygwin environment\index{platform!Windows!Cygwin~environment} on
\Windows\ have such a file, usually named \fn{tclConfig.sh}.  The
\Tcl\ configuration file records details about how \Tcl\ was built and
where it was installed.  On \Windows\ platforms, this information is
recorded in the
\Windows\ registry\index{platform!Windows!no~Tcl~configuration~file}, so
it is normal to have \cd{oommf.tcl +platform} report ``none found''.  If
\cd{oommf.tcl +platform} reports ``none found'', but you know that an
appropriate \Tcl\ configuration file is present on your system, you can
tell \OOMMF\ where to find the file by setting the environment variable
OOMMF\_TCL\_CONFIG\index{environment~variables!OOMMF\_TCL\_CONFIG} to
its absolute filename.  In unusual circumstances, \OOMMF\ may find a
\Tcl\ configuration file which doesn't correctly describe your
\Tcl\ installation.  In that case, use the environment variable
OOMMF\_TCL\_CONFIG to instruct \OOMMF\ to use a different file that you
specify, and, if necessary, edit that file to include a correct
description of your \Tcl\ installation.

Next, the \cd{oommf.tcl +platform} reports similar information about the
\cd{wish} and \Tk\ configuration.  The environment variables
\cd{OOMMF\_TK\_CONFIG}\index{environment~variables!OOMMF\_TK\_CONFIG}
and \cd{OOMMF\_WISH}\index{environment~variables!OOMMF\_WISH} may be
used to tell \OOMMF\ where to find the \Tk\ configuration file and the
\fn{wish} program, respectively.

\index{parallelization|(}
Following the \Tk\ information are some lines reporting ``thread'' build
and run status. Threads are used by \OOMMF\ to implement parallelism in
the Oxs (\cd{oxsii} and \cd{boxsi}) 3D solvers on
multi-processor/multi-core shared memory machines.  In order to build or
run a parallel version of OOMMF, you must have a thread-enabled version
of \Tcl.  The \Tcl\ thread status is indicated on the first thread
status line.  If \Tcl\ is thread enabled, then the default \OOMMF\ build
process will create a threaded version of \OOMMF.  You can override this
behavior if you wish to build a non-parallel version of \OOMMF\ by
editing the
\cd{oommf\_threads}\index{configuration~values!oommf\_threads} value in
the \fn{config/platforms/} file for your platform.

If \Tcl\ and \OOMMF\ threads are enabled, then the next line will show
the default number of threads run by the Oxs solvers.  (This value may
vary between machines, depending on the number of processors in the
machine.)  You can change this by setting the
\cd{thread\_count}\index{configuration~values!thread\_count} value in
the \fn{config/platforms/} file for your platform.  The default can be
overridden at run time by the environment variable
\cd{OOMMF\_THREADS}\index{environment~variables!OOMMF\_THREADS}, or by
the \cd{oxsii}/\cd{boxsi} command line option \cd{-threads}.  If
NUMA\index{NUMA} support is possible on your platform (see
\hyperrefhtml{below}{``Parallelization,'' Sec.~}{ below}{sec:parallel}),
then the next line of thread info will indicate whether or not the build
process will create NUMA-aware Oxs solvers.
\index{parallelization|)}

After the thread info, \cd{oommf.tcl +platform} reports the directory
that \OOMMF\ will use to write temporary files\index{temporary~files}.
This directory is used, for example, to transfer magnetization data from
the micromagnetic solvers to the \app{mmDisp} display module.  You must
have write access to this directory.  If you don't like the
\OOMMF\ default, you may change it via the
\cd{path\_directory\_temporary}\index{configuration~values!path\_directory\_temporary}
setting in the \fn{config/platforms/} file for your platform.  Or you
can set the environment variable
\cd{OOMMF\_TEMP}\index{environment~variables!OOMMF\_TEMP}, which will
override all other settings.

If any environment variables relevant to \OOMMF\ are set, then
\cd{oommf.tcl +platform} will report these next, followed finally by any
warnings about possible problems with your \Tcl/\Tk\ installation, such
as if you are missing important header files.

If \cd{oommf.tcl +platform} indicates problems with your \Tcl/\Tk\
installation,
it may be easiest to re-install \Tcl/\Tk\, taking care to perform a
conventional installation.  \OOMMF\ deals best with conventional
\Tcl/\Tk\ installations.  If you do not have the power to re-install
an existing broken \Tcl/\Tk\ installation (perhaps you are not
the sysadmin of your machine), you might still install your own
copy of \Tcl/\Tk\ in your own user space.  In that case, if your
private \Tcl/\Tk\ installation makes use of shared libraries,
take care that you do whatever is necessary on your platform to
be sure that your private \fn{tclsh} and \fn{wish} 
find and use your private
shared libraries instead of those from the system \Tcl/\Tk\ installation.
This might involve setting an environment variable (such as 
LD\_LIBRARY\_PATH\index{environment~variables!LD\_LIBRARY\_PATH}).
If you use a private \Tcl/\Tk\ installation, you also want to be sure
that there are no environment variables like 
TCL\_LIBRARY\index{environment~variables!TCL\_LIBRARY}
or TK\_LIBRARY\index{environment~variables!TK\_LIBRARY}
that still refer to the system \Tcl/\Tk\ installation.
\index{installation!Tcl/Tk|)}

\paragraph{Additional Configuration Issues on \Windows}

\index{platform!Windows!configuration|(}
A few other configurations should be checked on \Windows\ platforms.
\index{platform!Windows!file~path~separator|(}
First, note that absolute filenames on \Windows\ makes use of the
backslash (\bs) to separate directory names.  On \Unix\ and
within \Tcl\ the forward slash (\fs) is used to separate directory
names in an absolute filename.  In this manual we usually use the
\Tcl\ convention of forward slash as separator.  In portions of the
manual pertaining only to MS \Windows\ we use the backslash as
separator.  There may be instructions in this manual which do not
work exactly as written on \Windows\ platforms.  You may need to
replace forward slashes with backward slashes in pathnames when
working on \Windows.
\index{platform!Windows!file~path~separator|)}

\index{platform!Windows!hosts~file|(}
\OOMMF\ software needs networking support that recognizes
the host name \cd{localhost}.  It may be necessary
to edit a file which records that \cd{localhost} is a synonym
for the loopback interface (127.0.0.1).  If a file named \fn{hosts}
exists in your system area (for example, \fn{C:\bs Windows\bs hosts}),
be sure it includes an entry mapping 127.0.0.1 to \cd{localhost}.
If no \fn{hosts} file exists, but a \fn{hosts.sam} file exists,
make a copy of \fn{hosts.sam} with the name \fn{hosts}, and edit
the copy to have the \fn{localhost} entry.
\index{platform!Windows!hosts~file|)}

The directory that holds the \fn{tclsh} and \fn{wish} programs also
holds several {\fn{*.dll}} files that \OOMMF\ software needs to find to
run properly.  Normally when the \hyperrefhtml{\OOMMF\ bootstrap
  application}{\OOMMF\ bootstrap application (Sec.~}{)}{sec:cll} or
\hyperrefhtml{{\bf mmLaunch}}{{\bf mmLaunch} (Sec.~}{)}{sec:mmlaunch} is
used to launch \OOMMF\ programs, they take care of making sure the
necessary {\fn{*.dll}} files can be found.  As an additional measure,
you might want to add the directory which holds the \fn{tclsh} and
\fn{wish} programs to the list of directories stored in the
PATH\index{environment~variables!PATH} environment variable.  All the
directories in the PATH are searched for {\fn{*.dll}} files needed when
starting an executable.  \index{platform!Windows!configuration|)}

\subsubsection{Compiling and Linking}\label{sec:install.compile}

If you downloaded a distribution with pre-compiled executables, you may
skip this section.

When building \OOMMF\ software from source code, be sure the C++
compiler reported by \cd{oommf.tcl +platform} is properly configured.
In particular, if you are running on a \Windows\ system, please read
carefully the notes in \html{the}
\hyperrefhtml{Advanced Installation}{Advanced Installation,
Sec.~}{,}{sec:install.windows} \html{section} pertaining to your
compiler.

The compiling and linking of the C++ portions of \OOMMF\ software 
are guided by the application
\hyperrefhtml{pimake}{pimake (Sec.~}{)}{sec:pimake}
\index{application!pimake}
(``Platform Independent Make'') which 
is distributed as part of the \OOMMF\ release.
To begin building \OOMMF\ software with \fn{pimake}, first change
your working directory\index{working~directory}
to the root directory of the \OOMMF\ distribution:
\begin{verbatim}
cd .../path/to/oommf
\end{verbatim}

If you unpacked the new \OOMMF\ release into a directory \fn{oommf}
which contained an earlier \OOMMF\ release, 
use \fn{pimake} to build the target \cd{upgrade}
to clear away any source code files which were 
in a former distribution but are not part of the latest distribution:
\begin{verbatim}
tclsh oommf.tcl pimake upgrade
\end{verbatim}

Next, build the target \cd{distclean} to clear away any old executables
and object files which are left behind from the compilation of the
previous distribution:
\begin{verbatim}
tclsh oommf.tcl pimake distclean
\end{verbatim}

Next, to build all the \OOMMF\ software, run \fn{pimake} without
specifying a target:
\begin{verbatim}
tclsh oommf.tcl pimake
\end{verbatim}
On some platforms, you cannot successfully compile \OOMMF\ software if
there are \OOMMF\ programs running.  Check that all \OOMMF\ programs
have terminated (including those in the background) before trying to
compile and link \OOMMF.

When \fn{pimake} calls on a compiler or other software development
utility, the command line is printed, so that you may monitor the build
process.  
Assuming a proper configuration for your platform, \fn{pimake} should be
able to compile and link all the \OOMMF\ software without error.  If
\fn{pimake} reports errors, please first consult 
\hyperrefhtml{Troubleshooting}{Troubleshooting (Sec.~}{)}{sec:trouble}
to see if a fix is already documented.
If not, please send both the {\em complete} output 
from \fn{pimake} and the output from \cd{oommf.tcl +platform}
to the \OOMMF\ developers when you e-mail to ask for help.

\subsubsection{Installing}

The current \OOMMF\ release does not support an installation procedure.
For now, simply run the executables from the directories in which they
were unpacked/built.

\subsubsection{Using \OOMMF\ Software}

To start using \OOMMF\ software, run the 
\hyperrefhtml{\OOMMF\ bootstrap application}{\OOMMF\ bootstrap 
application (Sec.~}{)}{sec:cll}. This may be launched from the command
line interface:
\begin{verbatim}
tclsh oommf.tcl
\end{verbatim}

If you prefer, you may launch the \OOMMF\ bootstrap application
\fn{oommf.tcl} using whatever graphical ``point and click''
interface your operating system provides.  By default, the \OOMMF\
bootstrap application will start up a copy of the \OOMMF\ application
\hyperrefhtml{\app{mmLaunch}}{\app{mmLaunch} (Sec.~}{)}{sec:mmlaunch} in a
new window.

If you publish material created with the aid of \OOMMF, please refer to
\hyperrefhtml{Credits}{Credits (Sec.~}{)}{sec:credits}
for citation information.


\subsubsection{Reporting Problems}

If you encounter problems when installing or using \OOMMF, please report
them to the \OOMMF\ developers.  The \cd{oommf.tcl +platform} command
has been designed in large part to help \OOMMF\ developers debug
installation problems, so \textbf{PLEASE} be sure to include the
complete output from \cd{oommf.tcl~+platform} in your report.  See also
the section on
\hyperrefhtml{troubleshooting}{troubleshooting (Sec.~}{)}{sec:trouble}
for additional instructions.

\subsection{Advanced Installation}\label{sec:install.advanced}

The following sections provide instructions for some additional
installation options.

\subsubsection{Reducing Disk Space Usage}\label{sec:install.reducedisk}

\index{requirement!disk~space|(}
To delete the intermediate files created when building the \OOMMF\
software from source code, use 
\hyperrefhtml{pimake}{pimake (Sec.~}{)}{sec:pimake}
to build the target
\cd{objclean} in the root directory of the \OOMMF\ distribution.
\begin{verbatim}
tclsh oommf.tcl pimake objclean
\end{verbatim}
Running your platform \fn{strip} utility on the \OOMMF\ executable files
should also reduce their size somewhat.
\index{requirement!disk~space|)}

\subsubsection{Local Customizations}\label{sec:custom}
\index{customize}

\OOMMF\ software supports local customization of some of its
features.  All \OOMMF\ programs load the file
\fn{config/options.tcl}\index{options.tcl}\index{file!options.tcl}, which
contains customization commands as well as editing instructions.  As it
is distributed, \fn{config/options.tcl} directs programs to also load
the file \fn{config/local/options.tcl}, if it exists.  Because future
\OOMMF\ releases may overwrite the file
\fn{config/options.tcl}, permanent customizations should be made by
copying \fn{config/options.tcl} to \fn{config/local/options.tcl} and
editing the copy.  It is recommended that you leave in the file
\fn{config/local/options.tcl} only the customization commands necessary
to change those options you wish to modify.  Remove all other options so
that overwrites by subsequent \OOMMF\ releases are allowed to change the
default behavior.

Notable available customizations include the choice of which network
port the \hyperrefhtml{host service directory application} {host service
directory application (Sec.~}{)}{sec:arch} uses, and the choice of what
program is used for the display of help documentation.  By default, 
\OOMMF\ software uses the application
\hyperrefhtml{\app{mmHelp}}{\app{mmHelp} (Sec.~}{)}{sec:mmhelp}, which
is included in the \OOMMF\ release, but the help documentation files
are standard HTML, so any web browser (for example, 
Netscape Navigator\index{application!Netscape}
or Microsoft Internet Explorer\index{application!Internet~Explorer}) 
may be used instead.  Complete
instructions are in the file \fn{config/options.tcl}.

\subsubsection{Optimization}\label{sec:optimize}\index{optimization}

In the interest of successful compilation of a usable software package
``out of the box,'' the default configuration for \OOMMF\ does not
attempt to achieve much in terms of optimization.  However, in each
platform's configuration file (for example,
\fn{config/platforms/wintel.tcl}), there are alternative values for the
configuration's optimization flags, available as comments.  If you are
familiar with your compiler's command line options, you may experiment
with other choices as well.  You can edit the platform configuration
file to replace the default selection with another choice that
provides better computing performance.  For example, in
\fn{config/platforms/wintel.tcl}, alternative optimization flags for the
MSVC++ compiler are defined with the line:
\begin{verbatim}
$config SetValue program_compiler_c++_option_opt {format "/G5 /Ox"}
\end{verbatim}

The extensible solver, Oxs, can be compiled with debugging support
for extensive run-time code checks.  This will significantly reduce
computation performance.  In the standard \OOMMF\ distributions, these
checks should be disabled.  You may verify this by checking that
the following line appears in the file \fn{config/options.tcl}:
\begin{verbatim}
Oc_Option Add * Platform cflags {-def NDEBUG}
\end{verbatim}
To enable these checks, either comment/remove this line, or else add 
to the \fn{config/local/options.tcl} file a ``cflags'' option line
without ``-def NDEBUG'', such as
\begin{verbatim}
Oc_Option Add * Platform cflags {-warn 1}
\end{verbatim}
The \fn{config/local/options.tcl} file may be created if it does not
already exist.

\subsubsection{Parallelization}%
\label{sec:parallel}\index{parallelization|(}
The \OOMMF\ Oxs 3D solvers (\cd{oxsii} amd \cd{boxsi}) can be built
thread-enabled to allow parallel processing on
multi-processor/multi-core machines.  In order to build and run a
parallel version of \OOMMF, you must have a thread-enabled version of
\Tcl.  Most standard binary releases of \Tcl\ today are thread-enabled,
so \OOMMF\ releases that include pre-built executables are built
thread-enabled.  If you build \OOMMF\ from source, then by default
\OOMMF\ will be built thread-enabled if your \Tcl\ is thread-enabled.
As explained earlier, you can check thread build status with the
\cd{tclsh oommf.tcl +platform} command.  If you want to force a
non-threaded build of \OOMMF, then edit the \fn{config/platforms/} file
for your platform.  In the section labeled \cd{LOCAL CONFIGURATION}, you
will find a line that looks like
\begin{verbatim}
# $config SetValue oommf_threads 0
\end{verbatim}
Uncomment this line (i.e., remove the leading `\verb+#+' character) to
force a non-threaded build.  Then run
\begin{verbatim}
tclsh oommf.tcl pimake distclean
tclsh oommf.tcl pimake
\end{verbatim}
from the \OOMMF\ root directory to create a fresh build.

You can use the \cd{tclsh oommf.tcl +platform} command to see the
default number of compute threads that will be run by the Oxs 3D solver
programs \cd{oxsii} and \cd{boxsi}.  You can modify the default by
editing the \cd{oommf\_thread\_count} value in the
\fn{config/platforms/} file for your platform.  You can override the
default at run time by setting the environment variable
\cd{OOMMF\_THREADS}\index{environment~variables!OOMMF\_THREADS}, or by
using the command line option \cd{-threads} to \cd{oxsii} and
\cd{boxsi}.

\index{NUMA|(}Some multi-processor machines have a non-uniform memory
architecture (NUMA), which means that although each processor can access
all of system memory, some parts of memory can be accessed faster than
others.  Typically this is accomplished by dividing the system memory
and processors into ``nodes.''  Memory accesses within a node are faster
than accesses between nodes, and depending on the architecture access
latency and bandwidth may be different between different node pairs.
Examples of machines with NUMA include some multi-processor AMD Opteron
and Intel Nehalem Xeon boxes.

Computer programs such as \OOMMF\ can run on NUMA machines without
making any special allowances for the memory architecture.  However, a
program that is written to take advantage of the faster local
(intra-node) memory accesses can sometimes run significantly faster.
\OOMMF\ contains NUMA-aware code, but this code is highly operating
system specific.  At present, \OOMMF\ can be built with NUMA support
only on Linux (32- and 64-bit) systems.  To do this, you must install
the operating system NUMA support packages ``numactl'' and
``numactl-devel''.  The names may vary somewhat between Linux
distributions, but the first typically includes the executable
\fn{numactl} and the second includes the header file \fn{numa.h}.  Once
the numactl package is installed, you can run the command
\begin{verbatim}
numactl --hardware
\end{verbatim}
to get an overview of the memory architecture on your machine.  If this
shows you have only one node, then there is no advantage to making a
NUMA-aware build of \OOMMF.

The next step is to edit the \fn{config/platforms} for your platform.
For example, on a 64-bit Linux box this file is
\fn{config/platforms/linux-x86\_64.tcl}. 
In the section labeled \cd{LOCAL CONFIGURATION}, find the line
\begin{verbatim}
# $config SetValue use_numa 1
\end{verbatim}
Edit this to remove the leading `\verb+#+' character.  Alternatively
(and, actually, preferably), create a \fn{local} subdirectory and make a
local configuration file with the same platform name; e.g.,
\fn{config/platforms/local/linux-x86\_64.tcl} on a 64-bit Linux machine.  Add
the line
\begin{verbatim}
$config SetValue use_numa 1
\end{verbatim}
to this file.  (The advantage of using a \fn{config/platforms/local}
file is that you can make changes without modifying the original \OOMMF\
source code, which makes it easier to port your local changes to future
releases of \OOMMF.)  If this is done correctly, then the command
`\cd{tclsh oommf.tcl +platform}' will show that NUMA support is enabled.
Then simply run `\cd{tclsh oommf.tcl pimake distclean}' and
`\cd{tclsh oommf.tcl pimake}' from the \OOMMF\ root directory to build a
NUMA-aware version of \OOMMF.

To activate the NUMA-aware code, you must specify the \cd{-numanodes}
option on the \cd{oxsii}/\cd{boxsi} command line, or set the the
environment variable
\cd{OOMMF\_NUMANODES}\index{environment~variables!OOMMF\_NUMANODES}.
Check the \hyperrefhtml{Oxs documention}{Oxs documentation
(Sec.~}{)}{sec:oxs} for details.
\index{NUMA|)}\index{parallelization|)}

\subsubsection{Managing \OOMMF\ Platform Names}\label{sec:platformNames}
\index{platform!names|(}

\OOMMF\ software classifies computing platforms into different types
using the scripts in the directory \fn{config/names} relative to the
root directory of the \OOMMF\ distribution.  
Each type of computing platform is assigned a unique name.  
These names are used as directory names
and in index and configuration files so that a single 
\OOMMF\ installation may contain platform-dependent sections for many
different types of computing platforms.

To learn what name \OOMMF\ software uses to refer to your computing
platform, run 
\begin{verbatim}
tclsh oommf.tcl +platform
\end{verbatim}
in the \OOMMF\ root directory.

\paragraph{Changing the name \OOMMF\ assigns to your platform}

First, use 
\hyperrefhtml{pimake}{pimake (Sec.~}{)}{sec:pimake}
to build the target \cd{distclean} to
clear away any compiled executables built using the old platform
name.
\begin{verbatim}
tclsh oommf.tcl pimake distclean
\end{verbatim}
Then, to change the name \OOMMF\ software uses to describe your platform from 
\cd{foo} to \cd{bar}, simply rename the file
\begin{quote}
\fn{config/names/foo.tcl}
\hspace{1em} to \hspace{1em}
\fn{config/names/bar.tcl}
\end{quote}
and
\begin{quote}
\fn{config/platforms/foo.tcl}
\hspace{1em} to \hspace{1em}
\fn{config/platforms/bar.tcl}.
\end{quote}
After renaming your platform type, you should recompile your executables
using the new platform name.

\paragraph{Adding a new platform type}

If \cd{oommf.tcl +platform} reports the platform name
\cd{unknown}, then none of the scripts in \fn{config/names/}
recognizes your platform type.  As an example, to add the platform
name \cd{foo} to \OOMMF's vocabulary of platform names, create the
file \fn{config/names/foo.tcl}.  The simplest way to proceed is to
copy an existing file in the directory \fn{config/names} and edit it
to recognize your platform.

The files in \fn{config/names} include \Tcl\ code like this:

\begin{verbatim}
  Oc_Config New _ \
    [string tolower [file rootname [file tail [info script]]]] {
      # In this block place the body of a Tcl proc which returns 1
      # if the machine on which the proc is executed is of the
      # platform type identified by this file, and which returns 0
      # otherwise.
      #
      # The usual Tcl language mechanism for discovering details 
      # about the machine on which the proc is running is to 
      # consult the global Tcl variable 'tcl_platform'.  See the
      # existing files for examples, or contact the OOMMF
      # developers for further assistance.
  }
\end{verbatim}

After creating the new platform name file \fn{config/names/foo.tcl}, you
also need to create a new platform file \fn{config/platforms/foo.tcl}.
A reasonable starting point is to copy the file
\fn{config/platforms/unknown.tcl} for editing.  Contact the \OOMMF\
developers for assistance.

Please consider contributing your new platform recognition and 
configuration files to the \OOMMF\ developers for inclusion in 
future releases of \OOMMF\ software.

\paragraph{Resolving platform name conflicts}

If the script \cd{oommf.tcl +platform} reports ``Multiple platform names are
compatible with your computer'', then there are multiple files in the
directory \fn{config/names/} that return 1 when run on your computer.
For each compatible platform name reported, edit the corresponding
file in \fn{config/names/} so that only one of them returns 1.
Experimenting using \fn{tclsh} to probe the \Tcl\ variable
\cd{tcl\_platform} should assist you in this task.  If that fails, you
can explicitly assign a platform type corresponding to your computing
platform by matching its hostname.  For example, if your machine's
host name is {\tt foo.bar.net}:
\begin{verbatim}
  Oc_Config New _ \
    [string tolower [file rootname [file tail [info script]]]] {
      if {[string match foo.bar.net [info hostname]]} {
          return 1
      }
      # Continue with other tests...
  }
\end{verbatim}

Contact the \OOMMF\ developers if you need further assistance.
\index{platform!names|)}

\subsection{Platform Specific Installation Issues}\label{sec:install.platform}

The installation procedure discussed in the previous sections applies to
all platforms (\Unix, \Windows, \MacOSX).  There are, however, some
details which pertain only to a particular platform.  These issues are
discussed below.

\index{platform!Unix!configuration|(}
\subsubsection{\Unix\ Configuration}
\paragraph{Missing \Tcl/\Tk\ files}
The basic installation procedure should be sufficient to install \OOMMF\
on most \Unix\ systems.  Sometimes, however, the build will fail due to
missing \Tcl\ header files (\fn{tcl.h}, \fn{tk.h}) or libraries (e.g.,
\fn{libtcl.so}, \fn{libtk.so}).  This problem can usually be solved by
installing a ``development'' version of \Tcl/\Tk, which may be found on
the operating system installation disks, or may be available from the
system vender.  There are also binary releases of \Tcl/\Tk\ for a number
of systems available from ActiveState, under the name
\htmladdnormallinkfoot{ActiveTcl}{http://www.activestate.com/Products/ActiveTcl/}.
Alternatively, one may download the sources for \Tcl\ and \Tk\ from the
\htmladdnormallinkfoot{Tcl Developer
Xchange}{http://purl.org/tcl/home/}, and build and install \Tcl/\Tk\
from source.  The \Tcl/\Tk\ build follows the usual \Unix\
\fn{configure}, \fn{make}, \fn{make install} build convention.

\paragraph{Compiler Optimization Options}
On most systems, \OOMMF\ builds by default with relatively unaggressive
compiler optimization options.  As discussed earlier
(\hyperrefhtml{under Optimization}{``Optimiation,''
Sec.~}{)}{sec:optimize}), you may edit the appropriate
\fn{oommf/config/platforms/} file to change the default compilation options.
However, on some common systems (e.g., Linux, some BSD variants) \OOMMF\
will try to deduce the hardware architecture (i.e., the CPU subtype,
such as Pentium 3 vs. Pentium 4) and apply architecture-specific options
to the compile commands.  This is probably what you want if \OOMMF\ is
to be run only on the system on which it was built, or if it is run on a
homogeneous cluster.  If, instead, you intend to run \OOMMF\ on a
heterogeneous cluster you may need to restrict the compiler options to
those supported across your target machines.  In that case, open the
appropriate configuration file in the \fn{oommf/config/platforms/}
directory, and look for the lines
\begin{verbatim}
    # You can override the GuessCPU results by directly setting or
    # unsetting the cpuopts variable, e.g.,
    #
    #    set cpuopts [list -march=athlon]
    # or
    #    unset cpuopts
    #
\end{verbatim}
Uncomment either the ``unset cpuopts'' line to make a generic build, or
else edit the ``set cpuopts'' line to an appropriate common-denominator
architecture and uncomment that line.

In a similar vein, some compilers support a ``-fast'' switch, which
usually creates an architecture-specific executable.  The same
considerations apply in this case.

An advanced alternative would be to define separate \OOMMF\
``platforms'' for each CPU subtype in your cluster.  At a minimum, this
would involve creating separate platform name files in
\fn{oommf/config/names/} for each subtype, and then making copies of the
appropriate \fn{oommf/config/platforms} file for each new platform.  The
platform name files would have to be written so as to reliably detect
the CPU subtype on each machine.  See \hyperrefhtml{``Managing \OOMMF\
platform names''} {``Managing \OOMMF\ platform names''
(Sec.~}{)}{sec:platformNames} for details on creating platform name
files.  \index{platform!Unix!configuration|)}

\index{platform!Windows!configuration|(}
\subsubsection{Microsoft \Windows\ Options}\label{sec:install.windows}

This section lists installation options for Microsoft \Windows.

\paragraph{Using Microsoft Visual C++}
\index{platform!Windows!Microsoft~Visual~C++}
If you are building \OOMMF\ software from source using the Microsoft
Visual C++ command line compiler, \fn{cl.exe}, it is necessary to run
\fn{vcvars32.bat} to set up the path and some environment variables.
This file is distributed as part of Visual C++.  You may want to set up
your system so this batch file gets run automatically when you boot the
system, or open a command prompt.  See your compiler and system
documentation for details.

\paragraph{Using the Cygwin toolkit}\label{par:install.cygwin}
\index{platform!Windows!Cygwin~environment}
The \htmladdnormallinkfoot{Cygwin
Project}{http://www.cygwin.com/} is a free port of the GNU
development environment to \Windows, which includes the GNU C++ compiler
g++.  \OOMMF\ has been successfully built and
tested within the Cygwin environment; sample configuration files
\fn{config/names/cygtel.tcl} and
\fn{config/platforms/cygtel.tcl} are
included in the \OOMMF\ distribution.  \textbf{IMPORTANT:} Use a
standard \Windows\ build of Tcl/Tk (e.g., the
\htmladdnormallinkfoot{ActiveTcl
release}{http://www.activestate.com/Products/ActiveTcl/}) when
configuring, building, and launching \OOMMF\ software.  As of this
writing (Oct.\ 2004), the \fn{tclsh} distributed with Cygwin (i.e.,
\fn{/usr/bin/tclsh}) has problems involving sockets that make it
unsuitable for use with \OOMMF.  Unfortunately, standard \Windows\
versions of \Tcl\ are not acquainted with the Cygwin POSIX-style
pathnames, so in many cases you will have to use Windows-style pathnames
instead.

Note that \OOMMF\ software determines whether it is running under Cygwin
by examining the environment variables
OSTYPE\index{environment~variables!OSTYPE} and
TERM\index{environment~variables!TERM}.  If either is set to a value
beginning with \cd{cygwin}, the Cygwin environment is assumed.  If you
are using the Cygwin environment with different values for both OSTYPE
and TERM, you will have to modify the \fn{config/names/cygtel.tcl} file
accordingly.

\paragraph{Using Borland C++}
\index{platform!Windows!Borland~C++}
\OOMMF\ has been successfully built and tested using the
Borland C++ command line compiler version 5.5.
However, a couple preparatory steps are necessary before building
\OOMMF\ with this compiler.
\begin{enumerate}
\item Properly complete bcc55 compiler installation.

Be sure to read the \fn{readme.txt} file in the \fn{BCC55}
subdirectory of the Borland install directory.  In particular, check
that the \fn{bcc32.cfg} and \fn{ilink32.cfg} configuration files exist
in the \fn{BIN} subdirectory, and have appropriate contents.  If you
omit this step you will get error messages during the \OOMMF\ build
process relating to the inability of the Borland compiler to find
system header files and libraries.  You will probably also need to add
the Borland \fn{BIN} directory to your \cd{PATH} environment variable.
Some of the Borland tools are fragile with respect to spaces in their
pathnames, so you should either select the Borland install directory
to be one without spaces anywhere in the pathname (e.g., use
\fn{C:\bs Borland\bs} instead of
\fn{"C:\bs Program~Files\bs Borland\bs "}), or at least when setting
the \cd{PATH} use the ``8dot3'' style short name version of each
component of the Borland install directory, e.g.,
\begin{quote}
\begin{verbatim}
PATH=C:\Progra~1\Borland\BCC55\Bin;%PATH%
\end{verbatim}
\end{quote}
Use ``\cd{dir /x}'' to display both the short and long versions of
filenames.  The Borland Developer Studio 2006 install automatically sets
the path to include the long name version of the Borland \fn{BIN}
directory; you should manually change this via the System dialog box
from the Control Panel.  Select the Advanced tab, and pull up the
Environment Variables sub-dialog.  Edit the \cd{Path} variable as
discussed above; check both the ``User variables'' and the ``System
variables'' settings.  You will need to launch a new shell (command
prompt) for the changes to take effect.

\item Create Borland compatible \Tcl\ and \Tk\ libraries.

The import libraries distributed with \Tcl/\Tk, release 8.0.3 and later,
are not compatible with the Borland~C++ linker.  However, the command
line utility \fn{coff2omf}, which is distributed with
the Borland compiler, can be used to create suitable libraries from
the \Tcl/\Tk\ .lib's.  In the \Tcl/\Tk\ library directory (typically
\fn{C:\bs Tcl\bs lib} or \fn{"C:\bs Program~Files\bs Tcl\bs lib"}),
issue the following commands
\begin{quote}
\begin{verbatim}
coff2omf tcl84.lib tcl84bc.lib
coff2omf  tk84.lib  tk84bc.lib
\end{verbatim}
\end{quote}
Here \fn{tcl84.lib} and \fn{tk84.lib} are the input libraries (in COFF
format) and \fn{tcl84bc.lib} and \fn{tk84bc.lib} are the new libraries
(in OMF format).

If \fn{coff2omf} doesn't work, you can try creating the necessary import
libraries directly from the \Tcl/\Tk\ DLL's.  From the \Tcl/\Tk\ library
directory issue the following commands:
\begin{quote}
\begin{verbatim}
impdef -a tcl84bc.def ..\bin\tcl84.dll
implib tcl84bc.lib tcl84bc.def
\end{verbatim}
\end{quote}
This creates the Borland compatible import library \fn{tcl84bc.lib}.
Repeat with ``tk'' in place of ``tcl'' to create \fn{tk84bc.lib}.  The
``-a'' switch requests \fn{impdef} to add a leading underscore to
function names.  This is sufficient for the DLL's shipped with \Tcl/\Tk\
8.4, but other releases may require additional tweaking.  The module
definition file output by \fn{impdef}, e.g., \fn{tcl84bc.def} above,
is a plain text file.  You may need to edit this file to add or modify
entries.


\item Edit \fn{oommf\bs config\bs platforms\bs wintel.tcl}

At a minimum, you will have to change the \cd{program\_compiler\_c++}
value to point to the Borland C++ compiler.  The sample \fn{wintel.tcl}
file assumes the librarian \fn{tlib} and the linker \fn{ilink32}
are in the execution path, and that the Borland compatible import
libraries, with names as specified above, are in the \Tcl/\Tk\ library
directory.  If this is not the case then you will have to make
appropriate modifications.  Also, you may need to add the ``-o'' switch
to the linker command to force ordinal usage of the Borland compatible
\Tcl/\Tk\ libraries produced in the previous step.

\end{enumerate}
After this, continue with the instructions in \html{the}
\hyperrefhtml{Compiling and Linking}{Sec.~}{, Compiling and
Linking.}{sec:install.compile} \html{section.}

\paragraph{Using Digital Mars C++}
\index{platform!Windows!Digital~Mars~C++}

The \htmladdnormallinkfoot{Digital
Mars}{http://www.digitalmars.com/} C++ 
command line compilers (dmc) versions 8.50 and earlier do not
successfully build this release of \OOMMF.  The following notes may help
build \OOMMF\ with a later release of dmc.
\begin{enumerate}
\item Install the Digital Mars C++ compiler, tools, and STL.

  Unpack the dmc archive into a convenient location.  The default name
  for the root directory of the dmc installation area is ``dm''.  Unpack
  the STLport (C++ Standard Library) into the dmc installation area.
  The top-level directory in the STLport archive is ``dm'', so if you
  unzip this archive from the parent directory to the dmc installation
  area it will naturally unpack into its standard location.  Then modify
  the dmc configuration to include the STL header files.  The
  \fn{dm{\bs}bin{\bs}sc.ini} file should be edited so that the first
  element of the \cd{INCLUDE} path is \verb+"%@P%\..\stlport\stlport";+

  Next, use ``\cd{set INCLUDE}'' and ``\cd{set LIBRARY}'' from the DOS
  command prompt to check that these environment variables are either
  not set, or else set to values as needed by the Digital Mars compiler.
  (These variables names may be used by other applications, which will
  conflict with values expected by dmc.)  To unset these variables, use
  the commands ``\cd{set INCLUDE=}'' and ``\cd{set LIBRARY=}''.  For
  convenience, you probably also want to put the \fn{dm{\bs}bin}
  directory into your environment \cd{PATH} variable.

\item Create compatible \Tcl/\Tk\ import libraries.

  The Digital Mars linker uses the same library format as the Borland
  linker, and as in that case, you will have to build compatible import
  libraries for the \Tcl/\Tk\ libraries.  The free download from Digital
  Mars does not include a utility to create these import libraries.  If
  you have purchased the compiler from Digital Mars, you can use the
  coff2off or implib tools for this.  See the documentation for details.
  Another option is to use the Borland tools.  See the section above on
  using Borland C++ for details.

\item Edit \fn{oommf\bs config\bs platforms\bs wintel.tcl}.

  You will need to uncomment the entry for the dmc compiler, and comment
  out the other compiler selections.  (The comment character is '\#'.)
  The configuration file assumes that the dmc compiler and associated
  tools are in a directory included in your environment \cd{PATH}
  variable.

\end{enumerate}
After this, continue with the instructions in \html{the}
\hyperrefhtml{Compiling and Linking}{Sec.~}{, Compiling and
Linking.}{sec:install.compile} \html{section.}

\paragraph{Setting the TCL\_LIBRARY environment variable}
\index{platform!Windows!setting~environment~variables}

If you encounter difficulties during \OOMMF\ start up, you may need to set
the environment variable 
TCL\_LIBRARY\index{environment~variables!TCL\_LIBRARY}.  

\subparagraph{On \Windows~NT}
Bring up the Control Panel (e.g., by selecting 
\btn{Settings\pipe Control Panel} off the Start menu), and select 
\btn{System}.  Go to the \btn{Environment} tab, and enter
TCL\_LIBRARY as the Variable, and the name of the directory containing
\fn{init.tcl} for the Value, e.g.,
\begin{verbatim}
%SystemDrive%\Program Files\Tcl\lib\tcl8.0
\end{verbatim}
Click \btn{Set} and \btn{OK} to finish.

\subparagraph{On \Windows\ 9x}

Edit the file \fn{autoexec.bat}.  Add a line such as the following:
\begin{verbatim}
set TCL_LIBRARY=C:\Program Files\Tcl\lib\tcl8.0
\end{verbatim}

\paragraph{Checking \fn{\bf .tcl} file association on \Windows~NT}
\index{platform!Windows!file~extension~associations}
As part of the \Tcl/\Tk\ installation, files with the \fn{.tcl}
extension are normally associated with the \fn{wish} application.  This
allows \Tcl\ scripts to be launched from 
\Windows\ Explorer\index{application!Windows~Explorer} by
double-clicking on their icon, or from the NT command line without
specifying the \fn{tclsh} or \fn{wish} shells.  If this is not working,
you may check your installation from the NT command line as follows.
First, run the command ``\cd{assoc~.tcl}''.  This should return the file
type associated with the \fn{.tcl} extension, e.g., \cd{TclScript}.
Next, use the \cd{ftype} command to check the command line associated
with that file type, e.g.,
%
\begin{verbatim}
C:\> ftype TclScript
 "C:\Program Files\Tcl\bin\wish84.exe" "%1" %2 %3 %4 %5 %6 %7 %8 %9
\end{verbatim}
%
Note that the quotes are required as shown to protect spaces in
pathnames.  If either \cd{assoc} or \cd{ftype} are incorrect, view the
command line help information (``\cd{assoc~/?}'' and ``\cd{ftype~/?}'')
for details on making changes.

\paragraph{Adding an \OOMMF\ shortcut to your desktop}
\index{platform!Windows!desktop~shortcut}

Right mouse click on the desktop to bring up the configuration dialog,
and select \btn{New\pipe Shortcut}.  
Enter the command line necessary to bring up \OOMMF, e.g.,
\begin{verbatim}
tclsh84 c:\oommf\oommf.tcl
\end{verbatim}

Click \btn{Next\bca} and enter \cd{OOMMF} for the shortcut name.  
Select \btn{Finish}.

At this point the shortcut will appear on your desktop with either the
tclsh or wish icons.  Right mouse click on the icon and select
\btn{Properties}.  Select the \btn{ShortCut} tab, 
and bring up \btn{Change Icon\ldots}  Under \btn{File Name:} enter the
\OOMMF\ icon file, e.g.,
\begin{verbatim}
C:\oommf\oommf.ico
\end{verbatim}

Click \btn{OK}.  Back on the \btn{Shortcut} tab, change the 
\btn{Run:} selection to
Minimized.  Click \btn{OK} to exit the Properties dialog box.  Double
clicking on the \OOMMF\ icon should now bring up the 
\OOMMF\ application \app{mmLaunch}.
\index{platform!Windows!configuration|)}

\subsubsection{\MacOSX\ Configuration}\label{sec:install.macosx}
\index{platform!MacOSX!configuration|(}

This section lists installation options for \MacOSX.

\paragraph{Building \OOMMF\ on \MacOSX}

1) Check Tcl/Tk; pointer to ActiveTcl.
2) Check C++ compiler

Follow the basic installation instructions in the previous section.  If
the platform configuration check does not find a C++ compiler, then you
will have to install one.  Recent releases of \MacOSX include the XCode
development tools on the base OS X installation disk.  These tools
include the GNU C++ compiler (g++).  Install the XCode tools onto your
computer and then go back to the basic \OOMMF\ installation instructions.

\index{platform!MacOSX!configuration|)}

\chapter{Quick Start: Example \OOMMF\ Session}\label{sec:quickstart}
%% Based on notes by Dianne P. O'Leary

% LIST NOTES: If the outer list is make a \begin{itemize} list,
% then LaTeX complains about the innermost list being too deeply
% nested.  The \begin{description} list, or any list using the
% optional argument to the \item command, causes latex2html to use the
% <DL> glossary list, which in most browsers puts the label on a
% separate line (most browsers ignore the COMPACT option), which I'd
% prefer not to have.  The following is a workaround to these
% problems.

\newcounter{quickstartstep}
\newcounter{quickstartsubstep}

\newcommand{\blankspace}{\ }

\newcommand{\quickstartitemlabel}[1]{%
{\setcounter{quickstartsubstep}{0}\stepcounter{quickstartstep}%
\bf STEP \arabic{quickstartstep}: #1}}

\newcommand{\quickstartitemlabelsubstart}[1]{%
{\setcounter{quickstartsubstep}{1}\stepcounter{quickstartstep}%
\bf STEP \arabic{quickstartstep}\alph{quickstartsubstep}: #1}}

\newcommand{\quickstartitemlabelsubincr}[1]{%
{\stepcounter{quickstartsubstep}%
\bf STEP \arabic{quickstartstep}\alph{quickstartsubstep}: #1}}

\begin{list}{}{\setlength{\labelwidth}{0pt}
               \setlength{\leftmargin}{0pt}
               \setlength{\rightmargin}{\leftmargin}
               \setlength{\itemsep}{0pt}}
  \item \quickstartitemlabel{Start up the
               mmLaunch\index{application!mmLaunch} window.}
  \begin{itemize}
    \item At the command prompt, when you are in the \OOMMF\ root
          directory, type 
\begin{verbatim}
tclsh oommf.tcl
\end{verbatim}
    (The name of the \Tcl\ shell, rendered here as \verb+tclsh+, may
    \hyperrefhtml{vary between systems.}{vary between systems.
    This matter is discussed in Sec.~}{.}{sec:install.requirements})
    Alternatively, you may launch \verb+oommf.tcl+ using
    whatever ``point and click'' interface is provided by your operating
    system.
    \item This will bring up a small window labeled
          \app{mmLaunch}. It will come up in background mode, so you will get
          another prompt in your original window, even before the
          \app{mmLaunch} window appears.
  \end{itemize}
  \item \quickstartitemlabel{Gain access to other useful windows.}
  \begin{itemize}
  \item The \app{mmLaunch} window is divided into two columns. The
    right column provides a list of running applications. This will
    normally be empty when you first start \app{mmLaunch}. The
      left column, labeled ``Programs,'' provides a collection of
      buttons that launch applications when clicked:
    \begin{itemize}
       \item {\htmlonlyref{\bf{mmArchive:}}{sec:mmarchive}}\index{application!mmArchive}
                   auto-saves tabular and field data files
       \item {\htmlonlyref{\bf{mmDataTable:}}{sec:mmdatatable}}\index{application!mmDataTable}
                   displays current values of tabular (scalar) outputs
       \item {\htmlonlyref{\bf{mmDisp:}}{sec:mmdisp}}\index{application!mmDisp}
                   displays scalar and vector fields
       \item {\htmlonlyref{\bf{mmGraph:}}{sec:mmgraph}}\index{application!mmGraph}
                   makes x-y plots
       \item {\htmlonlyref{\bf{mmProbEd:}}{sec:mmprobed}}\index{application!mmProbEd}
                   problem editor for \app{mmSolve2D} or \app{Oxsii}
       \item {\htmlonlyref{\bf{mmSolve2D:}}{sec:mmsolve2d}}\index{application!mmSolve2D}
                   2D solver interactive interface
       \item {\htmlonlyref{\bf{Oxsii:}}{sec:oxsii}}\index{application!Oxsii}
                   3D solver interactive interface
    \end{itemize}
    \item Click on \btn{mmDisp}, \btn{mmGraph}, and/or
          \btn{mmDataTable}, depending on what form of output you
          want to view.  Use \btn{mmArchive} to save data to disk.
  \end{itemize}
  \item \quickstartitemlabelsubstart{Run a 2D problem.}
    \begin{description}
    \item[Load problem:]\blankspace
    \begin{itemize}
      \item In the \app{mmLaunch} window, click on the \btn{mmProbEd}
            button. 
      \item In the
            \hyperrefhtml{\app{mmProbEd}}{\app{mmProbEd}
              (Ch.~}{)}{sec:mmprobed}\index{application!mmProbEd}
            window, make menu selection
            \btn{File\pipe Open\ldots}  An \btn{Open File} dialog window
            will appear.  In this window:
      \begin{itemize}
        \item Double click in the \btn{Path} subwindow to change
              directories.  Several sample problems can be found in
              the directory \fn{oommf/app/mmpe/examples}.
        \item To load a problem, double click on a \fn{*.mif} file
              (e.g., prob1.mif) from the list above the \btn{Filter:}
              subwindow.
      \end{itemize}
      \item Modify the problem as desired by clicking on buttons from
            the main \app{mmProbEd} window (e.g., \btn{Material
            Parameters}), and fill out the pop-up forms.  A
            completely new problem may be defined this way.
      \item If desired, the defined problem may be stored to disk via
            the \btn{File\pipe Save as\ldots} menu selection. The 2D
            solver \app{mmSolve2D} reads problem definitions directly
            from \app{mmProbEd}, but \app{Oxsii} requires file input.
    \end{itemize}
    \item[Initialize solver:]\blankspace
    \begin{itemize}
      \item In the \app{mmLaunch} window, click on the \btn{mmSolve2D}
            button to launch an instance of the program
            \hyperrefhtml{\app{mmSolve2D}}{\app{mmSolve2D}
              (Sec.~}{)}{sec:mmsolve2d}\index{application!mmSolve2D}.
      \item Wait for the new solver instance to appear in the 
          \btn{Running Applications} column in the \app{mmLaunch} window.
      \item Check the box next to the \app{mmSolve2D} entry in the
          \btn{Running Applications}\index{running~applications}
          column.  A window containing an \app{mmSolve2D} interface
          will appear.
      \item In the \app{mmSolve2D} window:
      \begin{itemize}
        \item Check \btn{Problem Description} under \btn{Inputs}.
        \item Check \btn{mmProbEd} under \btn{Source Threads}.
        \item Click \btn{LoadProblem}.
        \item A status line will indicate the problem is loading.
        \item When the problem is fully loaded, more buttons appear.
        \item Check \btn{Scheduled Outputs}.
        \item For each desired output (\btn{TotalField}, \btn{Magnetization}, 
          and/or \btn{DataTable}), specify the frequency of update:
        \begin{enumerate}
          \item Check desired output.  This will exhibit the possible
                output destinations under the Destination Threads
                heading.  Output applications such as \btn{mmDisp},
                \btn{mmGraph}, and/or \btn{mmDataTable} must be running
                to appear in this list.
          \item Check the box next to the desired Destination Thread.  
                This will exhibit Schedule options.
          \item Choose a schedule\index{output~schedule}:
          \begin{itemize}
            \item {\bf Iteration:} fill in number and check the box.
            \item {\bf ControlPoint:} fill in number and check the box.
            \item {\bf Interactive:} whenever you click corresponding
                  Interactive output button.
          \end{itemize}
        \end{enumerate}
      \end{itemize}
    \end{itemize}
    \item[Start calculation:]\blankspace
    \begin{itemize}
      \item In the \app{mmSolve2D} window, start the calculation with
            \btn{Run} (which runs until problem completion) or
            \btn{Relax} (which runs until the next control point is
            reached)\index{simulation~2D!interactive~control}.
      \item If you requested mmDataTable output, check the boxes for the
            desired quantities on the
            \hyperrefhtml{\app{mmDataTable}}{\app{mmDataTable}
              (Ch.~}{)}{sec:mmdatatable}\index{application!mmDataTable}
            window under
            the \btn{Data} menu, so that they appear and are updated as
            requested in your schedule.
      \item Similarly, check the box for the desired X, Y1, and Y2
            quantities on the
            \hyperrefhtml{\app{mmGraph}}{\app{mmGraph}
              (Ch.~}{)}{sec:mmgraph}\index{application!mmGraph}
            window(s) under the \btn{X}, \btn{Y1} and \btn{Y2} menus.
    \end{itemize}
    \item[Save and/or display results:]\index{data!save}\blankspace
    \begin{itemize}
      \item Vector field data (magnetization and effective field) may be
            viewed using \hyperrefhtml{\app{mmDisp}}{\app{mmDisp}
              (Ch.~}{)}{sec:mmdisp}\index{application!mmDisp}.  You can
            manually save data to disk\index{file!vector~field} using
            the \btn{File\pipe Save as\ldots} menu option in
            \app{mmDisp}, or you can send scheduled output to
            \hyperrefhtml{\app{mmArchive}}{\app{mmArchive}
              (Ch.~}{)}{sec:mmarchive}\index{application!mmArchive} for
            automatic storage.  For example, to save the magnetization
            state at the end of each control point, start up an instance
            of \app{mmArchive} and select the {\bf
            ControlPoint}\index{simulation~2D!control~point} check box for
            \app{mmArchive} on the \btn{Magnetization} schedule in the
            solver.  This may be done before starting the calculation.
            (Control points are points in the simulation where the applied
            field is stepped.  These are typically
            equilibrium\index{simulation~2D!equilibrium} states, but
            depending on the input \fn{*.mif} file, may be triggered by
            elapsed simulation time or iteration count.)
      \item Tabular data\index{file!data~table} may be saved by
            sending scheduled output\index{output~schedule} from the
            solver to \app{mmArchive}\index{application!mmArchive},
            which automatically saves all the data it receives.
            Alternatively, \app{mmGraph} can be used to save a subset of
            the data: schedule output to \app{mmGraph} as desired, and
            use either the interactive or automated save functionality
            of \app{mmGraph}.  You can set up the solver data scheduling
            before the calculation is started, but you must wait for the
            first data point to configure \app{mmGraph} before saving
            any data.  As a workaround, you may configure \app{mmGraph}
            by sending it the initial solver state interactively, and
            then use the {\btn{Options\pipe clear Data}} menu item in
            \app{mmGraph} to remove the initializing data point.  If you
            want to inspect explict numeric values, use
            \app{mmDataTable}\index{application!mmDataTable}, which
            displays single sets of values in a tabular format.
            \app{mmDataTable} has no data save functionality.
    \end{itemize}
    \item[Midcourse control:]\blankspace
    \begin{itemize}
      \item In the \app{mmSolve2D} window, buttons can stop and restart the
            calculation\index{simulation~2D!interactive~control}:
      \begin{itemize}
        \item {\bf Reset:}  Return to beginning of problem.
        \item {\bf LoadProblem:} Restart with a new problem.
        \item {\bf Run:}    Apply a sequence of fields until all complete.
        \item {\bf Relax:}  Run the ODE at the current applied field until
            the next control point is reached.
        \item {\bf Pause:}  Click anytime to stop the solver.  Continue
            simulation from paused point with \btn{Run} or \btn{Relax}.
        \item {\bf\boldmath Field$-$:} Apply the previous field again.
        \item {\bf\boldmath Field$+$:} Apply the next field in the list.
      \end{itemize}
      \item Output options can be changed and new output windows opened.
      \item When the stopping criteria for the final control point are
            reached, \app{mmSolve2D} will pause to allow the user to
            interactively output final results.
    \end{itemize}
  \end{description}
  \item \quickstartitemlabelsubincr{Run a 3D problem.}
  \begin{description}
    \item[Launch solver:]\blankspace
    \begin{itemize}
      \item In the \app{mmLaunch} window, click on the \btn{Oxsii}
            button to launch an instance of the program
            \hyperrefhtml{\app{Oxsii}}{\app{Oxsii}
             (Sec.~}{)}{sec:oxsii}\index{application!oxsii}.
      \item Wait for the new solver instance to appear in the 
          \btn{Running Applications} column in the \app{mmLaunch} window.
      \item Check the box next to the \app{Oxsii} entry in the
          \btn{Running Applications}\index{running~applications}
          column.  A window containing an \app{Oxsii} interface will
          appear.
    \end{itemize}
    \item[Load problem:]\blankspace
    \begin{itemize}
      \item In the \app{Oxsii} window, select the
        \btn{File\pipe Load\ldots} menu option.  A \btn{Load Problem}
        dialog box will appear.  On this window:
        \begin{itemize}
          \item Double click in the \btn{Path} subwindow to change
                directories. Numerous sample problems can be found in
                the directory \fn{oommf/app/oxs/examples}.
          \item To load a problem, double click on a \fn{*.mif} file
                (e.g., stdprob1.mif) from the list above the \btn{Filter:}
                subwindow.
        \end{itemize}
        The native input format for the 3D solver is the
        \hyperrefhtml{\MIF~2}{\MIF~2 (Sec.~}{)}{sec:mif2format}
        format, which must be composed by hand using a plain text
        editor.  (See the \hyperrefhtml{Oxs\_Ext Child Class}{Oxs\_Ext
        Child Class (Sec.~}{)}{sec:oxsext} documentation for additional
        details.)  However, \MIF~1.1 (i.e., 2D problem) files are
        readable by \app{Oxsii}, or may be converted to the \MIF~2.1
        format using the command line tool
        \hyperrefhtml{\app{mifconvert}}{\app{mifconvert}
        (Sec.~}{)}{sec:mifconvert}\index{application!mifconvert}.
        \hyperrefhtml{\app{mmProbEd}}{\app{mmProbEd}
          (Ch.~}{)}{sec:mmprobed}\index{application!mmProbEd}
        also supports an extension to the \MIF~1.1
        format, namely \MIF~1.2, which provides limited 3D
        functionality.  \MIF~1.2 files may also be read directly by
        \app{Oxsii}.  Either way, to run in \app{Oxsii} a problem
        created by \app{mmProbEd}, the problem must first be saved to
        disk via the \btn{File\pipe Save as\ldots} menu option in
        \app{mmProbEd}.
      \item The status line in the \app{Oxsii} interface window will
        indicate the problem is loading. 
      \item When the problem is fully loaded, the status line will
        show ``Pause'', and the top row of buttons (\btn{Reload},
        \btn{Reset}, \ldots) will become active.  Also, the
        Output list will fill with available outputs.
      \item Set up scheduled outputs.  For each desired output
      \begin{enumerate}
         \item Select the source from the Output list.
         \item Select the receiver from the Destination list.
         \item Specify the frequency of update:
         \begin{itemize}
            \item {\bf Step:} fill in number and check the box.
            \item {\bf Stage:} fill in number and check the box.
            \item {\bf Done:} produces output when problem completes.
         \end{itemize}
         You can also transmit data interactively by clicking on the
         \btn{Send}.
      \end{enumerate}
      The items in the Output list will vary depending on the
      problem that was loaded.  The items in the Destination list
      reflect the OOMMF data display and archiving programs currently
      running.
    \end{itemize}
    \item[Start calculation:]\blankspace
    \begin{itemize}
       \item In the \app{Oxsii} window, start the calculation with
         \btn{Run}, \btn{Relax}, or
         \btn{Step}\index{simulation~3D!interactive~control}.
       \item If you requested mmDataTable output, check the boxes for the
         desired quantities on the
         \hyperrefhtml{\app{mmDataTable}}{\app{mmDataTable}
           (Ch.~}{)}{sec:mmdatatable}\index{application!mmDataTable}
         window under
         the \btn{Data} menu, so that they appear and are updated as
         requested in your schedule.
       \item Similarly, check the box for the desired X, Y1, and Y2
         quantities on the
         \hyperrefhtml{\app{mmGraph}}{\app{mmGraph}
           (Ch.~}{)}{sec:mmgraph}\index{application!mmGraph}
         window(s) under the \btn{X}, \btn{Y1} and \btn{Y2} menus.
    \end{itemize}
    \item[Save and/or display results:]\index{data!save}\blankspace
    \begin{itemize}
      \item Vector field data (magnetization and fields) may be
        viewed using
        \hyperrefhtml{\app{mmDisp}}{\app{mmDisp}
          (Ch.~}{)}{sec:mmdisp}\index{application!mmDisp}.
        You can 
        manually save data to disk\index{file!vector~field} using
        the \btn{File\pipe Save as\ldots} menu option in \app{mmDisp},
        or you can send scheduled output to
        \hyperrefhtml{\app{mmArchive}}{\app{mmArchive}
          (Ch.~}{)}{sec:mmarchive}\index{application!mmArchive} for
        automatic storage.  For example, to save the magnetization state
        at the end of each problem  stage, start up an instance of
        \app{mmArchive} and select the
        {\bf Stage}\index{simulation~3D!stage} check box for 
        the \cd{Magnetization} output, \cd{mmArchive} destination pair.
        (Stages denote points in the simulation where some
         significant event occurs, such as when an equilibrium is
         reached or some preset simulation time index is met.
         These criteria are set by the input \MIF\ file.)
      \item Tabular data\index{file!data~table} may be saved by sending
         scheduled output\index{output~schedule} from the solver to
         \app{mmArchive}\index{application!mmArchive}, which
         automatically saves all the data it receives.  Alternatively,
         \app{mmGraph} can be used to save a subset of the data:
         schedule output to \app{mmGraph} as desired, and use either the
         interactive or automated save functionality of \app{mmGraph}.
         You can set up the solver data scheduling before the
         calculation is started, but you must wait for the first data
         point to configure \app{mmGraph} before saving any data.  As a
         workaround, you may configure \app{mmGraph} by sending it the
         initial solver state interactively, and then use the
         {\btn{Options\pipe clear Data}} menu item in \app{mmGraph} to
         remove the initializing data point.  If you want to inspect
         explict numeric values, use
         \app{mmDataTable}\index{application!mmDataTable}, which
         displays single sets of values in a tabular format.
         \app{mmDataTable} has no data save functionality.
      \end{itemize}
    \item[Midcourse control:]\blankspace
    \begin{itemize}
       \item In the \app{Oxsii} window, buttons can stop and restart the
             calculation\index{simulation~3D!interactive~control}:
       \begin{itemize}
         \item {\bf Reload:} Reload the same file from disk.
         \item {\bf Reset:}  Return to problem start.
         \item {\bf Run:}    Step through all stages until complete.
         \item {\bf Relax:}  Run until the current stage termination
                             criteria are met.
         \item {\bf Step:}   Do one solver iteration and then pause.
         \item {\bf Pause:}  Click anytime to stop the solver.  Continue
             simulation from paused point with \btn{Run}, \btn{Relax} or
             \btn{Step}.
         \item {\bf Stage:}  Interactively change the current stage
             index by either typing the desired stage number (counting
             from 0) into the \btn{Stage} entry box or by moving the
             associated slider.
       \end{itemize}
       \item Output options can be changed and new output windows
             opened.  The \btn{Send} button in the \app{Oxsii} Schedule
             subwindow is used to interactively send output to the
             selected Output + Destination pair.
       \item When the stage termination (stopping) criteria of the
             final stage are met, \app{Oxsii} will pause to allow the
             user to interactively output final results via the
             \btn{Send} button. (This differs from the
             \hyperrefhtml{\app{Boxsi}}{\app{Boxsi}
             (Sec.~}{)}{sec:boxsi}\index{application!Boxsi}
             batch interface which terminates automatically when the
             final stage is complete.)
    \end{itemize}
  \end{description}
  \item \quickstartitemlabel{Exit \OOMMF.}
  \begin{itemize}
    \item Individual \OOMMF\ applications can be terminated by selecting 
       the \btn{File\pipe Exit} menu item from their interface window.
    \item Selecting \btn{File\pipe Exit} on the \app{mmLaunch} window
       will close the \app{mmLaunch} window, and also the interface
       windows for any \app{mmArchive}, \app{mmSolve2D}, and \app{Oxsii}
       applications.  However, those applications will continue to run
       in the background, and their interfaces may be re-displayed by
       starting a new \app{mmLaunch} instance.
    \item To kill all \OOMMF\ applications, select the
       \btn{File\pipe Exit All OOMMF} option from the \app{mmLaunch}
       menu bar.
  \end{itemize}
\end{list}


\chapter{OOMMF Architecture Overview}\label{sec:arch}
\index{architecture}

Before describing each of the applications which comprise
the \OOMMF\ software, it is helpful to understand how these
applications work together.  \OOMMF\ is not structured as
a single program.  Instead it is a collection of programs,
each specializing in some task needed as part of a
micromagnetic simulation system.  An advantage of this modular
architecture is that each program may be improved or even replaced 
without a need to redesign the entire system.

The \OOMMF\  programs work together by providing services\index{services}
to one another.  
The programs communicate using localhost Internet
(TCP/IP\index{TCP/IP}) connections.
\index{client-server~architecture|(}
When two \OOMMF\ applications are in
the relationship that one is requesting a service from the other,
it is convenient to introduce some clarifying terminology.  Let
us refer to the application that is providing a service as
the ``server application\index{server}'' and the application requesting the
service as the ``client application\index{client}.''  
Note that a single application
can be both a server application in one service relationship and a 
client application in another service relationship.  
\index{client-server~architecture|)}

\index{account~service~directory|(}
Each server application provides its service on a particular
Internet port, and needs to inform potential client applications 
how to obtain its service.  Each client application needs to be able
to look up possible providers of the service it needs.  The
intermediary which brings server applications and client applications
together is another application called the 
``account service directory.''
Each account service directory keeps track of all the services provided
by \OOMMF\ server applications running under its user account on its
host and the corresponding Internet ports at which those services
may be obtained.
\OOMMF\ server applications register their services with
the corresponding account service directory application.  \OOMMF\
client applications look up service providers running under a 
particular user ID\index{user~ID} in the corresponding account server directory 
application.  
\index{account~service~directory|)}

\index{host~service~directory|(}
The account service directory applications simplify the problem
of matching servers and clients, but they do not completely solve
it.  \OOMMF\ applications still need a mechanism to find out how
to obtain the service of the account service directory!
Another application, called the ``host service directory'' serves
this function.  Its sole purpose is to tell \OOMMF\ applications
where to obtain the services of account service directories on that
host. It provides this service on a ``well-known'' port that is
configured into the \OOMMF\ software.  By default, this is port 15136.
\OOMMF\ software can be 
\hyperrefhtml{customized}{customized (Sec.~}{)}{sec:install.custom}
\index{customize!host~server~port}
to use a different port number.
\index{host~service~directory|)}

\index{host~service~directory!launching~of|(}
\index{account~service~directory!launching~of|(}
These service directory applications are vitally important to the operation
of the total \OOMMF\ micromagnetic simulation system.  However, it would be
easy to overlook them.  They act entirely behind the scenes without a user
interface window.  Furthermore, they are usually not launched directly by
the user.  (The notable exception involves\index{application!launchhost}
\hyperrefhtml{\app{launchhost}}{\app{launchhost} (Sec.~}{)}{sec:launchhost},
which is used when multiple host servers are needed to isolate groups of
\OOMMF\ applications running on one machine.) When any server application
needs to register its service, if it finds that these service directory
applications are not running, it launches new copies of them.  In this way
the user can be sure that if any \OOMMF\ server application is running, then
so are the service directory applications needed to direct clients to its
service.  After all server applications terminate, and there are no longer
any services registered with a service directory application, it terminates
as well.  Similarly, when all service directory applications terminate, the
host service directory application exits. The command line
utility\index{application!pidinfo}
\hyperrefhtml{\app{pidinfo}}{\app{pidinfo} (Sec.~}{)}{sec:pidinfo}
can be used to check the current status of the host and account service
directory applications.
\index{host~service~directory!expires}
\index{account~service~directory!expires}
\index{host~service~directory!launching~of|)}
\index{account~service~directory!launching~of|)}

In the sections which follow, the \OOMMF\ applications are
described in terms of the services they provide and the services
they require.  



\chapter{Command Line Launching}\label{sec:cll}
\index{launch!from~command~line}\index{launch!with~bootstrap~application}

Some of the \OOMMF\ applications are platform-independent Tcl
scripts.  Some of them are Tcl scripts that require special
platform-dependent interpreters.  Others are platform-dependent,
compiled C++ applications.  It is possible that some of them will
change status in later releases of \OOMMF.  Each of these types
of application requires a different command line for launching.
Rather than require all \OOMMF\ users to manage this complexity,
we provide a pair of programs that provide simplified interfaces
for launching \OOMMF\ applications.

\index{application!bootstrap|(}
The first of these is used to launch \OOMMF\ applications from the
command line.  Because its function is only to start another
program, we refer to this program as the ``bootstrap application.''
The bootstrap application is the Tcl script \fn{oommf.tcl}.
In its simplest usage, it takes a single argument on the command line,
the name of the application to launch.  For example, to launch
\hyperrefhtml{\app{mmGraph}}{\app{mmGraph} (Ch.~}{)}{sec:mmgraph},
the command line is:
\begin{verbatim}
tclsh oommf.tcl mmGraph
\end{verbatim}
The search for an application matching the name is case-insensitive.
(Here, as elsewhere in this document, the current working
directory\index{working~directory} is assumed to be the \OOMMF\ root
directory.  For other cases, adjust the pathname to {\tt oommf.tcl} as
appropriate.)  As discussed \hyperrefhtml{earlier}{in
Sec.~}{}{sec:install.requirements}, the name of the \Tcl\ shell,
rendered here as \verb+tclsh+, may vary between systems.

If no command line arguments are passed to the bootstrap application,
by default it will launch the application
\hyperrefhtml{\app{mmLaunch}}{\app{mmLaunch} (Ch.~}{)}{sec:mmlaunch}.

\index{launch!command~line~arguments|(}
Any command line arguments to the bootstrap
application that begin with the character `\cd{+}' modify its
behavior.
For a summary of all command line options recognized by the bootstrap
application, run:
\begin{verbatim}
tclsh oommf.tcl +help
\end{verbatim}

\index{launch!foreground|(}
The command line arguments \cd{+bg} and \cd{+fg} control how the
bootstrap behaves after launching the requested application.  It
can exit immediately after launching the requested application
in background mode (\cd{+bg}), or it can block until the
launched application exits (\cd{+fg}).  Each application
registers within the \OOMMF\ system whether it prefers to be launched
in foreground or background mode.  If neither option is requested on
the command line, the bootstrap launches the requested application
in its preferred mode.
\index{launch!foreground|)}

\index{launch!version~requirement|(}
The first command line argument that does not begin with the
character \cd{+} is interpreted as a specification of which
application should be launched.  As described above, this is
usually the simple name of an application.
When a particular
version of an application is required, though, the bootstrap
allows the user to include that requirement as part of the
specification.  For example:
\begin{verbatim}
tclsh oommf.tcl "mmGraph 1.1"
\end{verbatim}
will guarantee that the instance of the application mmGraph it
launches is of at least version 1.1.  If no copy of mmGraph
satisfying the version requirement can be found, an error is
reported.
\index{launch!version~requirement|)}

The rest of the command line arguments that are not recognized by
the bootstrap are passed along as arguments to the application the
bootstrap launches.  Since the bootstrap recognizes command line
arguments that begin with \cd{+} and most other applications
recognize command line arguments that begin with \cd{-}, confusion
about which options are provided to which programs can be avoided.
For example,
\begin{verbatim}
tclsh oommf.tcl +help mmGraph
\end{verbatim}
prints out help information about the bootstrap and exits without
launching mmGraph.  However,
\begin{verbatim}
tclsh oommf.tcl mmGraph -help
\end{verbatim}
launches mmGraph with the command line argument \cd{-help}.
mmGraph then displays its own help message.

\index{launch!standard~options|(}
\index{standard~options|(}
Most \OOMMF\ applications accept the standard options listed below.
Some of the \OOMMF\ applications accept additional arguments when
launched from the command line, as documented in the corresponding
sections of this manual.  The \verb+-help+ command line option can
also be used to view the complete list of available options.  When an
option argument is specified as \verb+<0|1>+, \verb+0+ typically means
off, no or disable, and \verb+1+ means on, yes or enable.

\begin{description}
\item[\optkey{-console}]
Display a console widget in which Tcl
commands may be interactively typed into the application.
Useful for debugging.

\item[\optkey{-cwd directory}]
Set the current working directory of the application.

\item[\optkey{-help}]
Display a help message and exit.

\item[\optkey{-nickname \boa name\bca}]\index{nicknames}
Associates the specified \textit{name} as a nickname for the process.
The string \textit{name} should contain at least one non-numeric
character.  Nicknames can also be set at launch time via the
\htmlonlyref{\cd{Destination}}{html:destinationCmd} command\latex{
(Sec.~\ref{sec:mif2ExtensionCommands})} in \cd{MIF 2.x} files, or
after a process is running via the
\hyperrefhtml{\app{nickname}}{\app{nickname} (Sec.~}{)}{sec:nickname}
command line application.  Nicknames are used by the \MIF\ 2.x
\cd{Destination} command to associate \cd{Oxs} output streams with
particular application instances.  Multiple \cd{-nickname} options may
be used to set multiple nicknames.  (Technical detail: Nickname
functionality is only available to processes that connect to an
account server.)

\item[\optkey{-tk \boa 0\pipe 1\bca}]
Disable or enable Tk.  Tk must be enabled for an application to display
graphical widgets.  However, when Tk is enabled on Unix platforms
the application is dependent on an X Windows server.  If the
X Windows server dies, it will kill the application.  Long-running
applications that do not inherently use display widgets support
disabling of Tk with \verb+-tk 0+.
Other applications that must use display widgets are unable to run
with the option \verb+-tk 0+.  To run applications that require
\verb+-tk 1+ on a Unix system with no display, one might use
\htmladdnormallinkfoot{Xvfb}{https://www.x.org/archive/X11R7.6/doc/man/man1/Xvfb.1.xhtml}
\index{application!Xvfb}.

\item[\optkey{-version}]
Display the version of the application and exit.
\end{description}

In addition, those applications which enable Tk accept additional Tk
options, such as \verb+-display+.  See the Tk documentation for details.
\index{standard~options|)}
\index{launch!standard~options|)}
\index{launch!command~line~arguments|)}

The bootstrap application should be infrequently used by most users.
The application \hyperrefhtml{\app{mmLaunch}}{\app{mmLaunch}
(Ch.~}{)}{sec:mmlaunch} provides a more convenient graphical
interface for launching applications.  The main uses for the bootstrap
application are launching \app{mmLaunch}, launching \app{pimake},
launching programs which make up the \hyperrefhtml{OOMMF Batch
System}{OOMMF Batch System (Sec.~}{)}{sec:obs} and other programs
that are inherently command line driven, and in circumstances where
the user wishes to precisely control the command line arguments passed
to an \OOMMF\ application or the environment in which an \OOMMF\ application
runs.
\index{application!bootstrap|)}

\starsechead{Platform Issues}

\index{platform!Unix!executable~Tcl~scripts|(}
On most Unix platforms, if \fn{oommf.tcl} is marked executable,
it may be run directly, i.e., without specifying \cd{tclsh}.  This
works because the first few lines of the \fn{oommf.tcl} Tcl script are:
\begin{verbatim}
#!/bin/sh
# \
exec tclsh "$0" ${1+"$@"}
\end{verbatim}
When run, the first \fn{tclsh} on the execution path is invoked to
interpret the \fn{oommf.tcl} script.  If the Tcl shell program cannot be
invoked by the name \fn{tclsh} on your computer, edit the first lines of
\fn{oommf.tcl} to use the correct name.  Better still, use symbolic links
or some other means to make the Tcl shell program available by the name
\fn{tclsh}.  The latter solution will not be undone by file overwrites
from \OOMMF\ upgrades.

If in addition, the directory
\fn{.../path/to/oommf} is in the execution path, the command line can
be as simple as:
\begin{verbatim}
oommf.tcl appName
\end{verbatim}
from any working directory.
\index{platform!Unix!executable~Tcl~scripts|)}

\index{platform!Windows!file~extension~associations|(}
On Windows platforms, because \fn{oommf.tcl} has the file
extension \fn{.tcl}, it is normally associated by Windows with the
\fn{wish} interpreter.  The \fn{oommf.tcl} script has been
specially written so that either \fn{tclsh} or \fn{wish} is a suitable
interpreter.  This means that simply double-clicking on an icon
associated with the file \fn{oommf.tcl}
(say, in Windows Explorer\index{application!Windows~Explorer})
will launch the bootstrap application with no arguments.  This will
result in the default behavior of launching the application
\app{mmLaunch}, which is suitable for launching other \OOMMF\
applications.  (If this doesn't work, refer back to the
\hyperrefhtml{Windows Options}{Windows Options section in the
installation instructions, Sec.~}{}{sec:install.windows}\HTMLoutput{
section in the installation
instructions}.)\index{platform!Windows!file~extension~associations|)}

\section{OOMMF Launcher/Control Interface: mmLaunch}\label{sec:mmlaunch}
\index{application!mmLaunch}

\begin{center}
\includepic{mmlaunch-ss}{mmLaunch Screen Shot}
\end{center}

\ssechead{Overview}
The application \app{mmLaunch} launches, monitors, and controls other 
\OOMMF\ applications.  It is the \OOMMF\ application which is
most closely connected to the account service directory and host
service directory applications that run behind the scenes.  It also
provides user interfaces\index{mmLaunch~user~interface}
to any applications, notably
\hyperrefhtml{\app{Oxsii}}{\app{Oxsii}
(Sec.~}{)}{sec:oxsii} and
\hyperrefhtml{\app{mmSolve2D}}{\app{mmSolve2D}
(Sec.~}{)}{sec:mmsolve2d}, that do not have their own user interface
window.

\ssechead{Launching}
\app{mmLaunch} should be launched using the \hyperrefhtml{bootstrap
application}{bootstrap application (Sec.~}{)}{sec:cll}.  The command
line is
\begin{verbatim}
tclsh oommf.tcl mmLaunch [standard options]
\end{verbatim}

\ssechead{Controls}
Upon startup, \app{mmLaunch} displays a panel of checkbuttons, one for
each host service directory\index{host~service~directory}
to which it is connected.  In the current
release of \OOMMF\ there is only one checkbutton, named for the host on
which \app{mmLaunch} is running.
%Future releases of \app{mmLaunch} will be able to connect to remote
%hosts as well.  
If there is no host service directory running on the
localhost when \app{mmLaunch} is launched,
\app{mmLaunch} will start one.  In that circumstance, there may be 
some delay before the host checkbutton appears.

Toggling the host checkbutton toggles the display of an
interface to the host service directory.  The host service directory
interface consists of a row of checkbuttons, one for each account
service directory\index{account~service~directory}
registered with the host service directory.  Each
checkbutton is labeled with the user ID\index{user~ID}
of the corresponding account
service directory.  For most users, there will be only one
checkbutton, labeled with the user's own account
ID.
%ID, except on
%\Windows~9X, where the dummy account ID 
%``oommf''\index{platform!Windows!dummy~user~ID} may be displayed
%instead.
If there is no account service directory running for the
account under which {\bf mmLaunch} was launched, \app{mmLaunch} will
start one.  In that circumstance, there may be some delay before the
account checkbutton appears.

\index{launch!with~mmLaunch|(}
Toggling an account checkbutton toggles the display of an interface
to the corresponding account service directory.  The account
service directory interface consists of two columns.  The 
\btn{Programs} column contains buttons labeled with the names of
\OOMMF\ applications that may be launched under the account managed by
this account service directory.  Clicking on one of these buttons
launches the corresponding application.  Only one click is needed,
though there will be some delay before the launched application
displays a window to the user.  Multiple clicks will launch multiple
copies of the application.  Note: The launching is actually handled by
the \hyperrefhtml{account service directory application}{account
service directory application (Sec.~}{)}{sec:arch}, which sets the
initial working directory\index{working~directory} to the
\OOMMF\ root directory.
\index{launch!with~mmLaunch|)}

The \btn{Threads}\index{threads}
column is a list of all the \OOMMF\ applications
currently running under the account that are registered with the
account service directory.  The list includes both the application
name and an ID number by which multiple copies of the same application
may be distinguished.  This ID number is also displayed in the title
bar of the corresponding application's user interface window.  When an
application exits, its entry is automatically removed from the Threads
list.

\index{mmLaunch~user~interface|(}
Any of the running applications that do not provide their own
interface window will be displayed in the \btn{Threads} list with a
checkbutton.  The checkbutton toggles the display of an interface
which \app{mmLaunch} provides on behalf of that application.  The only
\OOMMF\ applications currently using this service are
the 3D solvers \hyperrefhtml{\app{Oxsii} and \app{Boxsi}}{\app{Oxsii}
and \app{Boxsi} (Sec.~}{)}{sec:oxs},
the 2D solvers \hyperrefhtml{\app{mmSolve2D and
\app{batchsolve}}}{\app{mmSolve2D} and \app{batchsolve} (Sec.~}{)}{sec:mmsolve}, and the archive application
\hyperrefhtml{\app{mmArchive}}{\app{mmArchive} (Sec.~}{)}{sec:mmarchive}.
These interfaces are described in the documentation for the
corresponding applications.
\index{mmLaunch~user~interface|)}

The menu selection \btn{File\pipe Exit} terminates the \app{mmLaunch}
application, and the \btn{File\pipe Exit All OOMMF} selection terminates
all applications in the Threads list, and then exits \app{mmLaunch}.
The menu \btn{Help} provides the usual help facilities.


\section{\OOMMF\ eXtensible Solver}\label{sec:oxs}

The \OOMMF\ eXtensible Solver (OXS) top level architecture is shown in
\hyperrefhtml{the class diagram below}{Fig.~}{}{fig:oxsclass}.
The ``Tcl Control Script'' block represents the user interface and
associated control code, which is written in \Tcl.  The
micromagnetic problem input file is the content of the ``Problem
Specification'' block.  The input file should be a valid \MIF~2.0 file
(see the \OOMMF\ User's Guide for details on the \MIF\ file formats),
which also happens to be a valid \Tcl\ script.  The rest of the
architecture diagram represents \Cplusplus\ classes.

\ofig{\includeimage{6in}{!}{oxsclass}{OXS top-level class diagram}}{OXS
top-level class diagram.}{fig:oxsclass}

All interactions between the \Tcl\ script level and the core solver are
routed through the Director object.  Aside from the Director, all other
classes in this diagram are examples of \cd{Oxs\_Ext}
objects---technically, \Cplusplus\ child classes of the abstract
\cd{Oxs\_Ext} class.  OXS is designed to be extended primarily by the
addition of new \cd{Oxs\_Ext} child classes.

The general steps involved in adding an \cd{Oxs\_Ext} child class to OXS
are:
\begin{enumerate}
\item Add new source code files to \fn{oommf/app/oxs/local} containing
your class definitions.  The \Cplusplus\ non-header source code file(s)
must be given the \cd{.cc} extension.  (Header files are typically
denoted with the \cd{.h} extension, but this is not mandatory.)
\item Run \app{pimake} to compile your new code and link it in to the OXS
executable.
\item Add the appropriate \cd{Specify} blocks to your input \MIF~2.0
files.
\end{enumerate}
The source code can usually be modeled after an existing \cd{Oxs\_Ext}
object.  Refer to the Oxsii section of the \OOMMF\ User's Guide for a
description of the standard \cd{Oxs\_Ext} classes, or
\hyperrefhtml{below}{Sec.~}{}{sec:energyexample} for an annotated example of
an \cd{Oxs\_Energy} class.  Base details on adding a new energy term are
\hyperrefhtml{also presented below}{presented in Sec.~}{}{sec:energynew}.

The \app{pimake} application automatically detects all files in the
\fn{oommf/app/oxs/local} directory with the \cd{.cc} extension, and searches
them for \cd{\lb include} requests to construct a build dependency tree.
Then \app{pimake} compiles and links them together with the rest of the
OXS files into the \app{oxs} executable.  Because of the automatic file
detection, no modifications are required to any files of the standard
\OOMMF\ distribution in order to add local extensions.

Local extensions are then activated by \cd{Specify} requests in the
input \MIF~2.0 files.  The object name prefix in the \cd{Specify} block
is the same as the \Cplusplus\ class name.  All \cd{Oxs\_Ext} classes in
the standard distribution are distinguished by an \cd{Oxs\_} prefix.  It
is recommended that local extensions use a local prefix to avoid name
collisions with standard OXS objects.  (\Cplusplus\ namespaces are not
currently used in \OOMMF\ for compatibility with some older \Cplusplus\
compilers.)  The \cd{Specify} block initialization string format is
defined by the \cd{Oxs\_Ext} child class itself; therefore, as the
extension writer, you may choose any format that is convenient.
However, it is recommended that you follow the conventions laid out in
the \MIF~2.0 file format section of the \OOMMF\ User's Guide.


\subsection{Sample \cd{Oxs\_Energy} Class}\label{sec:energyexample}
This sections provides an extended dissection of a simple
\cd{Oxs\_Energy} child class.  The computational details are kept as
simple as possible, so the discussion can focus on the \Cplusplus\ class
structural details.  Although the calculation details will vary between
energy terms, the class structure issues discussed here apply across the
board to all energy terms.

The particular example presented here is for simulating
uniaxial magneto-crystalline energy, with a single anisotropy constant,
\cd{K1}, and a single axis, \cd{axis}, which are uniform across the
sample.
\begin{htmlonly}
The \htmlref{class definition}{fig:energyexampledfn} (.h) and
\htmlref{code}{fig:energyexamplecode} (.cc) files are included below.
\end{htmlonly}
\begin{latexonly}
The class definition (.h) and code (.cc) are displayed in
Fig.~\ref{fig:energyexampledfn} and \ref{fig:energyexamplecode},
respectively.
\end{latexonly}

\begin{codelisting}{p}{fig:energyexampledfn}{Example energy class
definition.}{sec:energyexample}
\begin{verbatim}
/* FILE: exampleanisotropy.h
 *
 * Example anisotropy class definition.
 * This class is derived from the Oxs_Energy class.
 *
 */

#ifndef _OXS_EXAMPLEANISOTROPY
#define _OXS_EXAMPLEANISOTROPY

#include "energy.h"
#include "threevector.h"
#include "meshvalue.h"

/* End includes */

class Oxs_ExampleAnisotropy:public Oxs_Energy {
private:
  double K1;        // Primary anisotropy coeficient
  ThreeVector axis; // Anisotropy direction
public:
  virtual const char* ClassName() const; // ClassName() is
  /// automatically generated by the OXS_EXT_REGISTER macro.
  virtual BOOL Init();
  Oxs_ExampleAnisotropy(const char* name,  // Child instance id
			Oxs_Director* newdtr, // App director
			Tcl_Interp* safe_interp, // Safe interpreter
			const char* argstr);  // MIF input block parameters

  virtual ~Oxs_ExampleAnisotropy() {}

  virtual void GetEnergyAndField(const Oxs_SimState& state,
                                 Oxs_MeshValue<REAL8m>& energy,
                                 Oxs_MeshValue<ThreeVector>& field
                                 ) const;
};


#endif // _OXS_EXAMPLEANISOTROPY
\end{verbatim}
\end{codelisting}

\begin{codelisting}{p}{fig:energyexamplecode}{Example energy class
code.}{sec:energyexample}
\begin{verbatim}
/* FILE: exampleanisotropy.cc            -*-Mode: c++-*-
 *
 * Example anisotropy class implementation.
 * This class is derived from the Oxs_Energy class.
 *
 */

#include "exampleanisotropy.h"

// Oxs_Ext registration support
OXS_EXT_REGISTER(Oxs_ExampleAnisotropy);

/* End includes */

#define MU0           12.56637061435917295385e-7   /* 4 PI 10^7 */

// Constructor
Oxs_ExampleAnisotropy::Oxs_ExampleAnisotropy(
  const char* name,     // Child instance id
  Oxs_Director* newdtr, // App director
  Tcl_Interp* safe_interp, // Safe interpreter
  const char* argstr)   // MIF input block parameters
  : Oxs_Energy(name,newdtr,safe_interp,argstr)
{
  // Process arguments
  K1=GetRealInitValue("K1");
  axis=GetThreeVectorInitValue("axis");
  VerifyAllInitArgsUsed();
}

BOOL Oxs_ExampleAnisotropy::Init()
{ return 1; }

void Oxs_ExampleAnisotropy::GetEnergyAndField
(const Oxs_SimState& state,
 Oxs_MeshValue<REAL8m>& energy,
 Oxs_MeshValue<ThreeVector>& field
 ) const
{
  const Oxs_MeshValue<REAL8m>& Ms_inverse = *(state.Ms_inverse);
  const Oxs_MeshValue<ThreeVector>& spin = state.spin;
  UINT4m size = state.mesh->Size();

  for(UINT4m i=0;i<size;++i) {
    REAL8m field_mult = (2.0/MU0)*K1*Ms_inverse[i];
    if(field_mult==0.0) {
      energy[i]=0.0;
      field[i].Set(0.,0.,0.);
      continue;
    }
    REAL8m dot = axis*spin[i];
    field[i] = (field_mult*dot) * axis;
    if(K1>0) {
      energy[i] = -K1*(dot*dot-1.0); // Make easy axis zero energy
    } else {
      energy[i] = -K1*dot*dot; // Easy plane is zero energy
    }
  }
}
\end{verbatim}
\end{codelisting}


\subsection{Writing a New \cd{Oxs\_Energy} Extension}\label{sec:energynew}
Under construction.







\chapter{Micromagnetic Problem Editor: mmProbEd}\label{sec:mmprobed}

\begin{center}
\includepic{mmprobed-ss}{mmProbEd Screen Shot}
\end{center}

\starsechead{Overview}
The application \app{mmProbEd}\index{application!mmProbEd} provides a
user interface for creating and editing micromagnetic problem
descriptions in the \hyperrefhtml{\MIF~1.1}{\MIF~1.1 (Sec.~}{,
page~\pageref{sec:mif1format})}{sec:mif1format}\index{file!MIF~1.1} and
\hyperrefhtml{\MIF~1.2}{\MIF~1.2
(Sec.~}{)}{sec:mif12format}\index{file!MIF~1.2} formats.  \app{mmProbEd}
also acts as a server, supplying problem descriptions to running
\app{mmSolve2D} micromagnetic solvers.

\starsechead{Launching}
\app{mmProbEd} may be started either by selecting the
\btn{mmProbEd} button on \htmlonlyref{\app{mmLaunch}}{sec:mmlaunch}, or
from the command line via
\begin{verbatim}
tclsh oommf.tcl mmProbEd [standard options] [-net <0|1>]
\end{verbatim}

\begin{description}
\item[\optkey{-net \boa 0\pipe 1\bca}]
  Disable or enable a server which provides problem descriptions
  to other applications.  By default, the server is enabled.  When
  the server is disabled, \app{mmProbEd} is only useful for editing
  problem descriptions and saving them to files.
\end{description}

\starsechead{Inputs}
The menu selection \btn{File\pipe Open...} displays a dialog box
for selecting a file from which to load a \MIF\ problem 
description.  Several example files are included in the 
\OOMMF\ release in the directory \fn{oommf/app/mmpe/examples}.
At startup, \app{mmProbEd} loads the problem contained in
\fn{oommf/app/mmpe/init.mif} as an initial problem.  Note: When loading
a file, \app{mmProbEd} discards comments and moves records it does not
understand to the bottom of its output file.  Use the
\hyperrefhtml{\app{FileSource} application}{{\bf FileSource} application
(Ch.}{)}{sec:filesource} to serve unmodified problem descriptions.

\starsechead{Outputs}
The menu selection \btn{File\pipe Save as...} displays a dialog box for
selecting/entering a file in which the problem description currently
held by \app{mmProbEd} is to be saved.  Because the internal data format
use by \app{mmProbEd} is an unordered array that does not include
comments (or unrecognized records), the simple operation of reading in a
\MIF\ file and then writing it back out may alter the file.

Each instance of \app{mmProbEd} contains exactly one problem description
at a time.  When the option \cd{-net 1} is active (the default), each also
services requests from client applications
(typically solvers) for the problem description it contains.

\starsechead{Controls}
The \btn{Options} menu allows selection of \MIF\ output format; either
\MIF~1.1 or \MIF~1.2 may be selected.  This affects both
\btn{File\pipe Save as...} file and \app{mmSolve2D} server
output.  See the \hyperrefhtml{\MIF~1.2}{\MIF~1.2 (Sec.~}{,
page~\pageref{sec:mif12format})}{sec:mif12format} format documentation
for details on the differences between these formats.

The main panel in the \app{mmProbEd} window contains buttons
corresponding to the sections in a \MIF~1.x problem description.
Selecting a button brings up another window through which the contents
of that section may be edited.  The \MIF\ sections and the elements they
contain are described in detail in the
\htmlonlyref{\MIF~1.1}{sec:mif1format} and
\htmlonlyref{\MIF~1.2}{sec:mif12format} documentation.
Only one editing window is displayed at a time.  The windows may be
navigated in order using their \btn{Next} or
\btn{Previous} buttons.

The \btn{Material Parameters} edit window includes a pull-down list of
pre-configured material settings.  \textbf{NOTE:} These values should
\textit{not} be taken as standard reference values for these materials.
The values are only approximate, and are included for convenience,
and as examples for users who wish to supply their own material types
with symbolic names.  To introduce additional material types, edit the
\texttt{Material} \texttt{Name}, \texttt{Ms}, \texttt{A}, \texttt{K1}, and
\texttt{Anisotropy} \texttt{Type} values as desired, and hit the \btn{Add}
button.  (The \texttt{Damp} \texttt{Coef} and \texttt{Anistropy}
\texttt{Init} settings are not affected by the Material Types
selection.)  The Material Name entry will appear in red if it does not
match any name in the Material Types list, or if the name matches but
one or more of the material values differs from the corresponding value
as stored in the Material Types list.  You can manage the Material Types
list with the \btn{Replace} and \btn{Delete} buttons, or by directly
editing the file \fn{oommf/app/mmpe/local/materials}; follow the format
of other entries in that file.  The format is the same as in the default
\fn{oommf/app/mmpe/materials} file included with the \OOMMF\ distribution.

The menu selection \btn{File\pipe Exit} terminates the
\app{mmProbEd} application.  The menu \btn{Help} provides
the usual help facilities.


\chapter{Micromagnetic Problem File Source: FileSource}\label{sec:filesource}
\index{application!FileSource}

\begin{center}
\includepic{filesource-ss}{FileSource Screen Shot}
\end{center}

\starsechead{Overview}
The application \app{FileSource} provides the same service as
{\hyperrefhtml{\app{mmProbEd}}{\app{mmProbEd}
(Ch.~}{)}{sec:mmprobed}}\index{application!mmProbEd}, supplying
\MIF~1.x\index{file!MIF~1.x} problem descriptions to running \app{mmSolve2D}
micromagnetic solvers.  As the \MIF\ specification evolves,
\app{mmProbEd} may lag behind.  There may be new fields in the \MIF\
specification that \app{mmProbEd} is not capable of editing, or which
\app{mmProbEd} may not pass on to solvers after loading them in from a
file.  To make use of such fields, a \MIF\ file may need to be edited
``by hand'' using a general purpose text editor.  \app{FileSource} may
then be used to supply the \MIF\ problem description contained in a file
to a solver without danger of corrupting its contents.

\starsechead{Launching}
\app{FileSource} must be launched from the command line. You may specify
on the command line the \MIF\ problem description file it should serve
to client applications.  The command line is
\begin{verbatim}
tclsh oommf.tcl FileSource [standard options] [filename]
\end{verbatim}

Although \app{FileSource} does not appear on the list of
{\btn{Programs}} that \app{mmLaunch} offers to launch, running copies do
appear on the list of \btn{Threads} since they do provide a service
registered with the account service directory.

\starsechead{Inputs}
\app{FileSource} takes its \MIF\ problem description from the file named
on the command line, or from a file selected through the
\btn{File\pipe Open} dialog box.  No checking of the file contents
against the \MIF\ specification is performed.  The file contents are
passed uncritically to any client application requesting a problem
description.  Those client applications should raise errors when
presented with invalid problem descriptions.

\starsechead{Outputs}
Each instance of \app{FileSource} provides the contents of exactly one
file at a time.  The file name is displayed in the \app{FileSource}
window to help the user associate each instance of \app{FileSource} with
the data file it provides.  Each instance of \app{FileSource} accepts
and services requests from client applications (typically solvers) for
the contents of the file it exports.

The contents of the file are read at the time of the client request, so
if the contents of a file change between the time of the
\app{FileSource} file selection and the arrival of a request from a
client, the new contents will be served to the client application.

\starsechead{Controls}
The menu selection \btn{File\pipe Exit} terminates the 
\app{FileSource} application.  The \btn{Help} menu provides
the usual help facilities.

\chapter{The 2D Micromagnetic Solver}\label{sec:mmsolve}%
\index{simulation~2D}

%begin{latexonly}
\newcounter{msoldsecnumdepth}
\setcounter{msoldsecnumdepth}{\value{secnumdepth}}
\setcounter{secnumdepth}{4}
%end{latexonly}


The OOMMF 2D micromagnetic computation engine, mmSolve, is capable of
solving problems defined on a two-dimensional grid of square cells with
three-dimensional spins.  This solver is older, less flexible and less
extensible than the \hyperrefhtml{Oxs}{Oxs (Ch.~}{)}{sec:oxs} solver.
Users are encouraged to migrate to Oxs where possible.

There are two interfaces provided to mmSolve, the interactive
\hyperrefhtml{\app{mmSolve2D}}{\app{mmSolve2D} (Sec.~}{)}{sec:mmsolve2d}
interface and the command line driven
\hyperrefhtml{\app{batchsolve}}{\app{batchsolve} (Sec.~}{)}{sec:batchsolve}
interface which can be used in conjunction with the
\hyperrefhtml{OOMMF Batch System}{OOMMF Batch System (Sec.~}{)}{sec:obs}.

Problem definition for mmSolve is accomplished using input files in the
\hyperrefhtml{\MIF~1.1 format}{\MIF~1.1 format (Sec.~}{)}{sec:mif1format}.
Please note that this format is incompatible with the newer \MIF~2.x
format used by the Oxs solver.  However, the command line utility
\hyperrefhtml{\app{mifconvert}}{\app{mifconvert}
(Sec.~}{)}{sec:mifconvert} can be used to aid conversion from the
\MIF~1.1 format to \MIF~2.1.

mmSolve will also accept files in the \hyperrefhtml{\MIF~1.2
format}{\MIF~1.2 format (Sec.~}{)}{sec:mif12format} format, provided the
\texttt{CellSize} record meets the restriction that the $x$- and
$y$-dimensions are the same, and that the $z$-dimension equals the part
thickness.

Note on \Tk\ dependence: If a problem is loaded that uses a
{\hyperrefhtml{bitmap mask file}{bitmap mask file
(Sec.~}{)}{sec:partgeometry}}\index{file!mask}\index{file!bitmap}, and
if that mask file is not in the PPM P3 (text) format, then
\app{mmSolve2D} will launch {\hyperrefhtml{any2ppm}{\app{any2ppm}
(Sec.~}{)}{sec:any2ppm}}\index{application!any2ppm} to convert it into
the PPM P3 format.  Since \app{any2ppm}
requires\index{requirement!Tk}\index{requirement!display} \Tk, at the
time the mask file is read a valid display must be available.  See the
\app{any2ppm} documentation for details.

\section{The 2D Micromagnetic Interactive Solver: mmSolve2D}%
\label{sec:mmsolve2d}\index{simulation~2D}\index{application!mmSolve2D}

\begin{center}
\includepic{mmsolve2d-ss}{mmSolve2D Screen Shot}
\end{center}

\starssechead{Overview}
The application \app{mmSolve2D} is a micromagnetic computation engine
capable of solving problems defined on two-dimensional square grids of
three-dimensional spins.  Within the \hyperrefhtml{\OOMMF\
architecture}{\OOMMF\ architecture (see Ch.~}{)}{sec:arch},
\app{mmSolve2D} is both a server and a client application.
\app{mmSolve2D} is a client of
problem description server applications, data table display and storage
applications, and vector field display and storage applications.
\app{mmSolve2D} is the server of a solver control service for which the
only client is \hyperrefhtml{\app{mmLaunch}}{\app{mmLaunch}
(Ch.~}{)}{sec:mmlaunch}\index{application!mmLaunch}.  It is through
this service that \app{mmLaunch} provides a user interface window (shown
above) on behalf of \app{mmSolve2D}.

\starssechead{Launching}
\app{mmSolve2D} may be started either by selecting the
\btn{mmSolve2D} button on \htmlonlyref{mmLaunch}{sec:mmlaunch}, or from the
command line via
\begin{verbatim}
tclsh oommf.tcl mmSolve2D [standard options] [-restart <0|1>]
\end{verbatim}
\begin{description}
\item[\optkey{-restart \boa 0\pipe 1\bca\index{simulation~2D!restarting}}]
  Affects the behavior of the solver
  when a new problem is loaded.  Default value is 0.  When launched with
  \cd{-restart 1}, the solver will look for \fn{\textit{basename}.log}
  and \fn{\textit{basename*}.omf} files to restart a previous run from
  the last saved state (where \fn{\textit{basename}} is the ``Base
  Output Filename'' specified in the input {\NONHTMLoutput{\MIF~1.1 problem
  specification file
  (Sec.~\ref{sec:mif1format})).}}{\HTMLoutput{\htmlonlyref{\MIF~1.1}{sec:mif1outspec}
  problem specification file).}}  If these files cannot be found, then a
  warning is issued and the solver falls back to the default behavior
  (\cd{-restart 0}) of starting the problem from scratch.  The specified
  \cd{-restart} setting holds for \textbf{all} problems fed to the
  solver, not just the first.  (There is currently no interactive way to
  change the value of this switch.)
\end{description}

Since \app{mmSolve2D}\index{mmLaunch~user~interface} does not present
any user interface window of its own, it depends on
\app{mmLaunch}\index{application!mmLaunch} to provide an interface on
its behalf.  The entry for an instance of \app{mmSolve2D} in the
\btn{Threads}\index{threads} column of any running copy of
\app{mmLaunch} has a checkbutton next to it.  This button toggles the
presence of a user interface window through which the user may control
that instance of \app{mmSolve2D}.  The user interface window is divided
into panels, providing user interfaces to the
\btn{Inputs}, \btn{Outputs}, and \btn{Controls} of \app{mmSolve2D}.

\starssechead{Inputs}
The top panel of the user interface window may be opened and closed
by toggling the \btn{Inputs} checkbutton.  When open, the
\btn{Inputs} panel reveals two subpanels.  The left subpanel
contains a list of the inputs required by \app{mmSolve2D}.  There is
only one item in the list: \btn{ProblemDescription}.  When
\btn{ProblemDescription} is selected, the right subpanel (labeled
\btn{Source Threads}\index{threads}) displays a list of applications
that can supply a problem description.  The user selects from among the
listed applications the one from which \app{mmSolve2D} should request a
problem description.

\starssechead{Outputs}
When \app{mmSolve2D} has outputs available to be controlled, a
\btn{Scheduled Outputs} checkbutton appears in the user interface
window.  Toggling the \btn{Scheduled Outputs} checkbutton causes a
bottom panel to open and close in the user interface window.  When open,
the \btn{Scheduled Outputs} panel contains three subpanels.  The
\btn{Outputs} subpanel is filled with a list of the types of output
\app{mmSolve2D} can generate while solving the loaded problem.  The
three elements in this list are \btn{TotalField}, for the output of a
vector field\index{file!vector~field} representing the total effective
field, \btn{Magnetization}, for the output of a vector field
representing the current magnetization state of the grid of spins, and
\btn{DataTable}, for the output of a table of data
values\index{file!data~table} describing other quantities of interest
calculated by \app{mmSolve2D}.

Upon selecting one of the output types from the \btn{Outputs} subpanel,
a list of applications appears in the
\btn{Destination Threads}\index{threads} subpanel which provide a
display and/or storage service for the type of output selected.  The
user may select from this list those applications to which the selected
type of output should be sent.

For each application selected, a final interface is displayed in the
\btn{Schedule}\index{output~schedule} subpanel.  Through this interface
the user may set the schedule according to which the selected type of
data is sent to the selected application for display or storage.  The
schedule is described relative to events in \app{mmSolve2D}.  An
\btn{Iteration}~event occurs at every step in the solution of the ODE.
A \btn{ControlPoint}\index{simulation~2D!control~point} event occurs
whenever the solver determines that a control point specification is
met.  (Control point specs are discussed in the {\htmlonlyref{Experiment
parameters}{sec:expparams}} paragraph in the {\hyperrefhtml{\MIF~1.1
documentation}{\MIF~1.1 documentation (Sec.~}{)}{sec:mif1format}}, and
are triggered by solver equilibrium, simulation time, and iteration
count conditions.)  An {\btn{Interactive}} event occurs for a particular
output type whenever the corresponding ``Interactive Outputs'' button is
clicked in the {\btn{Runtime Control}} panel.  The \btn{Interactive}
schedule gives the user the ability to interactively force data to be
delivered to selected display and storage applications.  For the
\btn{Iteration} and \btn{ControlPoint} events, the granularity of the
output delivery schedule is under user control.  For example, the user
may elect to send vector field data describing the current magnetization
state to an \htmlonlyref{\app{mmDisp}}{sec:mmdisp}\index{application!mmDisp}
instance for display every 25 iterations of the ODE, rather than every
iteration.

The quantities included in \btn{DataTable} output produced by
\app{mmSolve2D} include:
\begin{itemize}
\item {\bf Iteration:}\index{iteration} The iteration count of the ODE
      solver.
\item {\bf Field Updates:}\index{field!update~count} The number of times the
      ODE solver has calculated the effective field.
\item {\bf Sim Time (ns):}\index{simulation~2D!time} The elapsed simulated
      time.
\item {\bf Time Step (ns):}\index{time~step} The interval of simulated
      time spanned by the last step taken in the ODE solver.
\item {\bf Step Size:}\index{step~size} The magnitude of the last step
        taken by the ODE solver as a normalized value.  (This is
        currently the time step in seconds, multiplied by the
        gyromagnetic ratio times the damping coefficient times $M_s$.)
\item {\bf Bx, By, Bz (mT):}\index{field!applied} The $x$, $y$, and $z$
        components of the \hyperrefhtml{nominal applied field}{nominal
        applied field (see Sec.~}{)}{sec:expparams}.
\item {\bf B (mT):} The magnitude of the nominal applied field (always non-negative).
\item {\bf \pipe m x h\pipe:}\index{simulation~2D!mxh}  The maximum of the
        point-wise quantity
        \html{$|\vM\times\vH_{eff}|/M_s^2$}\latex{$\|\vM\times\vH_{\rm
        eff}\|/M_s^2$} over all the spins.  This ``torque'' value is
        used to test convergence to an equilibrium state (and raise
        control point --torque events).
\item {\bf Mx/Ms, My/Ms, Mz/Ms:}\index{magnetization} The $x$,
        $y$, and $z$ components of the average magnetization of the
        magnetically active elements of the simulated part.
\item {\bf Total Energy (J/m${}^3$):}\index{energy!total} The total
        average energy density for the magnetically active elements of
        the simulated part.
\item {\bf Exchange Energy (J/m${}^3$):}\index{energy!exchange} The
        component of the average energy density for the magnetically
        active elements of the simulated part due to exchange
        interactions.
\item {\bf Anisotropy Energy (J/m${}^3$):}\index{energy!anisotropy} The
        component of the average energy density for the magnetically
        active elements of the simulated part due to crystalline and
        surface anisotropies.
\item {\bf Demag Energy (J/m${}^3$):}\index{energy!demag} The component
        of the average energy density for the magnetically active
        elements of the simulated part due to self-demagnetizing fields.
\item {\bf Zeeman Energy (J/m${}^3$):}\index{energy!Zeeman}  The
        component of average energy density for the magnetically active
        elements of the simulated part due to interaction with the
        applied field.
\item {\bf Max Angle:}\index{max~angle}  The maximum angle (in degrees)
        between the magnetization orientation of any pair of neighboring
        spins in the grid.  (The neighborhood of a spin is the same as
        that defined by the exchange energy calculation.)
\end{itemize}
In addition, the solver automatically keeps a log file\index{file!log}
that records the input problem specification and miscellaneous runtime
information.  The name of this log file is \fn{\textit{basename}.log},
where \fn{\textit{basename}} is the ``Base Output Filename'' specified
in the input problem specification.  If this file already exists, then
new entries are appended to the end of the file.

\starssechead{Controls}
The middle section of the user interface window contains a series of
buttons providing user control over the
solver\index{simulation~2D!interactive~control}.  After a problem
description server application has been selected, the \btn{LoadProblem}
button triggers a fetch of a problem description from the selected
server.  The \btn{LoadProblem} button may be selected at any time to
(re-)load a problem description from the currently selected server.
After loading a new problem the solver goes automatically into a paused
state.  (If no problem description server is selected when the
\btn{LoadProblem} button is invoked, nothing will happen.)  The
\btn{Reset} button operates similarly, except that the current problem
specifications are used.

Once a problem is loaded, the solver can be put into any of three
states: run, relax and pause.  Selecting \btn{Relax} puts the solver
into the ``relax'' state, where it runs until a control point is
reached, after which the solver pauses.  If the \btn{Relax} button is
reselected after reaching a control point, then the solver will simply
re-pause immediately.  The \btn{Field+} or \btn{Field--} button must be
invoked to change the applied field state. (Field state schedules are
discussed below.)  The \btn{Run} selection differs in that when a
control point is reached, the solver automatically steps the nominal
applied field to the next value, and continues.  In ``run'' mode the
solver will continue to process until there are no more applied field
states in the problem description.  At any time the \btn{Pause} button
may be selected to pause the solver.  The solver will stay in this state
until the user reselects either \btn{Run} or \btn{Relax}.  The current
state of the solver is indicated in the \btn{Status} line in the center
panel of the user interface window.

The problem description (\MIF~1.x format)\index{file!MIF~1.x}
specifies a fixed \hyperrefhtml{applied field schedule}{applied field
schedule (see Sec.~}{)}{sec:expparams}.  This schedule defines an
ordered list of applied fields, which the solver in ``run'' mode steps
through in sequence.  The \btn{Field--} and \btn{Field+} buttons allow
the user to interactively adjust the applied field sequence.  Each click
on the
\btn{Field+} button advances forward one step through the specified
schedule, while \btn{Field--} reverses that process.  In general, the
step direction is {\em not} related to the magnitude of the applied
field.  Also note that hitting these buttons does not generate a
``ControlPoint'' event.  In particular, if you are manually accelerating
the progress of the solver through a hysteresis loop, and want to send
non-ControlPoint data to a display or archive widget before advancing
the field, then you must use the appropriate ``Interactive Output''
button.

The second row of buttons in the interaction control panel,
\btn{TotalField}, \btn{Magnetization} and \btn{DataTable}, allow the
user to view the current state of the solver at any time.  These buttons
cause the solver to send out data of the corresponding type to all
applications for which the ``Interactive'' schedule button for that
data type has been selected, as discussed in the Outputs section above.

At the far right of the solver controls is the \btn{Exit} button, which
terminates\index{simulation~2D!termination} \app{mmSolve2D}.  Simply
closing the user interface window does not terminate \app{mmSolve2D},
but only closes the user interface window.  To kill the solver the
\btn{Exit} button must be pressed.

\starssechead{Details}
Given a problem description, \app{mmSolve2D} integrates the
Landau-Lifshitz equation\index{ODE!Landau-Lifshitz}~\cite{gilbert1955,landau1935}\\
\begin{equation}
\htmlimage{antialias}
  \frac{d\vM}{dt} = -|\bar{\gamma}|\,\vM\times\vH_{\rm eff}
   - \frac{|\bar{\gamma}|\alpha}{M_s}\,
     \vM\times\left(\vM\times\vH_{\rm eff}\right),
\label{eq:llode}
\end{equation}
where
\latex{
  \begin{displaymath}
    \begin{array}{rcl}
  \vM       && \textnormal{is the pointwise magnetization (A/m),} \\
  \vH_{\rm eff} && \textnormal{is the pointwise effective field (A/m),} \\
  \bar{\gamma}  && \textnormal{is the Landau-Lifshitz gyromagnetic ratio (m/(A$\cdot$s)),} \\
  \alpha    && \textnormal{is the damping coefficient (dimensionless).}
    \end{array}
  \end{displaymath}
} % close latex
% Use BLOCKQUOTE until mmHelp supports tables.
\html{
\begin{quotation}
  $\vM$              is the pointwise magnetization (A/m),\\
  $\vH_{\mbox{eff}}$ is the pointwise effective field (A/m),\\
  \abovemath{\bar{\gamma}} is the Landau-Lifshitz gyromagnetic ratio
                        (m/(A\begin{rawhtml}&#183;\end{rawhtml}s)),\\
  $\alpha$           is the damping coefficient (dimensionless).
\end{quotation}
} % close html
\NONHTMLoutput{(Compare to (\ref{eq:oxsllode}), page~\pageref{eq:oxsllode}.)}%
\HTMLoutput{(See also the discussion of the
\htmlonlyref{Landau-Lifshitz-Gilbert equations}{eq:oxsllode} in the
Oxs documentation.)}

% Note 1: Don't use transparent images, because mmHelp renders them
%   rather slowly.
% Note 2: makeimage (maybe +tabular?) breaks on newer systems, with
%   latex2html v1.71 and v1.68.  OTOH, v1.68 works on older systems,
%   so it is not clear what the problem is.  The error message is:
%          panic: end_shift at /usr/local/bin/latex2html line 11720.
%   This is with Perl v5.8.0.  So, we have to render the table
%   directly.  This is actually preferred with real HTML browsers,
%   but as of this writing (Dec-2004) mmHelp doesn't do tables.
The effective field\index{field!effective} is defined as
\html{\begin{center}}
%\begin{makeimage}
%\htmlimage{no_transparent}
\begin{displaymath}
  \vH_{\rm eff} = -\mu_0^{-1} \frac{\partial E}{\partial\vM}.
\end{displaymath}
%\end{makeimage}
\html{\end{center}}
The average energy density\index{energy!total} $E$ is a function of
$\vM$ specified by Brown's equations \cite{brown1963}, including
anisotropy\index{energy!anisotropy},
exchange\index{energy!exchange}, self-magnetostatic
(demagnetization)\index{energy!demag} and applied
field\index{energy!Zeeman} (Zeeman) terms.

The micromagnetic problem is impressed upon a regular 2D
grid\index{grid} of squares, with 3D magnetization spins positioned at
the centers of the cells.  Note that the constraint that the grid be
composed of square elements takes priority over the requested size of
the grid.  The actual size of the grid used in the computation will be
the nearest integral multiple of the grid's cell size to the requested
size.  It is important when comparing the results from grids with
different cell sizes to account for the possible change in size of the
overall grid.

The anisotropy and applied field energy terms are calculated
assuming constant magnetization in each cell.  The exchange energy is
calculated using the eight-neighbor bilinear interpolation described in
\cite{donahue1997}, with Neumann boundary conditions.  The more common
four-neighbor scheme is available as a compile-time option.  Details can
be found in the source-code file \fn{oommf/app/mmsolve/magelt.cc}.

The self-magnetostatic field is calculated as the convolution of the
magnetization against a kernel that describes the cell to cell
magnetostatic interaction.  The convolution is evaluated using fast
Fourier transform (FFT)\index{FFT} techniques.  Several kernels are
supported; these are selected as part of the problem description in
\hyperrefhtml{\MIF~1.x format}{\MIF~1.x format; for details see Sec.~}{:
Demag specification}{sec:mifdemagspec}.  Each kernel represents a different
interpretation of the discrete magnetization.  The recommended model is
\cd{ConstMag}, which assumes the magnetization is constant in each cell,
and computes the average demagnetization field through the cell using
formulae from \cite{newell1993} and \cite{aharoni1998}.

The Landau-Lifshitz ODE\index{ODE!Landau-Lifshitz} (\ref{eq:llode}) is
integrated using a second order
predictor-corrector\index{ODE!predictor-corrector} technique of the
Adams type.  The right side of (\ref{eq:llode}) at the current and
previous step is extrapolated forward in a linear fashion, and is
integrated across the new time interval to obtain a quadratic prediction
for $\vM$ at the next time step.  At each stage the spins are
renormalized to $M_s$ before evaluating the energy and effective
fields.  The right side of (\ref{eq:llode}) is evaluated at the
predicted $\vM$, which is then combined with the value at the current
step to produce a linear interpolation of $d\vM/dt$ across the new
interval.  This is then integrated to obtain the final estimate of $\vM$
at the new step.  The local (one step) error of this procedure should be
\html{$O(dt^3)$}\latex{$\mathcal{O}(\Delta t^3)$}.

The step is accepted if the total energy of the system decreases, and
the maximum error between the predicted and final $\vM$ is smaller than
a nominal value.  If the step is rejected, then the step size is reduced
and the integration procedure is repeated.  If the step is accepted,
then the error between the predicted and final $\vM$ is used to adjust
the size of the next step.  No fixed ratio between the previous and
current time step is assumed.

A fourth order Runge-Kutta\index{ODE!Runge-Kutta} solver is used to
prime the predictor-corrector solver, and is used as a backup in case
the predictor-corrector fails to find a valid step.  The Runge-Kutta
solver is not selectable as the primary solver at runtime, but may be so
selected at compile time by defining the \cd{RUNGE\_KUTTA\_ODE} macro.
See the file \fn{oommf/app/mmsolve/grid.cc} for all details of the
integration procedure.

For a given applied field, the integration continues until a
\htmlonlyref{control point}{sec:expparams}
\latex{(cf.\ Experiment parameters, Sec.~\ref{sec:expparams})} is
reached.  A control point\index{simulation~2D!control~point} event may
be raised by the ODE iteration count, elapsed simulation time, or by the
maximum value of {\html{$|\vM\times\vH_{\mbox{eff}}|/M_s^2$}}
{\latex{$\|\vM\times\vH_{\mbox{eff}}\|/M_s^2$}} dropping below a
specified control point --torque value (implying an equilibrium state
has been reached).

Depending on the problem size, \app{mmSolve2D} can require a good deal
of working memory.  The exact amount depends on a number of factors, but
a reasonable estimate is 5~MB + 1500~bytes per cell.  For example, a
1~\micrometer~$\times$~1~\micrometer\ part discretized with 5~nm cells will require
approximately 62~MB.

\starssechead{Known Bugs}
\app{mmSolve2D} requires the damping coefficient to be non-zero.
See the \hyperrefhtml{\MIF~1.1 documentation}{\MIF~1.1 documentation
(Sec.~}{)}{sec:mif1format} for details on specifying the damping
coefficient.

When multiple copies of
\app{mmLaunch}\index{mmLaunch~user~interface}\index{application!mmLaunch}
are used, each can have its own interface to a running copy of
\app{mmSolve2D}.  When the interface presented by one copy of
\app{mmLaunch} is used to set the output schedule in \app{mmSolve2D},
those settings are not reflected in the interfaces presented by other
copies of \app{mmLaunch}.  For example, although the first interface
sets a schedule that DataTable data is to be sent to an instance of
\app{mmGraph}\index{application!mmGraph} every third Iteration, there is
no indication of that schedule presented to the user in the second
interface window.  It is unusual to have more than one copy of
\app{mmLaunch} running simultaneously. However, this bug also appears
when one copy of \app{mmLaunch} is used to load a problem and start a
solver, and later a second copy of \app{mmLaunch} is used to monitor the
status of that running solver.

%%%%%%%%%%%%%%%%%%%%%%%%%%%%%%%%%%%%%%%%%%%%%%%%%%%%%%%%%%%%%%%%%%%%%%%%
\section{\OOMMF\ 2D Micromagnetic Solver Batch System}\label{sec:obs}
\index{simulation~2D}\index{application!OOMMF~Batch~System}

The \OOMMF\ Batch System (\OBS) provides a scriptable interface
to the same micromagnetic solver engine used by
\hyperrefhtml{\app{mmSolve2D}}{\app{mmSolve2D}
(Sec.~}{)}{sec:mmsolve2d}\index{application!mmSolve2D}, in the
form of three \Tcl\ applicatons
(\app{batchmaster}, \app{batchslave}, and
\app{batchsolve}) that provide support for complex job scheduling.
All \OBS\ script files are in the \OOMMF\ distribution directory
\fn{app/mmsolve/scripts}.

Unlike much of the \OOMMF\ package, the \OBS\ is meant to be
driven primarily from the command line or shell (batch) script.
\OBS\ applications are launched from the command line using the
\hyperrefhtml{bootstrap application}{bootstrap application
(Ch.~}{)}{sec:cll}.

\subsection{2D Micromagnetic Solver Batch Interface: batchsolve}%
\label{sec:batchsolve}

\starsssechead{Overview}
The application
\app{batchsolve}\index{application!batchsolve} provides a simple
command line interface to the \OOMMF\ 2D micromagnetic solver engine.

\starsssechead{Launching}
The application \app{batchsolve} is launched by the command line:
\begin{verbatim}
tclsh oommf.tcl batchsolve [standard options]
   [-end_exit <0|1>] [-end_paused] [-interface <0|1>] \
   [-restart <0|1>] [-start_paused] [file]
\end{verbatim}
where
\begin{description}
\item[\optkey{-end\_exit \boa 0\pipe 1\bca}]
  Whether or not to explicitly call exit at bottom of \fn{batchsolve.tcl}.
  When launched from the command line, the default is to exit after
  solving the problem in \cd{file}.  When sourced into another script,
  like \fn{batchslave.tcl}, the default is to wait for the caller script
  to provide further instructions.
\item[\optkey{-interface \boa 0\pipe 1\bca}]
  Whether to register with the account service
  directory\index{account~service~directory} application, so
  that \hyperrefhtml{\app{mmLaunch}}{\app{mmLaunch}
  (Ch.~}{)}{sec:mmlaunch}\index{application!mmLaunch}, can provide
  an interactive interface.  Default = 1 (do register), which will
  automatically start account service directory and
  host service directory applications as necessary.
\item[\optkey{-start\_paused}]
  Pause solver after loading problem.
\item[\optkey{-end\_paused}]
  Pause solver and enter event loop at bottom of {\fn{batchsolve.tcl}}
  rather than just falling off the end (the effect of which will
  depend on whether or not \Tk\ is loaded).
\item[\optkey{-restart \boa 0\pipe 1\bca\index{simulation~2D!restarting}}]
  Determines solver behavior when a new problem is loaded.  If 1, then
  the solver will look for \fn{\textit{basename}.log}\index{file!log}
  and \fn{\textit{basename*}.omf}\index{file!magnetization} files to
  restart a previous run from the last
  saved state (where \fn{\textit{basename}} is the ``Base Output
  Filename'' specified in the input problem specification).  If these
  files cannot be found, then a warning is issued and the solver falls
  back to the default behavior (equivalent to \cd{-restart 0}) of
  starting the problem from scratch.  The specified \cd{-restart}
  setting holds for \textbf{all} problems fed to the solver, not just
  the first.
\item[\optkey{file}]
  Immediately load and run the specified \MIF~1.x
  file\index{file!MIF~1.x}.
\end{description}

The input file {\bf\fn{file}} should contain a
\hyperrefhtml{Micromagnetic Input Format}{Micromagnetic Input
Format (Ch.~}{)}{sec:mifformat} 1.x problem
description\index{file!MIF~1.x}, such as produced by
\hyperrefhtml{\app{mmProbEd}}{\app{mmProbEd}
(Ch.~}{)}{sec:mmprobed}\index{application!mmProbEd}.  The batch solver
searches several directories for this file, including the current
working directory\index{working~directory}, the \fn{data} and
\fn{scripts} subdirectories, and parallel directories relative to the
directories \fn{app/mmsolve} and
\fn{app/mmpe} in the \OOMMF\ distribution.  Refer to the
\cd{mif\_path} variable in \fn{batchsolve.tcl} for the complete list.

If \cd{-interface} is set to 1 (enabled), \app{batchsolve} registers
with the account service directory\index{account~service~directory}
application, and \app{mmLaunch} will be able to provide an interactive
interface.  Using this interface, \app{batchsolve} may be controlled in
a manner similar to \hyperrefhtml{\app{mmSolve2D}}{\app{mmSolve2D}
(Sec.~}{)}{sec:mmsolve2d}\index{application!mmSolve2D}.  The
interface\index{simulation~2D!interactive~control} allows you to pause,
un-pause, and terminate\index{simulation~2D!termination} the current
simulation, as well as to attach data display applications to monitor
the solver's progress.  If more interactive control is needed,
\app{mmSolve2D} should be used.

If \cd{-interface} is 0 (disabled), \app{batchsolve} does not register,
leaving it without an interface, unless it is sourced into another
script (e.g., \fn{batchslave.tcl}\index{application!batchslave}) that
arranges for an interface on the behalf of \app{batchsolve}.

Use the {\bf\verb|-start_paused|} switch to monitor the progress of
\app{batchsolve} from the very start of a simulation.  With this
switch the solver will be paused immediately after loading the
specified \MIF\ file, so you can bring up the interactive interface
and connect display applications before the simulation begins.  Start the
simulation by selecting the \btn{Run} command from the interactive
interface.  This option cannot be used if \cd{-interface} is disabled.

The {\bf\verb|-end_paused|} switch insures that the solver does
not automatically terminate after completing the specified
simulation.  This is not generally useful, but may find application
when \app{batchsolve} is called from inside a \Tcl -only wrapper
script.

Note on \Tk\ dependence: If a problem is loaded that uses a
{\hyperrefhtml{bitmap mask file}{bitmap mask file
(Sec.~}{)}{sec:partgeometry}}\index{file!mask}\index{file!bitmap}, and
if that mask file is not in the PPM P3 (text) format, then
\app{batchsolve} will launch {\hyperrefhtml{any2ppm}{\app{any2ppm}
(Sec.~}{)}{sec:any2ppm}}\index{application!any2ppm} to convert it into
the PPM P3 format.  Since \app{any2ppm}
requires\index{requirement!Tk}\index{requirement!display} \Tk, at the
time the mask file is read a valid display must be available.  See the
\app{any2ppm} documentation for details.

\starsssechead{Output}
The output may be changed by a \Tcl\ \hyperrefhtml{wrapper
script}{wrapper script (see Sec.~}{)}{sec:batchsolvepi}, but the default
output behavior of \app{batchsolve} is to write tabular text data and
the magnetization state at the control point for each applied field
step.  The tabular data are appended to the file \fn{{\em
basename}.odt}\index{file!data~table}, where {\em basename} is the
``Base Output Filename'' specified in the input
\MIF~1.x file\index{file!MIF~1.x}.  See the routine \cd{GetTextData} in
\fn{batchsolve.tcl} for details, but at present the output consists of
the solver iteration count\index{iteration}, nominal applied field
\vB\index{field!applied}, reduced average
magnetization \vm\index{magnetization}, and total
energy\index{energy!total}.  This output is in the \ODT\ file format.

The magnetization data are written to a series of \OVF\ (\OOMMF\ Vector
Field) files\index{file!vector~field},
\fn{\textit{basename}.field\textit{nnnn}.omf}, where \fn{\textit{nnnn}}
starts at \fn{0000} and is incremented at each applied
field\index{field!applied} step.  (The ASCII text header inside each
file records the nominal applied field at that step.)  These files are
viewable using \hyperrefhtml{\app{mmDisp}}{\app{mmDisp}
(Ch.~}{)}{sec:mmdisp}\index{application!mmDisp}.

The solver also automatically appends the input problem specification
and miscellaneous runtime information to the log file
\fn{{\em basename}.log}\index{file!log}.


\starsssechead{Programmer's interface}\label{sec:batchsolvepi}
In addition to directly launching \app{batchsolve} from the command
line, \fn{batchsolve.tcl}  may also be sourced into another \Tcl\ script
that provides additional control structures.  Within the scheduling
system of \OBS,
\fn{batchsolve.tcl} is sourced into \app{batchslave}, which provides
additional control structures that support scheduling control by
\app{batchmaster}.
There are several variables and routines
inside \fn{batchsolve.tcl} that may be accessed and redefined from such
a wrapper script to provide enhanced functionality.

\starsssechead{Global variables}
\begin{description}
\item[\cd{\bf mif}] A \Tcl\ handle to a global \cd{mms\_mif}
  object holding the problem description defined by the input
  \MIF~1.x file\index{file!MIF~1.x}.
\item[\cd{\bf solver}] A \Tcl\ handle to the \cd{mms\_solver} object.
\item[\cd{\bf search\_path}] Directory search path used by the
\ptlink{\cd{FindFile} proc}{PTbatchsolveFindFile}\NONHTMLoutput{ (see below)}.

\end{description}
Refer to the source code and sample scripts for details on manipulation
of these variables.

\starsssechead{Batchsolve procs}\label{sec:batchsolveprocs}
The following \Tcl\ procedures are designed for external use and/or
redefinition:
\begin{description}
\item[\cd{\bf SolverTaskInit}]
   Called at the start of each task.
\item[\cd{\bf BatchTaskIterationCallback}]
   Called after each iteration in the simulation.
\item[\cd{\bf BatchTaskRelaxCallback}]
   Called at each control point reached in the simulation.
\item[\cd{\bf SolverTaskCleanup}]
   Called at the conclusion of each task.
\pttarget{PTbatchsolveFindFile}
\item[\cd{\bf FindFile}]
   Searches the directories specified by the global variable
   \cd{search\_path} for a specified file.  The default
   \cd{SolverTaskInit} proc uses this routine to locate the requested
   input \MIF\ file.
\end{description}
\cd{SolverTaskInit} and \cd{SolverTaskCleanup} accept an arbitrary
argument list (\cd{args}), which is copied over from the \cd{args}
argument to the \cd{BatchTaskRun} and \cd{BatchTaskLaunch} procs in
\fn{batchsolve.tcl}.  Typically one copies the default procs (as needed)
into a \htmlonlyref{task script}{sec:batchschedex}, and makes appropriate
modifications.  You may (re-)define these procs either before or after
sourcing \fn{batchsolve.tcl}\index{application!batchsolve}.
\latex{See Sec.~\ref{sec:batchschedex} for example scripts.}

\subsection{2D Micromagnetic Solver Batch Scheduling System}%
\label{sec:mmsolveBSS}

\starsssechead{Overview}
The \OBS\ supports complex scheduling of multiple batch jobs
with two applications, \app{batchmaster} and \app{batchslave}.
The user launches \app{batchmaster} and provides it with
a task script\index{task~script}.  The task script is a
\Tcl\ script that describes the set of tasks for \app{batchmaster}
to accomplish.  The work is actually done by instances of
\app{batchslave} that are launched by \app{batchmaster}.
The task script may be
modeled after the included {\fn{simpletask.tcl}} or {\fn{multitask.tcl}}
\hyperrefhtml{sample scripts}{sample scripts
(Sec.~}{)}{sec:batchschedex}.

The \OBS\ has been designed to control multiple sequential and
concurrent micromagnetic simulations, but
\app{batchmaster} and \app{batchslave} are completely general
and may be used to schedule other types of jobs as well.

\subsubsection{Master Scheduling Control: batchmaster}\par
The application \app{batchmaster}\index{simulation~2D!scheduling} is
launched by the command line:
\begin{verbatim}
tclsh oommf.tcl batchmaster [standard options] task_script \
      [host [port]]
\end{verbatim}
\begin{description}
\item[{\tt\bf task\_script}]
  is the user defined task (job) definition \Tcl\ script,
\item[{\tt\bf host}]
  specifies the network address for the master to use (default is {\em
  localhost}),
\item[{\tt\bf port}]
  is the port address for the master (default is {\em 0}, which
  selects an arbitrary open port).
\end{description}

When \app{batchmaster}\index{application!batchmaster} is run, it
sources the task script.  \Tcl\ commands in the task script
should modify the global object \cd{\$TaskInfo}
to inform the master what tasks to perform and
optionally how to launch slaves to perform those tasks.
The easiest way to create a task script is to modify one of the
\hyperrefhtml{included example scripts}{example scripts in
Sec.~}{}{sec:batchschedex}.  More detailed instructions are in
\html{the }\hyperrefhtml{Batch task
scripts}{Sec.~}{}{sec:batchschedtask}\html{ section}.

After sourcing the task script, \app{batchmaster} launches all the
specified slaves, initializes each with a slave initialization script,
and then feeds tasks sequentially from the task list to the slaves.
When a slave completes a task it reports back to the master and is given
the next unclaimed task.  If there are no more tasks, the slave is shut
down.  When all the tasks are complete, the master prints a summary of
the tasks and exits.

When the task script requests the launching and controlling of jobs off
the local machine, with slaves running on remote machines, then the
command line argument \fn{host} {\bf must} be set to the local machine's
network name, and the \cd{\$TaskInfo} methods \cd{AppendSlave} and
\cd{ModifyHostList} will need to be called from inside the task script.
Furthermore, \OOMMF\ does not currently supply any methods for launching
jobs on remote machines, so a task script which requests the launching
of jobs on remote machines requires a working
\verb+ssh+\index{application!ssh} command or
equivalent\index{requirement!ssh}.
\hyperrefhtml{(Details.)}{See Sec.~}{ for
details.}{sec:batchschedtask}

\subsubsection{Task Control: batchslave}\par
The application \app{batchslave} may be launched by the command line:
\begin{verbatim}
tclsh oommf.tcl batchslave [standard options] \
   host port id password [auxscript [arg ...]]
\end{verbatim}
\begin{description}
\item[{\tt\bf host, port}]
  Host and port at which to contact the master to serve.
\item[{\tt\bf id, password}]
  ID and password to send to the master for identification.
\item[{\tt\bf auxscript arg ...}]
  The name of an optional script to source (which actually performs the
  task the slave is assigned), and any arguments it needs.
\end{description}

In normal operation, the user does not launch
\app{batchslave}.  Instead, instances of \app{batchslave} are
launched by \app{batchmaster} as instructed by a task script.
Although \app{batchmaster} may launch any slaves requested
by its task script, by default it launches instances of
\app{batchslave}.

The function of \app{batchslave} is to make a connection to
a master program, source the \cd{auxscript} and pass it the
list of arguments \cd{aux\_arg ...}.  Then it receives commands
from the master, and evaluates them, making use of the
facilities provided by \cd{auxscript}.  Each command is typically a
long-running one, such as solving a complete micromagnetic problem.
When each command is complete, the \app{batchslave} reports back to
its master program, asking for the next command.  When the master
program has no more commands \app{batchslave} terminates.

Inside \app{batchmaster}, each instance of \app{batchslave} is
launched by evaluating a \Tcl\ command.  This command is called
the spawn command, and it may be redefined by the task script
in order to completely control which slave applications are launched
and how they are launched.  When \app{batchslave} is to be launched,
the spawn command might be:
\begin{rawhtml}
<BLOCKQUOTE>
\end{rawhtml}
%begin<latexonly>
\begin{quote}
%end<latexonly>
\begin{verbatim}
exec tclsh oommf.tcl batchslave -tk 0 -- $server(host) $server(port) \
   $slaveid $passwd batchsolve.tcl -restart 1 &
\end{verbatim}
%begin<latexonly>
\end{quote}
%end<latexonly>
\begin{rawhtml}
</BLOCKQUOTE>
\end{rawhtml}
The \Tcl\ command \cd{exec} is used to launch subprocesses.  When
the last argument to \cd{exec} is \cd{\&}, the subprocess runs in
the background.  The rest of the spawn command should look familiar
as the command line syntax for launching \app{batchslave}.

The example spawn command above cannot be completely provided by
the task script, however, because parts of it are only known
by \app{batchmaster}.  Because of this, the task script should
define the spawn command using ``percent variables'' which are
substituted by \app{batchmaster}.  Continuing the example, the task
script provides the spawn command:
\begin{rawhtml}
<BLOCKQUOTE>
\end{rawhtml}
%begin<latexonly>
\begin{quote}
%end<latexonly>
\begin{verbatim}
exec %tclsh %oommf batchslave -tk 0 %connect_info \
   batchsolve.tcl -restart 1
\end{verbatim}
%begin<latexonly>
\end{quote}
%end<latexonly>
\begin{rawhtml}
</BLOCKQUOTE>
\end{rawhtml}
\app{batchmaster} replaces \cd{\%tclsh} with the path to \fn{tclsh},
and \cd{\%oommf} with the path to the \OOMMF\ bootstrap application.
It also replaces \cd{\%connect\_info} with the five arguments from \verb+--+
through \cd{\$password} that provide \app{batchslave}
the hostname and port where \app{batchmaster} is waiting for
it to report to, and the ID and password it should pass back.
In this example, the task script instructs \app{batchslave} to source the
file \fn{batchsolve.tcl} and pass it the arguments \cd{-restart 1}.
Finally, \app{batchmaster} always appends the argument \cd{\&} to
the spawn command so that all slave applications are launched in the
background.

The communication protocol\index{communication~protocol} between
\app{batchmaster} and \app{batchslave} is evolving and is not
described here.  Check the source code for the latest details.

\subsubsection{Batch Task Scripts}\label{sec:batchschedtask}\par
The application \app{batchmaster}
creates an instance of a \cd{BatchTaskObj} object with
the name \cd{\$TaskInfo}.  The task script\index{task~script} uses
method calls to this object to set up tasks to be performed.  The only
required call is to the \cd{AppendTask} method, e.g.,
\begin{rawhtml}
<BLOCKQUOTE>
\end{rawhtml}
%begin<latexonly>
\begin{quote}
%end<latexonly>
\begin{verbatim}
$TaskInfo AppendTask A "BatchTaskRun taskA.mif"
\end{verbatim}
%begin<latexonly>
\end{quote}
%end<latexonly>
\begin{rawhtml}
</BLOCKQUOTE>
\end{rawhtml}
This method expects two arguments, a label for the task (here ``A'') and
a script to accomplish the task.
The script will be passed across a
network socket\index{network~socket} from
\app{batchmaster} to a slave application, and
then the script will be interpreted by the slave.  In particular, keep
in mind that the file system seen by the script will be that of the
machine on which the slave process is running.

This example uses the default \fn{batchsolve.tcl} procs to run the
simulation defined by the \fn{taskA.mif} \MIF~1.x
file\index{file!MIF~1.x}.  If you want to make changes to the \MIF\
problem specifications on the fly, you will need to modify the default
procs.  This is done by creating a slave initialization script, via the
call
\begin{rawhtml}
<BLOCKQUOTE>
\end{rawhtml}
%begin<latexonly>
\begin{quote}
%end<latexonly>
\begin{verbatim}
$TaskInfo SetSlaveInitScript { <insert script here> }
\end{verbatim}
%begin<latexonly>
\end{quote}
%end<latexonly>
\begin{rawhtml}
</BLOCKQUOTE>
\end{rawhtml}
The slave initialization script does global initializations, and also
usually redefines the \cd{SolverTaskInit} proc; optionally the
\cd{BatchTaskIterationCallback}, \cd{BatchTaskRelaxCallback} and
\cd{SolverTaskCleanup} procs may be redefined as well.  At the start of
each task \cd{SolverTaskInit} is called by \cd{BatchTaskRun} (in
\fn{batchsolve.tcl}), after each iteration
\cd{BatchTaskIterationCallback} is executed, at each control
point\index{simulation~2D!control~point} \cd{BatchTaskRelaxCallback} is
run, and at the end of each task \cd{SolverTaskCleanup} is called.
\cd{SolverTaskInit} and \cd{SolverTaskCleanup} are passed the arguments
that were passed to \cd{BatchTaskRun}.  A simple \cd{SolverTaskInit}
proc could be
\begin{rawhtml}
<BLOCKQUOTE>
\end{rawhtml}
%begin<latexonly>
\begin{quote}
%end<latexonly>
\begin{verbatim}
proc SolverTaskInit { args } {
   global mif basename outtextfile
   set A [lindex $args 0]
   set outbasename "$basename-A$A"
   $mif SetA $A
   $mif SetOutBaseName $outbasename
   set outtextfile [open "$outbasename.odt" "a+"]
   puts $outtextfile [GetTextData header \
         "Run on $basename.mif, with A=[$mif GetA]"]
}
\end{verbatim}
%begin<latexonly>
\end{quote}
%end<latexonly>
\begin{rawhtml}
</BLOCKQUOTE>
\end{rawhtml}
This proc receives the exchange constant \cd{A}
for this task on the argument list, and makes use of the global
variables \cd{mif} and \cd{basename}.  (Both should be initialized in
the slave initialization script outside the \cd{SolverTaskInit} proc.)
It then stores the requested value of \cd{A} in the
\cd{mif}\index{file!MIF~1.x} object, sets up the base filename to use for
output, and opens a text file to which tabular
data\index{file!data~table} will be appended.  The handle to this text
file is stored in the global \cd{outtextfile}, which is closed by the
default \cd{SolverTaskCleanup} proc.  A corresponding task script could
be
\begin{rawhtml}
<BLOCKQUOTE>
\end{rawhtml}
%begin<latexonly>
\begin{quote}
%end<latexonly>
\begin{verbatim}
$TaskInfo AppendTask "A=13e-12 J/m" "BatchTaskRun 13e-12"
\end{verbatim}
%begin<latexonly>
\end{quote}
%end<latexonly>
\begin{rawhtml}
</BLOCKQUOTE>
\end{rawhtml}
which runs a simulation with \cd{A} set to
\latex{$13\times 10^{-12}$~J/m.}\html{13e-12 J/m.}
This example is taken from the \fn{multitask.tcl}
\hyperrefhtml{sample script}{script in Sec.~}{}{sec:batchschedex}.  (For
commands accepted by \cd{mif} objects, see the file \fn{mmsinit.cc}.
Another object than can be gainfully manipulated is \cd{solver}, which
is defined in \fn{solver.tcl}.)

If you want to run more than one task at a time, then the
\cd{\$TaskInfo} method \cd{AppendSlave} will have to be invoked.  This
takes the form
\begin{rawhtml}
<BLOCKQUOTE>
\end{rawhtml}
%begin<latexonly>
\begin{quote}
%end<latexonly>
\begin{verbatim}
$TaskInfo AppendSlave <spawn count> <spawn command>
\end{verbatim}
%begin<latexonly>
\end{quote}
%end<latexonly>
\begin{rawhtml}
</BLOCKQUOTE>
\end{rawhtml}
where \cd{<spawn command>} is the command to launch the slave
process, and \cd{<spawn count>} is the number of slaves to launch
with this command.  (Typically \cd{<spawn count>} should not be
larger than the number of processors on the target system.)  The default
value for this item (which gets overwritten with the first call to
\cd{\$TaskInfo AppendSlave}) is
\begin{rawhtml}
<BLOCKQUOTE>
\end{rawhtml}
%begin<latexonly>
\begin{quote}
%end<latexonly>
\begin{verbatim}
 1 {Oc_Application Exec batchslave -tk 0 %connect_info batchsolve.tcl}
\end{verbatim}
%begin<latexonly>
\end{quote}
%end<latexonly>
\begin{rawhtml}
</BLOCKQUOTE>
\end{rawhtml}
The \Tcl\ command \cd{Oc\_Application Exec} is supplied by \OOMMF\
and provides access to the same application-launching capability
that is used by the \OOMMF\
\hyperrefhtml{bootstrap application}{bootstrap application
(Ch.~}{)}{sec:cll}.  Using a \cd{<spawn command>} of
\cd{Oc\_Application Exec} instead of \cd{exec \%tclsh \%oommf}
saves the spawning of an additional process.
The default \cd{<spawn command>}
launches the \app{batchslave}
application, with connection information provided by \app{batchmaster}, and
using the \cd{auxscript} \fn{batchsolve.tcl}.

Before evaluating the \cd{<spawn command>}, \app{batchmaster}
applies several percent-style substitutions useful in slave
launch scripts: \%tclsh, \%oommf, \%connect\_info, \%oommf\_root, and
\%\%.  The first is the \Tcl\ shell to use, the second is an absolute
path to the \OOMMF\ bootstrap program on the master machine, the third
is connection information needed by the \fn{batchslave} application, the
fourth is the path to the \OOMMF\ root directory on the master machine,
and the last is interpreted as a single percent.
\app{batchmaster} automatically appends the argument
\cd{\&} to the
\cd{<spawn command>} so that the slave applications
are launched in the background.

To launch \app{batchslave} on a remote host, use \fn{ssh}\index{application!ssh}
in the spawn command, e.g.,
\begin{rawhtml}
<BLOCKQUOTE>
\end{rawhtml}
%begin<latexonly>
\begin{quote}
%end<latexonly>
\begin{verbatim}
$TaskInfo AppendSlave 1 {exec ssh foo tclsh oommf/oommf.tcl \
      batchslave -tk 0 %connect_info batchsolve.tcl}
\end{verbatim}
%begin<latexonly>
\end{quote}
%end<latexonly>
\begin{rawhtml}
</BLOCKQUOTE>
\end{rawhtml}
This example assumes \fn{tclsh} is in the execution path on the remote
machine \fn{foo}, and \OOMMF\ is installed off of your home directory.
In addition, you will have to add the machine \fn{foo} to the host
connect list with
\begin{rawhtml}
<BLOCKQUOTE>
\end{rawhtml}
%begin<latexonly>
\begin{quote}
%end<latexonly>
\begin{verbatim}
$TaskInfo ModifyHostList +foo
\end{verbatim}
%begin<latexonly>
\end{quote}
%end<latexonly>
\begin{rawhtml}
</BLOCKQUOTE>
\end{rawhtml}
and \fn{batchmaster} must be run with the network interface specified
as the server host (instead of the default \fn{localhost}), e.g.,
\begin{rawhtml}
<BLOCKQUOTE>
\end{rawhtml}
%begin<latexonly>
\begin{quote}
%end<latexonly>
\begin{verbatim}
tclsh oommf.tcl batchmaster multitask.tcl bar
\end{verbatim}
%begin<latexonly>
\end{quote}
%end<latexonly>
\begin{rawhtml}
</BLOCKQUOTE>
\end{rawhtml}
where \fn{bar} is the name of the local machine.

This may seem a bit complicated, but the examples in the
next section should make things clearer.

\subsubsection{Sample task scripts}\label{sec:batchschedex}\par
The
\pttarget{PTbatchschedsimpletask}
\hyperrefhtml{first sample task script}{first sample task script
(Fig.~}{)}{fig:batchschedsimpletask} is a simple example that runs the
3 micromagnetic simulations described by the \MIF~1.x files
\fn{taskA.mif}, \fn{taskB.mif} and \fn{taskC.mif}\index{file!MIF~1.x}.  It
is launched with the command
\begin{rawhtml}
<BLOCKQUOTE>
\end{rawhtml}
%begin<latexonly>
\begin{quote}
%end<latexonly>
\begin{verbatim}
tclsh oommf.tcl batchmaster simpletask.tcl
\end{verbatim}
%begin<latexonly>
\end{quote}
%end<latexonly>
\begin{rawhtml}
</BLOCKQUOTE>
\end{rawhtml}
This example uses the default slave launch script, so a single slave is
launched on the current machine, and the 3 simulations will be run
sequentially.  Also, no slave initialization script is given, so the
default procs in \fn{batchsolve.tcl} are used.  Output will be magnetization
states\index{file!magnetization} and tabular data\index{file!data~table}
at each control point\index{simulation~2D!control~point}, stored in
files on the local machine with base names as specified in the \MIF\
files.

\begin{codelisting}{f}{fig:batchschedsimpletask}{Sample task script
  \fn{simpletask.tcl}.}{PTbatchschedsimpletask}{hyperlink}{hyperlink}
\begin{verbatim}
# FILE: simpletask.tcl
#
# This is a sample batch task file.  Usage example:
#
#   tclsh oommf.tcl batchmaster simpletask.tcl
#
# Form task list
$TaskInfo AppendTask A "BatchTaskRun taskA.mif"
$TaskInfo AppendTask B "BatchTaskRun taskB.mif"
$TaskInfo AppendTask C "BatchTaskRun taskC.mif"
\end{verbatim}
\end{codelisting}

\pttarget{PTbatchoctrltask}
\hyperrefhtml{second sample task script}{second sample task script
(Fig.~}{)}{fig:batchoctrltask} builds on the previous example by
defining \cd{BatchTaskIterationCallback} and
\cd{BatchTaskRelaxCallback} procedures in the slave init script.
The first set up to write tabular data every 10 iterations, while the
second writes tabular data on each control point event.  The data is
written to the output file specified by the \cd{Base Output Filename}
entry in the input \MIF\ files.  Note that there is no magnetization
vector field output in this example.  This task script is launched the
same way as the previous example:
\begin{rawhtml}
<BLOCKQUOTE>
\end{rawhtml}
%begin<latexonly>
\begin{quote}
%end<latexonly>
\begin{verbatim}
tclsh oommf.tcl batchmaster octrltask.tcl
\end{verbatim}
%begin<latexonly>
\end{quote}
%end<latexonly>
\begin{rawhtml}
</BLOCKQUOTE>
\end{rawhtml}

\begin{codelisting}{f}{fig:batchoctrltask}{Task script with
  iteration output \fn{octrltask.tcl}.}{PTbatchoctrltask}{hyperlink}
\begin{verbatim}
# FILE: octrltask.tcl
#
# This is a sample batch task file, with expanded output control.
# Usage example:
#
#        tclsh oommf.tcl batchmaster octrltask.tcl
#
# "Every" output selection count
set SKIP_COUNT 10

# Initialize solver. This is run at global scope
set init_script {
    # Text output routine
    proc MyTextOutput {} {
        global outtextfile
        puts $outtextfile [GetTextData data]
        flush $outtextfile
    }
    # Change control point output
    proc BatchTaskRelaxCallback {} {
        MyTextOutput
    }
    # Add output on iteration events
    proc BatchTaskIterationCallback {} {
        global solver
        set count [$solver GetODEStepCount]
        if { ($count % __SKIP_COUNT__) == 0 } { MyTextOutput }
    }
}

# Substitute $SKIP_COUNT in for __SKIP_COUNT__ in above "init_script"
regsub -all -- __SKIP_COUNT__ $init_script $SKIP_COUNT init_script
$TaskInfo SetSlaveInitScript $init_script

# Form task list
$TaskInfo AppendTask A "BatchTaskRun taskA.mif"
$TaskInfo AppendTask B "BatchTaskRun taskB.mif"
$TaskInfo AppendTask C "BatchTaskRun taskC.mif"
\end{verbatim}
\end{codelisting}

The
\pttarget{PTbatchschedmultitask}
 \hyperrefhtml{third task script}{third task script
(Fig.~}{)}{fig:batchschedmultitask} is a more complicated example
running concurrent processes\index{simulation~2D!scheduling} on two
machines.  This script should be run with the command
\begin{rawhtml}
<BLOCKQUOTE>
\end{rawhtml}
%begin<latexonly>
\begin{quote}
%end<latexonly>
\begin{verbatim}
tclsh oommf.tcl batchmaster multitask.tcl bar
\end{verbatim}
%begin<latexonly>
\end{quote}
%end<latexonly>
\begin{rawhtml}
</BLOCKQUOTE>
\end{rawhtml}
where \fn{bar} is the name of the local machine.

Near the top of the \fn{multitask.tcl} script several \Tcl\ variables
(\cd{RMT\_MACHINE} through \cd{A\_list}) are defined; these are used
farther down in the script.  The remote machine is specified as
\fn{foo}, which is used in the \cd{\$TaskInfo AppendSlave} and
\cd{\$TaskInfo ModifyHostList} commands.

There are two \cd{AppendSlave} commands, one to run two slaves on the
local machine, and one to run a single slave on the remote machine
(\fn{foo}).  The latter changes to a specified
working directory\index{working~directory}  before
launching the \fn{batchslave} application on the remote machine.  (For
this to work you must have \cd{ssh} configured properly\index{application!ssh}.)

Below this the slave initialization script is defined.  The \Tcl\
\cd{regsub} command is used to place the task script defined value of
\cd{BASEMIF} into the init script template.  The init script is run on
the slave when the slave is first brought up.  It first reads the base
\MIF\ file into a newly created \cd{mms\_mif} instance.  (The \MIF\ file
needs to be accessible by the slave process, irrespective of which
machine it is running on.)  Then replacement \cd{SolverTaskInit} and
\cd{SolverTaskCleanup} procs are defined.  The new \cd{SolverTaskInit}
interprets its first argument as a value for the exchange constant
\cd{A}.  Note that this is different from the default
\cd{SolverTaskInit} proc, which interprets its first argument as the
name of a \MIF~1.x file\index{file!MIF~1.x} to load.  With this task
script, a \MIF\ file is read once when the slave is brought up, and then
each task redefines only the value of \cd{A} for the simulation (and
corresponding changes to the output filenames and data table header).

Finally, the \Tcl\ loop structure
\begin{rawhtml}
<BLOCKQUOTE>
\end{rawhtml}
%begin<latexonly>
\begin{quote}
%end<latexonly>
\begin{verbatim}
foreach A $A_list {
    $TaskInfo AppendTask "A=$A" "BatchTaskRun $A"
}
\end{verbatim}
%begin<latexonly>
\end{quote}
%end<latexonly>
\begin{rawhtml}
</BLOCKQUOTE>
\end{rawhtml}
is used to build up a task list consisting of one task for each value
of \cd{A} in \cd{A\_list} (defined at the top of the task script).  For
example, the first value of \cd{A} is 10e-13, so the first task
will have the label \cd{A=10e-13} and the corresponding script is
\cd{BatchTaskRun 10e-13}.  The value 10e-13 is passed on by
\cd{BatchTaskRun} to the \cd{SolverTaskInit} proc, which has been
redefined to process this argument as the value for \cd{A}, as
described above.

There are 6 tasks in all, and 3 slave processes, so the first three
tasks will run concurrently in the 3 slaves.  As each slave finishes
it will be given the next task, until all the tasks are complete.

\begin{codelisting}{p}{fig:batchschedmultitask}{Advanced sample task
  script \fn{multitask.tcl}.}{PTbatchschedmultitask}{hyperlink}
\begin{verbatim}
# FILE: multitask.tcl
#
# This is a sample batch task file.  Usage example:
#
#   tclsh oommf.tcl batchmaster multitask.tcl hostname [port]
#
# Task script configuration
set RMT_MACHINE   foo
set RMT_TCLSH      tclsh
set RMT_OOMMF      "/path/to/oommf/oommf.tcl"
set RMT_WORK_DIR   "/path/to/oommf/app/mmsolve/data"
set BASEMIF taskA
set A_list { 10e-13 10e-14 10e-15 10e-16 10e-17 10e-18 }

# Slave launch commands
$TaskInfo ModifyHostList +$RMT_MACHINE
$TaskInfo AppendSlave 2 "exec %tclsh %oommf batchslave -tk 0 \
        %connect_info batchsolve.tcl"
$TaskInfo AppendSlave 1 "exec ssh $RMT_MACHINE \
        cd $RMT_WORK_DIR \\\;\
        $RMT_TCLSH $RMT_OOMMF batchslave -tk 0 %connect_info batchsolve.tcl"

# Slave initialization script (with batchsolve.tcl proc
# redefinitions)
set init_script {
    # Initialize solver. This is run at global scope
    set basename __BASEMIF__      ;# Base mif filename (global)
    mms_mif New mif
    $mif Read [FindFile ${basename}.mif]
    # Redefine TaskInit and TaskCleanup proc's
    proc SolverTaskInit { args } {
        global mif outtextfile basename
        set A [lindex $args 0]
        set outbasename "$basename-A$A"
        $mif SetA $A
        $mif SetOutBaseName $outbasename
        set outtextfile [open "$outbasename.odt" "a+"]
        puts $outtextfile [GetTextData header \
                "Run on $basename.mif, with A=[$mif GetA]"]
        flush $outtextfile
    }
    proc SolverTaskCleanup { args } {
        global outtextfile
        close $outtextfile
    }
}
# Substitute $BASEMIF in for __BASEMIF__ in above script
regsub -all -- __BASEMIF__ $init_script $BASEMIF init_script
$TaskInfo SetSlaveInitScript $init_script

# Create task list
foreach A $A_list {
    $TaskInfo AppendTask "A=$A" "BatchTaskRun $A"
}
\end{verbatim}
\end{codelisting}

%begin{latexonly}
\setcounter{secnumdepth}{\value{msoldsecnumdepth}}
%end{latexonly}

\chapter{Data Table Display: mmDataTable}\label{sec:mmdatatable}%
\index{application!mmDataTable}

\begin{center}
\includepic{mmdatatable-ss}{mmDataTable Screen Shot}
\end{center}

\starsechead{Overview}
The application \app{mmDataTable} provides a data display service to its
client applications.  It accepts data from clients which are displayed in
a tabular format in a top-level window.  Its typical use is to display
the evolving values of quantities computed by micromagnetic solver
programs.

\starsechead{Launching}
{\bf mmDataTable} may be started either by selecting the
\btn{mmDataTable} button on \htmlonlyref{{\bf mmLaunch}}{sec:mmlaunch},
or from the command line via
\begin{verbatim}
tclsh oommf.tcl mmDataTable [standard options] [-net <0|1>]
\end{verbatim}

\begin{description}
\item[\optkey{-net \boa 0\pipe 1\bca}]
  Disable or enable a server which allows the data displayed by
  \app{mmDataTable} to be updated by another application.
  By default, the server is enabled.  When the server is disabled,
  \app{mmProbEd} is only useful if it is embedded in another application.
\end{description}

\starsechead{Inputs}
The client application(s) that send data to {\bf mmDataTable} for 
display control the flow of data.  The user, interacting with
the {\bf mmDataTable} window, controls how the data is displayed.
Upon launch, {\bf mmDataTable} displays only a menubar.  Upon user
request, a display window below the menubar displays data values.

Each message from a client contains a list of
(name, value, units) triples containing data for display.  
For example, one element in the list might be 
\cd{{\rm\{}Magnetization 800000 A/m{\rm\}}}.  {\bf mmDataTable}
stores the latest value it receives for each name.  Earlier
values are discarded when new data arrives from a client.

\starsechead{Outputs}
\app{mmDataTable} does not support any data output or storage
facilities.  To save tabular data, use the
\hyperrefhtml{\app{mmGraph}}{\app{mmGraph}
(Ch.~}{)}{sec:mmgraph}\index{application!mmGraph}
or
\hyperrefhtml{\app{mmArchive}}{\app{mmArchive}
(Ch.~}{)}{sec:mmarchive}\index{application!mmArchive} applications.

\starsechead{Controls}
The \btn{Data} menu holds a list of all the data names for which
\app{mmDataTable} has received data.  Initially, \app{mmDataTable} has
received no data from any clients, so this menu is empty.  As data
arrives from clients, the menu fills with the list of data names.
Each data name on the list lies next to a checkbutton.  When the
checkbutton is toggled from off to on, the corresponding data name and
its value and units are displayed at the bottom of the display window.
When the checkbutton is toggled from on to off, the corresponding data
name is removed from the display window.  In this way, the user
selects from all the data received what is to be displayed.  Selecting
the dashed rule at the top of the \btn{Data} menu detaches it so the
user may easily click multiple checkbuttons.

Displayed data values can be individually selected (or deselected) with
a left mouse button click on the display entry.  Highlighting is used to
indicated which data values are currently selected.  The {\btn{Options}}
menu also contains commands to select or deselect all displayed values.
The selected values can be copied into the
cut-and-paste\index{cut-and-paste} (clipboard) buffer with the
\key{CTRL-c} key combination, or the {\btn{Options\pipe Copy}} menu
command.

The data value selection mechanism is also used for data value
formatting control.  The \btn{Options\pipe Format} menu command brings
up a \wndw{Format} dialog box to change the justification and format
specification string.  The latter is the conversion string passed to
the \Tcl\ \cd{format} command, which uses the \C\ \cd{printf} format
codes.  If the {\btn{Adjust:Selected}} radiobutton is active, then the
specification will be applied to only the currently selected
(highlighted) data values.  Alternately, if {\btn{Adjust:All}} is
active, then the specification will be applied to all data values,
and will additionally become the default specification.

A right mouse button click on a display entry will select that entry,
and bring up the \wndw{Format} dialog box with the justification and
format specifications of the selected entry.  These specifications, with
any revisions, may then be applied to all of the selected entries.

If a value cannot be displayed with the selected format specification
string, e.g., if a ``\%d'' integer format were applied to a string
containing a decimal point, then the value will be printed in red
in the form as received by \app{mmDataTable}, without any additional
formatting.

The menu selection \btn{File\pipe Reset} reinitializes the
\app{mmDataTable} application to its original state, clearing the
display and the \btn{Data} menu.  The reset operation is also
automatically invoked upon receipt of new data following a data set
close message from a solver application.  The menu selection
\btn{File\pipe Exit} terminates the application.  The menu \btn{Help}
provides the usual help facilities.

\chapter{Data Graph Display: mmGraph}\label{sec:mmgraph}%
\index{application!mmGraph}

\begin{center}
\includepic{mmgraph-ss}{mmGraph Screen Shot}
\end{center}

\starsechead{Overview}
The application \app{mmGraph} provides a data display service similar to
that of \hyperrefhtml{\app{mmDataTable}}{\app{mmDataTable}
(Ch.~}{)}{sec:mmdatatable}\index{application!mmDataTable}.  The usual
data source is a running solver, but rather than the textual output
provided by \app{mmDataTable}, \app{mmGraph} produces 2D line plots.
\app{mmGraph} also stores the data it receives, so it can produce
multiple views of the data and can save the data to disk.  Postscript
output is also supported.

\starsechead{Launching}
\app{mmGraph} may be started either by selecting the {\btn{mmGraph}}
button on \htmlonlyref{\app{mmLaunch}}{sec:mmlaunch} or from the command
line via
\begin{verbatim}
tclsh oommf.tcl mmGraph [standard options] [-net <0|1>] [loadfile ...]
\end{verbatim}

\begin{description}
\item[\optkey{-net \boa 0\pipe 1\bca}]
  Disable or enable a server which allows the data displayed by
  \app{mmGraph} to be updated by another application.
  By default, the server is enabled.  When the server is disabled,
  \app{mmGraph} may only input data from a file.
\item[\optkey{loadfile \ldots}]
  Optional list of data (\ODT) files to preload.
\end{description}

\starsechead{Inputs}
Input to \app{mmGraph} may come from either a file in the
\hyperrefhtml{\ODT\ format}{\ODT\ format
(Ch.~}{)}{sec:odtformat}\index{file!data~table},
or when \cd{-net 1} (the default) is active, from a client application
(typically a running solver).  The
\btn{File\pipe Open\ldots} dialog box is used to select an input file.
Receipt of data from client applications is the same as for
\hyperrefhtml{\app{mmDataTable}}{\app{mmDataTable}
(Ch.~}{)}{sec:mmdatatable}\index{application!mmDataTable}.  In either
case, input data are appended to any previously held data.

When reading from a file, \app{mmGraph} will automatically
decompress\index{compressed~files}
data using the \hyperrefhtml{local customization}{local customization
(Sec.~}{)}{sec:install.custom}\index{file!configuration}
``Nb\_InputFilter decompress'' option to \cd{Oc\_Option}.  For details,
see the discussion on file translation in the Inputs section of the
\hyperrefhtml{\app{mmDisp} documentation}{\app{mmDisp}
documentation (Ch.~}{)}{sec:mmdisp}.

Curve breaks\index{curve~break} (i.e., separation of a curve into
disjoint segments) are recorded in the data storage buffer via {\em
curve break records.}  These records are generated whenever a new data
table is detected by \app{mmGraph}, or when requested by the user using
the \app{mmGraph} \btn{Options\pipe Break Curves} menu option.

\starsechead{Outputs}

Unlike \app{mmDataTable}\index{application!mmDataTable}, \app{mmGraph}
internally stores the data sent to it.  These data may be written to
disk via the \btn{File\pipe Save As...} dialog box.  If the file
specified already exists, then {\bf mmGraph} output is appended to that
file.  The output is in the tabular \hyperrefhtml{\ODT\ format}{\ODT\
format described in Ch.~}{}{sec:odtformat}\index{file!data~table}.  The
data are segmented into separate \cd{Table Start/Table End} blocks
across each curve break record.

By default, all data currently held by \app{mmGraph} is written, but the
\btn{Save: Selected Data} option presented in the \btn{File\pipe Save
As...} dialog box causes the output to be restricted to those curves
currently selected for display.  In either case, the graph display
limits do {\em not} affect the output.

The save operation\index{data!save} writes records that are held by
\app{mmGraph} at the time the \btn{File\pipe Save As...} dialog box
\btn{OK} button is invoked.  Additionally, the \btn{Auto Save} option in
this dialog box may be used to automatically append to the specified
file each new data record as it is received by \app{mmGraph}.  The
appended fields will be those chosen at the time of the save operation,
i.e., subsequent changing of the curves selected for display does not
affect the automatic save operation.  The automatic save operation
continues until either a new output file is specified, the
\btn{Options\pipe Stop Autosave} control is invoked, or \app{mmGraph} is
terminated.

The \btn{File\pipe Print...}\index{data!print} dialog is used to
produce a Postscript file of the current graph.  On Unix systems, the
output may be sent directly to a
printer\index{platform!Unix!PostScript~to~printer}
by filling the \btn{Print to:}
entry with the appropriate pipe command, e.g., \texttt{|lpr}.  (The
exact form is system dependent.)

\starsechead{Controls}
Graphs are constructed by selecting any one item off the
\btn{X}-axis menu, and any number of items off the \btn{Y1}-axis and
\btn{Y2}-axis menus.  The y1-axis is marked on the left side of the
graph; the y2-axis on the right.  These menus may be detached by
selecting the dashed rule at the top of the list.  Sample results are
shown in the figure at the start of this section.

When \app{mmGraph} is first launched, all the axis menus are empty.
They are dynamically built based on the data received by \app{mmGraph}.
By default, the graph limits and labels are automatically set based on
the data.  The x-axis label is set using the selected item data label
and measurement unit (if any).  The y-axes labels are the measurement
unit of the first corresponding y-axis item selected.

The \btn{Options\pipe Configure...} dialog box allows the user to
override default settings.  To change the graph title, simply enter
the desired title into the \btn{Title} field.  To set the axis labels,
deselect the \btn{Auto Label} option in this dialog box, and fill in the
\btn{X Label}, \btn{Y1 Label} and \btn{Y2 Label} fields as desired.
The axis limits can be set in a similar fashion.  In addition, if an
axis limit is left empty, a default value (based on all selected data)
will be used.  Select the \btn{Auto Scale} option to have the axis
ranges automatically adjust to track incoming data.

Use the \btn{Auto Offset Y1} and \btn{Auto Offset Y2} to automatically
translate each curve plotted against the specified axis up or down so
that the first point on the curve has a y-value of zero for a linear
axis or one for a logarithmic axis.  This feature is especially useful
for comparing variations between different energy curves, because for
these curves one is typically interested in changes is values rather
than the absolute energy value itself.

By default the scaling on each axis is linear.  The \btn{Log scale axes}
check boxes enables logarithmic scaling on the selected axes.  If
logarithmic scaling is selected then any point with a zero value is
dropped and negative values are replaced with their corresponding
positive absolute value.

The size of the margin surrounding the plot region is computed
automatically.  Larger margins may be specified by filling in the
appropriate fields in the \btn{Margin Requests} section.  Units are
pixels.  Requested values smaller than the computed (default) values are
ignored.

The initial curve width is determined by the \cd{Ow\_GraphWin
  default\_curve\_width} setting in the \fn{config/options.tcl} and
\fn{config/local/options.tcl} files, following the usual method of
\hyperrefhtml{local customization}{local customization
  (Sec.~}{)}{sec:install.custom}\index{file!configuration}.  The current
curve width can be changed by specifying the desired width in the
\btn{Curve Width} entry in the {\btn{Options\pipe Configure...}} dialog
box.  The units are pixels.  Long curves will be rendered more quickly,
especially on \Windows, if the curve width is set to 1.

As mentioned earlier, \app{mmGraph} stores in memory all data it
receives.  Over the course of a long run, the amount of data stored can
grow to many megabytes\index{memory~use}.  This storage can be limited
by specifying a positive ($>0$) value for the \btn{Point buffer size}
entry in the {\btn{Options\pipe Configure...}} dialog box.  The oldest
records are removed as necessary to keep the total number of records
stored under the specified limit.  A zero value for \btn{Point buffer
size} is interpreted as no limit.  (The storage size of an individual
record depends upon several factors, including the number of items in
the record and the version of \Tcl\ being used.)  Data erasures may not
be immediately reflected in the graph display.  At any time, the point
buffer storage may be completely emptied with the \btn{Options\pipe
clear Data} command.  The \btn{Options\pipe Stop Autosave} selection
will turn off the auto save feature, if currently active.

Also on this menu is \btn{Options\pipe Rescale}, which autoscales the
graph axis limits from the selected data.  This command ignores but does
not reset the \btn{Auto Scale} setting in the \btn{Options\pipe
Configure...} dialog box.  The Rescale command may also be invoked by
pressing the \key{Home} key.

The \btn{Options\pipe Break Curves} item inserts a curve break record
into the point buffer, causing a break in each curve after the current
point.  This option may be useful if \app{mmGraph} is being fed data
from multiple sources.

The \btn{Options\pipe Key} selection toggles the key (legend) display on
and off.  The key may also be repositioned by dragging with the left
mouse button.  If curves are selected off both the y1 and y2 menus, then
a horizontal line in the key separates the two sets of curves, with the
labels for the y1 curves on top.

If the \btn{Options\pipe Auto Reset} selection is enabled, then when a
new table is detected all previously existing axis menu labels that are
not present in the column list of the new data set are deleted, along
with their associated data.  \app{mmGraph} will detect a new table when
results from a new problem are received, or when data is input from a
file.  If \btn{Options\pipe Auto Reset} is not selected, then no data or
axis menu labels are deleted, and the axes menus will show the union of
the old column label list and the new column label list.  If the axes
menus grow too long, the user may manually apply the
\btn{File\pipe Reset} command to clear them.

The last command on the options menu is \btn{Options\pipe Smooth}.  If
smoothing is disabled, then the data points are connected by straight
line segments.  If enabled, then each curve is rendered as a set of
parabolic splines, which do not in general pass through the data points.
This is implemented using the \cd{--smooth 1} option to the
\Tcl\ \cd{canvas create line} command; see that documentation for
details.

A few controls are available only using the mouse.  If the mouse pointer
is positioned over a drawn item in the graph, holding down the
\key{Control} key and the left mouse button will bring up the
coordinates of that point, with respect to the y1-axis.  Similarly,
depressing the \key{Control} key and the right mouse button, or
alternatively holding down the \key{Control}+\key{Shift} keys while
pressing the left mouse button will bring up the coordinates of the
point with respect to the y2-axis.  The coordinates displayed are the
coordinates of a point on a drawn line, which are not necessarily the
coordinates of a plotted data point.  (The data points are plotted at
the endpoints of each line segment.)  The coordinate display is cleared
when the mouse button is released while the \key{Control} key is down.

One vertical and one horizontal rule (line) are also available.
Initially, these rules are tucked and hidden against the left and bottom
graph axes, respectively.  Either may be repositioned by dragging with
the left or right mouse button.  The coordinates of the cursor are
displayed while dragging the rules.  The displayed y-coordinate
corresponds to the y1-axis if the left mouse button is used, or the
y2-axis if the right mouse button or the \key{Shift} key with the left
mouse button are engaged.

The graph extents may be changed by selecting a ``zoom box'' with the
mouse.  This is useful for examining a small portion of the graph in
more detail.  This feature is activated by clicking and dragging the
left or right mouse button.  A rectangle will be displayed that changes
size as the mouse is dragged.  If the left mouse button is depressed,
then the x-axis and y1-axis are rescaled to just match the extents of
the displayed rectangle.  If the right mouse button, or alternatively
the shift key + left mouse button, is used, then the x-axis and y2-axis
are rescaled.  An arrow is drawn against the rectangle indicating which
y-axis will be rescaled.  The rescaling may be canceled by positioning
the mouse pointer over the initial point before releasing the mouse
button.  The zoom box feature is similar to the mouse zoom control in
the \hyperrefhtml{\app{mmDisp}}{\app{mmDisp}
  (Ch.~}{)}{sec:mmdisp}\index{application!mmDisp} application, except
that here there is no ``un-zooming'' mouse control.  However, the
coordinate limits for each zoom are stored in a list, and the \key{Esc}
and \key{Shift+Esc} keys may be used to move backwards and forwards,
respectively, through this list.  The \key{Enter}/\key{Return} key will
copy the current coordinate limits to the list.  A pointer is kept to
the selected display configuration state, and new states are added after
that point; if a display state is added in the interior of the list
(i.e. ahead of the last state), then all configurations following the
new entry are deleted.  The entire list is cleared any time automatic
scaling is invoked.

The \key{PageUp} and \key{PageDown} keys may also be used to zoom the
display in and out.  Use in conjuction with the \key{Shift} key to jump
by larger steps, or with the \key{Control} key for finer control.  The
\btn{Options\pipe Rescale} command or the
\btn{Options\pipe Configure\ldots} dialog box may also be used to reset the
graph extents.

If \app{mmGraph} is being used to display data from a running solver,
and if \btn{Auto Scale} is selected in the
\btn{Options\pipe Configure\ldots} dialog box, then the graph extents
may be changed automatically when a new data point is received.  This is
inconvenient if one is simultaneously using the zoom feature to examine
some portion of the graph.  In this case, one might prefer to disable
the \btn{Auto Scale} feature, and manually pan the display using the
keyboard arrow keys.  Each key press will translate the display one half
frame in the indicated direction.  The \key{Shift} key used in
combination with an arrow keys double the pan step size, while the
\key{Control} key halves it.

The menu selection \btn{File\pipe Reset} reinitializes the
\app{mmGraph} application to its original state, releasing all data and
clearing the axis menus.  The menu selection \btn{File\pipe Exit}
terminates the application.  The menu \btn{Help} provides the usual help
facilities.

\starsechead{Details}
The axes menus are configured based on incoming data.  As a result,
these menus are initially empty.  If a graph widget is scheduled to
receive data only upon control point or stage done events in the solver,
it may be a long time after starting a problem in the solver before the
graph widget can be configured.  Because \app{mmGraph} keeps all data up
to the limit imposed by the \cd{Point buffer size}, data loss is usually
not a problem.  Of more importance is the fact that automatic data
saving\index{data!save} can not be set up until the first data point is
received.  As a workaround, the solver initial state may be sent
interactively as a dummy point to initialize the graph widget axes
menus.  Select the desired quantities off the axes menus, and use the
\btn{Options\pipe clear Data} command to remove the dummy point from
\app{mmGraph}'s memory.  The {\btn{File\pipe Save As...}} dialog box may
then be used---with the {\btn{Auto Save}} option enabled---to write out
an empty table with proper column header information.  Subsequent data
will be written to this file as they arrive.



\chapter{Vector Field Display: mmDisp}\label{sec:mmdisp}%
\index{application!mmDisp}

\begin{center}
\includepic{mmdisp-ss}{mmDisp Screen Shot}
\end{center}

\starsechead{Overview}
The application \app{mmDisp} displays two-dimensional slices of
three-dimensional spatial distributions of vector fields.  \app{mmDisp}
currently supports display of 1D (i.e., scalar) and 3D vector data.  It
can load field data from files in a variety of formats, or it can accept
data from client applications, such as a running solver.  \app{mmDisp}
offers a rich interface for controlling the display of vector field
data, and can also save the data to a file or produce \postscript\ print
output.

\starsechead{Launching}
\app{mmDisp} may be started either by selecting the \btn{mmDisp} button
on \htmlonlyref{\app{mmLaunch}}{sec:mmlaunch}, or from the command line via
\begin{verbatim}
tclsh oommf.tcl mmDisp [standard options] [-config file] \
   [-net <0|1>] [filename]
\end{verbatim}

\begin{description}
\item[\optkey{-config file}]
  User configuration file that specifies default display parameters.
  This file is discussed in
  \htmlonlyref{detail below}{sec:mmdispconfig}.
\item[\optkey{-net \boa 0\pipe 1\bca}]
  Disable or enable a server which allows the data displayed by
  \app{mmDisp} to be updated by another application.
  By default, the server is enabled.  When the server is disabled,
  \app{mmDisp} may only input data from a file.
\end{description}

If a filename is supplied on the command line, \app{mmDisp} takes
it to be the name of a file containing vector field data for display.
That file will be opened on startup.

\starsechead{Inputs}
Input to \app{mmDisp} may come from either a file or from a client
application (typically a running solver), in any of the
\hyperrefhtml{\OOMMF\ vector field formats}{vector field
formats described in Ch.~}{}{sec:vfformats}\index{file!vector~field}.
Other file formats can also be supported if a translation filter program
is available.

Client applications that send data to \app{mmDisp} control the flow of
data.  The user, interacting with the \app{mmDisp} window, determines
how the vector field data are displayed.

File input is initiated through the \btn{File\pipe Open\ldots}\ dialog
box.  Several example files are included in the \OOMMF\ release in the
directory \fn{app/mmdisp/examples}.  When the \btn{Browse} button is
enabled, the ``Open File'' dialog box will remain open after loading a
file, so that multiple files may be displayed in sequence.  The
{\btn{Auto}} configuration box determines whether the vector
subsampling, data scale, zoom and slice settings should be
determined automatically (based on the data in the file and the current
display window size), or whether their values should be held constant
while loading the file.

\index{customize!file~format~translation|(}

\app{mmDisp} permits local customization allowing for automatic
translation from other file formats into one of the
\hyperrefhtml{\OOMMF\ vector field formats}{vector field formats
(Ch.~}{)}{sec:vfformats} that \app{mmDisp} recognizes.  When loading
a file, \app{mmDisp} compares the file name to a list of extensions.
An example extension is \cd{.gz}.  The assumption is that the extension
identifies files containing data in a particular format.  For each
extension in the list, there is a corresponding translation program.
\app{mmDisp} calls on that program as a filter which takes data in one
format from standard input and writes to standard output the same data
in one of the formats supported by \app{mmDisp}.  In its default
configuration, \app{mmDisp} recognizes the patterns \cd{.gz}, \cd{.z},
and \cd{.zip}, and invokes the translation
program {\cd{gzip -dc}}\index{application!gzip} to perform the
``translation.''  In this way, support for reading
compressed\index{compressed~files} files is
``built in'' to \app{mmDisp} on any platform where the \app{gzip}
program is installed.

There are two categories of translations supported: decompression and
format conversion.  Both are modified by the usual method of
\hyperrefhtml{local customization}{local customization
(Sec.~}{)}{sec:install.custom}\index{file!configuration}.  The
command governing decompression in the customization file is of the form
\begin{rawhtml}
  <BLOCKQUOTE>
\end{rawhtml}
%begin<latexonly>
\begin{quote}
%end<latexonly>
\begin{verbatim}
Oc_Option Add * Nb_InputFilter decompress {{.gz .zip} {gzip -dc}}
\end{verbatim}
%begin<latexonly>
\end{quote}
%end<latexonly>
\begin{rawhtml}
  </BLOCKQUOTE>
\end{rawhtml}
The final argument in this command is a list with an even number of
elements.  The first element of each pair is the filename extension.
The second element in each pair is the command line for launching the
corresponding translation program.  To add support for bzip2 compressed
files, change this line to
\begin{rawhtml}
  <BLOCKQUOTE>
\end{rawhtml}
%begin<latexonly>
\begin{quote}
%end<latexonly>
\begin{verbatim}
Oc_Option Add * Nb_InputFilter decompress \
                {{.gz .zip} {gzip -dc} .bz2 bunzip2}
\end{verbatim}
%begin<latexonly>
\end{quote}
%end<latexonly>
\begin{rawhtml}
  </BLOCKQUOTE>
\end{rawhtml}
This option also affects other applications such as
\htmlonlyref{\app{mmGraph}}{sec:mmgraph} that support ``on-the-fly''
decompression.  In all cases the decompression program must accept
compressed input on standard input and write the decompressed output to
standard output.

There is also input translation support for filters that convert from
foreign (i.e., non-\OOMMF) file formats.  For example,
if a program \cd{foo} were known to translate a file format identified by the
extension \fn{.bar} into the
\OVF\ file format, that program could be made known to \app{mmDisp}
by setting the customization command:
\begin{rawhtml}
  <BLOCKQUOTE>
\end{rawhtml}
%begin<latexonly>
\begin{quote}
%end<latexonly>
\begin{verbatim}
Oc_Option Add * Nb_InputFilter ovf {.bar foo}
\end{verbatim}
%begin<latexonly>
\end{quote}
%end<latexonly>
\begin{rawhtml}
  </BLOCKQUOTE>
\end{rawhtml}
This assumes that the program \cd{foo} accepts input of the form
\fn{.bar} on standard input and writes the translated results to
standard output.
\index{customize!file~format~translation|)}


\starsechead{Outputs}
The vector field displayed by \app{mmDisp} may be saved to disk via the
{\btn{File\pipe Save As\ldots}}\index{data!save} dialog box.  The output
is in the \hyperrefhtml{\OVF\ format}{\OVF\ format
(Sec.~}{)}{sec:ovfformat}\index{file!vector~field}.  The \OVF\ file
options may be set by selecting the appropriate radio buttons in the
\OVF\ File Options panel.  The \btn{Title} and \btn{Desc} fields may be
edited before saving.  Enabling the \btn{Browse} button allows for
saving multiple files without closing the ``Save File'' dialog box.

The \btn{File\pipe Print\ldots}\index{data!print} dialog is used to
produce a \postscript\ file of the current display.  On Unix systems, the
output may be sent directly to a
printer\index{platform!Unix!PostScript~to~printer} by filling the
\btn{Print to:} entry with the appropriate pipe command, e.g.,
\pipe\texttt{lpr}.  (The exact form is system dependent.)  The other
print dialog box options are described in the
\htmlonlyref{configuration files}{sec:mmdispconfig} section below.

The \btn{File\pipe Write config\ldots}\index{file!configuration} dialog
allows one to save to disk a configuration file holding the current
display parameters.  This file can be used to affect startup display
parameters, or used as input to the \hyperrefhtml{avf2ppm}{\app{avf2ppm}
(Sec.~}{)}{sec:avf2ppm}\index{application!avf2ppm} and
\hyperrefhtml{avf2ps}{\app{avf2ps}
(Sec.~}{)}{sec:avf2ps}\index{application!avf2ps} command line utilities
that convert files from the \OVF\ format into bitmap images and
\postscript\ printer files, respectively.  (mmDisp does not provide
direct support for writing bitmap files.)  Details of the configuration
file are \htmlonlyref{discussed below}{sec:mmdispconfig}.

\starsechead{Controls}\label{sec:mmdispcontrols}

The menu selection \btn{File\pipe Clear} clears the display window.
The menu selection \btn{File\pipe Exit} terminates the
\app{mmDisp} application.  The menu \btn{Help} provides
the usual help facilities.

The \btn{View} menu provides high-level control over how the vector
field is placed in the display window.  The menu selection
{\btn{View\pipe Wrap Display}} resizes the display window so that it
just contains the entire vector field surrounded by a margin.
{\btn{View\pipe Fill Display}} resizes the vector field until it fills
the current size of the display window.  If the aspect ratio of the
display window does not match the aspect ratio of the vector field, a
larger than requested margin appears along one edge to make up the
difference.  {\btn{View\pipe Center Display}} translates the vector
field to put the center of view at the center of the display window.
{\btn{View\pipe Rotate ccw}} and {\btn{View\pipe Rotate cw}} rotate the
display one quarter turn counter-clockwise and clockwise respectively.
If the display size is not locked (see {\btn{Options\pipe Lock~size}}
below), then the display window also rotates, so that the portion of the
vector field seen and any margins are preserved (unless the display of
the control bar forces the display window to be wider).  {\btn{View\pipe
reDraw}} allows the user to invoke a redrawing of the display window.
The {\btn{View\pipe Viewpoint}} tearable submenu supports rotation of
the vector field out of the plane of the display, so that it may be
viewed from along a different axis.

The menu selection \btn{Options\pipe Configure\ldots}\ brings up a
dialog box through which the user may control many features of the
vector field display.  Vectors in the vector field may be displayed as
arrows, pixels, or both.  The \btn{Arrow} and \btn{Pixel} buttons in the
\btn{Plot type} column on the left of the dialog box enable each type of
display.

Columns 2--4 in the Configure dialog box control the use of color.  Both
arrows and pixels may be independently colored to indicate some
quantity\index{color!quantity}.  The \btn{Color Quantity} column
controls which scalar quantity the color of the arrow or pixel
represents.  Available color quantities include vector $x$, $y$, and $z$
components, total vector magnitude, slice depth, and angles as measured
in-plane from a fixed axis.  On regularly gridded data the vector field
divergence is also available for display.

The assignment of a color to a quantity value is determined by the
\btn{Colormap}\index{color!map} selected.  Colormaps are labeled by a
sequence of colors that are mapped across the range of the selected
quantity.  For example, if the ``Red-Black-Blue'' colormap is applied to
the {\btn{Color Quantity}} ``z'', then vectors pointing into the
$xy$-plane ($z<0$) are colored red, those lying in the plane ($z=0$) are
colored black, and those pointing out of the plane ($z>0$) are colored
blue.  Values between the extremes are colored with intermediate colors,
selected using a discretization determined by the \btn{\lb\ of Colors}
value.  This value governs the use of potentially limited color
resources, and can be used to achieve some special coloring effects.
(Note: The in-plane angle quantities are generally best viewed with a
colormap that begins and ends with the same color, e.g.,
``Red-Green-Blue-Red.'')  The ordering of the colormap can be reversed
by selecting the \btn{Reverse} checkbox.  For example, this would change
the ``Red-Black-Blue'' colormap to effectively ``Blue-Black-Red.''

Below the \btn{Reverse} checkbutton in the pixel plot type row is a
\btn{Opaque} checkbutton.  If this is selected then arrows below the top
row in the pixel slice range (see slice discussion below) will be hidden
by the pixel object.  If disabled, then the pixel object is translucent,
so objects further below are partially visible.

When there are many vectors in a vector field, a display of all of them
may be more confusing than helpful.  The \btn{Subsample} column
allows the user to request that only a sampling of vectors from the
vector field be displayed.  The \btn{Subsample} value is roughly
the number of vectors along one spatial dimension of the vector field
which map to a single displayed vector (arrow or pixel).  Each vector
displayed is an actual vector in the vector field---the selection of
vectors for display is a sampling process, not an averaging or
interpolation process.  The subsample rates for arrows and pixels may be
set independently.  A subsample rate of 0 is interpreted specially to
display all data.  (This is typically much quicker than subsampling at a
small rate, e.g., 0.1.)

The length of an arrow represents the magnitude of the vector field.
All arrows are drawn with a length between zero and ``full-scale.''
By default, the full-scale arrow length is computed
so that it covers the region of the screen that one displayed
vector is intended to represent, given the current subsample rate.
Following this default, arrows do not significantly overlap each other,
yet all non-zero portions of the vector field have
a representation in the display.  Similarly, pixels are drawn with
a default size that fills an area equal to the region of the screen
one pixel is intended to represent, given the pixel subsample rate.
The \btn{Size} column allows the user to (independently)
override the default size of pixels and full-scale arrows.
A value of 1 represents the default size.  By
changing to a larger or smaller \btn{Size} value, the user may
request arrows or pixels larger or smaller than the default size.

Below the arrow \btn{Size} box is the \btn{View scale} option. If this
is enabled (the default) then arrow scaling is adjusted so that a size
setting of 1 results in an in-viewplane vector having length
approximately equal to the smaller of the two in-plane view-cell
dimensions. (The view-cell is the discretization cell multiplied by the
subsample setting.) If \btn{View scale} is disabled then the arrow size
is scaled relative to the smallest of all three view-cell dimensions,
and is therefore fixed independent of view axis. Disabling both auto
subsampling and view scale may make comparisons between different view
axis directions easier.

Below the Arrow and Pixel Controls are several additional controls.  The
\btn{Data Scale}\index{data!scale}\label{html:mmdispdatascale} entry
affects the data value scaling.  As described above, all arrows are
displayed with length between zero and full-scale.  The full-scale arrow
length corresponds to some scalar value of the magnitude of the vector
field.  The \btn{Data Scale}\index{data!scale} entry allows the user to
set the value at which the drawn arrow length goes full-scale.  Any
vectors in the vector field with magnitude equal to or greater than the
data scale value will be represented by arrows drawn at full scale.
Other vectors will be represented by shorter arrows with length
determined by a linear scale between zero and the data scale value.
Similarly, the data scale value controls the range of values spanned by
the colormap used to color pixels.  The usual use of the
\btn{Data Scale}\index{data!scale} entry is to reduce the data
scale value so that more detail can be seen in those portions of
the vector field which have magnitude less than other parts of the
vector field.
If the data scale value is increased, then the length of the
arrows in the plot is reduced accordingly.  If the data scale value is
decreased, then the length of the arrows is increased, until they
reach full-scale.
An arrow representing a vector with magnitude larger than the
data scale value may be thought of as being truncated to the data scale
value.  The initial (default) data scale value is usually the maximum
vector magnitude in the field, so at this setting no arrows are
truncated.  Entering \key{0} into the data scale box will cause the data
scale to be reset to the default value.  (For \hyperrefhtml{\OVF\
files}{\OVF\ files (Sec.~}{)}{sec:ovfformat}\index{file!vector~field},
the default data scale value is set from the \cd{ValueRangeMaxMag}
header line.  This is typically set to the maximum vector magnitude, but
this is not guaranteed.)  The data scale control is intended primarily
for use with vector fields of varying magnitude (e.g., \vH -fields), but
may also be used to adjust the pixel display contrast for any field
type.

The \btn{Zoom}\index{data!zoom} entry controls the spatial scaling of
the display.  The value roughly corresponds to the number of pixels per
vector in the fully-sampled vector field.  (This value is not affected
by the subsampling rate.)

The \btn{Margin} entry specifies the margin size, in pixels, to be
maintained around the vector field.

The next row of entry boxes control slice\index{data!slice~selection}
display.  Slice selection allows display of that subset of the data
that is within a specified distance of a plane running perpendicular
to the view axis.  The location of that plane with respect to the view
axis is specified in the \btn{X-slice center}, \btn{Y-slice center} or
\btn{Z-slice center} entry, depending on the current view axis.
The thickness of the slice may be varied separately for arrow and
pixel displays, as specified in the next two entry boxes.  The slice
span boxes interpret specially the following values: 0 resets the
slice thickness to the default value, which is usually the thickness
of a single cell.  Any negative value sets the slice thickness to be
the full thickness of the mesh.  Values for all of the slice control
entries are specified in the fundamental mesh spatial unit, for
example, meters.  (Refer to the vector field
\hyperrefhtml{file format}{file format (Ch.~}{)}{sec:vfformats}
documentation for more on mesh spatial units.)

Below the slice contols are controls to specify whether or not a
bounding polygon\index{boundary} is displayed, and the background
color for the display window.

No changes made by the user in the {\btn{Options\pipe Configure\ldots}}\
dialog box affect the display window until either the
\btn{Apply} or \btn{OK} button is selected.  If the \btn{OK} button is
selected, the dialog box is also dismissed.  The {\btn{Close}}
button dismisses the dialog without changing the display window.

The next item under the \btn{Options} menu is a checkbutton that
toggles the display of a control bar.  The control bar offers
alternative interfaces to some of the operations available from the
\btn{Options\pipe Configure\ldots}\ dialog box and the \btn{View} menu.
On the left end of the control bar is a display of the coordinate axes.
These axes rotate along with the vector field in the display window to
identify the coordinate system of the display, and are color coded to
agree with the pixel (if active) or arrow coloring.  A click of the left
mouse button on the coordinate axes causes a counter-clockwise rotation.
A click of the right mouse button on the coordinate axes causes a
clockwise rotation.

To the right of the coordinate axes are two rows of controls.  The top
row allows the user to control the subsample rate and size of displayed
arrows.  The subsample rate may be modified either by direct entry of a
new rate, or by manipulation of the slider.  The second row controls the
current data scale value and zoom (spatial magnification).  A vertical
bar in the slider area marks the default data scale value.  Specifying
\key{0} for the data scale value will reset the data scale to the
default value.

The spatial magnification may be changed either by typing a value in the
Zoom box of the control bar, or by using the mouse inside the display
window.  A click and drag with the left mouse button displays a red
rectangle that changes size as the mouse is dragged.  When the left
mouse button is released, the vector field is rescaled so that the
portion of the display window within the red rectangle expands until it
reaches the edges of the display window.  Both dimensions are scaled by
the same amount so there is no aspect distortion of the vector field.
Small red arrows on the sides of the red rectangle indicate which
dimension will expand to meet the display window boundaries upon release
of the left mouse button.  After the rescaling, the red rectangle
remains in the display window briefly, surrounding the same region of
the vector field, but at the new scale.

A click and drag with the right mouse button displays a blue rectangle
that changes size as the mouse is dragged.  When the right mouse button
is released, the vector field is rescaled so that all of the vector
field currently visible in the display window fits the size of the
blue rectangle.  Both dimensions are scaled by the same amount so there
is no aspect distortion of the vector field.  Small blue arrows on the
sides of the blue rectangle indicate the dimension in which the vector
field will shrink to exactly transform the display window size to the
blue rectangle size.  After the rescaling, the blue rectangle remains in
the display window briefly, surrounding the same region of the vector
field, now centered in the display window, and at the new scale.

When the zoom value is large enough that a portion of the vector field
lies outside the display window, scrollbars appear that may be used to
translate the vector field so that different portions are visible in the
display window.  On systems that have a middle mouse button, clicking
the middle button on a point in the display window translates the vector
field so that the selected point is centered within the display window.

\app{mmDisp} remembers the previous zoom value and data scale values.
To revert to the previous settings, the user may hit the \key{ESC} key.
This is a limited ``Undo'' feature.

Below the data scale and zoom controls in the control bar is the slice
center selection control\index{data!slice~selection}.  This will be
labeled \btn{Z-slice}, \btn{X-slice}, or \btn{Y-slice}, depending on
which view axis is selected.  The thickness of the slice can be set
from the \btn{Options\pipe Configure\ldots}\ dialog box.

The final item under the \btn{Options} menu is the
\btn{Options\pipe Lock~size} checkbutton.  By default, when the
display is rotated in-plane, the width and height of the viewport are
interchanged so that the same portion of the vector field remains
displayed.  Selecting the \btn{Options\pipe Lock~size} checkbutton
disables this behavior, and also other viewport changing operations
(e.g., display wrap).

Several keyboard shortcuts are available as alternatives to menu- or
mouse-based operations.  (These are in addition to the usual keyboard
access to the menu.)  The effect of a key combination depends on which
subwindow of \app{mmDisp} is active.  The \key{TAB} key may be used to
change the active subwindow.  The \key{SHIFT-TAB} key combination also
changes the active subwindow, in reverse order.

When the active subwindow is the display window, the following
key combinations are active:
\begin{itemize}
\item \key{CTRL-o} -- same as menu selection \btn{File\pipe Open\ldots}
\item \key{CTRL-s} -- same as menu selection
                       {\btn{File\pipe Save as\ldots}\index{data!save}}
\item \key{CTRL-p} -- same as menu selection
                       {\btn{File\pipe Print\ldots}\index{data!print}}
\item \key{CTRL-c} -- same as menu selection
        \btn{Options\pipe Configure\ldots}
\item \key{CTRL-v} -- launches viewpoint selection menu,
        \btn{View\pipe Viewpoint}
\item \key{CTRL-w} -- same as menu selection {\btn{View\pipe Wrap Display}}
\item \key{CTRL-f} -- same as menu selection {\btn{View\pipe Fill Display}}
\item \key{HOME} -- First fill, then wrap the display.
\item \key{CTRL-space} --
 same as menu selection {\btn{View\pipe Center Display}}
\item \key{CTRL-r} -- same as menu selection \btn{View\pipe Rotate ccw}
\item \key{SHIFT-CTRL-r} -- same as menu selection \btn{View\pipe Rotate cw}
\item \key{INSERT} -- decrease arrow subsample by 1
\item \key{DEL} -- increase arrow subsample by 1
\item \key{SHIFT-INSERT} -- decrease arrow subsample by factor of 2
\item \key{SHIFT-DEL} -- increase arrow subsample by factor of 2
\item \key{PAGEUP} -- increase the zoom value by a factor of 1.149
\item \key{PAGEDOWN} -- decrease the zoom value by a factor of 1.149
\item \key{SHIFT-PAGEUP} -- increase the zoom value by factor of 2
\item \key{SHIFT-PAGEDOWN} -- decrease the zoom value by factor of 2
\item \key{ESC} -- revert to previous data scale and zoom values
\end{itemize}

When the active subwindow is the control bar's coordinate axes display,
the following key combinations are active:
\begin{itemize}
\item \key{LEFT} -- same as menu selection \btn{View\pipe Rotate ccw}
\item \key{RIGHT} -- same as menu selection \btn{View\pipe Rotate cw}
\end{itemize}

When the active subwindow is any of the control bar's value entry
windows -- arrow subsample, size, data scale or zoom, the following key
combinations are active:
\begin{itemize}
\item \key{ESC} -- undo uncommitted value (displayed in red)
\item \key{RETURN} -- commit entered value
\end{itemize}

When the active subwindow is in any of the control bar's sliders---arrow
subsample, data scale or slice---the following key combinations are
active:
\begin{itemize}
\item \key{LEFT} -- slide left (decrease value)
\item \key{RIGHT} -- slide right (increase value)
\item \key{ESC} -- undo uncommitted value (displayed in red)
\item \key{RETURN} -- commit current value
\end{itemize}

When any of the separate dialog windows are displayed (e.g., the
\btn{File\pipe Open\ldots} or \btn{Options\pipe Configure\ldots}
dialogs), the shortcut \key{CTRL-.} (control-period) will raise and
transfer keyboard focus back to the root \app{mmDisp} window.

\starsechead{Configuration files}\label{sec:mmdispconfig}
The various initial display parameters (e.g., window size, orientation,
colormap) are set by configuration files.  The default configuration
file
\begin{quote}
\fn{oommf/app/mmdisp/scripts/mmdisp.config}
\end{quote}
is read first, followed by the local customization file,
\begin{quote}
\fn{oommf/app/mmdisp/scripts/local/mmdisp.config}
\end{quote}
if it exists.  Lastly, any files passed as \cd{-config} options on the
command line are input.  The files must be valid \Tcl\ scripts, the main
purpose of which is to set elements of the \cd{plot\_config} and
\cd{print\_config} arrays, as illustrated in the \hyperrefhtml{default
configuration file}{default configuration file
(Fig.~}{,}{fig:mmdisp.config}\latex{ page~\pageref{fig:mmdisp.config})}.
(See the \Tcl\ documentation for details of the \cd{array set} command.)

There are several places in the configuration file where colors are
specified.  Colors may be represented using the symbolic names in
\fn{oommf/config/colors.config}, in any of the \Tk\ hexadecimal
formats, e.g., \cd{\#RRGGBB}, or as a shade of gray using the format
``grayD'' (or ``greyD''), where D is a decimal integer from 0-100,
inclusive.  Examples in the latter two formats are \cd{\#FFFF00} for
yellow, \cd{gray0} for black, and \cd{gray100} or \cd{\#FFFFFF} for
white.

Refer to the default configuration file as we discuss each element of
the \cd{plot\_config} array:
\begin{description}
\item[\optkey{arrow,status}]
  Set to 1 to display arrows, 0 to not draw arrows.
\item[\optkey{arrow,autosample}]
 If 1, then ignore the value of \cd{arrow,subsample}\index{sampling} and
 automatically determine a ``reasonable'' subsampling rate for the
 arrows.  Set to 0 to turn off this feature.
\item[\optkey{arrow,subsample}]
 If \cd{arrow,autosample} is 0, then subsample the input vectors at this
 rate when drawing arrows.  A value of 0 for \cd{arrow,subsample} is
 interpreted specially to display all data.
\item[\optkey{arrow,colormap}]
  Select the colormap to use when drawing arrows.  Should be one of the
  strings specified in the {\tt Colormap} section of the
  \btn{Options\pipe Configure\ldots} dialog.
\item[\optkey{arrow,colorcount}]
  Number of discretization\index{color!discretization} levels to use
  from the colormap.  A value of zero will color all arrows with the
  first color in the colormap.
\item[\optkey{arrow,quantity}]
 Scalar quantity the arrow color\index{color!quantity} is to represent.
 Supported values include \cd{x}, \cd{y}, \cd{z}, \cd{xy-angle},
 \cd{xz-angle}, \cd{yz-angle}, and \cd{slice}.  The
 \btn{Options\pipe Configure\ldots} dialog presents the complete list of
 allowed quantities, which may be image dependent.
\item[\optkey{arrow,colorreverse}]
 The \cd{colorreverse} value should be 1 or 0, signifying to reverse or
 not reverse, respectively.  If reverse is selected, then the colormap
 ordering is inverted, changing for example \cd{Blue-White-Red} into
 \cd{Red-White-Blue}.  This corresponds to the \cd{Reverse} control in
 the \btn{Options\pipe Configure\ldots}.
\item[\optkey{arrow,colorphase}]
 The phase is a real number between -1 and 1 that provides a translation
 in the selected \cd{colormap}.  For the \cd{xy-angle}, \cd{xz-angle}
 and \cd{yz-angle} color quantities, this displays as a rotation of the
 colormap, e.g., setting colorphase to 0.333 would effectively
 change the \cd{Red-Green-Blue-Red} colormap into
 \cd{Green-Blue-Red-Green}.  For the other color quantities, it simply
 shifts the display band, saturating at one end.  For example, setting
 colorphase to 0.5 changes the \cd{Blue-White-Red} colormap into
 \cd{White-Red-Red}.  If both inversion and phase adjustment are
 requested, then inversion is applied first.
\item[\optkey{arrow,size}]
 Size of the arrows relative to the default size (represented as 1.0).
\item[\optkey{arrow,viewscale}]
 Enables automatic scaling of arrows determined by the in-viewplane cell
 dimensions.
\item[\optkey{pixel,\ldots}]
 Most of the pixel configuration elements have analogous arrow
 configuration elements, and are interpreted in the same manner.  The
 exception is the \cd{pixel,opaque} element, which is discussed next.
 Note too that the auto subsampling rate for pixels is considerably
 denser than for arrows.
\item[\optkey{pixel,opaque}]
 If the opaque value is 1, then the pixel is drawn in a solid manner,
 concealing any arrows which are drawn under it.  If opaque is 0, then
 the pixel is drawn only partially filled-in, so objects beneath it can
 still be discerned.
\item[\optkey{misc,background}]
 Specify the background color.
\item[\optkey{misc,drawboundary}]
 If 1, then draw the bounding polygon\index{boundary}, if any, as
 specified in the input vector field file.
\item[\optkey{misc,boundarycolor}]
 String specifying the bounding polygon color, if drawn.
\item[\optkey{misc,boundarywidth}]
 Width of the bounding polygon, in pixels.
\item[\optkey{misc,margin}]
 The size of the border margin\index{margin}, in pixels.
\item[\optkey{misc,defaultwindowwidth}, \optkey{misc,defaultwindowheight}]
 Width and height of initial display viewport, in pixels.
\item[\optkey{misc,width}, \optkey{misc,height}]
 Width and height of displayed area.  This will be less than the
 viewport dimensions if scrollbars are present.  These values are
 ignored during \app{mmDisp} initialization, but are written out by the
 \btn{File\pipe Write config\ldots} command as a convenience for the
 \hyperrefhtml{avf2ppm}{\app{avf2ppm}
 (Sec.~}{)}{sec:avf2ppm}\index{application!avf2ppm} command line
 utility.
\item[\optkey{misc,rotation}]
 Counterclockwise rotation in degrees; either 0, 90, 180 or 270.
\item[\optkey{misc,zoom}]
 Scaling factor for the display.  This is the same value as shown in the
 ``zoom'' box in the \app{mmDisp} control bar,
 and corresponds roughly to the number of pixels per vector in the
 (original, fully-sampled) vector field.  If set to zero, then
 the scaling is set so the image, including margins, just fits inside
 the viewport dimensions.
\item[\optkey{misc,datascale}]
 Scale for arrow size and colormap ranges; equivalent to the
 \htmlonlyref{\cd{Data Scale} control}{html:mmdispdatascale}.  In general,
 this should be a positive real value, but a zero or empty value will
 set the scaling to its default setting.
\item[\optkey{misc,centerpt}]
 If specified, the value should be a three item list of real numbers
 specifying the center of the display, \cd{\ocb x y z\ccb}, in
 file mesh units (e.g., meters).
\item[\optkey{misc,relcenterpt}]
 If specified, the value should be a three item list of real numbers in
 the range $[0,1]$ specifying the center of the display in relative
 coordinates.  If both \cd{misc,relcenterpt} and \cd{misc,centerpt} are
 defined, then \cd{misc,centerpt} takes precedence.
\item[\optkey{viewaxis}]
 Select the view axis, which should be one of \cd{+z}, \cd{-z}, \cd{+y},
 \cd{-y}, \cd{+x}, or \cd{-x}.  This option is equivalent to the
 \btn{View\pipe Viewpoint} menu control.
\item[\optkey{viewaxis,xarrowspan}, \optkey{viewaxis,yarrowspan},
      \optkey{viewaxis,zarrowspan}]
 Specifies the thickness of the arrow display slice, for the
 corresponding view.  For example, if the view axis is \cd{+z} or
 \cd{-z}, then only \optkey{viewaxis,zarrowspan} is active.  The value
 for each element should be either a real value or an empty string.  If
 the value is zero or an empty string, then the thickness is set to the
 default value, which is typically the thickness of a single cell.  If
 the value is positive, then it specifies the slice range in file mesh
 units, e.g., in meters.  Lastly, if the value is negative, then the
 slice is set to the entire thickness of the mesh in that view
 direction.
\item[\optkey{viewaxis,xpixelspan}, \optkey{viewaxis,ypixelspan},
      \optkey{viewaxis,zpixelspan}]
 Identical interpretation and behavior as the corresponding arrow span
 elements, but with regards to pixel display.
\end{description}

The \cd{print\_config} array controls printing defaults, as displayed in
the \btn{File\pipe Print\ldots}\index{data!print} dialog box:
\begin{description}
\item[\optkey{orient}]
 Paper orientation, either landscape or portrait.
\item[\optkey{paper}]
 Paper type: letter, legal, A4 or A3.
\item[\optkey{hpos}, \optkey{vpos}]
 The horizontal and vertical positioning on the printed page.  Valid
 values for \cd{hpos} are left, center, or right, and for \cd{vpos} are
 top, center, or bottom.
\item[\optkey{units}]
 Units that the margin and print area dimensions are measured in;
 either in or cm.
\item[\optkey{tmargin}, \optkey{lmargin}]
 Top and left margin size, measured in the selected units.
\item[\optkey{pwidth}, \optkey{pheight}]
 Output print area dimensions, width and height, measured in the
 selected units.  The output will be scaled to meet the more restrictive
 dimension.  In particular, the x/y-scaling ratio remains 1:1.
\item[\optkey{croptoview}]
 Boolean value, either 0 or 1.  If 1 (true), then the print
 output is cropped to display only that portion of the vector field that
 is visible in the display window.  If 0, then the display is ignored
 and the output is scaled so that the entire vector field is printed.
\end{description}

If any of the above elements are set in multiple configuration files,
then the last value read takes precedence.

\begin{codelisting}{f}{fig:mmdisp.config}{Contents of default configuration
  file \fn{mmdisp.config}.}{sec:mmdispconfig}{ref}
\begin{verbatim}
array set plot_config {
  arrow,status       1                misc,background           white
  arrow,autosample   1                misc,drawboundary         1
  arrow,subsample    0                misc,boundarycolor        black
  arrow,colormap     Red-Black-Blue   misc,boundarywidth        1
  arrow,colorcount   256              misc,margin               10
  arrow,quantity     z                misc,defaultwindowwidth   640
  arrow,colorreverse 0                misc,defaultwindowheight  480
  arrow,colorphase   0                misc,width                0
  arrow,size         1                misc,height               0
  arrow,viewscale    1                misc,rotation             0
                                      misc,zoom                 0
  pixel,status       0                misc,datascale            0
  pixel,autosample   1                misc,relcenterpt     {0.5 0.5 0.5}
  pixel,subsample    0
  pixel,colormap     Blue-White-Red   viewaxis                  +z
  pixel,colorcount   256              viewaxis,xarrowspan       {}
  pixel,quantity     x                viewaxis,xpixelspan       {}
  pixel,colorreverse 0                viewaxis,yarrowspan       {}
  pixel,colorphase   0                viewaxis,ypixelspan       {}
  pixel,size         1                viewaxis,zarrowspan       {}
  pixel,opaque       1                viewaxis,zpixelspan       {}
}
array set print_config {
    orient   landscape                tmargin   1.0
    paper    letter                   lmargin   1.0
    hpos     center                   pwidth    6.0
    vpos     center                   pheight   6.0
    units    in                       croptoview 1
}
\end{verbatim}
\end{codelisting}

\starsechead{Details}

The selection of vectors for display determined by the
\btn{Subsample} value differs depending on whether or not the data
lie on a regular grid\index{grid}.  If so, \btn{Subsample} takes integer
values and determines the ratio of data points to displayed points.  For
example, a value of 5 means that every fifth vector on the grid is
displayed.  This means that the number of vectors displayed is 25 times
fewer than the number of vectors on the grid.

For an irregular grid of vectors, an average cell size is computed,
and the \btn{Subsample} takes values in units of 0.1 times the
average cell size.  A square grid of that size is overlaid on the
irregular grid.  For each cell in the square grid, the data vector
from the irregular grid closest to the center of the square grid
cell is selected for display.  The vector is displayed at its true
location in the irregular grid, not at the center of the square
grid cell.  As the subsample rate changes, the set of displayed
vectors also changes, which can in some circumstances substantially
change the appearance of the displayed vector field.

%\starsechead{Using \app{mmDisp} as a WWW browser helper application}
%
%You may configure your web browser\index{application!web~browser} to
%automatically launch \app{mmDisp} when downloading an \OVF\ file.  The
%exact means to do this depends on your browser, but a couple of examples
%are presented below.
%
%In Netscape Navigator 4.X\index{application!Netscape}, bring up the
%\btn{Edit\pipe Preferences\ldots} dialog box, and select the
%\cd{Category} \btn{Navigator\pipe Applications} subwindow.  Create a
%\btn{New Type}, with the following fields:
%\begin{description}
%\item[Description of type:] \OOMMF\ Vector Field
%\item[MIME Type:] application/x-oommf-vf
%\item[Suffixes:] ovf,omf,ohf,obf
%\item[Application:]
%{\em tclsh} {\em oommfroot}/oommf.tcl {+fg} mmDisp {-net} 0 ``{\em arg}''
%\end{description}
%
%On \Windows\ platforms, the \cd{Suffixes} field is labeled
%\cd{File Extension}, and only one file extension may be entered.
%Files downloaded from a web server are handled according to their
%MIME Type, rather than their file extension, so that restriction
%isn't important when web browsing.  If you wish to have files on the
%local disk with all the above file extensions recognized as
%\OOMMF\ Vector Field\index{file!vector~field} files, you must repeat the
%\btn{New Type} entry for each file extension.  In the \cd{Application}
%field, the values of {\em tclsh}, {\em oommfroot}, and {\em arg} vary
%with your platform configuration.  The value of {\em tclsh} is the full
%path to the \app{tclsh} application on your platform (see
%\hyperrefhtml{Command Line Launching}{Ch.~}{}{sec:cll}).  On Unix
%systems, {\em tclsh} may be omitted, assuming that the
%\fn{oommf.tcl} script is executable.  If {\em tclsh} is not omitted
%on Unix systems, Netscape may issue a security warning each time it
%opens an \OOMMF\ Vector Field file.  The value of {\em oommfroot} should
%be the full path to the root directory of your \OOMMF\ installation.
%The value of {\em arg} should be ``\%1'' on \Windows\ and ``\%s'' on
%Unix.  The MIME type ``application/x-oommf-vf'' must be configured on
%any HTTP server which provides \OOMMF\ Vector Field files as well.
%
%For Microsoft Internet
%Explorer\index{application!Internet~Explorer}~3.X, bring up the
%\btn{View\pipe Options\ldots} dialog box, and select the
%\btn{Program} tab.  Hit the
%\btn{File Types\ldots}\ button, followed by the
%\btn{New Type\dots} button. Fill the resulting dialog box with
%\begin{description}
%\item[Description of type:] \OOMMF\ Vector Field
%\item[Associated extension:] ovf
%\item[Content type (MIME):] application/x-oommf-vf
%\end{description}
%You may also disable the \btn{Confirm open after download} checkbutton
%if you want.  Then hit the \btn{New\ldots}\ button below the
%{\cd{Actions:}} window, and in the pop-up fill in
%\begin{description}
%\item[Action:] open
%\item[Application used to perform action:]\ \\
%{\em tclsh} {\em oommfroot}/oommf.tcl {+fg} mmDisp {-net} 0 ``\%1''
%\end{description}
%Hit \btn{OK}, \btn{Close}, \btn{Close} and \btn{OK}.  Replace {\em tclsh}
%and {\em oommfroot} with the appropriate paths on your system
%(cf.\ \hyperrefhtml{Command Line Launching}{Ch.~}{}{sec:cll}).  This will
%set up an association on files with the .ovf extension. Internet
%Explorer 3.X apparently ignores the HTML Content Type field, so you must
%repeat this process for each file extension (.ovf, .omf, .ohf, .obf and
%.svf) that you want to recognize.  This means, however, that Internet
%Explorer will make the appropriate association even if the HTML server
%does not properly set the HTML Content Type field.
%
%Microsoft Internet Explorer 4.X is integrated with the \Windows\ operating
%system.  Internet Explorer 4.X doesn't offer any means to set up
%associations between particular file types and the applications which
%should be used to open them.  Instead, this association is configured
%within the \Windows\ operating system.
%To set up associations for the
%\OOMMF\ Vector Field file type on \Windows~95 or \Windows~NT,
%select \btn{Settings\pipe Control Panel} from the \btn{Start} menu.
%The Control Panel window appears.  Select \btn{View\pipe Options\ldots}
%to display a dialog box.
%A \Windows~98 shortcut to the same dialog box is to select
%\btn{Settings\pipe Folder Options\ldots} from the \btn{Start} menu.
%Select the \btn{File Types} tab and proceed as
%described above for Internet Explorer 3.X.
%Depending on the exact release/service patch of your \Windows\ operating
%system, the exact instructions may vary.
%
%Once you have your browser configured, you can test with the following
%URL:
%\begin{center}
%\ifnotpdf{\htmladdnormallink{https://math.nist.gov/\~{}MDonahue/cubevortex.ovf}{https://math.nist.gov/\~{}MDonahue/cubevortex.ovf}}
%\pdfonly{\htmladdnormallink{https://math.nist.gov/\~{}MDonahue/cubevortex.ovf}{https://math.nist.gov/\%7EMDonahue/cubevortex.ovf}}
%\end{center}

\starsechead{Known Bugs}
The slice selection feature does not work properly with irregular
meshes.
%
%

\chapter{Data Archive: mmArchive}\label{sec:mmarchive}%
\index{application!mmArchive}\index{data!save}

\begin{center}
\includepic{mmarchive-ss}{mmArchive Screen Shot}
\end{center}

\starsechead{Overview}
The application \app{mmArchive} provides automated vector field and data
table storage services.  Although
\hyperrefhtml{\app{mmDisp}}{\app{mmDisp}
(Ch.~}{)}{sec:mmdisp}\index{application!mmDisp}
and
\hyperrefhtml{\app{mmGraph}}{\app{mmGraph}
(Ch.~}{)}{sec:mmgraph}\index{application!mmGraph}
are able to save such data under the direction of the user, there are
situations where it is more convenient to write data to disk without
interactive control.

\app{mmArchive} does not present a user interface window of its own,
but like the {\hyperrefhtml{\app{Oxs solvers}}{\app{Oxs solvers}
(Ch.~}{)}{sec:oxs}}\index{application!Oxs}
relies on {\hyperrefhtml{\app{mmLaunch}}{\app{mmLaunch}
(Ch.~}{)}{sec:mmlaunch}}\index{mmLaunch~user~interface} to
provide an interface on its behalf.  Because \app{mmArchive} does not
require a window, it is possible on \Unix\ systems to bring down the X
(window) server\index{platform!Unix!X~server} and still keep
\app{mmArchive} running in the background.

\starsechead{Launching}
\app{mmArchive} may be started either by selecting the
\btn{mmArchive} button on \app{mmLaunch}
by \app{Oxsii/Boxsi} via a
\htmlonlyref{\cd{Destination}}{html:destinationCmd}
command in a 
{\hyperrefhtml{\MIF~2 file}{\MIF~2 file
(Sec.~}{)}{sec:mif2format}}\index{Destination~command~(MIF)},
or from the command line via
\begin{verbatim}
tclsh oommf.tcl mmArchive [standard options]
\end{verbatim}

When the \btn{mmArchive} button of \app{mmLaunch} is invoked,
\app{mmArchive} is launched with the {{\tt -tk 0}} option.
This allows \app{mmArchive} to continue running if the X
window server is killed.  The {{\tt -tk 1}} option is useful
only for enabling the {{\tt -console}} option for debugging.

As noted above, \app{mmArchive} depends upon
\app{mmLaunch}\index{mmLaunch~user~interface} to provide an interface.
The entry for an instance of \app{mmArchive} in the
\btn{Threads} column of any running copy of \app{mmLaunch} has a
checkbutton next to it.  This button toggles the presence of a user
interface window through which the user may control that instance of
\app{mmArchive}.

\starsechead{Inputs}
\app{mmArchive} accepts vector field and data table style input from
client applications (typically running solvers) on its network (socket)
interface.

\starsechead{Outputs}\index{data!save}
The client applications that send data to \app{mmArchive} control the
flow of data.  \app{mmArchive} copies the data it receives into files
specified by the client.  There is no interactive control to select the
names of these output files.  A simple status line shows the most recent
vector file save, or data table file open/close event.

For data table output, if the output file already exists then the new
data is appended to the end of the file.  The data records for each
session are sandwiched between ``Table Start'' and ``Table End''
records.  See the \hyperrefhtml{ODT format documentation}{ODT format
documentation (Ch.~}{)}{sec:odtformat}\index{file!data~table} for
explanation of the data table file structure.  It is the
responsibility of the user to insure that multiple data streams are
not directed to the same data table file at the same time.

For vector field output, if the output file already exists then the
old data is deleted and replaced with the current data.  See the
\hyperrefhtml{OVF documentation}{OVF documentation
(Ch.~}{)}{sec:vfformats}\index{file!vector~field} for information
about the vector field output format.


\starsechead{Controls}

The display area inside the \app{mmArchive} window displays a log of
\app{mmArchive} activity.  The menu selection \btn{File\pipe Close
interface} closes the \app{mmArchive} window without terminating
\app{mmArchive}.  Use the \btn{File\pipe Exit mmArchive} option or the
window close button to terminate \app{mmArchive}.

If the \btn{Options\pipe Wrap lines} option is selected, then each log
entry is line wrapped.  Otherwise, each entry is on one line, and a
horizontal slider is provided at the bottom of the display window to
scroll through line.  The \btn{Options\pipe Clear buffer} command
clears the log display area.  This clears the buffer in that
\app{mmArchive} display window only.  If a new display window is
opened for the same \app{mmArchive} instance, the new display will
show the entire log backing store.  The last two items on the
\btn{Options} menu,  \btn{Enlarge font} and \btn{Reduce font}, adjust
the size of the font used in the log display area.

\starsechead{Known Bugs}
\app{mmArchive} appends data table output to the file specified by the
source client application (e.g., a running solver).  If, at the same
time, more than one source specifies the same file, or if the the same
source sends data table output to more than one instance of
\app{mmArchive}, then concurrent writes to the same file may corrupt the
data in that file.  It is the responsibility of the user to ensure this
does not happen; there is at present no file locking mechanism in
\OOMMF\ to protect against this situation.

\chapter{Documentation Viewer: mmHelp}\label{sec:mmhelp}
\index{application!mmHelp}

\begin{center}
\includepic{mmhelp-ss}{mmHelp Screen Shot}
\end{center}

\starsechead{Overview}
The application {\bf mmHelp} manages the display and navigation of
hypertext (HTML) help files\index{file!HTML}.  
%It presents an interface similar
%to that of World Wide Web browsers\index{application!web~browser}.
%
%Although {\bf mmHelp} is patterned after World
%Wide Web browsers, it does not have all of their capabilities.
{\bf mmHelp} displays only a simplified form of hypertext
required to display the \OOMMF\ help pages.
It is not able to display many of
the advanced features provided by modern World Wide Web
browsers.  
%In the current release, {\bf mmHelp} is not able to 
%follow {\tt http:} URLs.  It only follows {\tt file:} URLs\index{URL},
%such as
%\begin{center}
%file:/path/to/oommf/doc/userguide/userguide/Documentation\_Viewer\_mmHelp.html
%\end{center}
\OOMMF\ software can be \hyperrefhtml{customized}{customized (See
Sec.~}{)}{sec:install.custom} to use another program to display the HTML
help files\index{customize!help~file~browser}.

\starsechead{Launching}
{\bf mmHelp} may be launched from the command line via
\begin{verbatim}
tclsh oommf.tcl mmHelp [standard options] [URL]
\end{verbatim}
The command line argument {\tt URL} is the URL of the first
page (home page) to be displayed.  If no URL is specified,
{\bf mmHelp} displays the Table of Contents of the {\em \OOMMF\ User's
Guide} by default.

\starsechead{Controls}
Each page of hypertext is displayed in the main {\bf mmHelp} window.
Words which are underlined and colored blue are hyperlinks which {\bf
mmHelp} knows how to follow.  Words which are underlined and colored red
are hyperlinks which {\bf mmHelp} does not know how to follow.  Moving
the mouse over a hyperlink displays the target URL of the hyperlink in
the \btn{Link:} line above the display window.  Clicking on a blue
hyperlink will follow the hyperlink and display a new page of hypertext.

{\bf mmHelp} keeps a list of the viewed pages in order of view.
Using the \btn{Back} and \btn{Forward} buttons, 
the user may move backward and forward through 
this list of pages.
The \btn{Home} button causes the first page to be displayed, 
allowing the user to start again from the beginning.  These
three buttons have corresponding entries in the 
\btn{Navigate} menu.

Use the menu selection \btn{File\pipe Open} to directly select
a file from the file system to be displayed by {\bf mmHelp}.

The menu selection \btn{File\pipe Refresh}, or 
the \btn{Refresh} button causes {\bf mmHelp} to reload and
redisplay the current
page.  This may be useful if the display becomes corrupted,
or for repeatedly loading a hypertext file which is being
edited.

When {\bf mmHelp} encounters hypertext elements it does not
recognize, it will attempt to work around the problem.  
However, in some cases it will not be able to make sense of 
the hypertext, and will display an error message.  Documentation
authors should take care to use only the hypertext elements
supported by {\bf mmHelp} in their documentation files.  Users
should never see such an error message.

{\bf mmHelp} displays error messages in one of two ways: 
within the display window, or in a separate window.  
Errors reported in the display window replace the display 
of the page of hypertext.
They usually indicate that the hypertext page could not be
retrieved, or that its contents are not hypertext.  File
permission errors are also reported in this way.

Errors reported in a separate window are usually due to a
formatting error within the page of hypertext.  Selecting the 
\btn{Continue} button of the error window instructs {\bf mmHelp} to 
attempt to resume display of the hypertext page beyond the error.  
Selecting \btn{Abort} abandons further display.  

The menu selection \btn{Options\pipe Font scale...} brings up a
dialog box through which the user may select the scale of
the fonts to use in the display window, relative to their
initial size.

\blackhole{
The menu selection \btn{Options\pipe Strict} should not
normally be enabled.  Enabling it causes {\bf mmHelp} to
become much more finicky about the correctness of the HTML
in the pages it displays.  The main consequence of enabling
this option is that {\bf mmHelp} will display more error
messages.  This option is primarily useful when \OOMMF\
documentation writers are writing HTML files directly,
and wish to check their work.
} % end-blackhole

The menu selection \btn{File\pipe Exit} or the \btn{Exit} button
terminates the {\bf mmHelp} application.  
The menu \btn{Help} provides the usual help facilities.

\starsechead{Known Bugs}
\app{mmHelp} is pretty slow.  You may be happier using
\hyperrefhtml{local customization}{local
customization (Sec.~}{)}{sec:install.custom} methods to replace it
with another more powerful HTML browser.  Also, we have noticed that the
underscore character in the italic font is not displayed (or is
displayed as a space) at some font sizes on some platforms.


\chapter{Command Line Utilities}\label{sec:cmdutils}

This section documents a few utilities distributed with \OOMMF\ that are
run from the command line (\Unix\ shell or \Windows\ \DOS\ prompt).
They are typically used in pre- or post-processing of data associated
with a micromagnetic simulation.

%%%%%%%%%%%%%%%%%%%%%%%%%%%%%%%%%%%%%%%%%%%%%%%%%%%%%%%%%%%%%%%%%%%%%%%%

\section{Bitmap File Format Conversion:
          any2ppm}\label{sec:any2ppm}%
\index{file!bitmap}\index{file!ppm}\index{file!bmp}\index{file!gif}%
\index{application!any2ppm}\index{file!conversion}
The \app{any2ppm} program converts bitmap files from the Portable Pixmap
(PPM), Windows BMP, and GIF formats into the Portable Pixmap P3
(text) or P6 (binary) formats, or the uncompressed 24 bits-per-pixel BMP
binary format.  With \Tcl/\Tk\ 8.6 or later
the \textit{Portable Network Graphics} (PNG) format is also
supported.  Additional formats may be available if the \Tcl/\Tk\
\htmladdnormallinkfoot{\textit{Img}}{https://wiki.tcl-lang.org/page/Img}
package is installed on your system.  (Note: \OOMMF\ support for BMP
requires \Tk\ 8.0\index{requirement!Tk~8.0+} or later.)

\starssechead{Launching}
The \app{any2ppm} launch command is:
\begin{verbatim}
tclsh oommf.tcl any2ppm [standard options] [-f] [-format fmt] \
   [-noinfo] [-o outfile] [infile ...]
\end{verbatim}
where
\begin{description}
\item[\optkey{-f}]
  Force output.  If the \cd{-o} option is not specified, then the output
  filename is automatically generated by stripping the extension, if
  any, off of each input filename, and appending a format-specific
  extension (e.g., \fn{.ppm}).  If \cd{-f} is specified, that generated
  filename is used for the output filename.  If \cd{-f} is not
  specified, then a check is made to see if the generated filename
  already exists.  If so, then an additional ``-000'' or ``-001''
  \ldots\ suffix is appended to create an unused filename.  If the input
  is coming from stdin, i.e., there is no input filename, then the
  default output is to stdout.
\item[\optkey{-format fmt}]
  Output file format.  The default is \cd{PPM} or \cd{P3} which is the
  Portable Pixmap P3 (text) format; use \cd{P6} to get the binary PPM P6
  output.  Setting \texttt{fmt} to \cd{BMP} will produce files in the
  uncompressed Windows BMP\index{file!bmp} 24 bits-per-pixel format.
  Under \Tcl/\Tk\ 8.6 and later the \textit{Portable Network Graphics}
  format can be selected by setting \texttt{fmt} to \cd{PNG}.  If the
  \Tcl/\Tk\ Img package is installed, then additional formats, such as
  PNG (for pre-\Tcl/\Tk\ 8.6), JPEG and TIFF, will be available.  The
  default output file extension depends on the format selected, e.g.,
  \fn{.ppm} for PPM files and \fn{.bmp} for BMP files.
\item[\optkey{-noinfo}]
  Suppress writing of progress information to stderr.
\item[\optkey{-o outfile}]
  Write output to {\tt outfile}; use ``-'' to pipe output to stdout.
  Note that if {\tt outfile} is specified, then all output will go to
  this one file; in this case it is unlikely that one wants to specify
  more than one input file.
\item[\optkey{infile \ldots}]
  List of input files to process.  If none, or if an infile is the empty
  string, then read from stdin.
\end{description}

\textbf{Note:} If the output is to stdout, and the selected output
format is anything other than \cd{PPM}, then the output is first written
to a temporary file before being copied to stdout.  Under normal
operation the temporary file will be automatically deleted, but this is
not guaranteed if the program terminates abnormally.

\textbf{\Tk\ Requirement:}\index{requirement!Tk} \app{any2ppm} uses the
\Tk\ \cd{image} command in its processing.  This requires that \Tk\ be
properly initialized, which in particular means that a valid display
must be available.  This is not a problem on \Windows, where a desktop
is always present, but on \Unix\ this means that an \X\ server must be
running.  The
\htmladdnormallinkfoot{\textit{Xvfb}}{%
https://www.x.org/archive/X11R7.6/doc/man/man1/Xvfb.1.xhtml}\index{application!Xvfb}
virtual framebuffer can be used if desired.  (Xvfb is an X server
distributed with X11R6 that requires no display hardware or physical
input devices.)

%%%%%%%%%%%%%%%%%%%%%%%%%%%%%%%%%%%%%%%%%%%%%%%%%%%%%%%%%%%%%%%%%%%%%%%%

\section{Making Data Tables from Vector Fields:
          avf2odt}\label{sec:avf2odt}%
\index{file!vector~field}\index{file!data~table}\index{file!vio}%
\index{application!avf2odt}\index{file!conversion}
The \app{avf2odt} program converts \textit{rectangularly} meshed vector field
files in any of the
\hyperrefhtml{recognized formats}{recognized formats (\OVF, \VIO; see
  Ch.~}{)}{sec:vfformats}\HTMLoutput{ (\OVF, \VIO)}
into the \hyperrefhtml{\ODT~1.0}{\ODT~1.0
(Ch.~}{)}{sec:odtformat} data table format.  (Irregular meshes are
not supported by this command.  Note that any \OVF\ file using the
``irregular'' meshtype is considered to be using an irregular mesh, even
if the mesh nodes do in fact lie on a rectangular grid.)

\starssechead{Launching}
The \app{avf2odt} launch command is:
\begin{verbatim}
tclsh oommf.tcl avf2odt [standard options] \
   [-average <space|plane|line|point|ball>] [-axis <x|y|z>] \
   [-ball_radius brad] [-defaultpos <0|1>] [-defaultvals <0|1>] \
   [-extravals flag] [-filesort method] [-headers <full|collapse|none>] \
   [-index label units valexpr] [-ipat pattern] [-normalize <0|1>] \
   [-numfmt fmt] [-onefile outfile] [-opatexp regexp] [-opatsub sub] \
   [-region xmin ymin zmin xmax ymax zmax] \
   [-rregion rxmin rymin rzmin rxmax rymax rzmax] \
   [-truncate <0|1>] [-v level] [-valfunc label units fcnexpr] \
   [infile ...]
\end{verbatim}
where
\begin{description}
\item[\optkey{-average \boa space\pipe plane\pipe line\pipe point\pipe
    ball\bca}]
  Specify type of averaging.  Selection of \cd{Space} averaging
  results in the output of one data line (per input file) consisting
  of the average $v_x$, $v_y$ and $v_z$ field values in the selected
  region (see \cd{-region} option below).  For example, in
  magnetization files, $v_x$, $v_y$ and $v_z$ correspond to $M_x$,
  $M_y$ and $M_z$.  If \cd{plane} or \cd{line} is selected, then the
  output data table consists of multiple lines with 4 or 5 columns per
  line, respectively.  The last 3 columns in both cases are the $v_x$,
  $v_y$ and $v_z$ averaged over the specified axes-parallel affine
  subspace (i.e., plane or line).  In the \cd{plane} case, the first
  column specifies the averaging plane offset along the coordinate
  axis normal to the plane (see \cd{-axis} option below).  In the
  \cd{line} case, the first 2 columns specify the offset of the
  averaging line in the coordinate plane perpendicular to the line.
  If the averaging type is \cd{point}, then no averaging is done, and
  the output consists of lines of 6 column data, one line for each
  point in the selected region, where the first 3 columns are the
  point coordinates, and the last 3 are the $v_x$, $v_y$ and $v_z$
  values at the point.  If the type is \cd{ball}, then one line is
  output for each sample point for which a ball of radius \cd{brad}
  (see \cd{-ball\_radius} option) centered about that point lies
  entirely inside the selected region.  The output values consist of 6
  columns: the ball center point location and the $v_x$, $v_y$ and
  $v_z$ values averaged across the ball.  As a special case, if the
  spatial extent of the selected region is two-dimensional (e.g., all
  the sample locations have the same $z$-coordinate), then the
  averaging region is taken to be a disk instead of a ball.
  Similarly, if the spatial extent of the selected region is
  one-dimensional, then the averaging region is reduced to a
  one-dimensional line segment.  (Note: The output columns described
  above may be suppressed by the \cd{-defaultpos} and
  \cd{-defaultvals} options.  Additional columns may be introduced by
  the \cd{-index} and \cd{-valfunc} options.)  The default averaging
  type is \cd{space}.
\item[\optkey{-axis \boa x\pipe y\pipe z\bca}]
  For the \cd{-average plane} and \cd{-average line} averaging types,
  selects which subset of affine subspaces the averaging will be
  performed over.  In the \cd{plane} case, the \cd{-axis} represents
  the normal direction to the planes, while for \cd{line} it is the
  direction parallel to the lines.  This parameter is ignored if
  \cd{-average} is not either \cd{plane} or \cd{line}.  Default value
  is \cd{x}.
\item[\optkey{-ball\_radius brad}]
  This option is required if \cd{-average} is \cd{ball}, in which case
  \cd{brad} specifies the radius of the averaging ball in problem
  units (e.g., meters).  If  \cd{-average} is not \cd{ball}, then this
  option is ignored.
\item[\optkey{-defaultpos \boa 0\pipe 1\bca}]
  By default, the output data columns are as described in the
  description of the \cd{-average} option above.  However,
  \cd{-defaultpos 0} may be used to omit the columns indicating the
  averaging position.
\item[\optkey{-defaultvals \boa 0\pipe 1\bca}]
  By default, the output data columns are as described in the
  description of the \cd{-average} option above.  However,
  \cd{-defaultvals 0} may be used to omit the columns containing the
  averaged $v_x$, $v_y$ and $v_z$ values.  In particular, this may be
  useful in conjunction with the \cd{-valfunc} option.
\item[\optkey{-extravals \boa 0\pipe 1\bca}]
  Specify \cd{-extravals 1} to augment the output with columns
  for the average $L^1$ norm
  $\sum\left(|v_x| + |v_y| + |v_z|\right)/N$, the normalized $L^2$
  norm $\sqrt{\sum v^2/N}$, the minimum component absolute value,
  and the maximum component absolute value.
\item[\optkey{-filesort method}]
  Specifies the sorting order to apply to the input file list.  This
  order is important when using the \cd{-onefile} option, since it
  controls the order in which the rows from the various input files
  are concatenated.  Method should be either the keyword ``none'', or
  else a valid option string for the \Tcl\ command \cd{lsort}, e.g.,
  ``-ascii -decreasing''.  Note that the \cd{lsort} sort options all
  begin with a hyphen, ``-'', and that if you want to use multiple
  options they must be grouped as one element to \cd{filesort} (by,
  for example, placing quotes around the list).  The default value is
  ``-dictionary'' if the \cd{-ipat} option is specified, or ``none''
  otherwise.
\item[\optkey{-headers \boa full\pipe collapse\pipe none\bca}]
  Determines the style of headers written to the output \ODT\ file(s).
  The full style (default) provides the standard headers, as described
  in the \hyperrefhtml{\ODT\ documentation}{\ODT\ documentation
  (Ch.~}{)}{sec:odtformat}.  Specifying ``none'' produces raw data
  lines without any headers.  The collapse style is used with multiple
  input files and the \cd{-onefile} output option to concatenate
  output with no \ODT\ header information between the segments.
\item[\optkey{-index label units valexpr}]
  Adds an input file based index column to the output, where label is
  the column header, units is a string displayed as the column units
  header, and valexpr is a \Tcl\ \cd{expr} expression that may include
  the special variables \cd{\$i}, \cd{\$f1}, \cd{\$f2}, \ldots, \cd{\$d1},
  \cd{\$d2}, \ldots; here \cd{\$i} is the 0-based index of
  the file in the list of input files, \cd{\$f1} is the first number
  appearing in the input filename, \cd{\$f2} is the second number
  appearing in the input filename, \cd{\$d1} is the first number
  appearing in the ``Desc'' fields in the header of the input file,
  etc.  For example, if there are two input files named
  \fn{foo-100.ovf} and and \fn{foo-101.ovf}, then setting valexpr to
  \cd{abs(\$f1)+1} would yield a column with the value 101 for all lines
  coming from \fn{foo-100.ovf}, and the value 102 for all lines coming
  from  \fn{foo-101.ovf}.  (We use the \Tcl\ \cd{expr} function
  \cd{abs} because the leading hyphen in \fn{foo-100.ovf} gets
  interpreted as a minus sign, so \cd{\$f1} is extracted as -100.)
  On \Unix\ systems, the valexpr string should be surrounding by single
  quotes in order to forestall interpolation of the special variables
  by the shell.  On Windows, the valexpr string should be surrounded
  by double quotes as usual to protect embedded spaces.
  Multiple instances of the \cd{-index} option on the command line
  will result in multiple columns in the output file, in the order
  specified.  The index columns, if any, will be the first columns in
  the output file.
\item[\optkey{-ipat pattern}]
  Specify input files using a pattern with ``glob-style'' wildcards.
  Especially useful in \DOS.  Files must meet the \cd{infile}
  requirements (see below).
\item[\optkey{-normalize \boa 0\pipe 1\bca}]
  If 1, then the default averaged output values $v_x$, $v_y$ and $v_z$
  are divided by the maximum magnitude that would occur if all the
  vectors in the averaging manifold are aligned.  (In particular, the
  maximum magnitude of the output vector is 1.)  This option should
  be used carefully because the normalization is done independently for
  each output row.  For \cd{-normalize 0}
  (the default), averaged output values are in file units.
\item[\optkey{-numfmt fmt}]
  C-style output format for numeric data in the body of the output
  table.  Default value is ``\verb+%- #20.15g+''.
\item[\optkey{-onefile outfile}]
  Generally a \app{avf2odt} writes its output to a collection of files
  with names generated using the \cd{-opatexp} and \cd{-opatsub}
  specifications.  This option overrides that behavior and sends all
  output to one place.  If outfile is ``-'', then the output is sent
  to standard output, otherwise outfile is the name of the output file.
\item[\optkey{-opatexp regexp}]
  Specify the ``regular expression'' applied to input filenames to
  determine portion to be replaced in generation of output filenames.
  The default regular expression is: {\verb!(\.[^.]?[^.]?[^.]?$|$)!}
\item[\optkey{-opatsub sub}]
  The string with which to replace the portion of input filenames
  matched by the {\tt -opatexp regexp} during output filename
  generation.  The default is {\verb!.odt!}.
\item[\optkey{-region xmin ymin zmin xmax ymax zmax}]
  Axes-parallel rectangular box denoting region in the vector field
  file over which data is to be collected.  The locations are in
  problem units (typically meters).  A single hyphen, ``-'', may be
  specified for any of the box corner coordinates, in which case the
  corresponding extremal value from the input file is used.  Optional;
  the default, \cd{-region - - - - - -}, selects the entire input file.
\item[\optkey{-rregion rxmin rymin rzmin rxmax rymax rzmax}]
  This option is the same as \cd{-region}, except that the locations
  are specified in relative units, between 0 and 1.
\item[\optkey{-truncate \boa 0\pipe 1\bca}]
  When opening an existing file for output, the new output can either be
  appended to the file (\cd{-truncate 0}), or else the existing data
  can be discarded (\cd{-truncate 1}).  The default is \cd{-truncate 0}.
\item[\optkey{-v level}]
  Verbosity (informational message) level, with 0 generating only
  error messages, and larger numbers generating additional information.
  The {\tt level} value is an integer, defaulting to 1.
\item[\optkey{-valfunc label units fcnexpr}]
  Similar to the \cd{-index} option, \cd{-valfunc} adds an additional
  column to the output with label and units as the column header, and
  fcnexpr is a \Tcl\ \cd{expr} expression that may include special
  variables.  Here, however, the allowed special variables are
  \cd{\$x}, \cd{\$y}, \cd{\$z}, \cd{\$r},
  \cd{\$vx}, \cd{\$vy}, \cd{\$vz}, \cd{\$vmag}, where
  \cd{\$x}, \cd{\$y}, \cd{\$z}, and \cd{\$r} are sample location and
  magnitude, respectively ($r = \sqrt{x^2+y^2+z^2}$), and
  \cd{\$vx}, \cd{\$vy}, \cd{\$vz} and \cd{\$vmag} are vector component
  values and magnitude.  The output is the value of fcnexpr averaged
  across the manifold selected by the \cd{-average} option.  A couple
  of examples are
\begin{verbatim}
   -valfunc Ms   A/m '$vmag'
   -valfunc M110 A/m '($vx+$vy)/sqrt(2.)'
\end{verbatim}
  As with the valexpr string for \cd{-index}, the fcnexpr string
  should be surrounding by single quotes on \Unix\ in order to
  forestall interpolation of the special variables by the shell.  On
  Windows, the fcnexpr string should be surrounded by double quotes as
  usual to protect embedded spaces.
  The output value is not affected by the \cd{-normalize} option.
  Multiple instances of the
  \cd{-valfunc} option on the command line will result in multiple
  columns in the output file, in the order specified.  These
  additional columns will be append to the right of all other columns in
  the output file.
\item[\optkey{infile \ldots}]
  Input file list.  Files must be one of the recognized
  formats,  \OVF\ 1.0 or \VIO, in a rectangular mesh subformat.
\end{description}

The file specification options require some explanation.  Input files
may be specified either by an explicit list (\cd{infile ...}),
or by giving a wildcard pattern, e.g., \cd{-ipat *.omf}, which is
expanded in the usual way by \app{avf2odt} (using the \Tcl\ command
\cd{glob}).   \Unix\ shells (sh, csh, etc.) automatically expand
wildcards before handing control over to the invoked application, so the
\cd{-ipat} option is not usually needed---although it is useful in case of a
``command-line too long'' error.  \DOS\ does not do this expansion, so
you must use \cd{-ipat} to get wildcard
expansion\index{platform!Windows!wildcard~expansion} in \Windows.
The resulting file list is sorted  based on the \cd{-filesort}
specification as described above.

If \cd{-onefile} is not requested, then as each input file is
processed, a name for the corresponding output file is produced from
the input filename by rules determined by handing the \cd{-opatexp}
and \cd{-opatsub} expressions to the \Tcl\ \cd{regsub} command.  Refer
to the \Tcl\ \cd{regsub} documentation for details, but essentially
whatever portion of the input filename is matched by the \cd{-opatexp}
expression is removed and replaced by the \cd{-opatsub} string.  The
default \cd{-opatexp} expression matches against any filename
extension of up to 3 characters, and the default \cd{-opatsub} string
replaces this with the extension \fn{.odt}.

%%%%%%%%%%%%%%%%%%%%%%%%%%%%%%%%%%%%%%%%%%%%%%%%%%%%%%%%%%%%%%%%%%%%%%%%

\section{Vector Field File Format Conversion:
          avf2ovf}\label{sec:avf2ovf}%
\index{file!vector~field}\index{file!vio}%
\index{application!avf2ovf}\index{file!conversion}
The \app{avf2ovf} program converts vector field files from any of the
\hyperrefhtml{recognized formats}{recognized formats (\OVF, \VIO; see
  Ch.~}{)}{sec:vfformats}\HTMLoutput{ (\OVF, \VIO)}
into the \OOMMF\ \OVF\ or the Python NumPy NPY format.

\starssechead{Launching}
The \app{avf2ovf} launch command is:
\begin{verbatim}
tclsh oommf.tcl avf2ovf [standard options] \
   [-clip xmin ymin zmin xmax ymax zmax] [-dataformat <text|b4|b8>] \
   [-fileformat <ovf|npy> version] [-flip flipstr] [-grid <rect|irreg>] \
   [-info] [-keepbb] [-mag] [-pertran xoff yoff zoff] [-q] \
   [-resample xstep ystep zstep order] [-rpertran rxoff ryoff rzoff] \
   [-subsample period] [infile [outfile]]
\end{verbatim}
where
\begin{description}
\item[\optkey{-clip xmin ymin zmin xmax ymax zmax}]
  The 6 arguments specify the vertices of a bounding clip box.  Only mesh
  points inside the clip box are brought over into the output file.  Any
  of the arguments may be set to ``-'' to use the corresponding value
  from the input file, i.e., to not clip along that box face.
\item[\optkey{-dataformat \boa text\pipe b4\pipe b8\bca}]
  Specify output data format, either ASCII text (\cd{text}), 4-byte
  binary (\cd{b4}), or 8-byte binary (\cd{b8}). For \OOMMF\ \OVF\ output
  files, the default is text (note that the \OVF\ format has an ASCII
  text header in all cases). For Python NumPy NPY output files the
  default is 8-byte binary. For \OOMMF\ \OVF\ version 2 output, the text
  option can additionally include a C-style printf format string, e.g.,
  \verb|-dataformat "text %16.12e"| (note the quotes to keep this a single
  argument to \cd{-dataformat}).
\item[\optkey{-fileformat \boa ovf\pipe npy\bca\ version}]
  Specify the output file format and version, either
  \OOMMF\ \OVF\ version 1 (default) or 2, or the Python NumPy array
  file format version 1.
\item[\optkey{-flip flipstr}]
  Provides an axis coordinate transformation.  Flipstr has the form
  A:B:C, where A, B, C is a permutation of $x$, $y$, $z$, with an
  optional minus sign on each entry.  The first component A denotes the
  axis to which $x$ is mapped, B where $y$ is mapped, and $C$ where $z$
  is mapped.  The default is the identity map, {\tt x:y:z}.  To rotate
  $90^\circ$ about the $z$-axis, use ``-flip y:-x:z'', which sends $x$
  to the $+y$ axis, $y$ to the -$x$ axis, and leaves $z$ unchanged.
\item[\optkey{-grid \boa rect\pipe irreg\bca}]
  Specify output grid\index{grid} structure.  The default is \cd{rect},
  which will output a regular rectangular grid if the input is recognized
  as a regular rectangular grid.  The option ``-grid irreg'' forces
  irregular mesh style output.
\item[\optkey{-info}]
  Instead of converting the file, print information about the file, such
  as size, range, and descriptive text from the file header.
\item[\optkey{-keepbb}]
  If the \cd{-clip} option is used, then normally the spatial extent,
  i.e., the boundary, of the output is clipped to the specified clip
  box.  If \cd{-keepbb} (keep bounding box) is given, then the spatial
  extent of the output file is taken directly from the input file.
  Clipping is still applied to the data vectors; \cd{-keepbb} affects
  only the action of the clip box on the boundary.
\item[\optkey{-mag}]
  Write out a scalar valued field instead of a vector value field, where
  the scalar values are the magnitude $|v(r)|$ of the vector values at
  each point $r$. This option is only supported for \OOMMF\ \OVF\ version
  2 output.
\item[\optkey{-pertran xoff yoff zoff}]
  Translates field with respect to location coordiates, by amount
  $(\mathit{xoff},\mathit{yoff},\mathit{zoff})$, in a periodic
  fashion.  For example, if
  $(\mathit{xoff},\mathit{yoff},\mathit{zoff})$ is $(\mbox{50e-9},0,0)$,
  then a vector $v$ at position
  $(\mathit{rx},\mathit{ry},\mathit{rz})$
  in the original file is positioned instead at
  $(\mathit{rx} + \mbox{50e-9},\mathit{ry},\mathit{rz})$
  in the output file.  If the spatial extent of the
  $x$ coordinate in the input file runs from $\mathit{xmin}$ to
  $\mathit{xmax}$, and if
  $\mathit{rx} +\mbox{50e-9}$ is larger than $\mathit{xmax}$, then $v$
  will be placed at
  $\mathit{rx} + \mbox{50e-9} - \mathit{xmax} + \mathit{xmin}$
  instead.  Translations are rounded to the
  nearest full step; aside from any clipping, the output file has the
  exact same spatial extent and sample locations as the original file.
  If both translation and clipping are requested, then the clipping is
  applied after the translation.
\item[\optkey{-q}]
  Quiet operation --- don't print informational messages.
\item[\optkey{-resample xstep ystep zstep \boa 0\pipe 1\pipe 3\bca}]
  Resample grid using specified step sizes.  Each step size must exactly
  divide the grid extent in the corresponding direction, after any
  clipping.  (That is, the export mesh consists of full cells only.)
  The last argument specifies the polynomial interpolation order: 0 for
  nearest value, 1 for trilinear interpolation, or 3 for fitting with
  tricubic Catmull-Rom splines.  This control is only available for
  input files having a rectangular grid structure.  Default is no
  resampling.
\item[\optkey{-rpertran rxoff ryoff rzoff}]
  Similar to -pertran, except the offsets
  $(\mathit{rxoff},\mathit{ryoff},\mathit{rzoff})$ are
  interpreted as offsets in the range $[0,1]$ taken relative to the
  spatial extents of the $x$, $y$, and $z$ coordinates.  For example, if
  $\mathit{xmax} - \mathit{xmin} = \mbox{500e-9}$, then an
  $\mathit{rxoff}$ value of 0.1 is equivalent
  to an $\mathit{xoff}$ value of 50e-9.
\item[\optkey{-subsample period}]
  Reduce point density in output by subsampling input with specified
  period along $x$, $y$, and $z$ axes.  For example, if period is 2,
  then the output will have only 1/8th as many points as the input.
  This control is only available for input files having a rectangular
  grid structure.  Default value is 1, i.e., no subsampling.
\item[\optkey{infile}]
  Name of input file to process.  Must be one of the recognized
  formats, \OVF\ 0.0, \OVF\ 1.0, \OVF\ 2.0, or \VIO.  If no file is
  specified, reads from stdin.
\item[\optkey{outfile}]
  Name of output file.  If no file is specified, writes to stdout.
\end{description}

There are also two recognized but deprecated options,
\optkey{-format} and \optkey{-ovf}. The former is replaced by
\optkey{-dataformat} and the latter superceded by \optkey{-fileformat}.

The \cd{-clip} option is useful when one needs to do analysis on a
small piece of a large simulation.  The \cd{-info} option is helpful
here to discover the extents of the original mesh.  The \cd{-clip}
option can also be used with \cd{-resample} to enlarge the mesh.

The \cd{-flip} option can be used to align different simulations
to the same orientation.  It can also be used to change a file into its
mirror image; for example, ``-flip~-x:y:z'' reflects the mesh through
the $yz$-plane.

If multiple operations are specified, then the operation order is
clip, resample, subsample, flip, and translate.

The \cd{-dataformat text} and \cd{-grid irreg} options are handy for
preparing files for import into non-\OOMMF\ applications, since all
non-data lines are readily identified by a leading ``\verb+#+,'' and
each data line is a 6-tuple consisting of the node location and vector
value.  Pay attention, however, to the scaling of the vector value as
specified by ``\verb+# valueunit+'' and ``\verb+# valuemultiplier+''
header lines (OVF version 1 only).

For output format details, see the \hyperrefhtml{OVF file
description}{OVF file description (Sec.~}{,
page~\pageref{sec:ovfformat})}{sec:ovfformat}.


\starssechead{Known Bugs}
If the input file contains an explicit boundary polygon (cf.\ the
\cd{boundary} entry in the \htmlonlyref{Segment Header
block}{sec:ovfsegmentheader} subsection of the {\hyperrefhtml{OVF file
description}{OVF file description, Sec.~}{}{sec:ovfformat}}) then the
output file will also contain an explicit boundary polygon.  If clipping
is active, then the output boundary polygon is formed by moving the
vertices from the input boundary polygon as necessary to bring them into
the clipping region.  This is arguably not correct, in particular for
boundary edges that don't run parallel to a coordinate axis.


%%%%%%%%%%%%%%%%%%%%%%%%%%%%%%%%%%%%%%%%%%%%%%%%%%%%%%%%%%%%%%%%%%%%%%%%

\section{Making Bitmaps from Vector Fields:
            avf2ppm}\label{sec:avf2ppm}%
\index{file!bitmap}\index{file!vector~field}\index{file!conversion}%
\index{animations}

The \app{avf2ppm}\index{application!avf2ppm} utility converts a
collection of vector field files (e.g., \fn{.omf}, \fn{.ovf}) into color
bitmaps suitable for inclusion into documents or collating into movies.
The command line arguments control filename and format selection, while
plain-text configuration files, modeled after the
{\hyperrefhtml{\app{mmDisp}}{\app{mmDisp}
(Ch.~}{)}{sec:mmdisp}\index{application!mmDisp}} configuration dialog
box, specify display parameters.

\starssechead{Launching}
The \app{avf2ppm} launch command is:
\begin{verbatim}
tclsh oommf.tcl avf2ppm [standard options] [-config file] [-f] \
   [-filter program] [-format <P3|P6|B24|PNG>] [-ipat pattern] \
   [-opatexp regexp] [-opatsub sub] [-v level] [infile ...]
\end{verbatim}
where
\begin{description}
\item[\optkey{-config file}]
  User configuration file that specifies image display parameters.  This
  file is discussed in \htmlonlyref{detail below}{sec:avf2ppmconfig}.
\item[\optkey{-f}]
  Force overwriting of existing (output) files.  By default, if
  \app{avf2ppm} tries to create a file, say \fn{foo.ppm}, that already
  exists, it generates instead a new name of the form \fn{foo.ppm-000},
  or \fn{foo.ppm-001}, \ldots, or \fn{foo.ppm-999}, that doesn't exist
  and writes to that instead.  The {\tt -f} flag disallows alternate
  filename generation, and overwrites \fn{foo.ppm} instead.
\item[\optkey{-filter program}]
  Post-processing application to run on each \app{avf2ppm} output
  file.  May be a pipeline of several programs.
\item[\optkey{-format \boa P3\pipe P6\pipe B24\pipe PNG\bca}]
  Specify the output image file format.  Currently supported formats are
  the true color \textit{Portable Pixmap} (PPM)\index{file!ppm} formats P3
  (ASCII text) and P6 (binary), the uncompressed Windows
  BMP\index{file!bmp} 24 bits-per-pixel format, and the compressed
  \textit{Portable Network Graphics} (PNG) format.  Conversion to the
  PNG format requires either \Tk\ 8.6+, or earlier \Tk\ plus the
  \htmladdnormallinkfoot{\textit{Img}}{https://wiki.tcl-lang.org/page/Img}
  package.  The default format is P6.
\item[\optkey{-ipat pattern}]
  Specify input files using a pattern with ``glob-style'' wildcards.
  Mostly useful in \DOS.
\item[\optkey{-opatexp regexp}]
  Specify the ``regular expression'' applied to input filenames to
  determine portion to be replaced in generation of output filenames.
  The default regular expression is: {\verb!(\.[^.]?[^.]?[^.]?$|$)!}
\item[\optkey{-opatsub sub}]
  The string with which to replace the portion of input filenames
  matched by the {\tt -opatexp regexp} during output filename
  generation.  The default is {\verb!.ppm!} for type P3 and P6,
  {\verb!.bmp!} for B24, and {\verb!.png!} for PNG file output.
\item[\optkey{-v level}]
  Verbosity (informational message) level, with 0 generating only
  error messages, and larger numbers generating additional information.
  The {\tt level} value is an integer, defaulting to 1.
\item[\optkey{infile \ldots}]
  List of input files to process.
\end{description}

The file specification options require some explanation.  Input files
may be specified either by an explicit list (\cd{infile ...}),
or by giving a wildcard pattern, e.g., \cd{-ipat *.omf}, which is
expanded in the usual way by \app{avf2ppm} (using the \Tcl\ command
\cd{glob}).   \Unix\ shells (sh, csh, etc.) automatically expand
wildcards before handing control over to the invoked application, so the
\cd{-ipat} option is not needed (although it is useful in case of a
``command-line too long'' error).  \DOS\ does not do this expansion, so
you must use \cd{-ipat} to get
wildcard expansion\index{platform!Windows!wildcard~expansion} in \Windows.

As each input file is processed, a name for the output file is produced
from the input filename by rules determined by handing the
\cd{-opatexp} and \cd{-opatsub} expressions to the \Tcl\
\cd{regsub} command.  Refer to the \Tcl\ \cd{regsub} documentation for
details, but essentially whatever portion of the input filename is
matched by the \cd{-opatexp} expression is removed and replaced by
the \cd{-opatsub} string.  The default \cd{-opatexp} expression
matches against any filename extension of up to 3 characters, and the
default \cd{-opatsub} string replaces this with the extension
\fn{.ppm} or \fn{.bmp}.

If you have command line image processing ``filter'' programs, e.g.,
\app{ppmtogif}\index{application!ppmtogif} (part of the
NetPBM\index{NetPBM} package), then you can use the \cd{-filter}
option to pipe the output of \app{avf2ppm} through that filter before it
is written to the output file specified by the \cd{-opat*}
expressions.  If the processing changes the format of the file, (e.g.,
\app{ppmtogif} converts from PPM\index{file!ppm} to
GIF\index{file!gif}), then you will likely want to specify a
\cd{-opatsub} different from the default.

Here is an example that processes all files with the \fn{.omf}
extension, sending the output through \app{ppmtogif} before saving the
results in files with the extension \fn{.gif}:
\begin{verbatim}
tclsh oommf.tcl avf2ppm -ipat *.omf -opatsub .gif -filter ppmtogif
\end{verbatim}
(On \Unix, either drop the \cd{-ipat} flag, or use quotes to protect
the input file specification string from expansion by the shell, as in
\cd{-ipat '*.omf'}.)  You may also pipe together multiple filters, e.g.,
\cd{-filter "ppmquant 256 | ppmtogif"}\index{application!ppmquant}.

\starssechead{Configuration files\label{sec:avf2ppmconfig}}\index{file!configuration}
The details of the conversion process are specified by plain-text
configuration files, in the same format as the
\hyperrefhtml{\app{mmDisp} configuration file}{\app{mmDisp}
  configuration file (Sec.~}{,}{sec:mmdispconfig}\NONHTMLoutput{
  page~\pageref{sec:mmdispconfig})}.

Each of the configurable parameters is an element in an array named
\cd{plot\_config}.  The default values for this array are read first
from the main configuration file
\begin{quote}
\fn{oommf/app/mmdisp/scripts/avf2ppm.config}
\end{quote}
followed by the local customization file
\begin{quote}
\fn{oommf/app/mmdisp/scripts/local/avf2ppm.config}
\end{quote}
if it exists.  Lastly, any files passed as \cd{-config} options on the
command line are input.  Each of these parameters is interpreted as
explained in the \htmlonlyref{\app{mmDisp}}{sec:mmdispconfig}
documentation, except that \app{avf2ppm} ignores the
\cd{misc,defaultwindowwidth} and \cd{misc,defaultwindowheight}
parameters, and the following additional parameters are available:
\begin{description}
\item[\optkey{arrow,antialias}]
 If 1\index{antialias}, then each pixel along the edge of an arrow is
 drawn not with the color of the arrow, but with a mixture of the arrow
 color and the background color.  This makes arrow boundaries appear
 less jagged, but increases computation time.  Also, the colors used in
 the anti-aliased pixels are not drawn from the arrow or pixel colormap
 discretizations, so color allocation in the output bitmap may increase
 dramatically.
\item[\optkey{arrow,outlinewidth}]
 Width of a colored outline around each arrow; this can improve
 visibility of an arrow when it is overlayed against a background with
 color similar to that of the arrow.  Default value is zero, meaning no
 outline.  A value of 1 produces an outline with a recommended width,
 and other positive values are scaled relative to this.
\item[\optkey{arrow,outlinecolor}]
 If \cd{arrow,outlinewidth} is positive, then this is the color of the
 arrow outline.
\item[\optkey{misc,boundarypos}]
 Placement of the bounding polygon, either \cd{back} or \cd{front},
 i.e., either behind or in front of the rendered arrows and pixel
 elements.
\item[\optkey{misc,matwidth}]
 Specifies the width, in pixels, of a mat (frame) around the outer edge
 of the image.  The mat is drawn in front of all other objects.  To
 disable, set matwidth to 0.
\item[\optkey{misc,matcolor}]
 Color of the mat.
\item[\optkey{misc,width}, \optkey{misc,height}]
 Maximum width and height of the output bitmap, in pixels.  If
 \cd{misc,crop} is enabled, then one or both of these dimensions may be
 shortened.
\item[\optkey{misc,crop}]
 If disabled (0), then any leftover space in the bitmap (of dimensions
 \cd{misc,width} by \cd{misc,height}) after packing the image are filled
 with the background color.  If enabled (1), then the bitmap is cropped
 to just include the image (with the margin specified by
 \cd{misc,margin}).  {\bf NOTE:} Some movie formats require that
 bitmap dimensions be multiples of 8 or 16.  For such purposes, you
 should disable \cd{misc,crop} and specify appropriate dimensions
 directly with \cd{misc,width} and \cd{misc,height}.
\end{description}

The \hyperrefhtml{default configuration file}{default configuration file
shown in Fig.~}{}{fig:avf2ppm.config}\NONHTMLoutput{
(page~\pageref{fig:avf2ppm.config})} can be used as a starting point for
user configuration files.  You may also use configuration files produced
by the \btn{File\pipe Write config\ldots} command in
\htmlonlyref{\app{mmDisp}}{sec:mmdisp}, although any of the above
\app{avf2ppm}-specific parameters that you wish to use will have to be
added manually, using a plain text editor.  You may omit any entries
that you do not want to change from the default.  You may ``layer''
configuration files by specifying multiple user configuration files on
the command line.  These are processed from left to right, with the last
value set for each entry taking precedence.

\begin{codelisting}{f}{fig:avf2ppm.config}{Contents of default configuration
file \fn{avf2ppm.config}.}{sec:avf2ppmconfig}{ref}
\begin{verbatim}
array set plot_config {
    arrow,status       1                misc,background    #FFFFFF
    arrow,antialias    1                misc,drawboundary  1
    arrow,outlinewidth 0.0              misc,boundarywidth 1
    arrow,outlinecolor #000000          misc,boundarycolor #000000
    arrow,colormap     Red-Black-Blue   misc,boundarypos   front
    arrow,colorcount   100              misc,matwidth      0
    arrow,quantity     z                misc,matcolor      #FFFFFF
    arrow,colorphase   0                misc,margin        10
    arrow,colorreverse 0                misc,width         640
    arrow,autosample   1                misc,height        480
    arrow,subsample    10               misc,crop          1
    arrow,size         1                misc,zoom          0
    arrow,viewscale    1                misc,rotation      0
                                        misc,datascale     0
    pixel,status       1                misc,relcenterpt   {0.5 0.5 0.5}
    pixel,colormap     Teal-White-Red
    pixel,colorcount   100              viewaxis            +z
    pixel,opaque       1                viewaxis,xarrowspan {}
    pixel,quantity     x                viewaxis,xpixelspan {}
    pixel,colorphase   0                viewaxis,yarrowspan {}
    pixel,colorreverse 0                viewaxis,ypixelspan {}
    pixel,autosample   1                viewaxis,zarrowspan {}
    pixel,subsample    0                viewaxis,zpixelspan {}
    pixel,size         1
}
\end{verbatim}
\end{codelisting}

%%%%%%%%%%%%%%%%%%%%%%%%%%%%%%%%%%%%%%%%%%%%%%%%%%%%%%%%%%%%%%%%%%%%%%%%

\section{Making \postscript\ from Vector Fields:
            avf2ps}\label{sec:avf2ps}%
\index{file!PostScript}\index{file!vector~field}\index{file!conversion}

The \app{avf2ps}\index{application!avf2ps} utility creates a
collection of color \eps\ files from a collection of
vector field files (e.g., \fn{.omf}, \fn{.ovf}), which can be embedded
into larger \postscript\ documents or printed directly on a \postscript\
printer.  Operation of the \app{avf2ps} command is modeled after the
{\hyperrefhtml{\app{avf2ppm} command}{\app{avf2ppm} command
(Sec.~}{)}{sec:avf2ppm}\index{application!avf2ppm}} and the print
dialog box in {\hyperrefhtml{\app{mmDisp}}{\app{mmDisp}
(Ch.~}{)}{sec:mmdisp}\index{application!mmDisp}}.

\starssechead{Launching}
The \app{avf2ps} launch command is:
\begin{verbatim}
tclsh oommf.tcl avf2ps [standard options] [-config file] [-f] \
   [-filter program] [-ipat pattern] [-opatexp regexp] [-opatsub sub] \
   [-v level] [infile ...]
\end{verbatim}
where
\begin{description}
\item[\optkey{-config file}]
  User configuration file that specifies image display parameters.  This
  file is discussed in \htmlonlyref{detail below}{sec:avf2psconfig}.
\item[\optkey{-f}]
  Force overwriting of existing (output) files.  By default, if
  \app{avf2ps} tries to create a file, say \fn{foo.ps}, that already
  exists, it generates instead a new name of the form \fn{foo.ps-000},
  or \fn{foo.ps-001}, \ldots, or \fn{foo.ps-999}, that doesn't exist
  and writes to that instead.  The {\tt -f} flag disallows alternate
  filename generation, and overwrites \fn{foo.ps} instead.
\item[\optkey{-filter program}]
  Post-processing application to run on each \app{avf2ps} output
  file.  May be a pipeline of several programs.
\item[\optkey{-ipat pattern}]
  Specify input files using a pattern with ``glob-style'' wildcards.
  Mostly useful in \DOS.
\item[\optkey{-opatexp regexp}]
  Specify the ``regular expression'' applied to input filenames to
  determine portion to be replaced in generation of output filenames.
  The default regular expression is: {\verb!(\.[^.]?[^.]?[^.]?$|$)!}
\item[\optkey{-opatsub sub}]
  The string with which to replace the portion of input filenames
  matched by the {\tt -opatexp regexp} during output filename
  generation.  The default is {\verb!.eps!}.
\item[\optkey{-v level}]
  Verbosity (informational message) level, with 0 generating only
  error messages, and larger numbers generating additional information.
  The {\tt level} value is an integer, defaulting to 1.
\item[\optkey{infile \ldots}]
  List of input files to process.
\end{description}

The file specification options, \cd{-ipat}, \cd{-opatexp}, and
\cd{-opatsub}, are interpreted in the same manner as for the
\htmlonlyref{\app{avf2ppm} application}{sec:avf2ppm}.

If you have command line \postscript\ processing ``filter'' programs,
e.g., \app{ghostscript}\index{application!ghostscript}, then you can
use the \cd{-filter} option to pipe the output of \app{avf2ps} through
that filter before it is written to the output file specified by the
\cd{-opat*} expressions.  If the processing changes the format of the
file, (e.g., from \postscript\ to PDF\index{file!pdf}), then you will
likely want to specify a \cd{-opatsub} different from the default.

Here is an example that processes all files with the \fn{.ovf}
extension, sending the output through \app{ps2pdf} (part of the
ghostscript package) before saving the results in files with the
extension \fn{.pdf}:
\begin{verbatim}
tclsh oommf.tcl avf2ps -ipat *.ovf -opatsub .pdf -filter "ps2pdf - -"
\end{verbatim}
On \Unix, either drop the \cd{-ipat} flag, or use quotes to protect
the input file specification string from expansion by the shell, as in
\cd{-ipat '*.ovf'}.

\starssechead{Configuration files\label{sec:avf2psconfig}}\index{file!configuration}
The details of the conversion process are specified by plain-text
configuration files, in the same format as the
\hyperrefhtml{\app{mmDisp} configuration file}{\app{mmDisp} configuration file
(Sec.~}{,}{sec:mmdispconfig}\NONHTMLoutput{ page~\pageref{sec:mmdispconfig})}.

The arrays \cd{plot\_config} and \cd{print\_config} hold the
configurable parameters.  The default values for these arrays are read
first from the main configuration file
\begin{quote}
\fn{oommf/app/mmdisp/scripts/avf2ps.config}
\end{quote}
followed by the local customization file
\begin{quote}
\fn{oommf/app/mmdisp/scripts/local/avf2ps.config}
\end{quote}
if it exists.  Lastly, any files passed as \cd{-config} options on the
command line are input.  Each of these parameters is interpreted as
explained in the \htmlonlyref{\app{mmDisp}}{sec:mmdispconfig}
documentation, except that \app{avf2ps} ignores the
\cd{misc,defaultwindowwidth} and \cd{misc,defaultwindowheight}
parameters, and the following additional parameters are available:
\begin{description}
\item[\optkey{arrow,outlinewidth}]
 Width of a colored outline around each arrow; this can improve
 visibility of an arrow when it is overlayed against a background with
 color similar to that of the arrow.  Default value is zero, meaning no
 outline.  A value of 1 produces an outline with a recommended width,
 and other positive values are scaled relative to this.
\item[\optkey{arrow,outlinecolor}]
 If \cd{arrow,outlinewidth} is positive, then this is the color of the
 arrow outline.
\item[\optkey{misc,boundarypos}]
 Placement of the bounding polygon, either \cd{back} or \cd{front},
 i.e., either behind or in front of the rendered arrows and pixel
 elements.
\item[\optkey{misc,matwidth}]
 Specifies the width, in pixels, of a mat (frame) around the outer edge
 of the image.  The mat is drawn in front of all other objects.  To
 disable, set matwidth to 0.
\item[\optkey{misc,matcolor}]
 Color of the mat.
\item[\optkey{misc,width}, \optkey{misc,height}]
 Maximum width and height of the output bitmap, in pixels.  If
 \cd{misc,crop} is enabled, then one or both of these dimensions may be
 shortened.
\item[\optkey{misc,crop}]
 If disabled (0), then any leftover space in the bitmap (of dimensions
 \cd{misc,width} by \cd{misc,height}) after packing the image are filled
 with the background color.  If enabled (1), then the bitmap is cropped
 to just include the image (with the margin specified by
 \cd{misc,margin}).  {\bf NOTE:} Some movie formats require that
 bitmap dimensions be multiples of 8 or 16.  For such purposes, you
 should disable \cd{misc,crop} and specify appropriate dimensions
 directly with \cd{misc,width} and \cd{misc,height}.
\end{description}

The \hyperrefhtml{default configuration file}{default configuration file
shown in Fig.~}{}{fig:avf2ps.config}\NONHTMLoutput{
(page~\pageref{fig:avf2ps.config})} can be used as a starting point for
user configuration files.  You may also use configuration files produced
by the \btn{File\pipe Write config\ldots} command in
\htmlonlyref{\app{mmDisp}}{sec:mmdisp}, although any of the above
\app{avf2ps}-specific parameters that you wish to use will have to be
added manually, using a plain text editor.  You may omit any entries
that you do not want to change from the default.  You may ``layer''
configuration files by specifying multiple user configuration files on
the command line.  These are processed from left to right, with the last
value set for each entry taking precedence.

\begin{codelisting}{f}{fig:avf2ps.config}{Contents of default configuration
  file \fn{avf2ps.config}.}{sec:avf2psconfig}{ref}
\begin{verbatim}
array set plot_config {
  arrow,status       1                  misc,background    #FFFFFF
  arrow,antialias    1                  misc,drawboundary  1
  arrow,colormap     Red-Black-Blue     misc,boundarywidth 1
  arrow,colorcount   100                misc,boundarycolor #000000
  arrow,quantity     z                  misc,boundarypos   front
  arrow,colorphase   0                  misc,matwidth      0
  arrow,colorreverse 0                  misc,matcolor      blue
  arrow,autosample   1                  misc,margin        10
  arrow,subsample    10                 misc,width         640
  arrow,size         1                  misc,height        480
  arrow,viewscale    1                  misc,crop          1
  arrow,boundarywidth 0.0               misc,zoom          0
  arrow,boundarycolor #000000           misc,rotation      0
  arrow,outlinewidth 0.0                misc,datascale     0
  arrow,outlinecolor #000000            misc,relcenterpt   {0.5 0.5 0.5}

  pixel,status       1                  viewaxis            +z
  pixel,colormap     Teal-White-Red     viewaxis,xarrowspan {}
  pixel,colorcount   100                viewaxis,xpixelspan {}
  pixel,opaque       1                  viewaxis,yarrowspan {}
  pixel,quantity     x                  viewaxis,ypixelspan {}
  pixel,colorphase   0                  viewaxis,zarrowspan {}
  pixel,colorreverse 0                  viewaxis,zpixelspan {}
  pixel,autosample   1
  pixel,subsample    0
  pixel,size         1
}
array set print_config {
  orient   landscape                    tmargin   1.0
  paper    letter                       lmargin   1.0
  hpos     center                       pwidth    6.0
  vpos     center                       pheight   6.0
  units    in                           croptoview 1
}
\end{verbatim}
\end{codelisting}

%%%%%%%%%%%%%%%%%%%%%%%%%%%%%%%%%%%%%%%%%%%%%%%%%%%%%%%%%%%%%%%%%%%%%%%%

\section{Vector Field File Difference:
          avfdiff}\label{sec:avfdiff}%
\index{file!vector~field}\index{file!vector~field}\index{file!vio}%
\index{application!avfdiff}\index{file!difference}
The \app{avfdiff} program computes differences between vector field files
in any of the
\hyperrefhtml{recognized formats}{recognized formats (\OVF, \VIO; see
  Ch.~}{)}{sec:vfformats}\HTMLoutput{ (\OVF, \VIO)}.  The input data
must lie on rectangular meshes with identical dimensions.

\starssechead{Launching}
The \app{avfdiff} launch command is:
\begin{verbatim}
tclsh oommf.tcl avfdiff [standard options] [-cross] [-filesort method] \
   [-info] [-numfmt fmt] [-odt label units valexpr] \
   [-resample fileselect interp_order] file-0 file-1 [... file-n]
\end{verbatim}
where
\begin{description}
\item[\optkey{-cross}]
  Compute the pointwise vector cross product
  of each \cd{file-k} against \cd{file-0} instead of subtraction.
\item[\optkey{-filesort method}]
  Specifies the sorting order to apply to the target file list,
  \cd{file-1} through \cd{file-n}.  The order is important when using
  the \cd{-odt} option, because it controls the order of the rows in the
  output.  Parameter \cd{method} should be a valid option string for the
  \Tcl\ command \cd{lsort}, e.g., ``-ascii -decreasing''.  Note that the
  \cd{lsort} sort options all begin with a hyphen, ``-'', and that if
  you want to use multiple options they must be grouped as one element
  to \cd{-filesort} (by, for example, placing quotes around the list).
  If this option is not specified then the order is as presented on the
  command line (or as produced by wildcard expansion).
\item[\optkey{-info}]
  Prints statistics on file differences.  If this option is selected
  then no output files are created.
\item[\optkey{-numfmt fmt}]
  Parameter \cd{fmt} specifies a C-style output format for numeric data if
  \cd{-info} or \cd{-odt} is selected.  Default value is
  ``\verb+%- #20.15g+''.
\item[\optkey{-odt label units valexpr}]
  Computes the file differences, but instead of writing difference files
  to disk this option writes \OOMMF\ Data Table
  \hyperrefhtml{(\ODT)}{(\ODT\ format, Ch.~}{)}{sec:odtformat} output to
  stdout.  The \ODT\ output consists of eight columns.  The first column
  is an index column identifying the target file (\cd{file-1} through
  \cd{file-n}).  The \cd{label} parameter is a string specifying the
  label for this column, and likewise the \cd{units} parameter is a
  string specifying the units for the column.  The third parameter,
  \cd{valexpr}, is any valid \Tcl\ \cd{expr} expression that may
  include the special variables \cd{\$i}, \cd{\$f1}, \cd{\$f2}, \ldots,
  \cd{\$d1}, \cd{\$d2}, \ldots; here \cd{\$i} is the 0-based index of
  the file in the target file list (\cd{file-1} is index 0, \cd{file-2}
  is index 1, etc.), \cd{\$f1} is the first number appearing in the
  target filename, \cd{\$f2} is the second number appearing in the target
  filename, \cd{\$d1} is the first number appearing in the ``Desc''
  fields in the header of the target file, etc.  This control is
  analogous to the \cd{-index} option to
  \hyperrefhtml{\cd{avf2odt}}{\cd{avf2odt} (Sec.~}{)}{sec:avf2odt}.
  The next three columns are the sum of each of the vector components in
  the difference.  The last four columns are the averaged $L^1$ norm,
  the normalized $L^2$ norm, minimum component absolute value, and
  maximum component absolute value of the difference; these columns
  correspond to those produced by the \cd{-extravals} option to
  \htmlonlyref{\cd{avf2odt}}{sec:avf2odt}.
\item[\optkey{-resample \boa 0\pipe n\bca\  \boa 0\pipe 1\pipe 3\bca}]
  Resample either the base file (\cd{file-0}) to match the resolutions of
  the target files (\cd{file-1} through \cd{file-n}), or resample each
  target file to match the resolution of the base file.  Set
  \cd{fileselect} to 0 for the former, to n for the latter.  The second
  argument specifies the polynomial interpolation order: 0 for
  nearest value, 1 for trilinear interpolation, or 3 for fitting with
  tricubic Catmull-Rom splines.  Default is no resampling.
\item[\optkey{file-0}]
  Name of input file to subtract from other files.  Must be either
  an \OVF\ 1.0 file in the rectangular mesh subformat, or an \VIO\
  file.  Required.
\item[\optkey{file-1}]
  Name of first input file from which \cd{file-0} is to be subtracted.
  Must also be either an \OVF\ 1.0 file in the rectangular mesh
  subformat, or an \VIO\ file, and must have the same dimensions as
  \cd{file-0}.  Required.
\item[\optkey{\ldots\ file-n}]
  Optional additional files from which \cd{file-0} is to be
  subtracted, with the same requirements as \cd{file-1}.
\end{description}

If neither \cd{-info} nor \cd{-odt} are specified, then for each target
file \cd{file-1} through \cd{file-n} a separate output file is
generated, in the \OVF\ 1.0 format.  Each output file has a name based
on the name of corresponding input file, with a \cd{-diff} suffix.  If a
file with the same name already exists, it will be overwritten.

For output file format details, see the \hyperrefhtml{OVF file
description}{OVF file description (Sec.~}{)}{sec:ovfformat}.

%%%%%%%%%%%%%%%%%%%%%%%%%%%%%%%%%%%%%%%%%%%%%%%%%%%%%%%%%%%%%%%%%%%%%%%%
\section{Cyclic Redundancy Check: crc32\label{sec:crc32}}%
\index{application!crc32}\index{CRC-32}

The \app{crc32} application computes 32-bit cyclic redundancy checksums
(CRC-32) on files.

\starssechead{Launching}
The \app{crc32} command line is:
\begin{verbatim}
tclsh oommf.tcl crc32 [standard options] [-binary|-text] \
   [-decimal|-hex] [-v level] [file ...]
\end{verbatim}
where
\begin{description}
\item[\optkey{-binary\textmd{\pipe}-text}]
 Select binary (default) or text input mode.
\item[\optkey{-decimal\textmd{\pipe}-hex}]
 Output CRC value in decimal (default) or hexadecimal format.
\item[\optkey{-v level}]
  Verbosity (informational message) level, with 0 generating only error
  messages and minimal CRC output, and larger numbers generating
  additional information.  The {\tt level} value is an integer,
  defaulting to 1.
\item[\optkey{file \ldots}]
 List of files to process.  If no files are listed, then input is read
 from stdin.
\end{description}
For each file in the input file list, the CRC-32 is computed and output.
By default, the computation is on the raw byte stream (binary mode).
However, if text mode is selected, then text mode translations, e.g.,
carriage return + newline $\rightarrow$ newline conversion, is performed
before the CRC-32 computation.  Text mode translations usually have no
effect on \Unix\ systems.  For additional information on text mode, see
the \Tcl\ documentation for \cd{fconfigure}, specifically ``-translation
auto.''

If the verbosity level is 1 or greater, then the length of the byte
stream as processed by the CRC-32 computation is also reported.

%%%%%%%%%%%%%%%%%%%%%%%%%%%%%%%%%%%%%%%%%%%%%%%%%%%%%%%%%%%%%%%%%%%%%%%%
\section{Killing \OOMMF\ Processes: killoommf\label{sec:killoommf}}%
\index{application!killoommf}\index{processes!killing}

The \app{killoommf} application terminates running \OOMMF\ processes.

\starssechead{Launching}
The \app{killoommf} command line is:
\begin{verbatim}
tclsh oommf.tcl killoommf [standard options] [-account name] \
   [-hostport port] [-pid] [-q] [-show] [-shownames] [-test] \
   [-timeout secs] oid [...]
\end{verbatim}
where
\begin{description}
\item[\optkey{-account name}]
 Specify the account name.  The default is the same used by
 \hyperrefhtml{\app{mmLaunch}}{\app{mmLaunch} (Ch.~}{)}{sec:mmlaunch}:
 the current user login name, except on \Windows~9X, where the dummy
 account ID  ``oommf''\index{platform!Windows!dummy~user~ID} may be used
 instead.
\item[\optkey{-hostport port}]
 Use the host server listening on \cd{port}.  Default is set by the
 \cd{Net\_Host~port} setting in \fn{oommf/config/options.tcl}, or by
 the environment variable \cd{OOMMF\_HOSTPORT} (which, if set,
 overrides the former).  The standard setting is 15136.
\item[\optkey{-pid}]
 Select processes by system pid rather than OOMMF oid.
\item[\optkey{-q}]
 Quiet; don't print informational messages.
\item[\optkey{-show}]
 Don't kill anything, just print matching targets.
\item[\optkey{-shownames}]
 Don't kill anything, just print nicknames of matching targets, where
 nicknames are as set by the \MIF\ 2.1
 \htmlonlyref{\cd{Destination}}{html:destinationCmd}
 command\latex{ (Sec.~\ref{sec:mif2ExtensionCommands})}.
\item[\optkey{-test}]
 Don't kill anything, just test that targets are responding.
\item[\optkey{-timeout secs}]
 Maximum time to wait for response from servers, in seconds.  Default
 is five seconds.
\item[\optkey{oid \ldots}]
 List of one or more oids (\OOMMF\ ID's), application names,
 nicknames, or the keyword ``all''.  Glob-style wildcards may also be
 used.  This field is required (there are no default kill targets).  If
 the \cd{-pid} option is specified then numbers are interpreted as
 referring to system process ID's rather than \OOMMF\ ID's.
\end{description}
The \app{killoommf} command affects processes that listen to \OOMMF\
message traffic.  These are the same applications that are listed in
the ``Threads'' list of \hyperrefhtml{\app{mmLaunch}}{\app{mmLaunch}
(Ch.~}{)}{sec:mmlaunch}.  The command
\begin{verbatim}
tclsh oommf.tcl killoommf all
\end{verbatim}
is essentially equivalent to the ``\cd{File\pipe Exit All OOMMF}''
menu option in \app{mmLaunch}, except that \app{killoommf} does not
shut down any \app{mmLaunch} processes.

An \OOMMF\ application that does not respond to \app{killoommf} can be
killed by using the \OOMMF\ command line program
\hyperrefhtml{\app{pidinfo}}{\app{pidinfo} (Sec.~}{)}{sec:pidinfo} to
determine its PID (process identification) as used by the operating
system, and then using the system facilities for terminating processes
(e.g., \cd{kill} on \Unix, or the \cd{Windows Task Manager} on
\Windows).

%%%%%%%%%%%%%%%%%%%%%%%%%%%%%%%%%%%%%%%%%%%%%%%%%%%%%%%%%%%%%%%%%%%%%%%%

\section{Last Oxsii/Boxsi run:
            lastjob\label{sec:lastjob}}%
\index{application!lastjob}

The \app{lastjob} command reads through \hyperrefhtml{Oxs}{Oxs
(Ch.~}{)}{sec:oxs}\index{application!Oxs} log files and identifies
the last simulation run.  From information in the log file,
\app{lastjob} constructs a command equivalent to that used to launch the
last simulation and prints that command to stdout.  If that simulation
is not recorded as complete in the log file, and a restart is requested,
then the simulation will be restarted with the \cd{-restart 1} comand
line option.  If a restart (checkpoint) file exists for the simulation,
then the command will restart the simulation at the checkpoint state.
If a restart file cannot be found, then the job restart will fail.  (By
default, \app{oxsii} and \app{boxsi} write \arbtargetlink{checkpoint
files}{checkpoint files (Sec.~\ref{sec:oxsDrivers},
page~}{)}{PToxsdrivercheckpoint} to disk every fifteen minutes.  If a
simulation is aborted, for example by a system crash, then the
checkpoint file can be used to restart the simulation.)

\starssechead{Launching}
The \app{lastjob} launch command is:
\begin{verbatim}
tclsh oommf.tcl lastjob [-logfile logname] [-unfinished] [-v] <show|restart> \
   <oxsii|boxsi> [hostname] [username]
\end{verbatim}
where
\begin{description}
\item[\optkey{-logfile logname}]
The name of the file to look in to determine the last job.  Optional.
The default is to look in the \OOMMF\ root directory for either
\fn{oxsii.errors} or \fn{boxsi.errors}, corresponding to whether
\app{oxsii} or \app{boxsi} jobs are selected.
\item[\optkey{-unfinished}]
Restrict search to unfinished jobs.  Optional.
\item[\optkey{-v}]
Request verbose output.  Optional.
\item[\optkey{show\pipe restart}]
Selects whether to simply show the command or to attempt a restart.
Required.
\item[\optkey{oxsii\pipe boxsi}]
Selects \app{oxsii} or \app{boxsi} jobs.  Required.
\item[\optkey{hostname}]
The name of the host machine to look for jobs for.  This is optional,
with the default being the name of the current machine.  This option is
useful if the log file is on a shared drive used by multiple hosts.
This field is interpreted as a regular expression, so for example
``\cd{.*}'' can be used to find the last job for all hosts.
\item[\optkey{username}]
The name of the user to look for jobs for.  This is optional,
with the default being the name of the current user.  This option is
useful if the same log file is shared by multiple users.
This field is interpreted as a regular expression, so for example
``\cd{.*}'' can be used to find the last job by any user.
\end{description}
Note: If your command shell expands wildcards, as is common on Unix
systems, then you may need to escape or quote regular expressions to
protect them from expansion by the shell.


%%%%%%%%%%%%%%%%%%%%%%%%%%%%%%%%%%%%%%%%%%%%%%%%%%%%%%%%%%%%%%%%%%%%%%%%

\section{Launching the \OOMMF\ host server:
   launchhost\label{sec:launchhost}}%
\index{application!launchhost}\index{processes!host~server}\index{ports}

Under normal circumstances, the \OOMMF\ host server (also known as the
host service directory) is automatically launched in the background as
needed by client applications.  However, it can be useful, primarily in
batch compute environments, to launch the host server explicitly in
order to control the host server port address.

\starssechead{Launching}
The \app{launchhost} command line is:
\begin{verbatim}
tclsh oommf.tcl launchhost [standard options] port
\end{verbatim}
where
\begin{description}
\item[\optkey{port}]
Requested port number for host server to listen on.  For
non-privileged users, this usually has to be larger than 1024, or the
special value 0 which signals the host server to open on a random,
unused port.  On success, \app{launchhost} writes the host port number
actually used to stdout.
\end{description}
As described in the \hyperrefhtml{\OOMMF\ architecture
documentation}{\OOMMF\ architecture documentation
(Ch.~}{)}{sec:arch}, the host server (host service directory) plays a
vital role in allowing various \OOMMF\ applications to communicate
with one another.  To work, the host server port number must be known
to all \OOMMF\ applications.  Typically this port number is determined
by the \cd{Net\_Host~port} setting in the file
\fn{oommf/config/options.tcl}, although this
setting may be overridden by the environment variable
\cd{OOMMF\_HOSTPORT}.

In batch-mode settings, however, it can occur that one wants to run
multiple concurrent but independent \OOMMF\ sessions on a single
machine.  One way to accomplish this is to set the environment
variable \cd{OOMMF\_HOSTPORT} to distinct values in each session.
A difficulty here is the bookkeeping necessary to insure that each
session really gets a distinct value.  Using \app{launchhost} with
\cd{port} set to zero provides a straightforward solution to this
problem.  For example, consider the Bourne shell script:
\begin{alltt}
   #!/bin/sh
   OOMMF_HOSTPORT=\backtick{}tclsh oommf.tcl launchhost 0\backtick
   export OOMMF_HOSTPORT
   tclsh oommf.tcl mmArchive
   tclsh oommf.tcl boxsi sample.mif
   tclsh oommf.tcl killoommf all
\end{alltt}\html{\\}
The second line (\cd{OOMMF\_HOSTPORT=\ldots}) launches the host server
on a random port; the port selected is printed to stdout by
\app{launchhost} and sets the environment variable
\cd{OOMMF\_HOSTPORT}.  (Note in particular the backticks around the
\app{launchhost} command, which invoke command execution.)  The
subsequent commands launch an instance of \app{mmArchive} in the
background, and run \app{boxsi} on the problem described by
\fn{sample.mif}.  (By default, \app{boxsi} runs in the foreground.)
When \app{boxsi} returns, the \app{killoommf} command is used to
terminate all \OOMMF\ processes in this session.  (Alternatively, the
\app{boxsi} command option \cd{-kill} may be used to the same effect
as \app{killoommf}.)  For \app{csh} and derivatives, use
\begin{alltt}
   setenv OOMMF_HOSTPORT \backtick{}tclsh oommf.tcl launchhost 0\backtick
\end{alltt}\html{\\}
in place of the two \cd{OOMMF\_HOSTPORT} commands in the above
example.


%%%%%%%%%%%%%%%%%%%%%%%%%%%%%%%%%%%%%%%%%%%%%%%%%%%%%%%%%%%%%%%%%%%%%%%%

\section{Calculating \vH\ Fields from Magnetization:
            mag2hfield}\label{sec:mag2hfield}%
\index{file!magnetization}\index{file!vector~field}\index{file!conversion}

The \app{mag2hfield}\index{application!mag2hfield} utility takes a
\MIF~1.1 micromagnetic problem specification file
(\hyperrefhtml{\fn{.mif}}{\fn{.mif}, see Sec.~}{}{sec:mif1format}) and a
magnetization file (\hyperrefhtml{\fn{.omf}}{\fn{.omf}, see
Ch.~}{}{sec:vfformats}) and uses the \hyperrefhtml{mmSolve2D}{mmSolve2D
(Sec.~}{)}{sec:mmsolve2d} computation engine to calculate the resulting
component (self-magnetostatic, exchange, crystalline anisotropy, Zeeman)
and total energy and/or \vH\ fields.  The main use of this utility to study
the fields in a simulation using magnetization files generated by an
earlier
\app{mmSolve2D} run.

\starssechead{Launching}
The \app{mag2hfield} launch command is:
\begin{verbatim}
tclsh oommf.tcl mag2hfield [standard options]
   [-component [all,][anisotropy,][demag,][exchange,][total,][zeeman] \
   [-data [energy,][field]] [-energyfmt fmt] [-fieldstep #] \
   mif_file omf_file [omf_file2 ...]
\end{verbatim}
where
\begin{description}
\item[\optkey{-component
  [all,][anisotropy,][demag,][exchange,][total,][zeeman]}]
  Specify all energy/field components that are desired.  Optional;
  default is {\tt total}, which is the sum of the crystalline
  anisotropy, demagnetization (self-magnetostatic), exchange, and Zeeman
  (applied field) terms.
\item[\optkey{-data [energy,][field]}]
  Calculate energies, \vH\ fields, or both.  Energy values are printed
  to stdout, \vH\ fields are written to files as described below.
  Optional; the default is {\tt energy,field}.
\item[\optkey{-energyfmt fmt}]
  Output C printf-style format string for energy data.  Optional.  The
  default format string is \verb+"%s"+.
\item[\optkey{-fieldstep \lb}]
  Applied field step index, following the schedule specified in the
  input \MIF\ file (0 denotes the initial field).  Optional; default is
  0.
\item[\optkey{mif\_file}]
  \MIF\ micromagnetic problem specification file (.mif).  Required.
\item[\optkey{omf\_file}]
  Magnetization state file.  This can be in any of the formats accepted
  by the \cd{avfFile} record of the input \MIF\ file.  Required.
\item[\optkey{omf\_file2 \ldots}]
  Optional additional magnetization state files.
\end{description}

The \vH\ field output file format is determined by the
\latexhtml{\cd{Total Field Output Format} record of the input \MIF~1.1 file
(Sec.~\ref{sec:mif1format}).}{\htmlonlyref{Total Field Output
Format}{sec:mif1outspec} record of the input \MIF~1.1 file.}  The output
file names are constructed using the form \fn{{\em
basename}-hanisotropy.ohf}, \fn{{\em basename}-hzeeman.ohf}, etc., where
{\em basename} is the input \fn{.omf} magnetization file name, stripped
of any trailing \fn{.omf} or
\fn{.ovf} extension.

%%%%%%%%%%%%%%%%%%%%%%%%%%%%%%%%%%%%%%%%%%%%%%%%%%%%%%%%%%%%%%%%%%%%%%%%

\section{\MIF\ Format Conversion: mifconvert}\label{sec:mifconvert}%
\index{file!MIF}\index{file!conversion}

The \app{mifconvert}\index{application!mifconvert} utility converts any
of the
\hyperrefhtml{\MIF}{\MIF\ (Ch.~}{,
page~\pageref{sec:mifformat})}{sec:mifformat} formats into the
\MIF~2.1 format used by the
\hyperrefhtml{Oxs 3D solvers}{Oxs 3D solvers (Ch.~}{)}{sec:oxs}.
It can also convert between
the  \MIF~1.1 and \MIF~1.2
formats generated by micromagnetic problem editor,
\hyperrefhtml{mmProbEd}{mmProbEd (Ch.~}{)}{sec:mmprobed}.

As a migration aid, \app{mifconvert} will convert most files from the
obsolete \MIF~2.0 format used by \OOMMF\ 1.2a2 into the newer
\MIF~2.1 format.

\starssechead{Launching}
The \app{mifconvert} launch command is:
\begin{verbatim}
tclsh oommf.tcl mifconvert [-f|--force] [--format fmt]
   [-h|--help] [--nostagecheck] [-q|--quiet] [--unsafe]
   [-v|--version] input_file output_file
\end{verbatim}
where
\begin{description}
\item[\optkey{-f} or \optkey{--force}]
  Force overwrite of output file.  Optional.
\item[\optkey{--format fmt}]
  Specify output format, where \cd{fmt} is one of 1.1, 1.2, or 2.1.  The
  1.1 and 1.2 formats are available only if the input file is also in
  the 1.x format.  Conversion from the 2.1 format to the 1.x formats is
  not supported.  Optional; default setting is 2.1.
\item[\optkey{-h} or \optkey{--help}]
  Print help information and stop.
\item[\optkey{--nostagecheck}]
  Sets the \cd{stage\_count\_check} parameter in the output
 \hyperrefhtml{Oxs\_Driver
 Specify block}{Oxs\_Driver Specify block (Sec.~}{,
 page~\pageref{sec:oxsDrivers})}{sec:oxsDrivers} to 0;
 this disables stage count consistency checks inside the Oxs
 solver. Optional.
 This option is only active when the output \MIF\ format is 2.1.
\item[\optkey{-q} or \optkey{--quiet}]
  Suppress normal informational and warning messages.  Optional.
\item[\optkey{--unsafe}]
  Runs embedded Tcl scripts, if any, in unsafe interpreter.  Optional.
\item[\optkey{-v} or \optkey{--version}]
  Print version string and stop.
\item[\optkey{input\_file}]
  Name of the import micromagnetic problem specification file, in
  \MIF~1.1, \MIF~1.2, or \MIF~2.0 format.  Use ``-'' to read from stdin.
  Required.
\item[\optkey{output\_file}]
  Name of the export micromagnetic problem specification file. Use ``-''
  to write to stdout.  Required.
\end{description}

%%%%%%%%%%%%%%%%%%%%%%%%%%%%%%%%%%%%%%%%%%%%%%%%%%%%%%%%%%%%%%%%%%%%%%%%
\section{Process Nicknames: nickname\label{sec:nickname}}%
\index{application!pidinfo}\index{PID's}\index{OID's}\index{nicknames}

The \app{nickname} command associates nicknames to running instances
of \OOMMF\ applications.  These names are used by the \MIF\ 2.x
\htmlonlyref{\cd{Destination}}{html:destinationCmd} command\latex{
(Sec.~\ref{sec:mif2ExtensionCommands})}.

\starssechead{Launching}
The \app{nickname} command line is:
\begin{verbatim}
tclsh oommf.tcl nickname [standard options] [-account name] \
   [-hostport port] [-pid] [-timeout secs] oid nickname [nickname2 ...]
\end{verbatim}
where
\begin{description}
\item[\optkey{-account name}]
 Specify the account name.  The default is the same used by
 \hyperrefhtml{\app{mmLaunch}}{\app{mmLaunch} (Ch.~}{)}{sec:mmlaunch}:
 the current user login name, except on \Windows~9X, where the dummy
 account ID  ``oommf''\index{platform!Windows!dummy~user~ID} may be used
 instead.
\item[\optkey{-hostport port}]
 Use the host server listening on \cd{port}.  Default is set by the
 \cd{Net\_Host~port} setting in \fn{oommf/config/options.tcl}, or by
 the environment variable \cd{OOMMF\_HOSTPORT} (which, if set,
 overrides the former).  The standard setting is 15136.
\item[\optkey{-pid}]
 Specify application instance to nickname by system PID
 (process identifier) rather than OID (\OOMMF\ identifier).
\item[\optkey{-timeout secs}]
 Maximum time to wait for response from servers, in seconds.  Default
 is five seconds.
\item[\optkey{oid}]
 The \OOMMF\ ID of the running application instance to nickname, unless
 the \cd{-pid} option is specified, in which case the system PID is
 specified instead.
\item[\optkey{nickname}]
 One or more nicknames to associate with the specified application
 instance.  Each nickname must include at least one non-numeric
 character.
\end{description}
This command is used to associate nicknames with running instances of
\OOMMF\ applications.  The \MIF\ 2
\htmlonlyref{\cd{Destination}}{html:destinationCmd} command can then
use the nickname to link Oxs output to a given OOMMF application
instance at problem load time.  Nicknames for GUI applications can be
viewed in the application \cd{About} dialog box, or can be seen for
any application via the \cd{-names} option to the command line
application \htmlonlyref{\app{pidinfo}}{sec:pidinfo}.

Note that nicknames can also be associated with \OOMMF\ applications
when they are started via the standard \cd{-nickname} command line
option, or by using the \cd{application:nickname} syntax for
applications launched by the \MIF\ \cd{Destination} command.

%%%%%%%%%%%%%%%%%%%%%%%%%%%%%%%%%%%%%%%%%%%%%%%%%%%%%%%%%%%%%%%%%%%%%%%%

\section{\ODT\ Derived Quantity Calculator: odtcalc}\label{sec:odtcalc}%
\index{file!data~table}

The \app{odtcalc}\index{application!odtcalc} utility reads an
\hyperrefhtml{\ODT}{\ODT\ (Ch.~}{)}{sec:odtformat} file on stdin
that contains one or more tables, and prints to stdout an \ODT\ file
consisting of the same tables augmented by additional columns as
controlled by command line arguments.  This utility enables the calculation
and recording of new data table columns that can be computed from
existing columns.

\starssechead{Launching}
The \app{odtcalc} launch command is:
\begin{verbatim}
tclsh oommf.tcl odtcalc [standard options] [var expr unit ...] \
    <infile >outfile
\end{verbatim}
where
\begin{description}
\item[\optkey{var expr unit \ldots}]
  Each triplet of command line arguments determines the calculation
  to make for the production of a new column in the output data table.
  Each {\tt var} value becomes the new {\tt Columns:} entry in the
  data table header, labeling the new column of data.
  Each {\tt unit} value becomes the new {\tt Units:} entry in the
  data table header, reporting the measurement unit for the new
  column of data.
  Each {\tt expr} value is a Tcl expression to be evaluated to
  compute each new data value to be stored in the new column of data.
  See below for more details.
\item[\optkey{\boa infile}]
  \app{odtcalc} reads its input from stdin.  Use the redirection operator
  ``\boa'' to read input from a file.
\item[\optkey{\bca outfile}]
  \app{odtcalc} writes its output to stdout.  Use the redirection operator
  ``\bca'' to send output to a file.
\end{description}

The computation of a new data value for each row of each new column
of data is performed by passing the corresponding {\tt expr} command
line argument to Tcl's {\bf expr} command.  The standard collection
of operators and functions are available.  The value of other columns
in the same row may be accessed by use of the column label as a variable
name.  For example, the value of the {\tt Iteration} column can be used
in {\tt expr} by including the variable substitution {\tt \$Iteration}.
When column labels include colons, the {\tt expr} has the option of using
just the portion of the column label after the last colon as the variable
name.  For example, the value of the {\tt Oxs\_UZeeman::Bx} column can
be used in {\tt expr} by including the variable substitution {\tt \$Bx}.
When multiple triples specifying new data columns are provided, the
values of earlier new columns may be used to compute the values of later
new columns.  The order of command line arguments controls the order of
the new columns that are added to the right side of the data table.

\starssechead{Example}
Suppose \fn{ring.odt} contains hysteresis loop data from an \app{Oxsii}
simulation where the field was applied in the $xy$-plane at an angle of
$30^\circ$ from the $x$-axis.  The data file holds field and average
magnetization axis component values $B_x$, $B_y$, $m_x$, and $m_y$.  We
want field and magnetization data projected onto the applied field
axis.  We can create those values using \app{odtcalc} like so:
\begin{verbatim}
tclsh oommf.tcl odtcalc B '$Bx*0.86602540378443865+$By*0.5' mT \
   m '$mx*0.86602540378443865+$my*0.5' '' \
   < ring.odt > ring-augmented.odt
\end{verbatim}
Here $\cos(30^\circ)=0.8660254037844365$ and $\sin(30^\circ)=0.5$.  The
\fn{ring-augmented.odt} file will have all the data in the original
\fn{ring.odt} file, plus two additional columns, labeled B with
units of mT and m with empty units.  (Note: On \Windows\ replace the
single quotes in the above command with double quotes.  Also, the
\Windows\ command line uses the caret character \verb+^+ for line
continuation instead of the backslash \verb+\+.)

To extract just the B and m columns and prepare for import into a
spreadsheet program supporting CSV (comma separated value) format,
post-process with \app{odtcols}:
\begin{verbatim}
tclsh oommf.tcl odtcols -t csv B m < ring-augmented.odt > ring-export.dat
\end{verbatim}


%%%%%%%%%%%%%%%%%%%%%%%%%%%%%%%%%%%%%%%%%%%%%%%%%%%%%%%%%%%%%%%%%%%%%%%%

\section{\ODT\ Table Concatenation: odtcat}\label{sec:odtcat}%
\index{file!data~table}

The \app{odtcat}\index{application!odtcat} utility reads an
\hyperrefhtml{\ODT}{\ODT\ (Ch.~}{)}{sec:odtformat} file on stdin that
contains one or more tables, and concatenates them together into a
single table, creating a new \ODT\ file consisting of a single table.
When successive tables are joined, the tail of the first is truncated as
necessary so that the specified control column is monotonic across the
seam.

This tool is useful for fixing up \ODT\ output from a simulation that
was interrupted and restarted from checkpoint data one or more times.

\starssechead{Launching}
The \app{odtcat} launch command is:
\begin{verbatim}
tclsh oommf.tcl odtcat [standard options] [-b overlap_lines] \
   [-c control_column] [-o order] [-q] <infile >outfile
\end{verbatim}
where
\begin{description}
\item[\optkey{-b overlap\_lines}]
  Overlap window size.  This is the maximum number of lines to retain
  when looking for overlap between two adjacent tables.  This is also
  the upper limit on the number of lines that may be removed when
  two tables are joined.  The default value is 100.
\item[\optkey{-c control\_column}]
  Specifies control column, either by number or glob-string.  Default
  is the glob string
  \cd{\ocb Oxs\_TimeDriver:*:Simulation time\ccb\ Oxs\_MinDriver:*:Iteration}.
\item[\optkey{-o order}]
  Order selection: one of \cd{increase}, \cd{decrease}, \cd{auto}
  (default), or \cd{none}.
\item[\optkey{-q}]
  Quiet; don't write informational messages to stderr.
\item[\optkey{\boa infile}]
  \app{odtcat} reads its input from stdin.  Use the redirection operator
  ``\boa'' to read input from a file.
\item[\optkey{\bca outfile}]
  \app{odtcat} writes its output to stdout.  Use the redirection operator
  ``\bca'' to send output to a file.
\end{description}
The first table header is examined and compared against the control
column specification to identify the control column.  If multiple
columns match the control column specification, then an error is
reported and the process exits.  The \OOMMF\ command line utility
{\hyperrefhtml{\app{odtcols}}{\app{odtcols} (Sec.~}{)}{sec:odtcols}}
with the \texttt{\textbf{-s}} command line switch can be used to view
column headers before running \app{odtcat}.

Each table in the input stream is assumed to have the same layout as
the first; header information between tables is summarily eliminated.
As each table is encountered, a check is made that the new table has
the same number of columns as the first.  If not, an error is reported
and processing is halted.

When subsequent table headers are encountered, the values in the
control column in the tail of the preceding table and the head of the
succeeding table are compared.  The order selection is used to
determine the position of the start of the latter table inside the
tail of the former.  If the data are not compatible with the specified
ordering, then an error is reported an the program aborts.  If
identical values are discovered, then the matching lines in the
earlier table are excluded from the output stream.

If the \texttt{\textbf{-q}} flag is not specified, then after
processing is complete a report is written to stderr detailing the
number of tables merged and the number of data lines eliminated.

%%%%%%%%%%%%%%%%%%%%%%%%%%%%%%%%%%%%%%%%%%%%%%%%%%%%%%%%%%%%%%%%%%%%%%%%

\section{\ODT\ Column Extraction: odtcols}\label{sec:odtcols}%
\index{file!data~table}

The \app{odtcols}\index{application!odtcols} utility extracts column
subsets from \hyperrefhtml{\ODT}{\ODT\ (Ch.~}{)}{sec:odtformat} data
table files.

\starssechead{Launching}
The \app{odtcols} launch command is:
\begin{verbatim}
tclsh oommf.tcl odtcols [standard options] [-f format] \
   [-m missing] [-q] [-s] [-S] [-t output_type] \
   [-table select] [-no-table deselect] [-w colwidth] \
   [col ...] <infile >outfile
\end{verbatim}
where
\begin{description}
\item[\optkey{-f format}]
  C printf-style format string for each output item.  Optional.  The
  default format string is \verb+"%$15s"+. Multiple \cd{-f}
  options may be interspersed with column selections, in which case each
  format applies to subsequently selected columns.
\item[\optkey{-m missing}]
  String used on output to designate a missing value.  Default
  is the two character open-close curly brace pair, \ocb\ccb, as
  specified by the \hyperrefhtml{\ODT\ file format}{\ODT\ file format
  (Ch.~}{)}{sec:odtformat}.
\item[\optkey{-q}]
  Silences some meaningless error messages, such as "broken pipe" when
  using the \Unix\ head or tail utilities.
\item[\optkey{-s}]
  Produces a file summary instead of column extraction.  Output includes
  table titles, column and row counts, and the header for each specified
  column.  If no columns are specified, then the headers for all the
  columns are listed.
\item[\optkey{-S}]
  Same as  \texttt{-s} option, except the column list is ignored;
  headers for all columns are reported.
\item[\optkey{-t output\_type}]
  Specify the output format.  Here output\_type should be one of the
  strings \cd{odt}, \cd{csv}, or \cd{bare}.  The default is \cd{odt},
  the \hyperrefhtml{\ODT\ file format}{\ODT\ file format
  (Ch.~}{)}{sec:odtformat}.  Selecting \cd{csv} will yield a
  ``Comma-Separated Values'' (CSV) file, which can be read by many
  spreadsheet programs.  The \cd{bare} selection produces space separated
  numeric output, with no \ODT\ header, trailer, or comment lines.  The
  latter two options are intended as aids for transferring data to third
  party programs; in particular, such output is not in \ODT\ format, and
  there is no support in OOMMF for translating back from CSV or bare
  format to \ODT\ format.
\item[\optkey{-table select}]
  Select tables to include in output.  Tables are selected by index
  number; the first table in the file has index 0.  The select string
  consists of one or more selections separated by commas, where each
  selection is either an individual index number or a range with
  inclusive endpoints separated by a colon.  Example select string:
  \cd{0:3,7,9:12}.  Default is all tables.
\item[\optkey{-no-table deselect}]
  Specify tables to exclude from output.  The deselect string has the
  same format at the \cd{-table select} string.  Default is to print all
  tables, so the effective default deselect string is the empty set.
\item[\optkey{-w colwidth}]
  Minimum horizontal spacing to provide for each column on output.
  Optional.  Default value is 15. Negative \cd{colwidth} values will
  fill from the left, positive from the right. (This positions the
  post-formatted data string, retaining any space in the field width
  portion of the \cd{-f format} specification.) Multiple \cd{-w}
  options may be interspersed with column selections, in which case each
  width applies to subsequently selected columns.
\item[\optkey{col \ldots}]
  Output column selections.  These may either be integers representing
  the position of the column in the input data (with the first column
  numbered as 0), or else arbitrary strings used for case-insensitive
  glob-style matching against the column headers.  The columns are
  output in match order, obtained by processing the column selections
  from left to right.  If no columns are specified then by default all
  columns are selected.
\item[\optkey{\boa infile}]
  \app{odtcols} reads its input from stdin.  Use the redirection operator
  ``\boa'' to read input from a file.
\item[\optkey{\bca outfile}]
  \app{odtcols} writes its output to stdout.  Use the redirection operator
  ``\bca'' to send output to a file.
\end{description}
Commonly the \texttt{\textbf{-s}} switch is used in a first pass, to
reveal the column headers; specific column selections may then be made
in a second, separate invocation.  If no options or columns are
specified, then the help message is displayed.

%%%%%%%%%%%%%%%%%%%%%%%%%%%%%%%%%%%%%%%%%%%%%%%%%%%%%%%%%%%%%%%%%%%%%%%%

\section{Oxs package management:
            oxspkg\label{sec:oxspkg}}%
\index{application!oxspkg}

The \app{oxspkg} command is used to manage optional Oxs extension
packages.  Each package is stored in a separate directory under
\fn{oommf/app/oxs/contrib/}.  These packages can be ``installed'' and
``uninstalled'' to and from the \fn{oommf/app/oxs/local/} directory by
the \app{oxspkg} command.  The install is a simple copy that does not
automatically build the package or link it into the Oxs executable---a
separate invocation of
\hyperrefhtml{\app{pimake}}{\app{pimake} (Sec.~}{)}{sec:pimake}
is needed for that.

\starssechead{Launching}
The \app{oxspkg} launch command is:
\begin{duplex}
\item \verb+tclsh oommf.tcl oxspkg list+
\item[\textbf{or}]\html{\\}
\item \verb+tclsh oommf.tcl oxspkg listfiles pkg [pkg ...]+
\item[\textbf{or}]\html{\\}
\item \verb+tclsh oommf.tcl oxspkg readme pkg [pkg ...]+
\item[\textbf{or}]\html{\\}
\item \verb+tclsh oommf.tcl oxspkg requires pkg [pkg ...]+
\item[\textbf{or}]\html{\\}
\item \verb+tclsh oommf.tcl oxspkg install [-v] [-nopatch] pkg [pkg ...]+
\item[\textbf{or}]\html{\\}
\item \verb+tclsh oommf.tcl oxspkg uninstall pkg [pkg ...]+
\item[\textbf{or}]\html{\\}
\item \verb+tclsh oommf.tcl oxspkg copyout pkg [pkg ...] destination+
\end{duplex}
Glob-style wildcards (*, ?) or the keyword \cd{all} are accepted in
package specifications.  (If your command shell expands wildcards, as
is common on \Unix\ systems, then you may need to escape or quote the
wildcards so that they are passed unadulterated to the \app{oxspkg}
program.)  The first argument following the \app{oxspkg}
keyword is one of the sub-commands \cd{list}, \cd{listfiles},
\cd{readme}, \cd{requires}, \cd{install}, \cd{uninstall}, or
\cd{copyout}:
\begin{description}
\item[\optkey{list}]
  Lists all the packges available under
  \fn{oommf/app/oxs/contrib/}, how many (installable) files are in
  each package, and the package install status.
\item[\optkey{listfiles pkg [pkg \ldots]}]
  Lists each of the ``installable'' files for the selected package.
  (There may be additional files for the package, e.g.\ \fn{README} or
  files with versioning information, included in the
  \fn{oommf/app/oxs/contrib/\oab pkg\cab/} directory.  However, those
  files are ignored by the \app{oxspkg install} command.)
\item[\optkey{readme pkg [pkg \ldots]}]
  Prints to \fn{stdout} the contents of the \fn{README} file (if any)
  for each package.
\item[\optkey{requires pkg [pkg \ldots]}]
  Lists external libraries required for each package, as declared by any
  \fn{*.rules} files contained in the package. The user should ensure
  these libraries are installed on the system and included on the
  compiler and linker search paths before installing the Oxs extension
  package. In particular, check the
\begin{verbatim}
$config SetValue program_compiler_extra_include_dirs ...
$config SetValue program_linker_extra_lib_dirs ...
\end{verbatim}
settings in the \hyperrefhtml{platform configuration file}{platform
  configuration file (Sec.~}{)}{sec:install.custom}.  There may be
additional installation information in the package \fn{README} file.

\item[\optkey{install [-v] [-nopatch] pkg [pkg \ldots]}]
  Installs (copies) files for the selected package(s) from
  \fn{oommf/app/oxs/contrib/} to \fn{oommf/app/oxs/local/}.  This
  command \textbf{does not} compile the files or link them into the Oxs
  executable.  The user is responsible for making a separate call to
  \htmlonlyref{\app{pimake}}{sec:pimake} to build and link the
  package. If the user is running \OOMMF\ distributed with pre-built
  binaries, then the user should first \cd{cd} to the \OOMMF\ root
  directory and run
  \verb+tclsh oommf.tcl pimake distclean+
  to delete the distributed binaries before running
  \verb+tclsh oommf.tcl pimake+ to create a new build of \OOMMF. This
  ensures  new extension binaries are compatible with the rest of
  the \OOMMF\ install.

  For third-party packages included in \OOMMF\ distributions, the
  contents of the directory \fn{oommf/app/oxs/contrib/\oab pkg\cab/}
  will mirror some release of the package from the official source of
  the package.  If those files don't build properly with the
  \OOMMF\ distribution, then a patch file will be included in the parent
  \fn{oommf/app/oxs/contrib/} directory.  Normally that patch file (if
  any) is automatically applied as part of the installation procedure.
  The \cd{-nopatch} option skips the patch step. If there are no patches
  for a particular package, then the message \cd{No patches found} will
  be reported during the install process.

  The \cd{-v} option requests more verbose output.
\item[\optkey{uninstall pkg [pkg \ldots]}]
 Deletes all files in \fn{oommf/app/oxs/local/} associated with the
 selected package.  Here ``associated'' means a file name match with a
 file in the package directory \fn{oommf/app/oxs/contrib/\oab pkg\cab/}.
 There is no checking of contents or timestamps between the files.
\item[\optkey{copyout pkg [pkg \ldots] destination}]
  Selects files in the same manner as the \cd{uninstall} command, but
  rather than deleting the files instead copies them to the
  \fn{destination} directory.  This is intended as a development aid
  for creating patches for packages.
\end{description}
Most of the optional packages controlled by \app{oxspkg} are from
third-party contributors.  Some may originate with the \OOMMF\ core
development team, but are made optional because they require third-party
libraries or are considered too experimental to be included among the
standard Oxs extension classes.  The various
\fn{oommf/app/oxs/contrib/\oab pkg\cab/} directories may contain
\fn{README} files with additional details.

%%%%%%%%%%%%%%%%%%%%%%%%%%%%%%%%%%%%%%%%%%%%%%%%%%%%%%%%%%%%%%%%%%%%%%%%

\section{Oxs regression tests:
            oxsregression\label{sec:oxsregression}}%
\index{application!regression}

The \app{oxsregression} runs a test suite for the Oxs solver.  For
each test, an instance of
\hyperrefhtml{\app{boxsi}}{\app{boxsi} (Sec.~}{)}{sec:boxsi}\index{application!boxsi}
is run and the results are compared against reference results stored
in subdirectories under \fn{oommf/app/oxs/regression\_tests/}.

\starssechead{Launching}
The \app{oxsregression} launch command is:
\begin{verbatim}
tclsh oommf.tcl runtests [-autoadd] [-alttestdir] [-cleanup] [-ignoreextra]
   [-keepfail] [-listtests] [-loglevel level] [-noexcludes] [-parallel n]
   [-resultsfile stemname] [-showoutput] [-sigfigs digits] [-threads count]
   [-timeout seconds] [-updaterefdata] [-v] [testa testb ...]
\end{verbatim}
where
\begin{description}
\item[\optkey{-autoadd}]
 Automatically adds new tests from \MIF\ files found in the examples
 directory \fn{oommf/app/oxs/examples/}.
\item[\optkey{-alttestdir}]
\begin{flushleft}
Specify an alternative test directory to use in place of the default
directory list \fn{oommf/app/oxs/examples/},
\fn{oommf/app/oxs/regression\_tests/bug\_tests/}, and
\fn{oommf/app/oxs/regression\_tests/local\_tests/}.
\end{flushleft}
\item[\optkey{-cleanup}]
If \app{oxsregression} is killed or crashes mid-run, then some temporary
result files may be left on disk. This command searches for and offers to
delete these stray files.
\item[\optkey{-ignoreextra}]
 Ignore extra columns, if any, in test results as compared to
 reference results.  This is useful in development work when
 changes to a \MIF\ file introduce additional data table output.
\item[\optkey{-keepfail}]
  Keep results from failed tests.  Normally test results are
  automatically deleted.
\item[\optkey{-listtests}]
  List all selected tests and exit without running any tests.
\item[\optkey{-loglevel level}]
%\begin{flushleft}
  Controls the amount of log information written to
  \fn{oxsregression.log} in the regression test directory
  \fn{oommf/app/oxs/regression\_tests/}.  The default setting is 0.
%\end{flushleft}
\item[\optkey{-noexcludes}]
  Some tests suffer from various numerical problems.  These are
  excluded from testing, unless this option is specified.
\item[\optkey{-parallel n}]
  Run \cd{n} tests concurrently, with default \cd{n=1}. This option is
  only available when using \Tcl\ 8.6 or later.
\item[\optkey{-resultsfile stemname}]
  Test results are written to temporary files; by default these files
  have the stem \cd{oxsregression-test-output}.  If \app{oxsregression}
  is run simultaneously, perhaps on different machines on a shared
  file system, then overwriting of files from one process can
  interfere with the processing by another.  The \cd{-resultsfile}
  option can be used to cordon off results between simultaneous runs.
\item[\optkey{-showoutput}]
  If this switch is not specified, then \fn{stdout} and \fn{stderr}
  output from \app{boxsi} is swallowed by \app{oxsregression}.
\item[\optkey{-sigfigs digits}]
 Number of significant (decimal) digits to use in comparing test to
 reference results; the default setting is eight.
\item[\optkey{-threads count}]
 Number of threads to run \app{boxsi} with.  This option is available
 for threaded builds only.  The default is the default thread count
 for \app{boxsi}.
\item[\optkey{-timeout seconds}]
 Maximum number of seconds to wait for one test to finish; any
 individual test that fails to complete within this time span is summarily
 terminated.  The default time is 150 seconds; use 0 to indicate no timeout.
\item[\optkey{-updaterefdata}]
 For developer use only; this option causes the reference results to
 be replaced (overwritten) with new test results.
\item[\optkey{-v}]
 Enable verbose output.
\item[\optkey{testa testb \ldots}]
 Tests to run, with glob-style wildcards (*, ?) accepted.  If no tests
 are specified then all (non-excluded) tests are selected.  Subtest
 selection must be quoted with the test to appear as a single
 argument, e.g., \verb+"exch6ngbr 1,7,9"+.  If no subtests are
 specified then all subtests are run.
\end{description}
Take notice  of the interplay between the \cd{-parallel n} and
\cd{-threads count} options. The former is the number of tests run in
parallel, and the latter is the number of computation threads active in
each test run. The total number of threads active at one time can
therefore be as many as \cd{n\,$\times$\,count}.

%%%%%%%%%%%%%%%%%%%%%%%%%%%%%%%%%%%%%%%%%%%%%%%%%%%%%%%%%%%%%%%%%%%%%%%%

\section{OOMMF and Process ID Information: pidinfo\label{sec:pidinfo}}%
\index{application!pidinfo}\index{PID's}\index{OID's}

The \app{pidinfo} command prints a table mapping OOMMF ID's (OID's)
to system process ID's (PID's) and application names.

\starssechead{Launching}
The \app{pidinfo} command line is:
\begin{verbatim}
tclsh oommf.tcl pidinfo [standard options] [-account name] \
   [-hostport port] [-names] [-noheader] [-pid] [-ports] \
   [-timeout secs] [-wait secs] [-v] [oid ...]
\end{verbatim}
where
\begin{description}
\item[\optkey{-account name}]
 Specify the account name.  The default is the same used by
 \hyperrefhtml{\app{mmLaunch}}{\app{mmLaunch} (Ch.~}{)}{sec:mmlaunch}:
 the current user login name, except on \Windows~9X, where the dummy
 account ID  ``oommf''\index{platform!Windows!dummy~user~ID} may be used
 instead.
\item[\optkey{-hostport port}]
 Use the host server listening on \cd{port}.  Default is set by the
 \cd{Net\_Host~port} setting in \fn{oommf/config/options.tcl}, or by
 the environment variable \cd{OOMMF\_HOSTPORT} (which, if set,
 overrides the former).  The standard setting is 15136.
\item[\optkey{-names}]\index{nicknames}
 Display application nicknames, which are used by the \MIF\ 2.1
 \htmlonlyref{\cd{Destination}}{html:destinationCmd}
 command\latex{ (Sec.~\ref{sec:mif2ExtensionCommands})}.
\item[\optkey{-noheader}]
 Don't print column headers.
\item[\optkey{-pid}]
 Select processes by system pid rather than OOMMF oid.
\item[\optkey{-ports}]
 Display active server ports for each application.
\item[\optkey{-timeout secs}]
 Maximum time to wait for response from servers, in seconds.  Default
 is five seconds.
\item[\optkey{-v}]
 Display information about the host and account servers.
\item[\optkey{-wait secs}]
 If no match is found, then retry for up to \cd{secs} seconds.  Default
 is zero seconds, i.e., try once.
\item[\optkey{oid \ldots}]
 List of OOMMF ID's to display information about.  Default is all
 current applications.  If the \cd{-pid} option is specified then this
 selection is by system process ID's rather than \OOMMF\ ID's.
\end{description}
The title bar of running OOMMF applications typically displays the
application OID, which are used by OOMMF applications to identify one
another.  These ID's start at 0 and are incremented each time a newly
launched application registers with the account server.  The OID's are
independent of the operating system PID's.  The PID is needed to obtain
information, e.g., resource use, about a running process using system
utilities.  The PID may also be needed to invoke the operating system
``kill'' facility to terminate a rogue OOMMF application.  The
\app{pidinfo} application can be used to correspond OID's or \OOMMF\
application names to PID's for such purposes.

%%%%%%%%%%%%%%%%%%%%%%%%%%%%%%%%%%%%%%%%%%%%%%%%%%%%%%%%%%%%%%%%%%%%%%%%

\section{Platform-Independent Make:
            pimake\label{sec:pimake}}%
\index{application!pimake}

The application \app{pimake}\index{application!pimake} is used to
construct \OOMMF\ end-user executables (and optionally documentation)
from the distributed source code. It is similar in operation to the
\Unix\ utility program \app{make}\index{application!make}, but it is
written entirely in \Tcl\ so that it will run anywhere \Tcl\ is
installed.  Like \app{make}, \app{pimake} controls the building of a
{\em target} from other files.  Just as \app{make} is controlled by
rules in files named \fn{Makefile} or \fn{makefile}, \app{pimake} is
controlled by rules in files named \fn{makerules.tcl}. \app{pimake}
generally operates recursively, starting in the current working
directory and continuing downward through all subdirectories. (The chief
exception to this is the documentation under \fn{oommf/doc};
\app{pimake} build requests started in the root directory \fn{oommf}
ignore the \fn{oommf/doc} subdirectory. Building the
\OOMMF\ documentation requires a working installation of
\htmladdnormallinkfoot{\LaTeX}{https://www.latex-project.org} and either
\htmladdnormallinkfoot{\LaTeX2HTML}{https://www.latex2html.org} or
\htmladdnormallinkfoot{\LaTeXML}{http://dlmf.nist.gov/LaTeXML/}, so
\OOMMF\ distributions include pre-built versions of the full
\OOMMF\ documentation to relieve the end-user the burden of installing
those tools. On the other hand, clean requests, including in particular
\cd{maintainer-clean}, will recurse into \fn{oommf/doc}.) The
\OOMMF\ standard command line option \cd{-cwd} may be used to change the
starting directory.

\starssechead{Launching}
The \app{pimake} launch command is:
\begin{verbatim}
tclsh oommf.tcl pimake [standard options] \
   [-d] [-i] [-k] [-out file] [-root] [target]
\end{verbatim}
where
\begin{description}
\item[\optkey{-d}]
  Print verbose information about dependencies.
\item[\optkey{-i}]
  Normally an error halts operation.  When \cd{-i} is specified,
  ignore errors and try to continue updating all dependencies
  of target.
\item[\optkey{-k}]
  Normally an error halts operation.
  When \cd{-k} is specified, and an error is encountered, stop
  processing dependencies which depend on the error, but continue
  updating other dependencies of target.
\item[\optkey{-out file}]
  Write informational output to named file instead of to the standard
  output.
\item[\optkey{-root}]
  Change to the \OOMMF\ root directory. Command line options are
  processed from left to right, so if used in conjunction with the
  standard option \cd{-cwd}, \cd{-root} should come first.
\item[\optkey{target}]
  The file to build.  May also be (and usually is) a symbolic
  target name.  See below for standard symbolic targets.  By default,
  the first target in \fn{makerules.tcl} is built.
\end{description}

There are several targets which may be used as arguments to
\app{pimake} to achieve different tasks.
Each target builds in the current directory
and all subdirectories.  The standard targets are:
\begin{description}
\item[\optkey{upgrade}]
Used immediately after unpacking a distribution, it removes
any files which were part of a previous release, but are not
part of the unpacked distribution.
\item[\optkey{all}]
Creates all files created by the \fn{configure} target (see below).
Compiles and links all the executables and libraries.
Constructs all index files.
\item[\optkey{configure}]
Creates subdirectories with the same name as the platform type.
Constructs a \fn{ocport.h} file which includes C++ header information
specific to the platform.
\item[\optkey{objclean}]
Removes the intermediate object files created by the compile and
link steps.  Leaves working executables in place.  Leaves
\OOMMF\ in the state of its distribution with pre-compiled
executables.
\item[\optkey{clean}]
Removes the files removed by the \fn{objclean} target.
Also removes the executables and libraries created by the \fn{all}
target.  Leaves the files generated by the \fn{configure} target.
\item[\optkey{distclean}]
Removes the files removed by the \fn{clean} target.
Also removes all files and directories generated by
\fn{configure} target.  Leaves only the files which
are part of the source code distribution.
\item[\optkey{maintainer-clean}]
Remove all files which can possibly be generated from other files.  If
run from the \OOMMF\ root directory then all such files,
\textbf{including \OOMMF\ documentation files}, will be deleted.
Rebuilding may require specialized developer tools, so use with care.
This target is intended primarily for developers, not end-users.
\item[\optkey{help}]
Print a summary of the standard targets.
\end{description}

%%%%%%%%%%%%%%%%%%%%%%%%%%%%%%%%%%%%%%%%%%%%%%%%%%%%%%%%%%%%%%%%%%%%%%%%

%begin{latexonly}
\newcounter{ffoldsecnumdepth}
\setcounter{ffoldsecnumdepth}{\value{secnumdepth}}
\setcounter{secnumdepth}{4}
%end{latexonly}

\chapter{Problem Specification File Formats (\MIF)}\label{sec:mifformat}
Micromagnetic simulations are specified to the \OOMMF\ solvers using the
\OOMMF\ \textit{Micromagnetic Input Format} (\MIF)\index{file!MIF}.  There
are four distinct versions of this format.  The oldest format,
version 1.1, is used by the mmSolve 2D solvers
(\hyperrefhtml{\app{mmSolve2D}}{\app{mmSolve2D},
Sec.~}{}{sec:mmsolve2d}\index{application!mmSolve2D} and
\hyperrefhtml{\app{batchsolve}}{\app{batchsolve},
Sec.~}{}{sec:batchsolve}\index{application!batchsolve}) and the
\hyperrefhtml{\app{mmProbEd}}{\app{mmProbEd}
(Ch.~}{)}{sec:mmprobed}\index{application!mmProbEd} problem editor.
The \MIF~2.1 and \MIF~2.2 formats are the powerful, native format used by the
\hyperrefhtml{Oxs 3D solvers}{Oxs 3D solvers
(Ch.~}{)}{sec:oxs}.  The \MIF~1.2 format is a minor modification to
the 1.1 format, which can be used as a simple but restricted interface
to the Oxs solvers.  In all cases values are specified in \SI\ units.  A
command line utility \hyperrefhtml{\app{mifconvert}}{\app{mifconvert}
(Sec.~}{)}{sec:mifconvert}\index{application!mifconvert} is provided to
aid in converting \MIF~1.1 files to the \MIF~2.1 format.  For all
versions it is recommended that \MIF\ files be given names ending with
the \fn{.mif} file extension.
%%%%%%%%%%%%%%%%%%%%%%%%%%%%%%%%%%%%%%%%%%%%%%%%%%%%%%%%%%%%%%%%%%%%%%%%

\section{\MIF\ 1.1}\label{sec:mif1format}
The \MIF~1.1 format is an older micromagnetic problem specification
format used by the mmSolve 2D solvers.  It is not compatible with the
\MIF~2.1 format used by the Oxs 3D solvers.  However, the command line tool
\hyperrefhtml{\app{mifconvert}}{\app{mifconvert}
(Sec.~}{)}{sec:mifconvert} may be used as a conversion aid;
\app{mifconvert} is also called automatically by Oxs solvers when a
\MIF~1.x file is input to them.

A sample \MIF~1.1 file is \hyperrefhtml{included below}{presented in
Fig.~}{}{fig:mif1sample}.  The first line of a \MIF\ file must be of the
form ``\verb+#+ MIF x.y'', where x.y represents the format revision
number.  (The predecessor \MIF~1.0 format was not included in any
released version of \OOMMF.)

After the format identifier line, any line ending in a backslash,
`\bs', is joined to the succeeding line before any other processing
is performed.  Lines beginning with a `\verb+#+' character are comments
and are ignored.  Blank lines are also ignored.

All other lines must consist of a {\em Record Identifier} followed by
a parameter list.  The Record Identifier is separated from the
parameter list by one or more `:' and/or `=' characters.  Whitespace
and case is ignored in the Record Identifier field.

The parameter list must be a proper \Tcl\ list\index{Tcl~list}.  The
parameters are parsed (broken into separate elements) following normal
\Tcl\ rules; in short, items are separated by whitespace, except as
grouped by double quotes and curly braces.  Opening braces and quotes
must be whitespace separated from the preceding text.  The grouping
characters are removed during parsing.  Any `\lb' character that is
found outside of any grouping mechanism is interpreted as a comment
start character.  The `\lb' and all following characters on that line
are interpreted as a comment.

Order of the records in a \MIF~1.1 file is unimportant, except as
explicitly stated below.  If two or more lines contain the same Record
Identifier\index{record~identifier}, then the last one takes precedence,
with the exception of Field Range records, of which there may be several
active.  All records are required unless listed as optional.  Some of
these record types are not supported by \app{mmProbEd}, however you
may modify a \MIF~1.1 file using any plain text editor and supply it to
\hyperrefhtml{\app{mmSolve2D}}{\app{mmSolve2D}
(Sec.~}{)}{sec:mmsolve2d}\index{application!mmSolve2D}
using
\hyperrefhtml{\app{FileSource}}{\app{FileSource}
(Ch.~}{)}{sec:filesource}\index{application!FileSource}.

For convenience, the Record Identifier tags are organized into several
groups; these groups correspond to the top-level buttons presented by
\app{mmProbEd}.  We follow this convention below.

\subsection{Material parameters}\label{sec:mif1materials}\index{materials}
\begin{itemize}
\item {\bf\lb\ Material Name:} This is a convenience entry for
   \app{mmProbEd}; inside the \MIF~1.1 file it is a comment line.  It
   relates a symbolic name (e.g., Iron) to specific values to the next
   4 items.  Ignored by solvers.
\item {\bf Ms:} Saturation magnetization\index{saturation~magnetization}
   in A/m.
\item {\bf A:} Exchange stiffness\index{exchange~stiffness} in J/m.
\item {\bf K1:} Crystalline
 anisotropy\index{crystalline~anisotropy}\index{energy!crystalline~anisotropy}
   constant in J/m${}^3$.  If $K1>0$, then the anisotropy axis (or axes)
   is an easy axis; if $K1<0$ then the anisotropy axis is a hard axis.
\item {\bf Anisotropy Type:} Crystalline anisotropy type; One of
\texttt{<uniaxial|cubic>}.
\item {\bf Anisotropy Dir1:} Directional cosines of first crystalline
   anisotropy axis, taken with respect to the coordinate axes (3
   numbers).  Optional; Default is 1 0 0 (x-axis).
\item {\bf Anisotropy Dir2:} Directional cosines of second crystalline
   anisotropy axis, taken with respect to the coordinate axes (3
   numbers).  Optional; Default is 0 1 0 (y-axis).
   \html{\\}\latex{\par}
   For uniaxial materials it suffices to specify only Anisotropy
   Dir1.  For cubic materials one must also specify Anisotropy
   Dir2; the third axis direction will be calculated as the cross
   product of the first two.  The anisotropy directions will be
   automatically normalized if necessary, so for example 1 1 1 is
   valid input (it will be modified to .5774 .5774 .5774).  For cubic
   materials, Dir2 will be adjusted to be perpendicular to Dir1 (by
   subtracting out the component parallel to Dir1).
\item {\bf Anisotropy Init:}
   Method to use to set up directions of anisotropy axes, as a function
   of spatial location; This is a generalization of the Anisotropy
   Dir1/2 records.  The value for this record should be one of
   \texttt{<Constant|UniformXY|UniformS2>}.  \texttt{Constant} uses the
   values specified for Anisotropy Dir1 and Dir2, with no dispersion.
   \texttt{UniformXY} ignores the values given for Anisotropy Dir1 and
   Dir2, and randomly varies the anisotropy directions uniformly in the
   xy-plane.  \texttt{UniformS2} is similar, but randomly varies the
   anisotropy directions uniformly on the unit sphere ($S^2$).  This
   record is optional; the default value is \texttt{Constant}.
\item {\bf Edge K1:}  Anisotropy
   constant\index{edge~anisotropy}\index{energy!edge~anisotropy}
   similar to crystalline anisotropy constant K1 described above, but
   applied only along the edge surface of the part.  This is a uniaxial
   anisotropy, directed along the normal to the boundary surface.  Units
   are J/m${}^3$, with positive values making the surface normal an easy
   axis, and negative values making the surface an easy plane.  The
   default value for Edge K1 is 0, which disables the term.
\item {\bf Do Precess:}
   If 1, then enable the precession\index{precession} term in the
   Landau-Lifshitz ODE\index{ODE!Landau-Lifshitz}.  If 0, then do pure
   damping only.  (Optional; default value is 1.)
\item {\bf Gyratio:}
   The Landau-Lifshitz gyromagnetic ratio\index{gyromagnetic ratio}, in
   m/(A.s).  This is optional, with default value of \latex{$2.21\times
   10^5$}\html{2.21e5}.  See the discussion of the Landau-Lifshitz ODE
   under the Damp~Coef record identifier description.
\item {\bf Damp Coef:}
   The ODE solver in \OOMMF\ integrates the Landau-Lifshitz
   equation\index{ODE!Landau-Lifshitz}~\cite{gilbert1955,landau1935},
   written as
\begin{displaymath}
\htmlimage{antialias}
  \frac{d\vM}{dt} = -|\bar{\gamma}|\,\vM\times\vH_{\rm eff}
   - \frac{|\bar{\gamma}|\alpha}{M_s}\,
     \vM\times\left(\vM\times\vH_{\rm eff}\right),
\end{displaymath}
where
\latex{
\begin{eqnarray*}
  \bar{\gamma}  && \mbox{is the Landau-Lifshitz gyromagnetic ratio
    (m/(A$\cdot$s)),} \\
  \alpha    && \mbox{is the damping coefficient (dimensionless).}
\end{eqnarray*}
} % close latex
% Use BLOCKQUOTE until mmHelp supports tables.
\html{
\begin{quotation}
  \abovemath{\bar{\gamma}} is the Landau-Lifshitz gyromagnetic ratio
                        (m/(A.s)),\\
  $\alpha$           is the damping coefficient (dimensionless).
\end{quotation}
} % close html
  \latex{(Compare to (\ref{eq:oxsllode}), page~\pageref{eq:oxsllode}.)}%
  \html{(See also the discussion of the \htmlonlyref{Landau-Lifshitz-Gilbert
  equations}{eq:oxsllode} in the Oxs documentation.)}  Here $\alpha$ is
  specified by the ``Damp Coef'' entry in the \MIF~1.1 file.  If not
  specified, a default value of 0.5 is used, which allows the solver to
  converge in a reasonable number of iterations.  Physical materials
  will typically have a damping coefficient in the range 0.004 to
  0.15.  The 2D solver engine
  \hyperrefhtml{\app{mmSolve}}{\app{mmSolve}
  (Ch.~}{)}{sec:mmsolve}\index{application!mmSolve} requires a
  non-zero damping coefficient.
\end{itemize}

\subsection{Demag specification}\label{sec:mifdemagspec}
\begin{itemize}
   \item {\bf Demag Type:} Specify algorithm and demagnetization kernel
      used to calculate self-magnetostatic
      (demagnetization\index{demagnetization}) field.  Must be one of
      \begin{itemize}
         \item {\bf ConstMag:} Calculates the {\em average} field in
            each cell under the assumption that the magnetization is
            constant in each cell, using formulae from \cite{newell1993}.
            (The other demag options calculate the field at the center
            of each cell.)
         \item {\bf 3dSlab:}  Calculate the in-plane field components
            using offset blocks of constant (volume) charge.  Details
            are given in \cite{berkov1993}. Field components parallel to
            the $z$-axis are calculated using squares of constant
            (surface) charge on the upper and lower surfaces of the
            sample.
         \item {\bf 3dCharge:} Calculate the in-plane field component
            using rectangles of constant (surface) charge on each cell.
            This is equivalent to assuming constant magnetization in
            each cell.  The $z$-components of the field are calculated
            in the same manner as for the 3dSlab approach.
         \item {\bf FastPipe:} Algorithm suitable for simulations that
            have infinite extent in the $z$-direction.  This is a 2D
            version of the 3dSlab algorithm.
        \item {\bf None:} No demagnetization.  Fastest but least
            accurate method.  \verb+:-}+    %\html{ :-\}}
    \end{itemize}

    All of these algorithms except FastPipe and None require that the
    Part Thickness (cf.\ the \htmlonlyref{Part Geometry}{sec:partgeometry}
    section) be set.  Fast Fourier Transform (FFT)\index{FFT} techniques
    are used to accelerate the calculations.

\end{itemize}

\subsection{Part geometry}\label{sec:partgeometry}\index{part~geometry}
\begin{itemize}
   \item {\bf Part Width:} Nominal part width ($x$-dimension) in
      meters.  Should be an integral multiple of Cell~Size.
   \item {\bf Part Height:} Nominal part height ($y$-dimension) in
      meters.  Should be an integral multiple of Cell~Size.
   \item {\bf Part~Thickness:} Part thickness ($z$-dimension) in meters.
      Required for all demag types except FastPipe and None.
   \item {\bf Cell Size:}\index{cell~size} In-plane ($xy$-plane) edge
      dimension of base calculation cell.  This cell is a rectangular
      brick, with square in-plane cross-section and thickness given by
      Part~Thickness.  N.B.: Part~Width and Part~Height should be
      integral multiples of Cell~Size.  Part~Width and Part~Height will
      be automatically adjusted slightly (up to 0.01\%) to meet this
      condition (affecting a small change to the problem), but if the
      required adjustment is too large then the problem specification is
      considered to be invalid, and the solver will signal an error.
   \item {\bf Part Shape:} Optional.  Part shape in the $xy$-plane;
      must be one of the following:
   \begin{itemize}
      \item {\bf Rectangle}\\
         The sample fills the area specified by Part Width and Part
         Height. (Default.)
      \item {\bf Ellipse}\\
         The sample (or the magnetically active
         portion thereof) is an ellipse inscribed into the rectangular
         area specified by Part Width and Part Height.
      \item {\bf Ellipsoid}\\
         Similar to the Ellipse shape, but the part thickness is
         varied to simulate an ellipsoid, with axis lengths of
         Part Width, Part Height and Part Thickness.
      \item {\bf Oval r}\\
         Shape is a rounded rectangle, where each
         corner is replaced by a quarter circle with radius $r$, where
         \latex{$0\leq r\leq 1$}\html{$0 <= r <= 1$}
         is relative to the half-width of the rectangle.
      \item {\bf Pyramid overhang}\\
         Shape is a truncated pyramid, with ramp transition base
         width (overhang) specified in meters.
      \item {\bf Mask filename}\index{file!mask}\index{file!bitmap}%
         \label{html:mifvariablethickness}\\
         Shape and thickness are determined by a bitmap file, the name
         of which is specified as the second parameter.  The overall
         size of the simulation is still determined by Part Width and
         Part Height (above); the bitmap is spatially scaled to fit
         those dimensions.  Note that this scaling will not be square if
         the aspect ratio of the part is different from the aspect ratio
         of the bitmap.

         The given filename must be accessible to the solver
         application.  At present the bitmap file must be in either the
         PPM\index{file!ppm} (portable pixmap), GIF\index{file!gif}, or
         BMP\index{file!bmp} formats.  (Formats other than the PPM P3
         (text) format may be handled by spawning an
         {\hyperrefhtml{\app{any2ppm}}{\app{any2ppm}
         (Sec.~}{)}{sec:any2ppm}}
         subprocess\index{application!any2ppm}.)

         White areas of the bitmap are interpreted as being non-magnetic
         (or having 0 thickness); all other areas are assumed to be
         composed of the material specified in the ``Material
         Parameters'' section.  Thickness is determined by the relative
         darkness of the pixels in the bitmap.  Black pixels are given
         full nominal thickness (specified by the ``Part Thickness''
         parameter above), and gray pixels are linearly mapped to a
         thickness between the nominal thickness and 0.  In general,
         bitmap pixel values are converted to a thickness relative to
         the nominal thickness by the formula 1-(R+G+B)/(3M), where R, G
         and B are the magnitudes of the red, green and blue components,
         respectively, and M is the maximum allowed component magnitude.
         For example, black has R=G=B=0, so the relative thickness is 1,
         and white has R=G=B=M, so the relative thickness is 0.
   \end{itemize}
   The code does not perform a complete 3D evaluation of thickness
   effects.  Instead, the approximation discussed in \cite{porter2001} is
   implemented.
\end{itemize}

\subsection{Initial magnetization}\label{sec:mif1initmag}%
\index{magnetization~initial}
\begin{itemize}
   \item {\bf Init Mag:} Name of routine to use to initialize the
   simulation magnetization directions (as a function of position), and
   routine parameters, if any.  Optional, with default Random.  The list
   of routines is long, and it is easy to add new ones.  See the file
   \fn{oommf/app/mmsolve/maginit.cc} for details.  A few of the more
   useful routines are:
   \begin{itemize}
      \item {\bf Random}\\
         Random directions on the unit sphere.  This
         is somewhat like a quenched thermal demagnetized state.
      \item {\bf Uniform \latex{\boldmath$\theta$ $\phi$}\html{theta phi}}\\
         Uniform magnetization in the direction
         indicated by the two additional parameters, \latex{$\theta$ and
         $\phi$}\html{theta and phi}, where the first is the angle
         from the $z$-axis (in degrees), and the second is the angle
         from the $x$-axis (in degrees) of the projection onto the
         $xy$-plane.
      \item {\bf Vortex}\\
         Fits an idealized vortex\index{vortex} about the center of the
         sample.
      \item {\bf avfFile filename}\\
         The second parameter specifies an OVF/VIO (i.e., ``any''
         vector field) file\index{file!vector~field}\index{file!vio} to use to
         initialize the magnetization. The grid in the input file will
         be scaled as necessary to fit the grid in the current
         simulation.  The file must be accessible to the intended solver
         application.
   \end{itemize}
\end{itemize}

\subsection{Experiment parameters}\label{sec:expparams}
The following records specify the applied field schedule:
\begin{itemize}
   \item {\bf Field Range:}\index{field~range} Specifies a range of
      applied fields that are stepped though in a linear manner.  The
      parameter list should be 7 numbers, followed by optional control
      point (stopping criteria) specifications.  The 7 required fields
      are the begin field Bx By Bz in Tesla, the end field Bx By Bz in
      Tesla, and an integer number of steps (intervals) to take between
      the begin and end fields (inclusive).  Use as many Field Range
      records as necessary---they will be stepped through in order of
      appearance.  If the step count is 0, then the end field is ignored
      and only the begin field is applied.  If the step count is larger
      than 0, and the begin field is the same as the last field from the
      previous range, then the begin field is not repeated.

      The optional control point\index{simulation~2D!control~point} specs
      determine the conditions that cause the applied field to be
      stepped, or more precisely, end the simulation of the
      magnetization evolution for the current applied field.  The
      control point specs are specified as {\em --type value} pairs.
      There are 3 recognized control point types:
      {\bf --torque}\index{simulation~2D!mxh},
      {\bf --time}\index{simulation~2D!time}, and
      {\bf --iteration}\index{simulation~2D!iteration}.  If a --torque pair
      is given, then the simulation at the current applied field is
      ended when {\html{$|\vm\times\vh|$\ (i.e.,
      $|\vM\times\vH|/M_s^2$)}} {\latex{$\|\vm\times\vh\|$ (i.e.,
      $\|\vM\times\vH\|/M_s^2$)}} at all spins in the simulation is
      smaller than the specified --torque value (dimensionless).  If a
      --time pair is given, then the simulation at the current field is
      ended when the elapsed simulation time {\em for the current field
      step} reaches the specified --time value (in seconds).  Similarly,
      an --iteration pair steps the applied field when the iteration
      count for the current field step reaches the --iteration value.
      If multiple control point specs are given, then the applied field
      is advanced when any one of the specs is met.  If no control point
      specs are given on a range line, then the {\bf Default Control
      Point Spec} is used.

      For example, consider the following Field Range line:
\begin{verbatim}
   Field Range: 0 0 0 .05 0 0  5  -torque 1e-5 -time 1e-9
\end{verbatim}
      This specifies 6 applied field values, (0,0,0), (0.01,0,0),
      (0.02,0,0), \ldots, (0.05,0,0) (in Tesla), with the advancement
      from one to the next occurring whenever
      {\html{$|\vm\times\vh|$}}{\latex{$\|\vm\times\vh\|$}}
      is smaller than 1e-5 for all spins,
      or when 1 nanosecond (simulation time) has elapsed at the current
      field. (If --torque was not specified, then the applied field
      would be stepped at 1, 2, 3 4 and 5~ns in simulation time.)

      The Field Range record is optional, with a default value of 0 0 0
      0 0 0 0.

   \item {\bf Default Control Point Spec:} List of control
      point\index{simulation~2D!control~point} {\em --type value} pairs to
      use as stepping criteria for any field range with no control point
      specs.  This is a generalization of and replacement for the {\em
      Converge $|$mxh$|$ Value} record.  Optional, with default
      ``\cd{-torque~1e-5}.''

   \item {\bf Field Type:} Applied (external) field\index{field!applied}
     routine and parameters, if any. This is optional, with default
     Uniform.  At most one record of this type is allowed, but the Multi
     type may be used to apply a collection of fields.  The nominal
     applied field (NAF) is stepped through the Field Ranges described
     above, and is made available to the applied field routines which
     use or ignore it as appropriate.  \html{\\}\latex{\par} The
     following Field Type routines are available:
     \begin{itemize}
        \item {\bf Uniform}\\
          Applied field is uniform with value specified by the NAF.
        \item {\bf Ribbon relcharge x0 y0 x1 y1 height}\\
          Charge ``Ribbon,'' lying perpendicular to the $xy$-plane.
          Here relcharge is the charge strength relative to Ms, and
          (x0,y0), (x1,y1) are the endpoints of the ribbon (in
          meters).  The ribbon extends height/2 above and below the
          calculation plane.  This routine ignores the NAF.
        \item {\bf Tie rfx rfy rfz x0 y0 x1 y1 ribwidth}\\
          The points (x0,y0) and (x1,y1) define (in meters) the
          endpoints of the center spine of a rectangular ribbon of
          width ribwidth lying in the $xy$-plane.  The cells with
          sample point inside this rectangle see an applied field of
          (rfx,rfy,rfz), in units relative to Ms.  (If the field is
          large, then the magnetizations in the rectangle will be
          ``tied'' to the direction of that field.)  This routine
          ignores the NAF.
        \item {\bf OneFile filename multiplier}\\
          Read B field (in Tesla) in from a file.  Each value in the
          file is multiplied by the ``multiplier'' value on input.  This
          makes it simple to reverse field direction (use -1 for the
          multiplier), or to convert H fields to B fields (use
          1.256637e-6).  The input file may be any of the vector field
          file types recognized by \app{mmDisp}.  The input dimensions
          will be scaled as necessary to fit the simulation grid, with
          zeroth order interpolation as necessary.  This routine ignores
          the NAF.
        \item {\bf FileSeq filename procname multiplier}\\
          This is a generalization of the OneFile routine that reads in
          fields from a sequence of files.  Here ``filename'' is the
          name of a file containing Tcl code to be sourced during
          problem initialization, and ``procname'' is the name of a Tcl
          procedure defined in filename, which takes the nominal B field
          components (in Tesla) and field step count values as imports
          (4 values total), and returns the name of the vector field
          file that should be used as the applied B field for that field
          step.  The B field units in the vector field file should be
          Tesla.
        \item {\bf Multi routinecount \bs\\
                   param1count name1 param1 param2 \ldots \bs\\
                   param2count name2 param1 param2 \ldots \bs\\
                   \ldots}\\
          Allows a conglomeration of several field type routines.  All
          entries must be on the same logical line, i.e., end physical
          lines with '\bs' continuation characters as necessary.
          Here routinecount is the number of routines, and param1count
          is the number parameters (including name1) needed by the
          first routine, etc.
     \end{itemize}
     Note that all lengths are in meters.  The coordinates in the
     simulation lie in the first octant, running from (0,0,0) to
     (Part~Width, Part~Height, Part~Thickness).
\end{itemize}

\subsection{Output specification}\label{sec:mif1outspec}
\begin{itemize}
   \item {\bf Base Output Filename:} Default base name used to
      construct output filenames.
   \item {\bf Magnetization Output Format:}\index{file!magnetization}
      Format to use in the \hyperrefhtml{\OVF}{\OVF\
      (Sec.~}{)}{sec:ovfformat} data block for exported magnetization
      files.  Should be one of ``binary~4'' (default), ``binary 8'', or
      ``text {\em format-spec}'', where {\em format-spec} is a C
      \cd{printf}-style format code, such as ``\cd{\%\lb~.17g}''.
       Optional.
   \item {\bf Total Field Output Format:}\index{file!vector~field}
      Analogous to the Magnetization Output Format, but for total field
      output files.  Optional, with default ``binary~4''.
   \item {\bf Data Table Output Format:}\index{file!data~table}
      Format to use when producing data table style scalar output, such
      as that sent to
      \hyperrefhtml{\app{mmDataTable}}{\app{mmDataTable}
      (Ch.~}{)}{sec:mmdatatable},
      \hyperrefhtml{\app{mmGraph}}{\app{mmGraph}
      (Ch.~}{)}{sec:mmgraph}, and
      \hyperrefhtml{\app{mmArchive}}{\app{mmArchive}
      (Ch.~}{)}{sec:mmarchive}.
      Should specify a C \cd{printf}-style format code, such as the
      default ``\cd{\%.16g}''.  Optional.
\end{itemize}

\subsection{Miscellaneous}\label{sec:mifmisc}
\begin{itemize}
   \item {\bf\boldmath Converge $|$mxh$|$ Value:}\index{simulation~2D!mxh}
      Nominal value to use as a stopping criterion: When
      {\html{$|\vm\times\vh|$\ (i.e., $|\vM\times\vH|/M_s^2$)}}
      {\latex{$\|\vm\times\vh\|$ (i.e., $\|\vM\times\vH\|/M_s^2$)}}
      at all spins in the simulation is smaller than this value, it is
      assumed that a relaxed (equilibrium) state has been reached for
      the current applied field.  This is a dimensionless value.\\
      {\bf NOTE:} This Record Identifier is deprecated.  Use {\em
      Default Control Point Spec} instead.
  \item {\bf Randomizer Seed:} Value with which to seed random
      number\index{random~numbers} generator.  Optional.  Default value
      is 0, which uses the system clock to generate a semi-random seed.
  \item {\bf Max Time Step:} Limit the maximum ODE step
      size\index{ODE!step~size} to no larger than this amount, in
      seconds.  Optional.
  \item {\bf Min Time Step:} Limit the minimum ODE step size to no
      less than this amount, in seconds.  Optional.
  \item {\bf User Comment:} Free-form comment string that may be used
      for problem identification.  Optional.
\end{itemize}

\begin{codelisting}{p}{fig:mif1sample}{Example \MIF~1.1
      file.}{sec:mif1format}{ref}
\begin{verbatim}
# MIF 1.1
#
# Example from the OOMMF User's Guide.
#
# All units are SI.
#
################# MATERIAL PARAMETERS ######################
Ms:  800e3                # Saturation magnetization in A/m.
A:   13e-12               # Exchange stiffness in J/m.
K1:  0.5e3                # Anisotropy constant in J/m^3.
Anisotropy Type: uniaxial # One of <uniaxial|cubic>.
Anisotropy Dir1: 1 0 0    # Directional cosines wrt to
                          # coordinate axes

################# DEMAG SPECIFICATION ######################
Demag Type: ConstMag # One of <ConstMag|3dSlab|2dSlab
                     #         |3dCharge|FastPipe|None>.

#################### PART GEOMETRY #########################
Part Width:     0.25e-6    # Nominal part width in m
Part Height:    1.0e-6     # Nominal part height in m
Part Thickness: 1e-9       # Part thickness in m.
Cell Size:      7.8125e-9  # Cell size in m.
#Part Shape:    # One of <Rectangle|Ellipse|Oval|Mask>.
                # Optional.

################ INITIAL MAGNETIZATION #####################
Init Mag: Uniform 90 45 # Initial magnetization routine
                        # and parameters

################ EXPERIMENT PARAMETERS #####################
# Field Range:  Start_field  Stop_field  Steps
Field Range: -.05 -.01 0.  .05  .01 0. 100
Field Range:  .05  .01 0. -.05 -.01 0. 100
Field Type: Multi 4 \
 7 Ribbon 1 0 1.0e-6 0.25e-6 1.0e-6 1e-9 \
 7 Ribbon 1 0 0      0.25e-6 0      1e-9 \
 9 Tie 100 0 0 0.12e-6 0.5e-6 0.13e-6 0.5e-6 8e-9 \
 1 Uniform
# The above positions ribbons of positive charge along the
# upper and lower edges with strength Ms, applies a large
# (100 Ms) field to the center cells, and also applies a
# uniform field across the sample stepped from
# (-.05,-.01,0.) to (.05,.01,0.) (Tesla), and back, in
# approximately 0.001 T steps.

Default Control Point Spec: -torque 1e-6
# Assume equilibrium has been reached, and step the applied
# field, when the reduced torque |mxh| drops below 1e-6.

################ OUTPUT SPECIFICATIONS #####################
Base Output Filename: samplerun
Magnetization Output Format: binary 8 # Save magnetization
# states in binary format with full (8-byte) precision.

#################### MISCELLANEOUS #########################
Randomizer Seed: 1   # Random number generator seed.
User Comment: Example MIF 1.1 file, with lots of comments.
\end{verbatim}
\end{codelisting}

%%%%%%%%%%%%%%%%%%%%%%%%%%%%%%%%%%%%%%%%%%%%%%%%%%%%%%%%%%%%%%%%%%%%%%%%
\section{\MIF\ 1.2}\label{sec:mif12format}
The \MIF~1.2 format is a minor modification to the \MIF~1.1 format,
which supports limited 3D problem specification.  It can be read by the
Oxs 3D solvers, and, with certain restrictions, by the mmSolve 2D
solvers as well.  The \app{mmProbEd} problem editor can read and
write this format.  The
\hyperrefhtml{\app{mifconvert}}{\app{mifconvert}
(Sec.~}{)}{sec:mifconvert} command line tool can be used to convert
between the \MIF~1.1 and \MIF~1.2 formats, and to convert from the
\MIF~1.x formats to the Oxs \MIF~2.1 format.  \app{mifconvert} is also
called automatically by Oxs solvers when a \MIF~1.x file is input to
them, so questions about the details of Oxs interpretation of \MIF~1.x
files can be answered by running \app{mifconvert} separately on the
input \MIF~1.x file.

There are four differences between the \MIF~1.1 and 1.2 formats.  In
the \MIF~1.2 format:
\begin{enumerate}
\item The first line reads: \verb+# MIF 1.2+
\item The \texttt{CellSize} record takes three parameters:
      $x$-dimension, $y$-dimension, and $z$-dimension (in meters).
\item The \texttt{3dSlab}, \texttt{2dSlab}, \texttt{3dCharge}, and
      \texttt{FastPipe} parameters of the \texttt{DemagType} record
      are deprecated.
\item The new record \texttt{SolverType} is introduced.  Valid values
      are \texttt{Euler}, \texttt{RK2}, \texttt{RK4}, \texttt{RKF54},
      and \texttt{CG}, requesting a first order Euler, second order
      Runge-Kutta, fourth order Runge-Kutta, fifth(+fourth) order
      Runge-Kutta-Fehlberg, and Conjugate-Gradient solvers,
      respectively.  This record is optional, with default value of
      \texttt{RKF54}.
\end{enumerate}
If the \texttt{CellSize} record has only one parameter, then it is
interpreted in the same manner as in the \MIF~1.1 format, i.e., the
parameter is taken as the $x$- and $y$-dimension of the computation
cell, and the $z$-dimension is set to the part thickness.

The mmSolve 2D solvers will accept files in the \MIF~1.2 format provided
the \texttt{CellSize} record meets the restrictions of those solvers,
namely, the $x$- and $y$-dimensions must be the same, and the
$z$-dimension must equal the part thickness.  The \texttt{SolverType}
record, if any, is ignored.

The Oxs 3D solvers will read files in the \MIF~1.2 format, but support
only the \texttt{ConstMag} and \texttt{None} demagnetization kernels.
All other \texttt{DemagType} records are silently converted
to \texttt{ConstMag}.  The \texttt{SolverType} record is converted into
the appropriate solver+driver pair.

\section{\MIF\ 2.1}\label{sec:mif2format}
The \MIF~2.x format was introduced with the
\hyperrefhtml{Oxs 3D solver}{Oxs 3D solver
(Ch.~}{)}{sec:oxs}\index{application!Oxsii}\index{application!Boxsi}.
It is \textit{not} backwards compatible with the \MIF~1.x formats,
however a conversion utility,
\hyperrefhtml{\app{mifconvert}}{\app{mifconvert}
(Sec.~}{)}{sec:mifconvert}\index{application!mifconvert}, is available
to aid in converting \MIF~1.x files to the \MIF~2.1 format.

\subsection{\MIF\ 2.1 File Overview}%
\label{sec:mif2overview}\index{MIF~2.1~Overview}

The first line of a \MIF\ file must be of the form ``\verb+#+ MIF x.y'',
where x.y represents the format revision number, here 2.1.  Unlike
\MIF~1.1 files, the structure of \MIF~2.1 files are governed by the
requirement that they be valid \Tcl\ scripts, albeit with a handful of
extensions.  These files are evaluated inside a \Tcl\ interpreter, which
may be a ``safe'' interpreter, i.e., one in which disk and other system
access is disabled.  (Refer to the documentation of the \Tcl\
\cd{interp} command for details on safe interpreters.)  The security
level is controlled by the \cd{MIFinterp} option in the \fn{options.tcl}
\hyperrefhtml{customization file}{customization file
(Sec.~}{)}{sec:install.custom}.
The default setting is
\begin{rawhtml}
<BLOCKQUOTE>
\end{rawhtml}
%begin{latexonly}
\begin{quote}
%end{latexonly}
\begin{verbatim}
Oc_Option Add Oxs* MIFinterp safety custom
\end{verbatim}
%begin{latexonly}
\end{quote}
%end{latexonly}
\begin{rawhtml}
</BLOCKQUOTE>
\end{rawhtml}
which enables all the \Tcl\ interpreter extensions described in
\html{the}
\hyperrefhtml{\MIF~2.1 Extension Commands}{\MIF~2.1 Extension Commands
(Sec.~}{)}{sec:mif2ExtensionCommands}
\html{section}
below, but otherwise provides a standard \Tcl\ safe interpreter.  The
keyword \cd{custom} above may be replaced with either \cd{safe} or
\cd{unsafe}.  The \cd{safe} selection is similar to \cd{custom}, except
that the \cd{DateSort}, \cd{ReadFile} and \cd{RGlob} extensions are
not provided, thereby eliminating all disk access at the \MIF\ script
level.  At the other extreme, choosing \cd{unsafe} provides an
unrestricted \Tcl\ interpreter.  This option should be used with
caution, especially if you are working with \MIF\ files from an
unknown or untrusted source.

After the first line, there is considerable flexibility in the layout of
the file.  Generally near the top of the file one places any
\htmlonlyref{\cd{OOMMFRootDir}}{html:mif2oommfrootdir},
\htmlonlyref{\cd{Parameter}}{html:mif2parameter}, and
\htmlonlyref{\cd{RandomSeed}}{html:mif2randomseed} statements, as
desired.

This is followed by the major content of the file, the various
\htmlonlyref{\cd{Specify}}{html:specifyCmd} blocks, which initialize
\hyperrefhtml{\cd{Oxs\_Ext} objects}{\cd{Oxs\_Ext} objects
(Sec.~}{, page~\pageref{sec:oxsext})}{sec:oxsext}:
\begin{itemize}
  \item Atlas (one or more)
  \item Mesh (one)
  \item Energy terms (one or more)
  \item Evolver (one)
  \item Driver(one)
\end{itemize}
The \cd{Specify} blocks are processed in order, so any block that is
referred to by another block must occur earlier in the file.  For that
reason, the main atlas object, which is referenced in many other
\cd{Specify} blocks, is generally listed first.  The atlas object
defines the spatial extent of the simulation, and optionally declares
subregions inside the simulation volume.

The mesh object details the spatial discretization of the simulation
volume.  Conventionally its \cd{Specify} block follows the \cd{Specify}
block for the main atlas object; the mesh is referenced by the driver,
so in any event the mesh \cd{Specify} block needs to precede the driver
\cd{Specify} block.

The energy terms describe the typical micromagnetic energies and fields
that determine the evolution of the simulation, such as exchange energy,
magnetostatic fields, and anisotropy energies.  Material parameters,
such as the anisotropy constant \cd{K1} and the exchange constant
\cd{A}, are generally specified inside the \cd{Specify} block for the
relevant energy, e.g., \cd{Oxs\_UniaxialAnisotropy} or
\cd{Oxs\_Exchange6Ngbr}.  The exception to this is saturation
magnetization, \cd{Ms}, which is declared in the driver \cd{Specify} block.
The initial magnetization, \cd{m0}, is also specified in the driver
\cd{Specify} block.  In many cases these material parameters may be
varied spatially by defining them using scalar or vector
\hyperrefhtml{field objects}{field objects (Sec.~}{,
page~\pageref{sec:oxsFieldObjects})}{sec:oxsFieldObjects}.  As discussed
in the section on \hyperrefhtml{Specify Conventions}{Specify Conventions
(Sec.~}{)}{sec:specConventions}, auxiliary objects such as scalar and
vector fields can be defined either inline (i.e., inside the body of the
referencing \cd{Specify} block) or in their own, standalone top-level
\cd{Specify} blocks.  In the latter case, the auxiliary
\cd{Specify} blocks must precede the referencing \cd{Specify} blocks in
the \MIF~2.1 file.

Given the energies and fields, the evolver and driver form a matched
pair that advance the magnetic state from an initial configuration,
obeying either Landau-Lifshitz-Gilbert (LLG) dynamics or direct energy
minimization.  For energy minimization studies, the driver must be an
\cd{Oxs\_MinDriver} object, and the evolver must be a minimization
evolver.  At the time of this writing, the only minimization evolver
packaged with \OOMMF\ is the \cd{Oxs\_CGEvolve} conjugate-gradient
evolver.  For time-evolution (LLG) simulations, the driver must be an
\cd{Oxs\_TimeDriver} object, and the evolver must be a time evolver,
such as \cd{Oxs\_RungeKuttaEvolve}.  The evolver to be used is cited
inside the driver \cd{Specify} block, so the evolver must precede the
driver in the \MIF~2.1 file.  As noted above, the pointwise saturation
magnetization \cd{Ms} and initial magnetization configuration \cd{m0}
are declared inside the driver \cd{Specify} block as well.

The pre-specified outputs, indicated by zero or more
\htmlonlyref{\cd{Destination}}{html:destinationCmd} and
\htmlonlyref{\cd{Schedule}}{html:scheduleCmd} commands, are
conventionally placed after the \cd{Specify} blocks.  Output selection
can also be modified at runtime using the
\hyperrefhtml{\app{Oxsii}}{\app{Oxsii} (Sec.~}{,
page~\pageref{sec:oxsii})}{sec:oxsii} or
\hyperrefhtml{\app{Boxsi}}{\app{Boxsi} (Sec.~}{,
page~\pageref{sec:boxsi})}{sec:boxsi} interactive interfaces.

Auxiliary \Tcl\ procs may be placed anywhere in the file, but
commonly either near their point of use or else at the bottom of the
\MIF\ file.  If a proc is only referenced from inside \cd{Specify} blocks, then
it can be placed anywhere in the file.  On the other hand, if a proc is
used at the top level of the \MIF\ file, for example to dynamically
create part of the problem specification ``on-the-fly,'' then it must be
defined before it is used, in the normal \Tcl\ manner.

A sample \MIF~2.1 file is \hyperrefhtml{included below.}{presented in
Fig.~}{ (Sec.~\ref{sec:mif2sample},
pages~\pageref{sec:mif2sample}--\pageref{fig:mif2sample}).}{fig:mif2sample}
More details on the individual \cd{Oxs\_Ext} objects can be found in the
\hyperrefhtml{Standard Oxs\_Ext Child Classes section}{Standard Oxs\_Ext
Child Classes portion (Sec.~}{, page~\pageref{sec:oxsext})}{sec:oxsext}
of the Oxs documentation.


\subsection{\MIF~2.1 Extension Commands}%
\label{sec:mif2ExtensionCommands}\index{MIF~2.1~Commands}
In addition to the standard \Tcl\ commands (modulo the safe \Tcl\
restrictions outlined above), a number of additional commands are
available in \MIF~2.1 files:
\htmlonlyref{\cd{Specify}}{html:specifyCmd},
\htmlonlyref{\cd{ClearSpec}}{html:mif2ClearSpec},
\htmlonlyref{\cd{DateSort}}{html:mif2datesort},
\htmlonlyref{\cd{Destination}}{html:destinationCmd},
\htmlonlyref{\cd{GetStateData}}{html:GetStateData},
\htmlonlyref{\cd{Ignore}}{html:mif2Ignore},
\htmlonlyref{\cd{OOMMFRootDir}}{html:mif2oommfrootdir},
\htmlonlyref{\cd{Parameter}}{html:mif2parameter},
\htmlonlyref{\cd{Random}}{html:mif2Random},\NONHTMLoutput{\linebreak}
\htmlonlyref{\cd{RandomSeed}}{html:mif2randomseed},
\htmlonlyref{\cd{Report}}{html:MifReport},
\htmlonlyref{\cd{ReadFile}}{html:ReadFile},
\htmlonlyref{\cd{RGlob}}{html:mif2rglob},
and \htmlonlyref{\cd{Schedule}}{html:scheduleCmd}.

\begin{description}
\item[Specify\label{html:specifyCmd}]\index{Specify~block~(MIF)}
An Oxs simulation is built as a collection of \cd{Oxs\_Ext} (Oxs
Extension) objects.  In general, \cd{Oxs\_Ext} objects are specified and
initialized in the input \MIF~2.1 file using the \cd{Specify} command,
making Specify blocks the primary component of the problem
definition.  The \cd{Specify} command takes two arguments: the name of
the \cd{Oxs\_Ext} object to create, and an {\it initialization string}
that is passed to the \cd{Oxs\_Ext} object during its construction.
The objects are created in the order in which they appear in the \MIF\
file. Order is important in some cases; for example, if one \cd{Oxs\_Ext}
object refers to another in its initialization string, then the referred
to object must precede the referrer in the \MIF\ file.

Here is a simple Specify block:
\begin{rawhtml}
<BLOCKQUOTE>
\end{rawhtml}
%begin{latexonly}
\begin{quote}
%end{latexonly}
\begin{verbatim}
Specify Oxs_EulerEvolve:foo {
  alpha 0.5
  start_dm 0.01
}
\end{verbatim}
%begin{latexonly}
\end{quote}
%end{latexonly}
\begin{rawhtml}
</BLOCKQUOTE>
\end{rawhtml}
The name of the new \cd{Oxs\_Ext} object is ``Oxs\_EulerEvolve:foo.''
The first part of this name, up to the colon, is the the \Cplusplus\
class name of the object.  This must be a child of the \cd{Oxs\_Ext}
class.  Here, \cd{Oxs\_EulerEvolve} is a class that integrates the
Landau-Lifshitz ODE using a simple forward Euler method.  The second
part of the name, i.e., the part following the colon, is
the \textit{instance name}\index{instance~name~(MIF~2)} for
this particular instance of the object.  In general, it is possible to
have multiple instances of an \cd{Oxs\_Ext} child class in a simulation,
but each instance must have a unique name.  These names are used for
identification by output routines, and to allow one Specify block to
refer to another Specify block appearing earlier in the \MIF\ file.  If
the second part of the name is not given, then as a default the empty
string is appended.  For example, if instead of ``Oxs\_EulerEvolve:foo''
above the name was specified as just ``Oxs\_EulerEvolve'', then the
effective full name of the created object would be
``Oxs\_EulerEvolve:''.

The second argument to the \cd{Specify} command, here everything between
the curly braces, is a string that is interpreted by the new
\cd{Oxs\_Ext} (child) object in its constructor.  The format of this
string is up to the designer of the child class, but there are a number
of conventions that designers are encouraged to follow.  These
conventions are described in\html{ the}
\hyperrefhtml{Specify Conventions}{Specify Conventions,
Sec.~}{,}{sec:specConventions}\html{ section} below.

\item[ClearSpec\label{html:mif2ClearSpec}]
This command is used to disable one or all preceding \cd{Specify}
commands.  In particular, one could use \cd{ClearSpec} to nullify a
Specify block from a base \MIF\ file that was imported using the
\cd{ReadFile} command.  Sample usage is
\begin{rawhtml}
<BLOCKQUOTE>
\end{rawhtml}
%begin{latexonly}
\begin{quote}
%end{latexonly}
\begin{verbatim}
ClearSpec Oxs_EulerEvolve:foo
\end{verbatim}
%begin{latexonly}
\end{quote}
%end{latexonly}
\begin{rawhtml}
</BLOCKQUOTE>
\end{rawhtml}
where the parameter is the full name (here \cd{Oxs\_EulerEvolve:foo}) of
the Specify block to remove.  If no parameter is given, then all
preceding Specify blocks are removed.

\item[DateSort\label{html:mif2datesort}]
Given a list of filenames, returns the list sorted by modification
time, with oldest first. The files must all exist. Sample usage is
\begin{rawhtml}
  <BLOCKQUOTE>
\end{rawhtml}
%begin{latexonly}
\begin{quote}
%end{latexonly}
\begin{verbatim}
set last_omf [lindex [DateSort [RGlob *omf]] end]
\end{verbatim}
%begin{latexonly}
\end{quote}
%end{latexonly}
\begin{rawhtml}
  </BLOCKQUOTE>
\end{rawhtml}
Here \cd{last\_omf} will be set to the \fn{*.omf} file in the current
directory with the most recent modification time.

\item[Destination\label{html:destinationCmd}]
\index{Destination~command~(MIF)}\index{nicknames}
The format for the \cd{Destination} command is
\begin{rawhtml}
<BLOCKQUOTE>
\end{rawhtml}
%begin{latexonly}
\begin{quote}
%end{latexonly}
\begin{verbatim}
Destination <desttag> <appname> [new]
\end{verbatim}
%begin{latexonly}
\end{quote}
%end{latexonly}
\begin{rawhtml}
</BLOCKQUOTE>
\end{rawhtml}
This command associates a symbolic \textit{desttag} with an
application.  The tags are used by the
\htmlonlyref{\cd{Schedule}}{html:scheduleCmd} command\latex{ (see below)}
to refer to specific application instances.  The \textit{appname} may
either be an \OOMMF\ application name, e.g., mmDisp, or else a
specific application instance in the form application:nickname, such
as mmDisp:Spock.  In the first case, the tag is associated with the
running instance of the requested application (here \cd{mmDisp}) with
the lowest \OOMMF\ ID (OID) that has not yet been associated with
another tag.  If no running application can be found that meets these
criteria, then a new instance of the application is launched.

If the \textit{appname} refers to a specific application instance, then
the tag is associated with the running instance of the application (say
\app{mmDisp}) that has been assigned the specified nickname.  Name
matching is case insensitive.  If there is no running copy of the
application meeting this condition, then a new instance of the
application is launched and it is assigned the specified nickname.  The
\OOMMF\ \htmlonlyref{account service directory}{sec:arch} guarantees
that there is never more than one instance of an application with a
given nickname.  However, as with the object name in the \cd{Specify}
command, instances of two different applications, e.g., \app{mmDisp} and
\app{mmGraph}, are allowed to share nicknames, because their full
instance names, say mmDisp:Spock and mmGraph:Spock, are unique.

The \cd{Destination} commands are processed in the order in which they
appear in the the \MIF\ file.  No \textit{desttag} may appear in more
than one \cd{Destination} command, and no two destination tags may refer
to the same application instance.  To insure the latter, the user is
advised to place all \cd{Destination} commands referring to specific
instances (e.g., mmDisp:Spock) before any \cd{Destination} commands
using generic application references (e.g., mmDisp).  Otherwise a
generic reference might be associated to a running application holding a
nickname that is referenced by a later \cd{Destination} command.

The tag association by the \cd{Destination} command is only known to the
solver reading the \MIF\ file.  In contrast, assigned instance nicknames
are recognized across applications.  In particular, multiple solvers may
reference the same running application by nickname.  For example,
several sequential solver runs could send stage output to the same
\app{mmGraph} widget, to build up overlapping hysteresis loops.

The last parameter to the Destination command is the optional
\cd{new} keyword.  If present, then a fresh copy of the requested
application is always launched for association with the given tag.  The
\cd{new} option can be safely used with any generic application
reference, but caution must be taken when using this option with
specific instance references, because an error is raised if the
requested nickname is already in use.

\item[GetStateData\label{html:GetStateData}]
The \cd{GetStateData} command retrieves data attached to a specific
magnetization state:
\begin{rawhtml}
<BLOCKQUOTE>
\end{rawhtml}
%begin{latexonly}
\begin{quote}
%end{latexonly}
\begin{verbatim}
GetStateData [-glob|-exact|-regexp] [-pairs] [--] <state_id> \
             [pattern ...]
\end{verbatim}
%begin{latexonly}
\end{quote}
%end{latexonly}
\begin{rawhtml}
</BLOCKQUOTE>
\end{rawhtml}
The data associated with a state are stored as key-value pairs.
If no patterns are specified then \cd{GetStateData} returns the list of
keys available for the given state.  If one or more patterns are
specified, then all values with keys matching some pattern are
collected.  If the \cd{-pairs} option is specified then the return is an
even length list of keys and values interleaved.  If \cd{-pairs} is not
specified then the return is a list of just the values.  The values are
returned in key-match order.  Key matching style is controlled by the
first slate of options, with default being glob.  Two hyphens may be
used to denote the end of options.

The \cd{state\_id} is a positive integer identifying the state.  This is
generally obtained via a \cd{script\_args} option in the
\cd{Specify} block of a conforming \cd{Oxs\_Ext} object.  For example,
\begin{rawhtml}
<BLOCKQUOTE>
\end{rawhtml}
%begin{latexonly}
\begin{quote}
%end{latexonly}
\begin{verbatim}
proc SpinMag { stage_time state_id } {
   lassign [GetStateData $state_id *:Mx *:My *:Mz] Mx My Mz
   ...
}
Specify Oxs_ScriptUZeeman {
  script SpinMag
  script_args {stage_time base_state_id}
}
\end{verbatim}
%begin{latexonly}
\end{quote}
%end{latexonly}
\begin{rawhtml}
</BLOCKQUOTE>
\end{rawhtml}
Typically two states may be accessed this way: the step base state
and a candidate (test) state.  The former, accessed
as \cd{base\_state\_id}, corresponds to the last valid, accepted
magnetization state.  The latter, accessed as \cd{current\_state\_id},
is the latest working state from the evolver object.  In some cases
these two states may coincide.

The keys associated with a state vary with the details of the
simulation.  The following keys are always available:
\begin{rawhtml}
<BLOCKQUOTE>
\end{rawhtml}
%begin{latexonly}
\begin{quote}
%end{latexonly}
\begin{tabular}{l@{\hskip 2em}l@{\hskip 2em}l}
\cd{state\_id}            & \cd{previous\_state\_id}     & \cd{iteration\_count}\\
\cd{stage\_number}        & \cd{stage\_iteration\_count} & \cd{stage\_start\_time}\\
\cd{stage\_elapsed\_time} & \cd{total\_elapsed\_time}    & \cd{last\_timestep}\\
\cd{step\_done}           & \cd{stage\_done}            & \cd{run\_done}\\
\cd{max\_absMs}
\end{tabular}
%begin{latexonly}
\end{quote}
%end{latexonly}
\begin{rawhtml}
</BLOCKQUOTE>
\end{rawhtml}
Times are in seconds, the \cd{step}/\cd{stage}/\cd{run\_done} values are
one of 1 (done), 0 (not done) or -1 (not yet determined),
and \cd{max\_absMs} is in A/m.

Additional key-value pairs may be attached to a state by \cd{Oxs\_Ext}
objects.  For example,
\htmlonlyref{\cd{Oxs\_RungeKuttaEvolve}}{html:RungeKuttaEvolve}
adds the average magnetization $x$-component under the key name
\cd{Oxs\_RungeKuttaEvolve:{\oab}instance\_name{\cab}:Mx}.
(Here \cd{{\oab}instance\_name{\cab}} is the instance name of the
object; this is typically an empty string or something like
``evolver''.)  See the documentation for the various \cd{Oxs\_Ext}
objects for details.

Moreover, the keys available for a state may depend on the simulation
status or processing step.  In particular, the current state indexed
by \cd{current\_state\_id} typically only has the default keys from the
table above available.  For this reason, and additionally because
the \cd{current\_state\_id} is only a test state that may be rejected,
user scripts should generally avoid using data tied to the current state
in favor of data collected from the base state.  Likewise, the available
keys may be different for a state (even a base state) marking the start
of a new stage as compared to states arising inside a stage.  In case of
problems, a \cd{Report} command inside a script proc can be used to
dump state information to the Oxsii console, for example,
\begin{rawhtml}
<BLOCKQUOTE>
\end{rawhtml}
%begin{latexonly}
\begin{quote}
%end{latexonly}
\begin{verbatim}
proc SpinMag { stage_time state_id } {
   Report "State $state_id, Keys: [GetStateData $base_state_id]"
   lassign [GetStateData $state_id *:Mx *:My *:Mz] Mx My Mz
   ...
}
\end{verbatim}
%begin{latexonly}
\end{quote}
%end{latexonly}
\begin{rawhtml}
</BLOCKQUOTE>
\end{rawhtml}
or
\begin{rawhtml}
<BLOCKQUOTE>
\end{rawhtml}
%begin{latexonly}
\begin{quote}
%end{latexonly}
\begin{verbatim}
proc SpinMag { stage_time state_id } {
   set report {}
   foreach {key value} [GetStateData $state_id *] {
      append report [format "%42s : $value\n" $key]
   }
   Report "--- State data ---\n$report"
   lassign [GetStateData $state_id *:Mx *:My *:Mz] Mx My Mz
   ...
}
\end{verbatim}
%begin{latexonly}
\end{quote}
%end{latexonly}
\begin{rawhtml}
</BLOCKQUOTE>
\end{rawhtml}

For example use of \cd{GetStateData}, see the sample files
\fn{spinmag.mif} and \fn{spinmag2.mif} in the
directory \fn{oommf/app/oxs/examples/}.


\item[Ignore\label{html:mif2Ignore}]
The \cd{Ignore} command takes an arbitrary number of arguments, which
are thrown away without being interpreted.  The primary use of
\cd{Ignore} is to temporarily ``comment out'' (i.e., disable) Specify
blocks.

\item[OOMMFRootDir\label{html:mif2oommfrootdir}]
This command takes no arguments, and returns the full directory path of
the \OOMMF\ root directory.  This is useful in conjunction with the
\cd{ReadFile} command for locating files within the \OOMMF\ hierarchy,
and can also be used to place output files.  File paths must be created
directly since the \Tcl\ \cd{file} command is not accessible inside safe
interpreters.  For example
\begin{rawhtml}
<BLOCKQUOTE>
\end{rawhtml}
%begin{latexonly}
\begin{quote}
%end{latexonly}
\begin{verbatim}
set outfile [OOMMFRootDir]/data/myoutput
\end{verbatim}
%begin{latexonly}
\end{quote}
%end{latexonly}
\begin{rawhtml}
</BLOCKQUOTE>
\end{rawhtml}
In this context one should always use \Tcl\ path conventions.  In
particular, use forward slashes, ``\fs'', to separate directories.

\item[Parameter\label{html:mif2parameter}]
The Oxs interfaces
(\hyperrefhtml{Oxsii}{Oxsii, Sec.~}{}{sec:oxsii} and
\hyperrefhtml{Boxsi}{Boxsi, Sec.~}{}{sec:boxsi})
allow specified variables in the \MIF\ file to be set from the command
line via the \cd{-parameters} option.  This functionality is enabled
inside the \MIF\ file via the \cd{Parameter} command:
\begin{quote}
\cd{Parameter} \textit{varname} \textit{optional\_default\_value}
\end{quote}
Here \textit{varname} is the name of a variable that may be set from
the command line.  If it is not set on the command line then the
variable is set to the optional default value, if any, or otherwise an
error is raised.  An error is also raised if a variable set on the
command line does not have a corresponding \cd{Parameter} command in the
\MIF\ file.  See also the notes on
\hyperrefhtml{variable substitution}{variable substitution
(Sec.~}{)}{sec:varSubst} below.

\item[Random\label{html:mif2Random}]
Returns a pseudo-random number in the interval $[0,1]$, using a
C-library random number generator.  This random number generator is
specified by the \cd{OMF\_RANDOM} macro in the \fn{ocport.h} file found in
the system-specific subdirectory of \fn{oommf/pkg/oc/}.  The standard
\Tcl\ \cd{expr rand()} command is also available.

\item[RandomSeed\label{html:mif2randomseed}]
Initializes both the \Tcl\ and the C-library random number generators.
If no parameter is given, then a seed is drawn from the system clock.
Otherwise, one integer parameter may be specified to be used as the
seed.

\item[Report\label{html:MifReport}]
Intended primarily as a \MIF\ debugging aid, \cd{Report} takes one
string argument that is printed to the solver interface console and the
Oxs log file.  It is essentially a replacement for the standard \Tcl\
\cd{puts} command, which is not available in safe interpreters.

\item[ReadFile\label{html:ReadFile}]
The \Tcl\ \cd{read} command is absent from safe interpreters.  The
\cd{ReadFile} command is introduced as a replacement available in
``custom'' and ``unsafe'' interpreters.  \cd{ReadFile} takes two
arguments, the file to be read and an optional translation
specification.  The file may either be specified with an absolute path,
i.e., one including all its directory components, or with a relative
path interpreted with respect to the directory containing the \MIF\
file.  The \cd{OOMMFRootDir} command can be used to advantage to locate
files in other parts of the \OOMMF\ directory tree.

The translation specification should be one of \cd{binary}, \cd{auto}
(the default), \cd{image} or \cd{floatimage}\index{file!bitmap}.  The
first two translation modes provide the functionality of the
\cd{-translation} option of the \Tcl\ \cd{fconfigure} command.  Refer to
the \Tcl\ documentation for details.  Specifying \cd{image} causes the
input file to be read as an image file.  The file will be read
directly if it in the PPM P3 (text), PPM P6 (binary), or uncompressed
BMP formats; otherwise it is filtered through the \OOMMF\
\hyperrefhtml{\app{any2ppm}}{\app{any2ppm}(Sec.~}{)}{sec:any2ppm}
program.  (Note that \app{any2ppm} requires Tk, and Tk requires a
display.)  The input file is converted into a string that mimics a PPM
P3 text file, minus the leading ``P3''.  In particular, after conversion
the first 3 whitespace separated values are image width, height and
maxvalue, followed by an array of 3 $\times$ width $\times$ height
values, where each triplet corresponds to the red, green and blue
components of an image pixel, sequenced in normal English reading order.
Each component is in the range $[0,maxvalue]$.  This output contains no
comments, and may be treated directly as a \Tcl\ list.  The
\cd{floatimage} option is very similar to the \cd{image} option, except
that the third value (i.e., maxvalue) in the resulting string is always
``1'', and the succeeding red, green and blue values are floating point
values in the range $[0,1]$.

In all cases, the return value from the \cd{ReadFile} command is a
string corresponding to the contents of the (possibly translated) file.
For example,
\begin{rawhtml}
<BLOCKQUOTE>
\end{rawhtml}
%begin{latexonly}
\begin{quote}
%end{latexonly}
\begin{verbatim}
eval [ReadFile extra_mif_commands.tcl]
\end{verbatim}
%begin{latexonly}
\end{quote}
%end{latexonly}
\begin{rawhtml}
</BLOCKQUOTE>
\end{rawhtml}
can be used to embed a separate \Tcl\ file into a \MIF~2.1 file.

\index{file!bitmap|(}%
Here's a more complicated example that uses a color image file to
create a vector field:
\begin{rawhtml}
<BLOCKQUOTE>
\end{rawhtml}
%begin{latexonly}
\begin{quote}
%end{latexonly}
\begin{verbatim}
set colorimage [ReadFile mirror.ppm floatimage]
set imagewidth [lindex $colorimage 0]
set imageheight [lindex $colorimage 1]
set imagedepth [lindex $colorimage 2] ;# Depth value should be 1
if {$imagedepth != 1} {
   Report "ReadFile returned unexpected list value."
}

proc ColorField { x y z } {
     global colorimage imagewidth imageheight
     set i [expr {int(floor(double($x)*$imagewidth))}]
     if {$i>=$imagewidth} {set i [expr {$imagewidth-1}]}
     set j [expr {int(floor(double(1-$y)*$imageheight))}]
     if {$j>=$imageheight} {set j [expr {$imageheight-1}]}
     set index [expr {3*($j*$imagewidth+$i)+3}]  ;# +3 is to skip header
     set vx [expr {2*[lindex $colorimage $index]-1}] ; incr index  ;# Red
     set vy [expr {2*[lindex $colorimage $index]-1}] ; incr index  ;# Green
     set vz [expr {2*[lindex $colorimage $index]-1}] ; incr index  ;# Blue
     return [list $vx $vy $vz]
}

Specify Oxs_ScriptVectorField:sample {
   atlas :atlas
   norm 1.0
   script ColorField
}
\end{verbatim}
%begin{latexonly}
\end{quote}
%end{latexonly}
\begin{rawhtml}
</BLOCKQUOTE>
\end{rawhtml}
\index{file!bitmap|)}
If the input image is large, then it is best to work with the image list
(i.e., the variable \cd{colorimage} in the preceding example) directly,
as illustrated above.  The image list as returned by \cd{ReadFile} is in
a packed format; if you make modifications to the list values then the
memory footprint of the list can grow substantially.

The \cd{ReadFile} command is not available if the \cd{MIFinterp safety}
option is set to \cd{safe} in the \fn{options.tcl}
\hyperrefhtml{customization file}{customization file
(Sec.~}{)}{sec:install.custom}.

\item[RGlob\label{html:mif2rglob}]
This command is modeled on the \Tcl\ \cd{glob} command (q.v.), but
is restricted to the current working directory, that is, the
directory holding the \MIF\ file.  The syntax is
\begin{rawhtml}
<BLOCKQUOTE>
\end{rawhtml}
%begin{latexonly}
\begin{quote}
%end{latexonly}
\begin{verbatim}
RGlob [-types typelist] [--] <pattern> [...]
\end{verbatim}
%begin{latexonly}
\end{quote}
%end{latexonly}
\begin{rawhtml}
</BLOCKQUOTE>
\end{rawhtml}
The optional \texttt{typelist} restricts the match to files meeting the
typelist criteria.  The optional \verb+--+ switch marks the end of
options.  The one or more \texttt{pattern}'s should be glob-style
patterns (strings containing asterisks and question marks) intended to
match filenames in the current working directory.  See the \Tcl\
\cd{glob} documentation for details on the \texttt{-types} option and
glob pattern details.

One use of this command is to identify files created by earlier runs of
Oxs.  For example, suppose we wanted to use the mmArchive magnetization
output from the third stage of a previous MIF file with basename ``sample''.
Output files are tagged by stage number (here ``2'' since stages are
counted from 0) and iteration.  The iteration is generally not known a
priori, but assuming the output files are in the same directory as the
current MIF file, we could use a command like
\begin{rawhtml}
<BLOCKQUOTE>
\end{rawhtml}
%begin{latexonly}
\begin{quote}
%end{latexonly}
\begin{verbatim}
set file [RGlob sample-Oxs_MinDriver-Magnetization-02-???????.omf]
\end{verbatim}
%begin{latexonly}
\end{quote}
%end{latexonly}
\begin{rawhtml}
</BLOCKQUOTE>
\end{rawhtml}
to determine the name of the magnetization file.  If more than one
magnetization state was saved for that stage, then the variable
\cd{file} will hold a list of filenames.  In this case the \Tcl\ \cd{lsort}
command can be used to select the one with the highest iteration number.
The \cd{file} variable can be used in conjunction with the
\htmlonlyref{\cd{Oxs\_FileVectorField}}{item:FileVectorField}
class to import the magnetization into the new simulation, for example
to set the initial magnetization configuration.

The \cd{RGlob} command is not available if the \cd{MIFinterp safety}
option is set to \cd{safe} in the \fn{options.tcl}
\hyperrefhtml{customization file}{customization file
(Sec.~}{)}{sec:install.custom}.  If \cd{MIFinterp safety} is set to
\cd{unsafe}, then the standard (and more capable) \Tcl\
\cd{glob} command will be available.

\item[Schedule\label{html:scheduleCmd}]\index{Schedule~command~(MIF)}
The \cd{Schedule} command is used to setup outputs from the \MIF\ file.
This functionality is critical for solvers running in batch mode, but is
also useful for setting up default connections in interactive mode.

The syntax for the \cd{Schedule} command is
\begin{rawhtml}
<BLOCKQUOTE>
\end{rawhtml}
%begin{latexonly}
\begin{quote}
%end{latexonly}
\begin{verbatim}
Schedule <outname> <desttag> <event> <frequency>
\end{verbatim}
%begin{latexonly}
\end{quote}
%end{latexonly}
\begin{rawhtml}
</BLOCKQUOTE>
\end{rawhtml}
The \cd{Schedule} command mirrors the interactive output scheduling
provided by the
\htmlonlyref{\app{Oxsii}}{sec:oxsii} and
\htmlonlyref{\app{Boxsi}}{sec:boxsi}
graphical interfaces\latex{ (Ch.~\ref{sec:oxs})}.  The first parameter
to the \cd{Schedule} command is the name of the output being scheduled.
These names are the same as those appearing in the ``Outputs'' list in
the Oxs graphical interfaces, for example ``DataTable'' or
``Oxs\_CubicAnisotropy:Nickel:Field.''  The name must be presented to
the \cd{Schedule} command as a single argument; if the name includes one
or more spaces then use double quotes to protect the spaces.  Aside from
the DataTable output which is always present, the
\textit{outname}'s are \MIF\ file dependent.

The second parameter to the \cd{Schedule} command is a destination tag.
This is a tag associated to a running application by a previous
\htmlonlyref{\cd{Destination} command}{html:destinationCmd}\latex{ (see
above)}.  The symbolic destination tag replaces the application:OID
nomenclature used in the graphical interface, because in general it is
not possible to know the \OOMMF\ ID for application instances at the
time the \MIF\ file is composed.  In fact, some of the applications may
be launched by \cd{Destination} commands, and so don't even have OID's
at the time the \cd{Destination} command is processed.

The \textit{event} parameter should be one of the keywords \cd{Step},
\cd{Stage}, or \cd{Done}.  For \cd{Step} and \cd{Stage} events the
\textit{frequency} parameter should be a non-negative integer, representing
with what frequency of the specified event should output be dispatched.
For example, if \cd{Step 5} is given, then on every fifth step of the
solver (iterations 0, 5, 10, \ldots) output of the indicated type will
be sent to the selected destination.  Set \textit{frequency} to 1 to
send output each time the event occurs. A value of 0
for \textit{frequency} results in output on only the very first event of
that type; in particular, \cd{Step 0} will output the simulation initial
state, but will not fire on any subsequent \cd{Step} events.
The \cd{Done} event occurs at the successful completion of a simulation;
as such, there is at most one ``Done'' event per simulation.
Accordingly, the \textit{frequency} parameter for \cd{Done} events is
optional; if present it should be the value 1.

There are examples of scheduling with the \cd{Destination} and
\cd{Schedule} commands in the \hyperrefhtml{sample \MIF~2.1
file}{sample \MIF~2.1 file presented in Fig.~}{}{fig:mif2sample}\latex{
(Sec.~\ref{sec:mif2sample},
pages~\pageref{sec:mif2sample}--\pageref{fig:mif2sample})}.  There,
three destinations are tagged.  The first refers to a possibly already
running instance of \app{mmGraph}, having nickname Hysteresis.  The
associated \cd{Schedule} command sends DataTable output to this
application at the end of each Stage, so hysteresis graphs can be
produced.  The second destination tag references a different copy of
\app{mmGraph} that will be used for monitoring the run.  To make sure
that this output is rendered onto a blank slate, the \texttt{new}
keyword is used to launch a fresh copy of \app{mmGraph}.  The
\cd{Schedule} command for the monitor destination delivers output to
the monitoring \app{mmGraph} every 5 iterations of the solver.  The last
\cd{Destination} command tags an arbitrary \app{mmArchive} application,
which is used for file storage of DataTable results at the end of each
stage, and snapshots of the magnetization and total field at the end of
every third stage.  Note that double quotes enclose the
``Oxs\_EulerEvolve::Total field'' output name.  Without the quotes, the
\cd{Schedule} command would see five arguments,
``Oxs\_EulerEvolve::Total'', ``field'', ``archive'', ``Stage'', and
``3''.
\end{description}

\subsection{Specify Conventions}%
\label{sec:specConventions}\index{Specify~conventions~(MIF)}
The Specify blocks in the input \MIF\ file determine the collection
of \cd{Oxs\_Ext} objects defining the Oxs simulation.  As explained
above, the \htmlonlyref{\cd{Specify}}{html:specifyCmd} command takes two
arguments, the name of the \cd{Oxs\_Ext} object to create, and an
initialization string.  The format of the initialization string can
be arbitrary, as determined by the author of the \cd{Oxs\_Ext} class.
This section presents a number of recommended conventions which
\cd{Oxs\_Ext} class authors are encouraged to follow.  Any \cd{Oxs\_Ext}
classes that don't follow these conventions should make that fact
explicitly clear in their documentation.
Details on the standard \cd{Oxs\_Ext} classes included with \OOMMF\ can
be found in the
\hyperrefhtml{Oxs section}{Oxs documentation (Ch.~}{)}{sec:oxs}\html{
of this document}.

\starsssechead{Initialization string format}%
\label{par:specInitString}\index{Specify~initialization~string~(MIF)}
Consider again the simple Specify block presented above:
\begin{rawhtml}
<BLOCKQUOTE>
\end{rawhtml}
%begin{latexonly}
\begin{quote}
%end{latexonly}
\begin{verbatim}
Specify Oxs_EulerEvolve:foo {
  alpha 0.5
  start_dm 0.01
}
\end{verbatim}
%begin{latexonly}
\end{quote}
%end{latexonly}
\begin{rawhtml}
</BLOCKQUOTE>
\end{rawhtml}
The first convention is that the initialization string be structured as
a \Tcl\ list\index{Tcl~list} with an even number of elements, with
consecutive elements consisting of a label + value pairs.  In the above
example, the initialization string consists of two label + value pairs,
``alpha 0.5'' and ``start\_dm 0.01''.  The first specifies that the
damping parameter $\alpha$ in the Landau-Lifshitz ODE is 0.5.  The
second specifies the initial step size for the integration routine.
Interested parties should refer to a \Tcl\ programming reference (e.g.,
\cite{welch2000}) for details on forming a proper \Tcl\ list, but in short
the items are separated by whitespace, and grouped by double quotes or
curly braces (``\verb+{+'' and ``\verb+}+'').  Opening braces and quotes
must be whitespace separated from the preceding text.  Grouping
characters are removed during parsing.  In this example the list as a
whole is set off with curly braces, and individual elements are white
space delimited.  Generally, the ordering of the label + value pairs in
the initialization string is irrelevant, i.e., \cd{start\_dm 0.01} could
equivalently precede \cd{alpha 0.5}.

Sometimes the value portion of a label + value pair will itself be a list,
as in this next example:
\begin{latexonly}
\begin{quote}\tt
Specify Oxs\_BoxAtlas:myatlas \ocb\\
\bi\bi\ldots\\
\ccb\\
\\
Specify Oxs\_RectangularMesh:mymesh \ocb \\
\bi cellsize \ocb\ 5e-9 5e-9 5e-9 \ccb\\
\bi atlas Oxs\_BoxAtlas:myatlas\\
\ccb
\end{quote}
\end{latexonly}
\begin{rawhtml}<BLOCKQUOTE><DL><DT>
<TT>Specify Oxs_BoxAtlas:myatlas {</TT>
<DD>...
<DT><TT>}</TT><P>
<DT><DT><TT>Specify Oxs_RectangularMesh:mymesh {</TT>
<DD><TT>cellsize { 5e-9 5e-9 5e-9 }</TT>
<DD><TT>atlas Oxs_BoxAtlas:myatlas</TT>
<DT><TT>}</TT></DL>
</BLOCKQUOTE><P>
\end{rawhtml}
Here the value associated with ``cellsize'' is a list of 3 elements,
which declare the sampling rate along each of the coordinate axes, in
meters.  (\cd{Oxs\_BoxAtlas} is a particular type of \cd{Oxs\_Atlas},
and ``\ldots'' mark the location of the \cd{Oxs\_BoxAtlas}
initialization string, which is omitted because it is not pertinent to
the present discussion.)

\starsssechead{\cd{Oxs\_Ext} referencing}%
\label{par:oxsExtReferencing}\index{Oxs\_Ext~referencing~(MIF)}
The ``atlas'' value in the mesh Specify block of the preceding example
refers to an earlier \cd{Oxs\_Ext} object, ``Oxs\_BoxAtlas:myatlas''.
It frequently occurs that one \cd{Oxs\_Ext} object needs access to
another \cd{Oxs\_Ext} object.  In this example the mesh object
\cd{:mymesh} needs to query the atlas object \cd{:myatlas} in order to
know the extent of the space that is to be gridded.  The atlas object is
defined earlier in the \MIF\ input file by its own, separate, top-level
Specify block, and the mesh object refers to it by simply specifying its
name.  Here the full name is used, but the short form \cd{:myatlas}
would suffice, provided no other \cd{Oxs\_Ext} object has the same short
name.

Alternatively, the \cd{Oxs\_RectangularMesh} object could define an
\cd{Oxs\_BoxAtlas} object inline:
\begin{latexonly}
\begin{quote}\tt
Specify Oxs\_RectangularMesh:mymesh \ocb \\
\bi atlas \ocb\\
\bi\bi Oxs\_BoxAtlas \ocb\\
\bi\bi\bi\ldots\\
\bi\bi\ccb\\
\bi\ccb\\
\bi cellsize \ocb\ 5e-9 5e-9 5e-9 \ccb\\
\ccb
\end{quote}
\end{latexonly}
\begin{rawhtml}<BLOCKQUOTE><DL><DT>
<DT><DT><TT>Specify Oxs_RectangularMesh:mymesh {</TT>
<DD><TT>atlas { Oxs_BoxAtlas {</TT>
<DD><DL><DD>...</DL>
<DD><TT>}}</TT>
<DD><TT>cellsize { 5e-9 5e-9 5e-9 }</TT>
<DT><TT>}</TT></DL>
</BLOCKQUOTE><P>
\end{rawhtml}
In place of the name of an external atlas object, a
two item list is provided consisting of the type of object (here
\texttt{Oxs\_BoxAtlas}) and the corresponding initialization
string.  The initialization string is provided as a sublist, with the
same format that would be used if that object were initialized via a
separate Specify block.

More commonly, embedded \cd{Oxs\_Ext} objects are used to initialize
spatially varying quantities.  For
example,
\begin{rawhtml}
<BLOCKQUOTE>
\end{rawhtml}
%begin{latexonly}
\begin{quote}
%end{latexonly}
\begin{verbatim}
Specify Oxs_UniaxialAnisotropy {
  axis { Oxs_RandomVectorField {
           min_norm 1
           max_norm 1
  }}
  K1  { Oxs_UniformScalarField { value 530e3 } }
}
\end{verbatim}
%begin{latexonly}
\end{quote}
%end{latexonly}
\begin{rawhtml}
</BLOCKQUOTE>
\end{rawhtml}
\sloppy
The magneto-crystalline anisotropy class \cd{Oxs\_UniaxialAnisotropy}
supports cellwise varying K1 and anisotropy axis directions.  In this
example, the anisotropy axis directions are randomly distributed.  To
initialize its internal data structure, \cd{Oxs\_UniaxialAnisotropy}
creates a local \cd{Oxs\_RandomVectorField} object.  This object is
also a child of the \cd{Oxs\_Ext} hierarchy, which allows it to be
constructed using the same machinery invoked by the \cd{Specify}
command.  However, it is known only to the enclosing
\cd{Oxs\_UniaxialAnisotropy} object, and no references to it are
possible, either from other Specify blocks or even elsewhere inside
the same initialization string.  Because it cannot be referenced, the
object does not need an instance name.  It does need an initialization
string, however, which is given here as the 4-tuple ``min\_norm 1
max\_norm 1''.  Notice how the curly braces are nested so that this
4-tuple is presented to the \cd{Oxs\_RandomVectorField} initializer as a
single item, while ``Oxs\_RandomVectorField'' and the associated
initialization string are wrapped up in another \Tcl\ list, so that the
value associated with ``axis'' is parsed at that level as a single item.

\fussy
The value associated with ``K1'' is another embedded \cd{Oxs\_Ext}
object.  In this particular example, K1 is desired uniform (homogeneous)
throughout the simulation region, so the trivial
\cd{Oxs\_UniformScalarField} class is used for initialization (to the value
\latex{$530\times 10^{3}$~J/m${}^3$}\html{530e3 J/m^3}).  In the case of
uniform fields, scalar or vector, a shorthand notation is available that
implicitly supplies a uniform \cd{Oxs\_Ext} field class:
\begin{rawhtml}
<BLOCKQUOTE>
\end{rawhtml}
%begin{latexonly}
\begin{quote}
%end{latexonly}
\begin{verbatim}
Specify Oxs_UniaxialAnisotropy {
  axis { 1 0 0 }
  K1  530e3
}
\end{verbatim}
%begin{latexonly}
\end{quote}
%end{latexonly}
\begin{rawhtml}
</BLOCKQUOTE>
\end{rawhtml}
which is equivalent to
\begin{rawhtml}
<BLOCKQUOTE>
\end{rawhtml}
%begin{latexonly}
\begin{quote}
%end{latexonly}
\begin{verbatim}
Specify Oxs_UniaxialAnisotropy {
  axis { Oxs_UniformVectorField {
           vector { 1 0 0 }
  }}
  K1  { Oxs_UniformScalarField { value 530e3 } }
}
\end{verbatim}
%begin{latexonly}
\end{quote}
%end{latexonly}
\begin{rawhtml}
</BLOCKQUOTE>
\end{rawhtml}

While embedding \cd{Oxs\_Ext} objects inside Specify blocks can be
convenient, it is important to remember that such objects are not
available to any other \cd{Oxs\_Ext} object---only objects declared via
top-level Specify blocks may be referenced from inside other
Specify blocks.  Also, embedded \cd{Oxs\_Ext} objects cannot
directly provide user output.  Furthermore, the only \cd{Oxs\_Energy}
energy objects included in energy and field calculations are those
declared via top-level Specify blocks.  For this reason
\cd{Oxs\_Energy} terms are invariably created via top-level Specify
blocks, and not as embedded objects.

\starsssechead{Grouped lists}%
\label{par:groupedLists}\index{grouped~lists~(MIF)}
As noted earlier, sometimes the value portion of a label + value pair
will be a list.  Some Oxs objects support \textit{grouped lists}, which
provide a type of run-length encoding for lists.  Consider the sample
list
\begin{verbatim}
   { 1.1  1.2  1.2  1.2  1.2  1.3 }
\end{verbatim}
In a grouped list the middle run of 1.2's may be represented as a
sublist with a repeat count of 4, like so
\begin{verbatim}
   { 1.1  { 1.2  4 } 1.3 :expand: }
\end{verbatim}
Here the \cd{:expand:} keyword, when appearing as the last element of
the top level list, enables the group expansion mechanism.  Any
preceding element, such as \cd{\ocb\ 1.2 4 \ccb}, that 1) is a proper
sublist, and 2) has a positive integer as the last element, is treated
as a grouped sublist with repeat count given by the last element.  No
element of the top-level list is ever interpreted as a repeat count.
For example, the short form of the list
\begin{verbatim}
   { 1e-9 1e-9 1e-9 1e-9 1e-9 1e-9 }
\end{verbatim}
is
\begin{verbatim}
   { { 1e-9 6 } :expand: }
\end{verbatim}
Note the additional level of brace grouping.  Grouped lists may also be
nested, as in this example
\begin{verbatim}
   { 5.0 { 5.1 { 5.2 3 } 5.3 2 } :expand: }
\end{verbatim}
which is equivalent to
\begin{verbatim}
   { 5.0  5.1  5.2  5.2  5.2  5.3  5.1  5.2  5.2  5.2  5.3 }
\end{verbatim}
There are some difficulties with this mechanism when the list components
are strings, such as filenames, that may contain embedded spaces.  For
example, consider the list
\begin{verbatim}
   { "file 3" "file 3" "5 file" }
\end{verbatim}
If we tried to write this as
\begin{verbatim}
   { { "file 3" 2 } "5 file" :expand: }
\end{verbatim}
we would find that, because of the nested grouping rules, this grouped
list gets expanded into
\begin{verbatim}
   { file file file file file file "5 file" }
\end{verbatim}
Here the trailing ``3'' in ``file 3'' is interpreted as a repeat count.
Following normal \Tcl\ rules, the double quotes are treated as equivalents
to braces for grouping purposes.  However, the keyword \cd{:noexpand:}
may be used to disable further expansion, like so
\begin{verbatim}
   { { {"file 3" :noexpand:} 2 } "5 file" :expand: }
\end{verbatim}
The \cd{:noexpand:} keyword placed at the end of a list disables all
group expansion in that list.  Although it is an unlikely example,
if one had a flat, i.e., non-grouped list with last element ``:expand:'',
then one would have to disable the grouping mechanism that would
otherwise be invoked by appending \cd{:noexpand:} to the list.  In
flat lists generated by program code, it is recommended to append
\cd{:noexpand:} just to be certain that the list is not expanded.

As a matter of nomenclature, standard (i.e., flat) lists and single
values are also considered grouped lists, albeit trivial ones.  Any Oxs
object that accepts grouped lists in its Specify block should explicitly
state so in its documentation.

\starsssechead{Comments}%
\label{par:specifyComments}\index{Specify~comments~(MIF)}
The standard \Tcl\ commenting mechanism treats all text running from an
initial \lb\ symbol through to the end of a line as a comment.  You may
note in the above examples that newlines are treated the same as
other whitespace inside the curly braces delimiting the \cd{Specify}
initialization string.  Because of this and additional reasons, \Tcl\
comments cannot be used inside Specify blocks.  Instead, by
convention any label + value pair where label is ``comment'' is treated
as a comment and thrown away.  For example:
\begin{rawhtml}
<BLOCKQUOTE>
\end{rawhtml}
%begin{latexonly}
\begin{quote}
%end{latexonly}
\begin{verbatim}
Specify Oxs_UniaxialAnisotropy {
  axis { 1 0 0 }
  comment {K1  4500e3}
  K1 530e3
  comment { 530e3 J/m^3 is nominal for Co }
}
\end{verbatim}
%begin{latexonly}
\end{quote}
%end{latexonly}
\begin{rawhtml}
</BLOCKQUOTE>
\end{rawhtml}
Pay attention to the difference between ``comment'' used here as the
label portion of a label + value pair, and the \MIF\ extension command
``Ignore'' used outside Specify blocks.  In particular, \cd{Ignore}
takes an arbitrary number of arguments, but the value element associated
with a comment label must be grouped as a single element, just as any
other value element.

\starsssechead{Attributes}%
\label{par:specifyAttributes}\index{Specify~attributes~(MIF)}
Sometimes it is convenient to define label + value pairs outside a
particular Specify block, and then import them using the
``attributes'' label.  For example:
\begin{rawhtml}
<BLOCKQUOTE>
\end{rawhtml}
%begin{latexonly}
\begin{quote}
%end{latexonly}
\begin{verbatim}
Specify Oxs_LabelValue:probdata {
  alpha 0.5
  start_dm 0.01
}

Specify Oxs_EulerEvolve {
  attributes :probdata
}
\end{verbatim}
%begin{latexonly}
\end{quote}
%end{latexonly}
\begin{rawhtml}
</BLOCKQUOTE>
\end{rawhtml}
The \cd{Oxs\_LabelValue} object is an \cd{Oxs\_Ext} class that does
nothing except hold label + value pairs.  The ``attributes'' label acts as an
include statement, causing the label + value pairs contained in the
specified \cd{Oxs\_LabelValue} object to be embedded into the enclosing
\cd{Specify} initialization string.  This technique is most useful if the
label + value pairs in the \cd{Oxs\_LabelValue} object are used in
multiple Specify blocks, either inside the same \MIF\ file, or
across several \MIF\ files into which the \cd{Oxs\_LabelValue} block is
imported using the \cd{ReadFile} \MIF\ extension command.

\starsssechead{User defined support procedures}%
\label{par:supportProcs}\index{Specify~support~procs~(MIF)}
A number of \cd{Oxs\_Ext} classes utilize user-defined \Tcl\
procedures (procs) to provide extended runtime functionality.  The
most common examples are the various field initialization script
classes, which call a user specified \Tcl\ proc for each point in the
simulation discretization mesh.  The proc returns a value, either
scalar or vector, which is interpreted as some property of the
simulation at that point in space, such as saturation magnetization,
anisotropy properties, or an external applied field.

Here is an example proc that may be used to set the initial
magnetization configuration into an approximate vortex state, with a
central core in the positive $z$ direction:
\begin{rawhtml}
<BLOCKQUOTE>
\end{rawhtml}
%begin{latexonly}
\begin{quote}
%end{latexonly}
\begin{verbatim}
proc Vortex { x_rel y_rel z_rel } {
   set xrad [expr {$x_rel-0.5}]
   set yrad [expr {$y_rel-0.5}]
   set normsq [expr {$xrad*$xrad+$yrad*$yrad}]
   if {$normsq <= 0.0125} {return "0 0 1"}
   return [list [expr {-1*$yrad}] $xrad 0]
}
\end{verbatim}
%begin{latexonly}
\end{quote}
%end{latexonly}
\begin{rawhtml}
</BLOCKQUOTE>
\end{rawhtml}
The return value in this case is a 3D vector representing the spin
direction at the point \texttt{(x\_rel,y\_rel,z\_rel)}.  Procs
that are used to set scalar properties, such as saturation
magnetization $M_s$, return a scalar value instead.  But in both
cases, the import argument list specifies a point in the simulation
mesh.

In the above example, the import point is specified relative to the
extents of the simulation mesh.  For example, if \texttt{x\_rel} were
0.1, then the $x$-coordinate of the point is one tenth of the way
between the minimum $x$ value in the simulation  and the maximum $x$
value.  In all cases \texttt{x\_rel} will have a value between 0 and
1.

In most support proc examples, relative coordinates are the most
flexible and easiest representation to work with.  However, by
convention, scripting \cd{Oxs\_Ext} classes also support absolute
coordinate representations.  The representation used is selected in the
\cd{Oxs\_Ext} object Specify block by the optional \cd{script\_args}
entry.  The \Tcl\ proc itself is specified by the \cd{script} entry,
as seen in this example:
\begin{rawhtml}
<BLOCKQUOTE>
\end{rawhtml}
%begin{latexonly}
\begin{quote}
%end{latexonly}
\begin{verbatim}
proc SatMag { x y z } {
   if {$z < 20e-9} {return 8e5}
   return 5e5
}

Specify ScriptScalarField:Ms {
   atlas :atlas
   script_args { rawpt }
   script SatMag
}
\end{verbatim}
%begin{latexonly}
\end{quote}
%end{latexonly}
\begin{rawhtml}
</BLOCKQUOTE>
\end{rawhtml}
The value associated with the label \cd{script\_args} should in this
case be a subset of \cd{\ocb relpt rawpt minpt maxpt span scalars
vectors\ccb}, as explained in the
\htmlonlyref{\cd{Oxs\_ScriptScalarField}}{item:ScriptScalarField}
documentation\latex{ (page~\pageref{item:ScriptScalarField})}.  Here
\cd{rawpt} provides the point representation in problem coordinates,
i.e., in meters.  Other \cd{Oxs\_Ext} objects support a different list
of allowed \cd{script\_args} values.  Check the documentation of the
\cd{Oxs\_Ext} object in question for details.  Please note that the
names used in the proc argument lists above are for exposition purposes
only.  You may use other names as you wish.  It is the order of the
arguments that is important, not their names.  Also, \MIF~2.1 files are
parsed first in toto before the Specify blocks are evaluated, so the
support procs may be placed anywhere in a \MIF~2.1 file, regardless of
the location of the referencing Specify blocks.  Conversely,
\hyperrefhtml{\MIF~2.2}{\MIF~2.2 (Sec.~}{)}{sec:mif22format}
files are parsed in a single pass, with Specify blocks evaluated as they
are read.  Therefore for \MIF~2.2 files it is generally best to place
proc definitions ahead of Specify blocks in which they are referenced.

The command call to the \Tcl\ support proc is actually built up by
appending to the \cd{script} value the arguments as specified by the
\cd{script\_args} value.  This allows additional arguments to the
\Tcl\ proc to be specified in the \cd{script} value, in which case
they will appear in the argument list in front of the
\cd{script\_args} values.  The following is equivalent to the
preceding example:
\begin{rawhtml}
<BLOCKQUOTE>
\end{rawhtml}
%begin{latexonly}
\begin{quote}
%end{latexonly}
\begin{verbatim}
proc SatMag { zheight Ms1 Ms2 x y z } {
   if {$z < $zheight} {return $Ms1}
   return $Ms2
}

Specify ScriptScalarField:Ms {
   script_args { rawpt }
   script {SatMag 20e-9 8e5 5e5}
}
\end{verbatim}
%begin{latexonly}
\end{quote}
%end{latexonly}
\begin{rawhtml}
</BLOCKQUOTE>
\end{rawhtml}
Notice in this case that the \cd{script} value is wrapped in curly
braces so that the string \cd{SatMag 20e-9 8e5 5e5} will be treated as
the single value associated with the label \cd{script}.

As seen in the earlier example using the \cd{Vortex} \Tcl\ proc,
support procedures in \MIF~2.1 files will frequently make use of the
\Tcl\ \cd{expr} command.  If you are using \Tcl\ version 8.0 or later,
then the cpu time required by the potentially large number of calls to
such procedures can be greatly reduced by grouping the arguments to
\cd{expr} commands in curly braces, as illustrated in the
\cd{Vortex} example.  The braces aid the operation of the \Tcl\
byte code compiler, although there are a few rare situations involving
multiple substitution where such bracing cannot be applied.  See the
\Tcl\ documentation for the \cd{expr} command for details.

\index{file!mask|(}%
Sometimes externally defined data can be put to good use inside a
\Tcl\ support proc, as in this example:
\begin{rawhtml}
<BLOCKQUOTE>
\end{rawhtml}
%begin{latexonly}
\begin{quote}
%end{latexonly}
\begin{verbatim}
# Lay out a 6 x 16 mask, at global scope.
set mask {
   1 1 1 1 1 1 1 1 1 1 1 1 1 1 1 1
   1 1 1 1 1 1 1 1 1 1 1 1 1 1 1 1
   1 1 1 1 1 1 1 1 1 1 1 1 1 1 1 1
   1 1 1 1 1 0 0 0 0 0 0 1 1 1 1 1
   1 1 1 1 1 0 0 0 0 0 0 1 1 1 1 1
   1 1 1 1 1 0 0 0 0 0 0 1 1 1 1 1
}

proc MyShape { xrel yrel znotused } {
   global mask  ;# Make mask accessible inside proc
   set Ms 8e5   ;# Saturation magnetization of element
   set xindex [expr {int(floor($xrel*16))}]
   set yindex [expr {5 - int(floor($yrel*6))}]
   set index [expr {$yindex*16+$xindex}]
   # index references point in mask corresponding
   # to (xrel,yrel)
   return [expr {[lindex $mask $index]*$Ms}]
}
\end{verbatim}
%begin{latexonly}
\end{quote}
%end{latexonly}
\begin{rawhtml}
</BLOCKQUOTE>
\end{rawhtml}
The variable \cd{mask} holds a \Tcl\ list of 0's and 1's defining a part
shape.  The mask is brought into the scope of the \cd{MyShape} proc via
the \Tcl\ \cd{global} command.  The relative $x$ and $y$ coordinates are
converted into an index into the list, and the proc return value is
either 0 or 8e5 depending on whether the corresponding point in the mask
is 0 or 1.  This is essentially the same technique used in the
\cd{ColorField} proc example presented in the
\htmlonlyref{\cd{ReadFile}}{html:ReadFile} \MIF\ extension command
documented above\latex{ (Sec.~\ref{sec:mif2ExtensionCommands})},
except that there the data structure values are built from a separate
image file rather than from data embedded inside the \MIF\ file.
\index{file!mask|)}

\starsssechead{User defined scalar outputs}%
\label{par:userScalarOutputs}\index{Specify~user~scalar~outputs~(MIF)}
\OOMMF\ \cd{Oxs\_Ext} objects support a general method to allow users to
define scalar (DataTable) outputs derived from vector field outputs.
These scalar outputs are defined by ``\cd{user\_output}'' sub-blocks inside
Specify blocks.  The format is:
      \begin{latexonly}
      \begin{quote}\tt
      user\_output \ocb\\
       \bi name \oxsval{output\_name}\\
       \bi source\_field \oxsval{source}\\
       \bi select\_field \oxsval{weighting}\\
       \bi normalize \oxsval{norm\_request}\\
       \bi exclude\_0\_Ms \oxsval{novacuum}\\
       \bi user\_scaling \oxsval{scale}\\
       \bi units \oxsval{units}\\
      \ccb
      \end{quote}
      \end{latexonly}
      \begin{rawhtml}<BLOCKQUOTE><DL><DT>
      <TT>user_output {</TT>
      <DD><TT> name </TT><I>output_name</I>
      <DD><TT> source_field </TT><I>source</I>
      <DD><TT> select_field </TT><I>weighting</I>
      <DD><TT> normalize </TT><I>norm_request</I>
      <DD><TT> exclude_0_Ms </TT><I>novacuum</I>
      <DD><TT> user_scaling </TT><I>scale</I>
      <DD><TT> units </TT><I>units</I>
      <DT><TT>}</TT></DL></BLOCKQUOTE><P>
      \end{rawhtml}
The first parameter, \oxslabel{name}, specifies the label attached to
this output in the DataTable results; the full name will be the Specify
block instance name, followed by \oxsval{:output\_name}.  This label
must follow the rules for \ODT\ column labels; in particular, embedded
newlines and carriage returns are not allowed.

The second parameter, \oxslabel{source\_field}, specifies the vector
field output that the output is derived from.  The \oxsval{source} value
should match the label for the source field output as displayed in the
``Output'' pane of the Oxsii or Boxsi interactive interface; this can
also be found in the documentation for the source field \cd{Oxs\_Ext}
class.  If the source field is from the same class as the user output,
then \oxsval{source} can use the short form of the name (i.e., the
component following the last ``:''); otherwise the full name must be
used.

The third parameter, \oxslabel{select\_field}, references a field that
is used to weight the source field to create the scalar output.  The
output is computed according to
\begin{equation}
\sum_i W_{\rm select}[i]\cdot V_{\rm source}[i]
/\sum_i \|W_{\rm select}[i]\|
\label{eq:UserScalarOutput}
\end{equation}
where the sums are across all cells in the simulation, $W_{\rm
select}[i]$ is the value of the select field at cell $i$,
$V_{\rm source}[i]$ is the value of the source field at cell $i$,
and ``$\cdot$'' indicates the scalar (dot) product.

The first three parameters are required, the remaining parameters are
optional.  The first of the optional parameters, \oxslabel{normalize},
affects the denominator in (\ref{eq:UserScalarOutput}).  If
\oxsval{norm\_request} is 1 (the default), then the output is computed
as shown in (\ref{eq:UserScalarOutput}).  If \oxsval{norm\_request} is 0,
then instead the denominator is replaced with the number of cells in the
simulation, i.e., $\sum_i 1$.

The second optional parameter, \oxslabel{exclude\_0\_Ms}, is a
convenience operator; if \oxsval{novacuum} is 1, then the select field
is reset so that it is the zero vector at all cells in the simulation
where the saturation magnetization is zero.  This is especially useful
when you want to compute the average magnetization of a shaped part.
The change to the select field is made before the denominator in
(\ref{eq:UserScalarOutput}) is computed, so setting
\oxslabel{exclude\_0\_Ms} to 1 is equivalent to defining the select
field as being zero where $M_s$ is zero in the first place.  The default
value for this parameter is 0, which results in the select field being
used exactly as defined.

The \oxslabel{user\_scaling} parameter (default value 1.0) allows the
user to define a constant factor that is multiplied against the result
of (\ref{eq:UserScalarOutput}).  This can be used, for example, in
conjuction with the \oxslabel{units} parameter to implement unit
conversion.  The \oxsval{units} value is an arbitrary string (for
example, A/m) that is placed in the DataTable output.  This label must
follow the rules for \ODT\ unit labels; in particular, embedded newlines
and carriage returns are not allowed.  If \oxsval{units} is not set,
then the units string is copied from the units for the source field.

The following is a simple example showing two user outputs based
off the demagnetization field:
% The extra BLOCKQUOTE's here are a workaround for an apparent
% latex2html bug
\begin{rawhtml}
<BLOCKQUOTE>
\end{rawhtml}
%begin{latexonly}
\begin{quote}
%end{latexonly}
\begin{verbatim}
Specify Oxs_BoxAtlas:atlas [subst {
  xrange {0 $cube_edge}
  yrange {0 $cube_edge}
  zrange {0 $cube_edge}
}]

Specify Oxs_BoxAtlas:octant [subst {
  xrange {0 [expr {$cube_edge/2.}]}
  yrange {0 [expr {$cube_edge/2.}]}
  zrange {0 [expr {$cube_edge/2.}]}
}]

Specify Oxs_AtlasVectorField:octant_field_y {
   atlas :octant
   default_value {0 0 0}
   values {
      octant {0 1 0}
   }
}

Specify Oxs_Demag {
  user_output {
     name "Hdemag_x"
     source_field Field
     select_field {1 0 0}
  }
  user_output {
     name "octant Hdemag_y"
     source_field Field
     select_field :octant_field_y
  }
}
\end{verbatim}
%begin{latexonly}
\end{quote}
%end{latexonly}
\begin{rawhtml}
</BLOCKQUOTE>
\end{rawhtml}
The first user output, ``\cd{Hdemag\_x},'' returns the
$x$-component of the demagnetization field, averaged across the entire
simulation volume.  This output will appear in DataTable output with the
label ``\cd{Oxs\_Demag::Hdemag\_x}.''  The \cd{source\_field} parameter
``Field'' refers to the ``Field'' output of the surrounding
\cd{Oxs\_Ext} object, which here means \cd{Oxs\_Demag::Field}.
The select field is $(1,0,0)$, uniform across the simulation volume.
The second output, ``\cd{octant Hdemag\_y},'' is similar, but the
average is of the $y$ component of the demagnetization field, and is
averaged across only the first octant of the simulation volume.  The
averaging volume and component selection are defined by the
\cd{:octant\_field\_y} field object, which is $(0,1,0)$ in the first
octant and $(0,0,0)$ everywhere else.

The source code for user defined scalar outputs can be found in the
files \fn{ext.h} and \fn{ext.cc} in the directory
\fn{oommf/app/oxs/base/}.  Example \MIF\ files include \fn{cube.mif},
\fn{pbcbrick.mif}, \fn{spinvalve.mif}, and \fn{stdprob2.mif} in
\fn{oommf/app/oxs/examples/}.

\subsection{Variable Substitution}%
\label{sec:varSubst}\index{variable~substitution~(MIF)}
One powerful consequence of the evaluation of \MIF~2.1 input files by
\Tcl\ is the ability to define and use variables.  For example,
the Oxs interfaces (\hyperrefhtml{Oxsii}{Oxsii,
Sec.~}{}{sec:oxsii} and \hyperrefhtml{Boxsi}{Boxsi, Sec.~}{}{sec:boxsi})
use the \cd{-parameter} command line option in conjunction with
the \MIF\ \htmlonlyref{Parameter}{html:mif2parameter} command to
set variables from the command line for use inside the \MIF\ input file.
Variables in \Tcl\ are evaluated (i.e., value substituted)
by prefixing the variable name with the symbol ``\$''.  For example, if
\cd{cellsize} is a variable holding the value \cd{5e-9}, then
\cd{\$cellsize} evaluates to \cd{5e-9}.

Unfortunately, there are complications in using variables inside
Specify blocks.  Consider this simple example:
\begin{rawhtml}
<BLOCKQUOTE>
\end{rawhtml}
%begin{latexonly}
\begin{quote}
%end{latexonly}
\begin{verbatim}
Parameter cellsize 5e-9
Specify Oxs_RectangularMesh:BadExample {
  comment {NOTE: THIS DOESN'T WORK!!!}
  cellsize {$cellsize $cellsize $cellsize}
  atlas :atlas
}
\end{verbatim}
%begin{latexonly}
\end{quote}
%end{latexonly}
\begin{rawhtml}
</BLOCKQUOTE>
\end{rawhtml}
This \textbf{doesn't work}, because the curly braces used to set off the
\cd{Specify} initialization string also inhibit variable substitution.
There are several ways to work around this, but the easiest is usually
to embed the initialization string inside a \cd{subst} (substitution)
command:
\begin{rawhtml}
<BLOCKQUOTE>
\end{rawhtml}
%begin{latexonly}
\begin{quote}
%end{latexonly}
\begin{verbatim}
Parameter cellsize 5e-9
Specify Oxs_RectangularMesh:GoodExample [subst {
  comment {NOTE: This works.}
  cellsize {$cellsize $cellsize $cellsize}
  atlas :atlas
}]
\end{verbatim}
%begin{latexonly}
\end{quote}
%end{latexonly}
\begin{rawhtml}
</BLOCKQUOTE>
\end{rawhtml}
Here the square brackets, ``\texttt{[}'' and ``\texttt{]}'',
cause \Tcl\ to perform \textit{command substitution}, i.e.,
execute the string inside the square brackets as a \Tcl\
command, in this case the \cd{subst} command.  See the \Tcl\
documentation for \cd{subst} for details, but the default
usage illustrated above performs variable, command and
backslash substitutions on the argument string.

One more example, this time involving both variable and command
substitution:
\begin{rawhtml}
<BLOCKQUOTE>
\end{rawhtml}
%begin{latexonly}
\begin{quote}
%end{latexonly}
\begin{verbatim}
set pi [expr {4*atan(1.0)}]
set mu0 [expr {4*$pi*1e-7}]
Specify Oxs_UZeeman [subst {
  comment {Set units to mT}
  Hscale [expr {0.001/$mu0}]
  Hrange {
     {  0  0  0   10  0  0   2 }
     { 10  0  0  -10  0  0   2 }
  }
}]
\end{verbatim}
%begin{latexonly}
\end{quote}
%end{latexonly}
\begin{rawhtml}
</BLOCKQUOTE>
\end{rawhtml}
Note that the \cd{subst} command is evaluated at global scope, so that
the global variable \cd{mu0} is directly accessible.

\subsection{Sample \MIF~2.1 File}\label{sec:mif2sample}
\begin{codelisting}{p}{fig:mif2sample}{Example \MIF~2.1
   file.}{sec:mif2format}{ref}
\begin{verbatim}
# MIF 2.1
#
# All units are SI.
#
# This file must be a valid Tcl script.
#

# Initialize random number generators with seed=1
RandomSeed 1

# Individual Oxs_Ext objects are loaded and initialized via
# Specify command blocks.  The following block defines the
# extents (in meters) of the volume to be modeled.  The
# prefix "Oxs_BoxAtlas" specifies the type of Oxs_Ext object
# to create, and the suffix ":WorldAtlas" is the name
# assigned to this particular instance.  Each object created
# by a Specify command must have a unique full name (here
# "Oxs_BoxAtlas:WorldAtlas").  If the suffix is not
# explicitly given, then the default ":" is automatically
# assigned.  References may be made to either the full name,
# or the shorter suffix instance name (here ":WorldAtlas")
# if the latter is unique. See the Oxs_TimeDriver block for
# some reference examples.
Specify Oxs_BoxAtlas:WorldAtlas {
  xrange {0 500e-9}
  yrange {0 250e-9}
  zrange {0 10e-9}
}

# The Oxs_RectangularMesh object is initialized with the
# discretization cell size (in meters).
Specify Oxs_RectangularMesh:mesh {
  cellsize {5e-9 5e-9 5e-9}
  atlas :WorldAtlas
}

# Magnetocrystalline anisotropy block.   The setting for
# K1 (500e3 J/m^3) implicitly creates an embedded
# Oxs_UniformScalarField object.  Oxs_RandomVectorField
# is an explicit embedded Oxs_Ext object.
Specify Oxs_UniaxialAnisotropy {
  K1  530e3
  axis { Oxs_RandomVectorField {
           min_norm 1
           max_norm 1
  } }
}

# Homogeneous exchange energy, in J/m.  This may be set
# from the command line with an option like
#    -parameters "A 10e-12"
# If not set from the command line, then the default value
# specified here (13e-12) is used.
Parameter A 13e-12
Specify Oxs_UniformExchange:NiFe [subst {
  A  $A
}]

# Define a couple of constants for later use.
set PI [expr {4*atan(1.)}]
set MU0 [expr {4*$PI*1e-7}]

# The Oxs_UZeeman class is initialized with field ranges in A/m.
# The following block uses the multiplier option to allow ranges
# to be specified in mT.  Use the Tcl "subst" command to enable
# variable and command substitution inside a Specify block.
Specify Oxs_UZeeman:AppliedField [subst {
  multiplier [expr 0.001/$MU0]
  Hrange {
    {  0  0  0   10  0  0   2 }
    { 10  0  0  -10  0  0   2 }
    {  0  0  0    0 10  0   4 }
    {  1  1  1    5  5  5   0 }
  }
}]

# Enable demagnetization (self-magnetostatic) field
# computation.  This block takes no parameters.
Specify Oxs_Demag {}

# Runge-Kutta-Fehlberg ODE solver, with default parameter values.
Specify Oxs_RungeKuttaEvolve {}

# The following procedure is used to set the initial spin
# configuration in the Oxs_TimeDriver block.  The arguments
# x, y, and z are coordinates relative to the min and max
# range of each dimension, e.g., 0<=x<=1, where x==0
# corresponds to xmin, x==1 corresponds to xmax.
proc UpDownSpin { x y z } {
  if { $x < 0.45 } {
    return "0 1 0"
  } elseif { $x > 0.55 } {
    return "0 -1 0"
  } else {
    return "0 0 1"
  }
}

Specify Oxs_TimeDriver {
 evolver Oxs_RungeKuttaEvolve
 stopping_dm_dt 0.01
 mesh :mesh
 Ms 8e5   comment {implicit Oxs_UniformScalarField object}
 m0 { Oxs_ScriptVectorField {
        script {UpDownSpin}
        norm  1
        atlas :WorldAtlas
 } }
 basename example
 comment {If you don't specify basename, then the default
          is taken from the MIF filename.}
}

# Default outputs
Destination hystgraph mmGraph:Hysteresis
Destination monitor   mmGraph   new
Destination archive   mmArchive

Schedule DataTable hystgraph Stage 1
Schedule DataTable monitor   Step 5
Schedule DataTable archive   Stage 1
Schedule Oxs_TimeDriver::Magnetization archive Stage 3
Schedule "Oxs_RungeKuttaEvolve::Total field" archive Stage 3
\end{verbatim}
\end{codelisting}

%%%%%%%%%%%%%%%%%%%%%%%%%%%%%%%%%%%%%%%%%%%%%%%%%%%%%%%%%%%%%%%%%%%%%%%%

\section{\MIF\ 2.2}\label{sec:mif22format}
The \MIF~2.2 format, introduced with \OOMMF~1.2a4, is a minor
modification to the \MIF~2.1 format.  \MIF~2.2 provides a few additional
commands, and is mostly backwards compatible with \MIF~2.1, except as
detailed below.
\subsection{Differences between \MIF~2.2 and \MIF~2.1 Formats}\label{sec:mif22diffs}
\begin{enumerate}
\item The first line of a \MIF~2.2 file must be ``\verb+#+ MIF 2.2''.
\item The \cd{basename}, \cd{scalar\_output\_format} and
 \cd{vector\_field\_output\_format} options to the \cd{Oxs\_TimeDriver}
 and \cd{Oxs\_MinDriver} objects are no longer supported.  Instead,
 there is a new top-level extension command, \cd{SetOptions}, where
 these options are declared.  The \cd{SetOptions} block also
 supports new options for controlling output vector field mesh type
 (rectangular or irregular) and scalar field output format.
\item In the \MIF~2.1 format, \MIF\ files are
 processed in a two pass mode.  During the first pass, \cd{Specify}
 commands simply store the contents of the \cd{Specify} blocks without
 creating any \cd{Oxs\_Ext} objects.  The \cd{Oxs\_Ext} objects
 associated with each \cd{Specify} block are created in the second pass
 from the data stored in the first pass.  In the \MIF~2.2 format, this is
 replaced with a one pass mode, where \cd{Oxs\_Ext} objects are created
 at the time that the \cd{Specify} commands are parsed.  This processing
 model is more intuitive for \MIF\ file authors, but has two main
 consequences.  The first is that in \MIF~2.1 format files, \Tcl\ procs that
 are used only inside \cd{Specify} commands can be placed anywhere inside
 the \MIF\ file (for example, commonly at the end), because they won't be
 called during the first pass.  As long as they are defined at any point
 during the first pass, they will be available for use in the second
 pass.  In contrast, in the \MIF~2.2 format, \Tcl\ procs definitions must
 generally be moved forward, before any references in \cd{Specify}
 blocks.  The second consequence is that \cd{Oxs\_Ext} objects defined by
 \cd{Specify} commands are available for use inside the \MIF\ file.  This
 allows support for the new commands discussed next.
%\item Two new top-level extension commands are supported in the \MIF~2.2
% format, \cd{EvalScalarField} and \cd{EvalVectorField}.  These commands
% are described in detail below.
\end{enumerate}

\subsection{\MIF~2.2 New Extension Commands}%
\label{sec:mif22ExtensionCommands}\index{MIF~2.2~Commands}
In addition to the commands available in
\hyperrefhtml{\MIF~2.1 files}{\MIF~2.1 files (Sec.~}{)}{sec:mif2ExtensionCommands},
\MIF~2.2 introduces the following new commands:
%
\htmlonlyref{\cd{GetMifFilename}}{html:mif2GetMifFilename},
\htmlonlyref{\cd{GetMifParameters}}{html:mif2GetMifParameters},
\htmlonlyref{\cd{GetOptions}}{html:mif2GetOptions},
\htmlonlyref{\cd{SetOptions}}{html:mif2SetOptions},
\htmlonlyref{\cd{EvalScalarField}}{html:mif2EvalScalarField},
\htmlonlyref{\cd{EvalVectorField}}{html:mif2EvalVectorField},
\htmlonlyref{\cd{GetAtlasRegions}}{html:mif2GetAtlasRegions},
and \htmlonlyref{\cd{GetAtlasRegionByPosition}}{html:mif2GetAtlasRegionByPosition}.

\begin{description}
\item[GetMifFilename\label{html:mif2GetMifFilename}]\index{GetMifFilename~command~(MIF)}
The \cd{GetMifFilename} command returns the full (absolute) name of the
\MIF\ file being read.  This command takes no parameters.

\item[GetMifParameters\label{html:mif2GetMifParameters}]\index{GetMifParameters~command~(MIF)}
This command takes no parameters, and returns an even numbered list of
``Parameter'' label + value pairs as set on the command line or in the
Load Problem dialog box.  If no parameters were specified, then the return
will be an empty list.

\item[GetOptions\label{html:mif2GetOptions}]
The \cd{GetOptions} command takes no parameters.  It returns the
accumulated contents of all preceding \cd{SetOptions} blocks, as an even
numbered list of label + value pairs.

\item[SetOptions\label{html:mif2SetOptions}]\index{SetOptions~command~(MIF)}
In \MIF~2.1 files, the output basename and output file formats are
specified inside the driver's Specify block.  In \MIF~2.2 these
specifications are moved to a separate \cd{SetOptions} block.  This
block can be placed anywhere in the \MIF\ file, but is typically placed
near the start of the file so that it affects all output
initializations.  The \cd{SetOptions} command takes a single argument,
which is a list of label + value pairs.  The default labels are:
\begin{itemize}
\item \cd{basename}
\item \cd{scalar\_output\_format}
\item \cd{scalar\_field\_output\_format}
\item \cd{scalar\_field\_output\_meshtype}
\item \cd{vector\_field\_output\_format}
\item \cd{vector\_field\_output\_meshtype}
\end{itemize}
The \cd{basename} value is used as a prefix for output filename
construction by the data output routines.  If \cd{basename} is not
specified, then the default value is taken from the filename of the
input MIF file.  The \cd{scalar\_output\_format} value is a C-style
printf string specifying
the output format for DataTable output.  This is
optional, with default value ``\cd{\%.17g}''.  The values associated with
\cd{scalar\_field\_output\_format} and
\cd{vector\_field\_output\_format} should be two element lists
that specify the style and precision for scalar and vector field
output sent to \hyperrefhtml{\app{mmDisp}}{\app{mmDisp}
(Ch.~}{)}{sec:mmdisp}\index{application!mmDisp} and
\hyperrefhtml{\app{mmArchive}}{\app{mmArchive}
(Ch.~}{)}{sec:mmarchive}\index{application!mmArchive}.  The first
element in the list should be one of \cd{binary} or \cd{text}, specifying
the output style.  If binary output is selected, then the second element
specifying precision should be either 4 or 8, denoting component binary
output length in bytes.  For text output, the
second element should be a C-style printf string like that used by
\cd{scalar\_output\_format}.  The default value for both
\cd{scalar\_field\_output\_format} and
\cd{vector\_field\_output\_format} is ``\cd{binary 8}''.
For \cd{scalar\_field\_output\_meshtype} and
\cd{vector\_field\_output\_meshtype} the values should be either
``rectangular'' (default) or ``irregular'', specifying the grid type for
the corresponding field output files.

Multiple \cd{SetOptions} blocks are allowed.  Label values specified
in one \cd{SetOption} block may be overwritten by a later
\cd{SetOption} block.  Output formats for a given output are
set during the processing of the \cd{Specify} block for the enclosing
\cd{Oxs\_Ext} object.  Therefore, one can specify different formats
for outputs in different \cd{Oxs\_Ext} objects by strategic placement
of \cd{SetOptions} blocks.

Additional label names may be added in the future, and may be
\cd{Oxs\_Ext} class dependent.  At present there is no checking for
unknown label names, but that policy is subject to change.

An example \cd{SetOptions} block:
\begin{rawhtml}
<BLOCKQUOTE>
\end{rawhtml}
%begin{latexonly}
\begin{quote}
%end{latexonly}
\begin{verbatim}
SetOptions {
 basename fubar
 scalar_output_format %.12g
 scalar_field_output_format {text %.4g}
 scalar_field_output_meshtype irregular
 vector_field_output_format {binary 4}
}
\end{verbatim}
%begin{latexonly}
\end{quote}
%end{latexonly}
\begin{rawhtml}
</BLOCKQUOTE>
\end{rawhtml}


\item[EvalScalarField\label{html:mif2EvalScalarField}]\index{EvalScalarField~command~(MIF)}
This command allows access in a \MIF\ file to values from a scalar field
defined in a preceding \cd{Specify} block.  For example,
\begin{rawhtml}
<BLOCKQUOTE>
\end{rawhtml}
%begin{latexonly}
\begin{quote}
%end{latexonly}
\begin{verbatim}
   Oxs_AtlasScalarField:Ms {
      atlas :atlas
      default_value 0
      values {
         Adisks 520e3
         Bdisks 520e3
      }
   }}

   set Ms_a [EvalScalarField :Ms 50e-9 20e-9 2e-9]
\end{verbatim}
%begin{latexonly}
\end{quote}
%end{latexonly}
\begin{rawhtml}
</BLOCKQUOTE>
\end{rawhtml}
The four arguments to \cd{EvalScalarField} are a reference to the scalar
field (here \cd{:Ms}), and the three coordinates of the point where you
want the field evaluated.  The coordinates are in the problem coordinate
space, i.e., in meters.

\item[EvalVectorField\label{html:mif2EvalVectorField}]\index{EvalVectorField~command~(MIF)}
This command is the same as the \cd{EvalScalarField} command, except
that the field reference is to a vector field, and the return value is a
three item list representing the three components of the vector field at
the specified point.

\item[GetAtlasRegions\label{html:mif2GetAtlasRegions}]
This command takes one argument, which is a reference to an atlas, and
returns an ordered list of all the regions in that atlas.  The first
item on the returned list will always be ``universe'', which includes
all points not in any of the other regions, including in particular any
points outside the nominal bounds of the atlas.  Sample usage:
\begin{rawhtml}
<BLOCKQUOTE>
\end{rawhtml}
%begin{latexonly}
\begin{quote}
%end{latexonly}
\begin{verbatim}
   set regions_list [GetAtlasRegions :atlas]
\end{verbatim}
%begin{latexonly}
\end{quote}
%end{latexonly}
\begin{rawhtml}
</BLOCKQUOTE>
\end{rawhtml}


\item[GetAtlasRegionByPosition\label{html:mif2GetAtlasRegionByPosition}]
This command takes four arguments: a reference to atlas, followed by the
x, y, and z coordinates of a point using problem coordinates (i.e.,
meters).  The return value is the name of the region containing the
specified point.  This name will match exactly one of the names on the
list returned by the \cd{GetAtlasRegions} command for the given atlas.
Note that the return value might be the ``universe'' region.  Sample
usage:
\begin{rawhtml}
<BLOCKQUOTE>
\end{rawhtml}
%begin{latexonly}
\begin{quote}
%end{latexonly}
\begin{verbatim}
   set rogue_region [GetAtlasRegionByPosition :atlas 350e-9 120e-9 7.5e-9]
\end{verbatim}
%begin{latexonly}
\end{quote}
%end{latexonly}
\begin{rawhtml}
</BLOCKQUOTE>
\end{rawhtml}


\end{description}

\subsection{Sample \MIF~2.2 File}\label{sec:mif22sample}
\begin{codelisting}{p}{fig:mif22sample}{Example \MIF~2.2
  file.}{sec:mif22format}{ref}
\begin{verbatim}
# MIF 2.2

###############
# Constants
set pi [expr 4*atan(1.0)]
set mu0 [expr 4*$pi*1e-7]


###############
# Command-line controls
Parameter seed 1
Parameter thickness 6e-9
Parameter stop 1e-2

# Texturing angle, phideg, in degrees, from 0 to 90; 0 is all z.
Parameter phideg 10;


###############
# Output options
SetOptions [subst {
   basename "polyuniaxial_phi_$phideg"
   scalar_output_format %.12g
   scalar_field_output_format {text %.4g}
   scalar_field_output_meshtype irregular
   vector_field_output_format {binary 4}
}]


###############
# Rogue grain:
# If RoguePt is an empty string, then no rogue grain is selected.  OTOH,
# If RoguePt is set to a three item list consisting of x, y, and z coords
#   in the problem coordinate system (i.e., in meters), then the grain
#   containing that point is individually set as specified below.
Parameter RoguePt {263.5e-9 174.5e-9 3e-9}


###############
# Support procs:
proc Ellipse { Ms x y z} {
   set x [expr {2*$x-1.}]
   set y [expr {2*$y-1.}]
   if {$x*$x+$y*$y<=1.0} {
      return $Ms
   }
   return 0.0
}


###############
# Material constants
set Ms 1.40e6
set Ku 530e3
set A  8.1e-12


###############
# Atlas and mesh
set xsize 400e-9
set ysize 400e-9
set xycellsize 1.0e-9
set zcellsize  3.0e-9

set grain_count 260
set grain_map polycrystal-map-mif.ppm

set colormap {}
for {set i 0} {$i<$grain_count} {incr i} {
   lappend colormap [format "#%06x" $i]
   lappend colormap $i
}

Specify Oxs_ImageAtlas:world [subst {
   xrange {0 $xsize}
   yrange {0 $ysize}
   zrange {0 $thickness}
   viewplane xy
   image $grain_map
   colormap {
      $colormap
   }
   matcherror 0.0
}]

Specify Oxs_RectangularMesh:mesh [subst {
   cellsize {$xycellsize $xycellsize $zcellsize}
   atlas :world
}]


#################################
# Uniaxial Anisotropy

# Generate TEXTURED random unit vector
set phirange [expr {1-cos($phideg*$pi/180.)}]
proc Texture {} {
   global pi phirange

   set theta [expr {(2.*rand()-1.)*$pi}]
   set costheta [expr {cos($theta)}]
   set sintheta [expr {sin($theta)}]

   set cosphi [expr {1.-$phirange*rand()}]
   set sinphi [expr {1.0-$cosphi*$cosphi}]
   if {$sinphi>0.0} { set sinphi [expr {sqrt($sinphi)}] }

   set x [expr {$sinphi*$costheta}]
   set y [expr {$sinphi*$sintheta}]
   set z [expr {$cosphi}]

   return [list $x $y $z]
}


# Set a random unit vector for each grain region
set axes {}
for {set i 0} {$i<$grain_count} {incr i} {
   lappend axes $i
   lappend axes [Texture]
}

# Sets the rogue grain ($Rogue < $grain_count)
if {[llength $RoguePt] == 3} {
   # The :Regions field maps region name (which is a number)
   # to the corresponding number.
   set regionmap {}
   for {set i 0} {$i<$grain_count} {incr i} {lappend regionmap $i $i }
   Specify Oxs_AtlasScalarField:Regions [subst {
      atlas :world
      values [list $regionmap]
   }]
   foreach {x y z} $RoguePt { break }
   set Rogue [EvalScalarField :Regions $x $y $z]
   set item_number [expr 2*$Rogue+1]
   set axes [lreplace $axes $item_number $item_number {1 0 0}]
}

Specify Oxs_AtlasVectorField:axes [subst {
   atlas :world
   norm 1.0
   values [list $axes]
}]

Specify Oxs_UniaxialAnisotropy [subst {
   K1 $Ku
   axis :axes
}]


#################################
# Exchange
set A_list {}
for {set i 0} {$i<$grain_count} {incr i} {
   lappend A_list $i $i $A
}

Specify Oxs_Exchange6Ngbr [subst {
   default_A $A
   atlas world
   A   [list $A_list]
}]


#################################
# Zeeman (applied) field
set field 10000         ;# Maximum field (in Oe)
Specify Oxs_UZeeman [subst {
   multiplier [expr (1./($mu0*1e4))*$field]
   Hrange  {
      { 0 0 0   0 0 1   10}
   }
}]


#################################
# Driver and Evolver

Specify Oxs_CGEvolve:evolve {}

Specify Oxs_MinDriver [subst {
   evolver evolve
   stopping_mxHxm $stop
   mesh :mesh
   Ms { Oxs_ScriptScalarField {
      atlas :world
      script_args {relpt}
      script {Ellipse $Ms}
   } }
   m0 { 0 0 -1 }
}]
\end{verbatim}
\end{codelisting}


%%%%%%%%%%%%%%%%%%%%%%%%%%%%%%%%%%%%%%%%%%%%%%%%%%%%%%%%%%%%%%%%%%%%%%%%
\section{Tips for writing \MIF\ 2.x files}\label{sec:mif2tips}
\MIF\ 2.x files are \Tcl\ scripts, and so composing a \MIF\ file is a
programming exercise, with all the pitfalls that entails. In this
section we detail some tips for authoring \MIF\ files.

Generally a good place to start is to take an existing \MIF\ file,
either one you've written earlier or one from the
\cd{oommf/app/oxs/examples/} directory, that has some similarity to the
one you want to write. Make a copy of that and start to edit. If you
need any complex functionality then an elementary understanding of \Tcl\
is essential. In particular, you should be familiar with \Tcl\ lists,
arrays, and some of the basic \Tcl\ commands such as \cd{set},
\cd{expr}, \cd{for}, \cd{foreach}, \cd{if/else}, \cd{lrange}, and
\cd{subst}. You can find a list of \Tcl\ tutorials on the Tcler's Wiki
\htmladdnormallinkfoot{Online Tcl and Tk
  Tutorials}{https://wiki.tcl-lang.org/page/Online+Tcl+and+Tk+Tutorials}
page. The \textit{Tcl Tutorial} by Chris verBurg and \textit{Learn Tcl
  in Y Minutes} linked on that page are good places to start. The
\htmladdnormallinkfoot{Tcl
  Dodekalogue}{https://wiki.tcl-lang.org/page/Dodekalogue} is also a
handy reference for figuring out confusing error messages.  Many more
resources, including online manuals, can be found at the
\htmladdnormallinkfoot{Tcl Developer
  Xchange}{https://www.tcl-lang.org/}.

Although \MIF\ files are \Tcl\ scripts, they also rely on a number of
\hyperrefhtml{\MIF\ extension commands}{\MIF\ extension commands
  (Sec.~}{)}{sec:mif2ExtensionCommands}. The most prominent of these is
the \htmlonlyref{\cd{Specify}}{html:specifyCmd} command, used to
initialize \cd{Oxs\_Ext} (Oxs extension) objects, such as region
definitions (atlases), mesh discretization, energy terms (exchange,
anisotropy, dipole-dipole, applied fields), and solution method (energy
minimization or LLG integration). The contents of each \cd{Specify}
block, which depend upon the particular \cd{Oxs\_Ext} being initialized,
are documented
\html{in the }\hyperrefhtml{Standard Oxs\_Ext Child
Classes section}{under Standard Oxs\_Ext Child Classes
(Sec.~}{)}{sec:oxsext}.

Other commonly used \MIF\ extension commands are \cd{Parameter},
\cd{Destination}, and \cd{Schedule}.
\htmlonlyref{\cd{Parameter}}{html:mif2parameter} operates similar to
the \Tcl\ \cd{set} command, but allows the user to change the default
value at runtime, through either the \cd{Params} box in the \cd{Oxsii
  Load Problem} dialog, or via the \cd{-parameters} command line option
to \cd{Boxsi}. The
\htmlonlyref{\cd{Destination}}{html:destinationCmd} and
\htmlonlyref{\cd{Schedule}}{html:scheduleCmd} commands
are used primarily for non-interactive set up of simulation output
for \cd{Boxsi}.

As in any programming activity, bugs happen. Here are a few tips to aid
in debugging your MIF code:
\begin{enumerate}
\setcounter{enumi}{-1}
\item Check to see if your favorite programming editor (FPE) has support
    for \Tcl\ syntax highlighting and indenting. This can help spot a lot
    of errors, particularly brace nesting issues, before you hand the
    file over to the \Tcl\ interpreter inside \OOMMF.

  \item For development purposes, set the problem dimensions and cell
    sizes so the total number of cells is relatively small. This will
    allow problems to load and run faster, making it easier to detect
    and correct errors. Use the \cd{Oxsii\pipe Help\pipe About} menu to
    see the number of cells in the simulation. For larger simulations
    you may want to temporarily disable the \cd{Oxs\_Demag} module,
    because that module can take some time to initialize and is usually
    the slowest module to run. If the cellsize magnetization is set
    correctly, then \cd{Oxs\_Demag} will usually be okay; you might want
    to take a little extra care when working with periodic boundary
    conditions, however, because demagnetization effects across PBC's
    can be nonintuitive.

  \item When the \MIF\ file is ready, launch \cd{Oxsii} and bring up the
    \cd{File\pipe Load} dialog box. Select the \cd{Browse} check box, as
    otherwise the \cd{Load} dialog box is automatically closed when you
    select \cd{OK} to load the file. The \cd{Browse} option allows you
    to correct syntax errors in your \MIF\ file without having to
    repeatedly open the \cd{File\pipe Load} dialog box and reselect your
    \MIF\ file and parameters.

  \item If the load fails, open the \MIF\ file in your FPE and use the
    error message to help locate the point of failure in the
    file. Correct, save to disk, and try loading again.

  \item If you can't figure out an error message, or to just
    double-check that your file is being interpreted as you intend, use
    the \MIF\ \cd{Report} command to print the contents of variables or
    other state information to the \cd{Oxsii} console. (The \cd{Oxsii}
    console is launched via the \cd{File\pipe Show Console} menu item on the
    main \cd{Oxsii} window. If you are running \cd{Boxsi}, \cd{Report}
    output goes to the \cd{Boxsi} log file, \fn{oommf/boxsi.errors}.)

  \item Once the file loads without errors, send \cd{Magnetization}
    output from \cd{Oxsii} to \cd{mmDisp} for a visual check of the
    simulation structure. Use \cd{Ctrl-} or \cd{Shift-Ctrl-\oab left
    mouse click\cab} in \cd{mmDisp} to view magnetization component
    values at various locations. Bring up the \cd{mmDisp
    Options\pipe Configure} dialog box to adjust the coloring, pixel and
    arrow selection, etc. The \cd{\# of Colors} and \cd{Data Scale}
    settings in particular can make it easier to see small differences
    in the selected \cd{Color Quantity}. The \cd{Arrow span} setting can
    be used to control the number of levels of arrows that get displayed
    in the slice view. For example, if the cell dimension in the
    out-of-view-plane direction is 4~nm, then setting \cd{Arrow span} to
    \cd{4e-9} will cause a single slice of the magnetization to be
    displayed as you adjust the slice control slider in the main
    \cd{mmDisp} window. If you set \cd{Arrow span} to \cd{8e-9} you'll
    see two layers of overlapping arrows, which can be helpful for
    checking interface conditions.

  \item By default \cd{mmDisp} opens with a top-down view along the
    $z$-axis. For multilayer structures it is helpful to see
    cross-sectional views along other axes too; these are available from
    the \cd{View\pipe Viewpoint} submenu in the main \cd{mmDisp} window.
    Also, the \cd{mmDisp} rendering of high aspect ratio cells can be
    rather poor. In this situation you may want to enable pixel display
    in the \cd{mmDisp} configuration dialog with pixel size set smaller
    than 1, and adjust the background color to make the individual cells
    visible.

  \item The \cd{Magnetization} output viewed in \cd{mmDisp} allows you
    to check that you have set \cd{Ms} and \cd{m0} properly in the
    driver \cd{Specify} block. You can send other fields, such as
    anisotropy and exchange, to \cd{mmDisp} to check parameter settings
    for those energy terms as well. In this context it can be helpful to
    adjust \cd{m0} to align in specific directions for testing. You may
    want to open two instances of \cd{mmDisp}, with \cd{Magnetization}
    displayed in one and an energy field in the other, to help
    correspond energy field values with magnetization structure.

  \item If you are having problems implementing some functionality, do a
    search through the \MIF\ files in \fn{oommf/app/oxs/examples} for
    something similar. For example, a text search for ``pulse'' will
    turn up matches in \fn{pillar.mif}, \fn{pingpillar.mif}, and
    \fn{pulse.mif}, each illustrating how to implement an applied field
    pulse of various shapes.

\end{enumerate}


%%%%%%%%%%%%%%%%%%%%%%%%%%%%%%%%%%%%%%%%%%%%%%%%%%%%%%%%%%%%%%%%%%%%%%%%
\chapter{Data Table File Format (\ODT)}\label{sec:odtformat}
\pttarget{PTsecodtformat} % This anchor is needed for LaTeXML if files
                          % are split at the chapter level.

Textual output from solver applications that is not of the vector field
variety is output in the {\em \OOMMF\ Data Table} (\ODT)
format\index{file!data~table}.  This is an \ASCII\ text file format,
with column information in the header and one line of data per record.
Any line ending in a '\bs' character is joined to the succeeding
line before any other processing is performed.  Any leading `\lb'
characters on the second line are removed.

As with the \hyperrefhtml{\OVF\ format}{\OVF\ format
(Sec.~}{)}{sec:ovfformat}, all non-data lines begin with a `\lb'
character, comments with two `\lb' characters.  (This makes it easier
to import the data into external programs, for example, plotting
packages.)  An example is \hyperrefhtml{included below.}{shown in
Fig.~}{.}{fig:odtsample}

The first line of an \ODT\ file should be the file type descriptor
\begin{verbatim}
# ODT 1.0
\end{verbatim}
It is also recommended that \ODT\ files be given names ending
in the file extension \fn{.odt} so that \ODT\ files may be
easily identified.

The remaining lines of the \ODT\ file format should be comments,
data, or any of the following 5 recognized descriptor tag lines:
\begin{itemize}
\item {\bf\ \lb\ Table Start:} Optional, used to segment a file
   containing multiple data table blocks.  Anything after the colon is
   taken as an optional label for the following data block.
\item {\bf\ \lb\ Title:} Optional; everything after the colon is
   interpreted as a title for the table.
\item {\bf\ \lb\ Columns:} Required.  One parameter per column,
   designating the label header for that column.  Spaces may be embedded
   in a column label by using the normal \Tcl\ grouping mechanisms
   (i.e., double-quotes and braces).
\item {\bf\ \lb\ Units:} Optional.  If given, it should have one
   parameter for each column, giving a unit label for the
   corresponding column.  Spaces may be embedded in the unit labels, in
   the same manner as for column headers.
\item {\bf\ \lb\ Table End}: Optional, no parameters.  Should be paired
   with a corresponding Table Start record.
\end{itemize}
Data may appear anywhere after the Columns descriptor record
and before any Table End line, with one record per line.
The data should be numeric values separated by whitespace.  The two
character open-close curly brace pair, \ocb\ccb, is used to indicate
a missing value.

Embedded newlines and carriage returns are not allowed in the title,
columns, or units records.

The command line utility,
\hyperrefhtml{\app{odtcols}}{\app{odtcols} (Sec.~}{)}{sec:odtcols},
can be a useful tool for examining and partitioning \ODT\ files.

% The following code listing points back to sec:odtformat, which
% is a chapter heading that LaTeXML 0.8.6 drops. So for LaTeXML
% use instead an anchor tied to the line following the chapter
% start. For the other processors use the section link.
\latexmlonly{%
\begin{codelisting}{f}{fig:odtsample}{Sample \ODT\
  file.}{PTsecodtformat}{hyperlink}}%
\notlatexmlonly{%
\begin{codelisting}{f}{fig:odtsample}{Sample \ODT\
  file.}{sec:odtformat}{ref}}
\begin{verbatim}
# ODT 1.0
# Table Start
# Title: This is a small sample ODT file.
#
## This is a sample comment.  You can put anything you want
## on comment lines.
#
# Columns: Iteration "Applied Field"  {Total Energy}    Mx
# Units:      {}          "mT"           "J/m^3"       "A/m"
              103          50            0.00636      787840
             1000          32            0.00603      781120
            10300       -5000            0.00640     -800e3
# Table End
\end{verbatim}
\end{codelisting}

%%%%%%%%%%%%%%%%%%%%%%%%%%%%%%%%%%%%%%%%%%%%%%%%%%%%%%%%%%%%%%%%%%%%%%%%

\chapter{Vector Field File Format (\OVF)}\label{sec:vfformats}
Vector field files\index{file!vector~field} specify vector quantities
(e.g., magnetization or magnetic flux density) as a function of spatial
position.  The \textit{\OOMMF\ Vector Field} (\OVF) format is the output
vector field file format used by both the
\hyperrefhtml{2D}{2D (Ch.~}{)}{sec:mmsolve} and
\hyperrefhtml{3D}{3D (Ch.~}{)}{sec:oxs}
micromagnetic solvers.  It is also the input data type read by
\hyperrefhtml{\app{mmDisp}}{\app{mmDisp} (Ch.~}{)}{sec:mmdisp}.  There
are three versions of the \OVF\ format supported by \OOMMF.  The
\OVF~1.0 and 2.0 formats are preferred formats and the only ones
written by \OOMMF\ software.  They support both rectangular and
irregular meshes, in binary and \ASCII\ text.

The \OVF~0.0 format (formerly SVF\index{file!svf})
is an older, simpler format that can be useful for importing
three-dimensional vector
field data into \OOMMF\ from other programs.  (A fourth format, the
\textit{VecFil\index{file!VecFil}} or \textit{Vector Input/Output}
(VIO\index{file!vio}) format, was used by some precursors to the \OOMMF\
code.  Although \OOMMF\ is able to read the VIO format, its use is
deprecated.)

In all these formats, the field domain (i.e., the spatial extent) lies
across three dimensions, with units typically expressed in meters or
nanometers.  In all the formats except the \OVF~2.0 format, the field
values are also three dimensional; the value units are more varied, but
are most often Tesla or A/m.  The \OVF~2.0 format is more general, in
that the field values can be of any arbitrary dimension $N>0$.  (This
dimension, however, is fixed within the file.)  If $N=3$, then the
\OVF~2.0 format supports the same types of data as the \OVF~1.0
format, but another common case is $N=1$, which represents scalar
fields, such as energy density (in say, J/m${}^3$).

The recommended file extensions for \OVF\ files are \fn{.omf} for
magnetization files, \fn{.ohf} for magnetic field (\vH) files, \fn{.obf}
for magnetic flux density (\vB) files,  \fn{.oef} for energy density
files, or \fn{.ovf} for generic files.

\section{The \OVF\ 0.0 format}\label{sec:svfformat}
The \OVF\ 0.0 format\index{file!vector~field} is a simple \ASCII\ text
format supporting irregularly sampled data.  It is intended as an aid
for importing data from non-\OOMMF\ programs, and is backwards
compatible with the format used for problem submissions for the
\ifnotpdf{\htmladdnormallinkfoot{first \mumag\ standard
problem}{https://www.ctcms.nist.gov/\~{}rdm/stdprob\_1.html}.}
\pdfonly{\htmladdnormallinkfoot{first \mumag\ standard
problem}{https://www.ctcms.nist.gov/\%7Erdm/stdprob\_1.html}.}

Users of early releases of \OOMMF\ may recognize the \OVF\ 0.0 format
by its previous name, the Simple Vector Field (\SVF)\index{file!svf}
format.  It came to the attention of the \OOMMF\ developers that the
file extension \fn{.svf} was already registered in several MIME systems
to indicate the
\htmladdnormallinkfoot{Simple Vector Format}{http://www.softsource.com/svf/},
a vector graphics format.  To avoid conflict, we have stopped using
the name Simple Vector Field format, although \OOMMF\ software still
recognizes the \fn{.svf} extension and you may still find example
files and other references to the \SVF\ format.

A sample \OVF\ 0.0 file is shown \hyperrefhtml{below}{in
Fig.~}{}{fig:svfsample}.  Any line beginning with a `\lb' character is
a comment, all others are data lines.  Each data line is a whitespace
separated list of 6 elements: the $x$, $y$ and $z$ components of a
node position, followed by the $x$, $y$ and $z$ components of the
field at that position.  Input continues until the end of the file is
reached.

It is recommended (but not required) that the first line of an \OVF\
file be
\begin{verbatim}
# OOMMF: irregular mesh v0.0
\end{verbatim}
This will aid automatic file type detection.  Also, three special
(extended) comments in \OVF\ 0.0 files are recognized by \app{mmDisp}:
\begin{verbatim}
## File: <filename or extended filename>
## Boundary-XY: <boundary vertex pairs>
## Grid step: <cell dimension triple>
\end{verbatim}
All these lines are optional.  The ``File'' provides a preferred
(possibly extended) filename to use for display identification.  The
``Boundary-XY'' line specifies the ordered vertices of a bounding
polygon in the $xy$-plane.  If given, \app{mmDisp} will draw a frame
using those points to ostensibly indicate the edges of the simulation
body.  Lastly, the ``Grid step'' line provides three values
representing the average $x$, $y$ and $z$ dimensions of the volume
corresponding to an individual node (field sample).  It is used by
\app{mmDisp} to help scale the display.

Note that the data section of an \OVF\ 0.0 file takes the simple
form of columns of \ASCII\ formatted numbers.  Columns of whitespace
separated numbers expressed in \ASCII\ are easy to import
into other programs that process numerical datasets, and
are easy to generate, so the \OVF\ 0.0 file format is useful for
exchanging vector field data between \OOMMF\ and non-\OOMMF\ programs.
Furthermore, the data section of an \OVF\ 0.0 file is consistent
with the data section of an \OVF\ 1.0 file that has been saved
as an irregular mesh using text data representation.  This means that
even though \OOMMF\ software now writes only the \OVF\ 1.0 format
for vector field data, simple interchange of vector field data
with other programs is still supported.

\begin{codelisting}{f}{fig:svfsample}{Example \OVF\ 0.0
  file.}{sec:svfformat}{ref}
\begin{verbatim}
# OOMMF: irregular mesh v0.0
## File: sample.ovf
## Boundary-XY: 0.0 0.0 1.0 0.0 1.0 2.0 0.0 2.0 0.0 0.0
## Grid step: .25 .5 0
#  x      y      z        m_x      m_y      m_z
  0.01   0.01   0.01   -0.35537  0.93472 -0.00000
  0.01   1.00   0.01   -0.18936  0.98191 -0.00000
  0.01   1.99   0.01   -0.08112  0.99670 -0.00000
  0.50   0.50   0.01   -0.03302  0.99945 -0.00001
  0.99   0.05   0.01   -0.08141  0.99668 -0.00001
  0.75   1.50   0.01   -0.18981  0.98182 -0.00000
  0.99   1.99   0.01   -0.35652  0.93429 -0.00000
\end{verbatim}
\end{codelisting}

%begin{latexonly}
\setcounter{secnumdepth}{\value{ffoldsecnumdepth}}
%end{latexonly}

\section{The \OVF\ 1.0 format}\label{sec:ovfformat}
A commented sample \OVF\ 1.0 file is provided
\hyperrefhtml{below}{in Fig.~}{}{fig:ovfsample}.
An \OVF\ file has an \ASCII\ header and trailer, and a data block that
may be either \ASCII\ or binary.  All non-data lines begin with a `\lb'
character; double `\lb\lb' mark the start of a comment, which
continues until the end of the line.  There is no line continuation
character.  Lines starting with a `\lb' but containing only whitespace
characters are ignored.

All non-empty non-comment lines in the file header are structured as
label+value pairs.  The label tag consists of all characters after the
initial `\lb' up to the first colon (`:') character.  Case is ignored,
and all space and tab characters are eliminated.  The value consists
of all characters after the first colon, continuing up to a `\verb+##+'
comment designator or the end of the line.

The first line of an \OVF\ file should be a file type identification
line, having the form
\begin{verbatim}
# OOMMF: rectangular mesh v1.0
\end{verbatim}
or
\begin{verbatim}
# OOMMF: irregular mesh v1.0
\end{verbatim}
where the value ``rectangular mesh v1.0'' or ``irregular mesh v1.0''
identifies the mesh type and revision.  While the \OVF\ 1.0 format was
under development in earlier \OOMMF\ releases, the revision strings
\cd{0.99} and \cd{0.0a0} were sometimes recorded on the file type
identification line. \OOMMF\ treats all of these as synonyms for
\cd{1.0} when reading \OVF\ files.

The remainder of the file is conceptually broken into Segment
blocks\index{segment block}, and each Segment block is composed of a
(Segment) Header block and a Data block.  Each block begins with a
``\verb+# Begin: <block type>+'' line, and ends with a corresponding
``\verb+# End: <block type>+'' line.  The number of Segment blocks is
specified in the
\begin{verbatim}
# Segment count: 1
\end{verbatim}
line.  Currently only 1 segment is allowed.  This may be changed in
the future to allow for multiple vector fields per file.
This is followed by
\begin{verbatim}
# Begin: Segment
\end{verbatim}
to start the first segment.

\subsection{Segment Header block}\label{sec:ovfsegmentheader}
The Segment Header block start is marked by the line
``\lb~Begin: Header'' and the end by ``\lb~End: Header''.  Everything
between these lines should be either comments or one of the following
file descriptor lines.  They are order independent.  All are required
unless otherwise stated.  Numeric values are floating point values
unless ``integer'' is explicitly stated.
\begin{itemize}
\item {\bf title:} Long file name or title.
\item {\bf desc:} Description line.  Optional.  Use as many as desired.
   Description lines may be displayed by post-processing programs,
   unlike comment lines which are ignored by all automated processing.
\item {\bf meshunit:} Fundamental mesh spatial unit, treated as a
   label.  The comment marker `\verb+##+' is not allowed in this label.
   Example value: ``nm''.
\item {\bf valueunit:} Fundamental field value unit, treated as a
   label.  The comment marker `\verb+##+' is not allowed in this label.
   Example: ``kA/m.''
\item {\bf valuemultiplier:} Values in the data block are multiplied by
   this to get true values in units of ``valueunit.''  This simplifies
   the use of normalized values.
\item {\bf xmin, ymin, zmin, xmax, ymax, zmax:} Six separate lines,
   specifying the bounding box for the mesh, in units of ``meshunit.''
   This may be used by display programs to limit the display area,
   and may be used for drawing a boundary frame if ``boundary'' is not
   specified.
\item {\bf boundary:} List of {\it (x,y,z)} triples specifying the
   vertices of a boundary frame.  Optional.
\item {\bf ValueRangeMaxMag, ValueRangeMinMag:}  The maximum and
   minimum field magnitudes in the data block, in the same
   units and scale as used in the data block.  These are for optional
   use as hints by postprocessing programs; for example, \app{mmDisp}
   will not display any vector with magnitude smaller than
   ValueRangeMinMag.  If both ValueRangeMinMag and ValueRangeMaxMag
   are zero, then the values should be ignored.
\item {\bf meshtype:}\index{grid} Grid structure; should be either
   ``rectangular'' or
   ``irregular.''  Irregular grid files should specify ``pointcount''
   in the header; rectangular grid files should specify instead
   ``xbase, ybase, zbase,'' ``xstepsize, ystepsize, zstepsize,'' and
   ``xnodes, ynodes, znodes.''
\item {\bf pointcount:} Number of data sample points/locations, i.e.,
   nodes (integer).  For irregular grids only.
\item {\bf xbase, ybase, zbase:} Three separate lines, denoting the
   position of the first point in the data section, in units of
   ``meshunit.''  For rectangular grids only.
\item {\bf xstepsize, ystepsize, zstepsize:} Three separate lines,
   specifying the distance between adjacent grid points, in units
   of ``meshunit.''  Required for rectangular grids, but may be
   specified as a display hint for irregular grids.
\item {\bf xnodes, ynodes, znodes:} Three separate lines, specifying
   the number of nodes along each axis (integers).  For
   rectangular grids only.
\end{itemize}

\subsection{Data block}\label{sec:ovfdatablock}
The data block start is marked by a line of the form
\begin{verbatim}
# Begin: data <representation>
\end{verbatim}
where \texttt{<representation>} is one of ``text'', ``binary 4'', or
``binary 8''.  Text mode uses the \ASCII\ specification, with
individual data items separated by an arbitrary amount of whitespace
(spaces, tabs and newlines).  Comments are not allowed inside binary
mode data blocks, but are permitted inside text data blocks.

The binary representations are IEEE floating point in network byte
order (MSB).  To insure that the byte order is correct, and to provide
a partial check that the file hasn't been sent through a non 8-bit
clean channel, the first datum is a predefined value: 1234567.0 (Hex:
49 96 B4 38) for 4-byte mode, and 123456789012345.0 (Hex: 42 DC 12 21
83 77 DE 40) for 8-byte mode.  The data immediately follow the check
value.

The structure of the data depends on whether the ``meshtype'' declared
in the header is ``irregular'' or ``rectangular''.  For irregular
meshes, each data element is a 6-tuple, consisting of the $x$, $y$ and
$z$ components of the node position, followed by the $x$, $y$ and $z$
components of the field at that position.  Ordering among the nodes is
not relevant.  The number of nodes is specified in the ``pointcount''
line in the segment header.

For rectangular meshes, data input is field values only, in $x$, $y$,
$z$ component triples.  These are ordered with the $x$ index
incremented first, then the $y$ index, and the $z$ index last.  This
is nominally Fortran order, and is adopted here because commonly $x$
will be the longest dimension, and $z$ the shortest, so this order is
more memory-access efficient than the normal C array indexing of $z$,
$y$, $x$.  The size of each dimension is specified in the ``xnodes,
ynodes, znodes'' lines in the segment header.

In any case, the first character after the last data item should be a
newline, followed by
\begin{verbatim}
# End: data <representation>
\end{verbatim}
where \texttt{<representation>} must match the value in the ``Begin:
data'' line.  This is followed by a
\begin{verbatim}
# End: segment
\end{verbatim}
line that ends the segment, and hence the file.

Note: An \OVF~1.0 file with \ASCII\ data and irregular meshtype is
also a valid \OVF~0.0 (\SVF\index{file!svf}) file, although as a
\OVF~0.0 file the value scaling as specified by
``\verb+# valueunit+'' and ``\verb+# valuemultiplier+'' header lines is
inactive.

\begin{codelisting}{p}{fig:ovfsample}{Commented \OVF\ sample
  file.}{sec:ovfformat}{ref}
\begin{verbatim}
# OOMMF: rectangular mesh v1.0
#
## This is a comment.
## No comments allowed in the first line.
#
# Segment count: 1   ## Number of segments.  Should be 1 for now.
#
# Begin: Segment
# Begin: Header
#
# Title: Long file name or title goes here
#
# Desc: 'Description' tag, which may be used or ignored by postprocessing
# Desc: programs. You can put anything you want here, and can have as many
# Desc: 'Desc' lines as you want.  The ## comment marker is disabled in
# Desc: description lines.
#
## Fundamental mesh measurement unit.  Treated as a label:
# meshunit: nm
#
# meshtype: rectangular
# xbase: 0.      ## (xbase,ybase,zbase) is the position, in
# ybase: 0.      ## 'meshunit', of the first point in the data
# zbase: 0.      ## section (below).
#
# xstepsize: 20. ## Distance between adjacent grid pts.: on the x-axis,
# ystepsize: 10. ## 20 nm, etc.  The sign on this value determines the
# zstepsize: 10. ## grid orientation relative to (xbase,ybase,zbase).
#
# xnodes: 200    ## Number of nodes along the x-axis, etc. (integers)
# ynodes: 400
# znodes:   1
#
# xmin:    0.    ## Corner points defining mesh bounding box in
# ymin:    0.    ## 'meshunit'.  Floating point values.
# zmin:  -10.
# xmax: 4000.
# ymax: 4000.
# zmax:   10.
#
## Fundamental field value unit, treated as a label:
# valueunit: kA/m
# valuemultiplier: 0.79577472  ## Multiply data block values by this
#                              ## to get true value in 'valueunits'.
#
# ValueRangeMaxMag:  1005.3096  ## These are in data block value units,
# ValueRangeMinMag:  1e-8       ## and are used as hints (or defaults)
#     ## by postprocessing programs.  The mmDisp program ignores any
#     ## points with magnitude smaller than ValueRangeMinMag, and uses
#     ## ValueRangeMaxMag to scale inputs for display.
#
# End: Header
#
## Anything between '# End: Header' and '# Begin: data text',
## '# Begin: data binary 4' or '# Begin: data binary 8' is ignored.
##
## Data input is in 'x-component y-component z-component' triples,
## ordered with x incremented first, then y, and finally z.
#
# Begin: data text
1000 0 0 724.1 0. 700.023
578.5 500.4 -652.36
<...data omitted for brevity...>
252.34 -696.42 -671.81
# End: data text
# End: segment
\end{verbatim}
\end{codelisting}

\section{The \OVF\ 2.0 format}\label{sec:ovf20format}
The \OVF~2.0 format is a modification to the \OVF~1.0 format that also
supports fields across three spatial dimensions but having values of
arbitrary (but fixed) dimension.  In the \OVF~2.0 format:
\begin{enumerate}
\item The first line reads: \verb+# OOMMF OVF 2.0+ for both
regular and irregular meshes.
\item In the Segment Header block, the new record \cd{valuedim} is
required.  This must specify an integer value, $N$, bigger or equal to
one.
\item In the Segment Header block, the new \cd{valueunits} record
replaces the \cd{valueunit} record of \OVF~1.0.  Instead of a single
unit value, \cd{valueunits} should be a (\Tcl) list of value units, each
treated as an unparsed label.  The list should either have length $N$
(as specified by \cd{valuedim}), in which case each element denotes
the units for the corresponding dimension index, or else the list
should have length one, in which case the single element is applied to
all dimension indexes.  The old \cd{valueunit} record is not allowed
in \OVF~2.0 files.
\item In the Segment Header block, the new \cd{valuelabels} record is
required.  This should be an $N$-item (\Tcl) list of value labels, one
for each value dimension.  The labels identify the quantity in each
dimension.  For example, in an energy density file, $N$ would be 1,
\cd{valueunits} could be J/m${}^3$, and \cd{valuelabels} might be
``Exchange energy density''.
\item In the Segment Header block, the records
\cd{valuemultiplier}, \cd{boundary}, \cd{ValueRangeMaxMag} and
\cd{ValueRangeMinMag} of the \OVF~1.0 format are not supported.
\item In the Data block, for regular meshes each record consists of
$N$ values, where $N$ is the value dimension as specified by the
\cd{valuedim} record in the Segment Header.  The node ordering is
the same as for the \OVF~1.0 format.  For irregular meshes, each
record consists of $N+3$ values, where the first three values are the
$x$, $y$ and $z$ components of the node position.  For data blocks
using text representation with $N = 3$, the Data block in
\OVF~1.0 and \OVF~2.0 files are exactly the same.
\item The data layout for data blocks using binary representation is
also the same as in \OVF~1.0 files, except that all binary values are
written in a little endian (LSB) order, as compared to the MSB order
used in the \OVF~1.0 format.  This includes the initial check value
(IEEE floating point value 1234567.0 for 4-byte format, corresponding
to the LSB hex byte sequence 38 B4 96 49, and 123456789012345.0 for
8-byte format, corresponding to the LSB hex byte sequence 40 DE 77 83
21 12 DC 42) as well as the subsequent data records.
\end{enumerate}
In all other respects, the \OVF~1.0 and \OVF~2.0 are the same.  An
example \OVF~2.0 file for an irregular mesh with $N=2$
\hyperrefhtml{follows}{follows (Fig.~}{)}{fig:ovf20sample}.

\begin{codelisting}{p}{fig:ovf20sample}{Commented \OVF~2.0 sample
  file.}{sec:ovf20format}{ref}
\begin{verbatim}
# OOMMF OVF 2.0
#
# Segment count: 1
#
# Begin: Segment
# Begin: Header
#
# Title: Long file name or title goes here
#
# Desc: Optional description line 1.
# Desc: Optional description line 2.
# Desc: ...
#
## Fundamental mesh measurement unit.  Treated as a label:
# meshunit: nm
#
# meshtype: irregular
# pointcount: 5      ## Number of nodes in mesh
#
# xmin:    0.    ## Corner points defining mesh bounding box in
# ymin:    0.    ## 'meshunit'.  Floating point values.
# zmin:    0.
# xmax:   10.
# ymax:    5.
# zmax:    1.
#
# valuedim: 2    ## Value dimension
#
## Fundamental field value units, treated as labels (i.e., unparsed).
## In general, there should be one label for each value dimension.
# valueunits:  J/m^3  A/m
# valuelabels: "Zeeman energy density"  "Anisotropy field"
#
# End: Header
#
## Each data records consists of N+3 values: the (x,y,z) node
## location, followed by the N value components.  In this example,
## N+3 = 5, the two value components are in units of J/m^3 and A/m,
## corresponding to Zeeman energy density and a magneto-crystalline
## anisotropy field, respectively.
#
# Begin: data text
0.5 0.5 0.5  500.  4e4
9.5 0.5 0.5  300.  5e3
0.5 4.5 0.5  400.  4e4
9.5 4.5 0.5  200.  5e3
5.0 2.5 0.5  350.  2.1e4
# End: data text
# End: segment
\end{verbatim}
\end{codelisting}





\chapter{Troubleshooting}\label{sec:trouble}

The \OOMMF\ developers rely on reports from \OOMMF\ users to alert them
to problems\index{reporting~bugs} with the software and its
documentation, and to guide the selection and implementation of new
features.  See the \hyperrefhtml{Credits}{Credits
(Ch.~}{)}{sec:credits} for instructions on how to contact the
\OOMMF\ developers.

The more complete your report, the fewer followup messages will be
required to determine the cause of your problem.  Usually when a problem
arises there is an error message produced by the \OOMMF\ software.  A
stack trace may be offered that reveals more detail about the error.
When reporting an error, it will help the developers diagnose the
problem if users cut and paste into their problem report the error
message and stack trace exactly as reported by \OOMMF\ software.  In
addition, \textbf{PLEASE} include a copy of the output generated by
\cd{tclsh oommf.tcl +platform}\index{platform}; this is important
because it will help \OOMMF\ developers identify problems that are
installation or platform dependent.

Before making a report to the \OOMMF\ developers, please check the
following list of fixes for known problems.  Additional problems
discovered after release will be posted to version specific ``patch''
pages at the
\htmladdnormallink{\OOMMF\ web site.}{https://math.nist.gov/oommf/}
\begin{enumerate}
\item When
\hyperrefhtml{compiling}{compiling
(Sec.~}{)}{sec:install.compile},
there is an error about being unable to open system header files like
\fn{stdlib.h}, \fn{time.h}, \fn{math.h}, or system libraries and related
program startup code.  This usually indicates a bad compiler
installation.  If you running on Windows and building with the the
Microsoft Visual C++ command line compiler, did you remember to run
\fn{vcvars32.bat} to set up the necessary environment variables?  If
you are using the Borland C++ compiler, are the \fn{bcc32.cfg} and
\fn{ilink32.cfg} files properly configured?  In all cases, check
carefully any notes in the chapter
\hyperrefhtml{Advanced Installation}{Advanced Installation
(Sec.~}{)}{sec:install.advanced} pertaining to your compiler.

\item When
\hyperrefhtml{compiling}{compiling
(Sec.~}{)}{sec:install.compile},
there is an error something like:
\begin{rawhtml}
<BLOCKQUOTE>
\end{rawhtml}
%begin<latexonly>
\begin{quote}
%end<latexonly>
\begin{verbatim}
<30654> pimake 1.x.x.x MakeRule panic:
Don't know how to make '/usr/include/tcl.h'
\end{verbatim}
%begin<latexonly>
\end{quote}
%end<latexonly>
\begin{rawhtml}
</BLOCKQUOTE>
\end{rawhtml}

This means the header file \fn{tcl.h} is missing from your
\Tcl\ installation.  Other missing header files might be
\fn{tk.h} from the \Tk\ installation, or \fn{Xlib.h} from an
X Window System installation on \Unix.  In order to compile \OOMMF, you
need to have the development versions of \Tcl, \Tk, and (if needed) X
installed.  The way to achieve that is platform-dependent.  On \Windows\
you do not need an X installation, but when you install \Tcl/\Tk\ be
sure to request a ``full'' installation, or one with ``header and
library files''.  On Linux, be sure to install developer packages
%(for example, the XFree86-devel RPM)
as well as user packages.  Other
platforms are unlikely to have this problem.  In the case of
\fn{Xlib.h}, it is also possible that the \fn{tkConfig.sh} file has an
incorrect entry for \cd{TK\_XINCLUDES}.  A workaround for this is to
add the following line to your
\fn{oommf/config/platforms/}\textit{platform} file:
\begin{rawhtml}
<BLOCKQUOTE>
\end{rawhtml}
%begin<latexonly>
\begin{quote}
%end<latexonly>
\begin{verbatim}
$config SetValue TK_XINCLUDES "-I/usr/X11R6/include"
\end{verbatim}
%begin<latexonly>
\end{quote}
%end<latexonly>
\begin{rawhtml}
</BLOCKQUOTE>
\end{rawhtml}
Adjust the include directory as appropriate for your system.

\item When
\hyperrefhtml{compiling}{compiling
(Sec.~}{)}{sec:install.compile}, there is an error
indicating that exceptions are not supported.

Parts of \OOMMF\ are written in C++, and exceptions have been part
of the C++ language for many years.  If your compiler does not
support them, it is time to upgrade to one that does.  \OOMMF\ 1.2
requires a compiler capable of compiling source code which uses
C++ exceptions.

\item \hyperrefhtml{Compiling}{Compiling
(Sec.~}{)}{sec:install.compile} with gcc produces syntax errors
on lines involving \cd{auto\_ptr} templates.

This is known to occur on RedHat 5.2 systems.  The \cd{auto\_ptr}
definition in the system STL header file \cd{memory} (located on RedHat
5.2 systems in the directory \cd{/usr/include/g++}) is disabled by two
\cd{\lb if} statements.  One solution is to edit this file to turn off
the \cd{\lb if} checks.  If you do this, you will also have to fix two
small typos in the definition of the \cd{release()} member function.

\item When \hyperrefhtml{compiling}{compiling
(Sec.~}{)}{sec:install.compile} there is an error message arising from
system include directories being too early in the include search path.
Try adding the offending directories to the
\cd{program\_compiler\_c++\_system\_include\_path} property in the
platform file, e.g.,
\begin{rawhtml}
<BLOCKQUOTE>
\end{rawhtml}
%begin<latexonly>
\begin{quote}
%end<latexonly>
\begin{verbatim}
$config SetValue program_compiler_c++_system_include_path \
      [list /usr/include /usr/local/include]
\end{verbatim}
%begin<latexonly>
\end{quote}
%end<latexonly>
\begin{rawhtml}
</BLOCKQUOTE>
\end{rawhtml}

\item On Solaris, gcc reports many errors like 
\begin{quote}
\cd{ANSI C++ forbids declaration `XSetTransientForHint' with no type}
\end{quote}

On many Solaris systems, the header files for the X Window System
are not ANSI compliant, and gcc complains about that.  To work around
this problem, edit the file \fn{oommf/config/platforms/solaris.tcl} to add the
option \cd{-fpermissive} to the gcc command line.

\item On Windows, when first starting \cd{oommf.tcl}, there is an error:
\begin{rawhtml}
<BLOCKQUOTE>
\end{rawhtml}
%begin<latexonly>
\begin{quote}
%end<latexonly>
\begin{verbatim}
Error launching mmLaunch version 1.x.x.x:
        couldn't execute "...\omfsh.exe": invalid argument
\end{verbatim}
%begin<latexonly>
\end{quote}
%end<latexonly>
\begin{rawhtml}
</BLOCKQUOTE>
\end{rawhtml}

This cryptic message most likely means that the pre-compiled
\OOMMF\ binaries which were downloaded are for a different version
of \Tcl/\Tk\ than is installed on your system.  Download \OOMMF\ again,
taking care this time to retrieve the binaries that match the
release of \Tcl/\Tk\ you have installed.

\item When first starting \cd{oommf.tcl}, there is an error:
\begin{quote}
\raggedright
\cd{Error in startup script: Neither Omf\_export nor Omf\_export\_list set in}
\end{quote}

The file \fn{oommf/pkg/net/omfExport.tcl} may be missing from your 
\OOMMF\ installation.  If necessary, download and install \OOMMF\ again.

\item When launching \cd{oommf.tcl} on \Unix\ systems, there is an error
of the form: 
\begin{quote}
\raggedright
\cd{error while loading shared library: libtk8.4.so: cannot
open shared object file: No such file or directory}
\end{quote}

This typically happens because the libtk\lb.\lb.so (and/or
libtcl\lb.\lb.so) files are installed in a directory not included in the
ld.so runtime linker/loader search path. One way to fix this is to add
that directory (say /usr/local/lib) to the \cd{LD\_LIBRARY\_PATH}
environment variable. For example, include
\begin{rawhtml}
<BLOCKQUOTE>
\end{rawhtml}
%begin<latexonly>
\begin{quote}
%end<latexonly>
\begin{verbatim}
export LD_LIBRARY_PATH=$LD_LIBRARY_PATH:/usr/local/lib 
\end{verbatim}
%begin<latexonly>
\end{quote}
%end<latexonly>
\begin{rawhtml}
</BLOCKQUOTE>
\end{rawhtml}
in your \cd{\~{}/.bashrc} file (bash shell users) or
\begin{rawhtml}
<BLOCKQUOTE>
\end{rawhtml}
%begin<latexonly>
\begin{quote}
%end<latexonly>
\begin{verbatim}
setenv LD_LIBRARY_PATH ${LD_LIBRARY_PATH}:/usr/local/lib 
\end{verbatim}
%begin<latexonly>
\end{quote}
%end<latexonly>
\begin{rawhtml}
</BLOCKQUOTE>
\end{rawhtml}
in your \cd{\~{}/.cshrc} file (csh or tcsh shell users).  Another option
is to modify the ld.so cache; see the ld.so and ldconfig man pages for
details.

%\item \index{sockets}When launching multiple \OOMMF\ applications on
%  \Windows~9X, there is an error:
%\begin{quote}
%\raggedright
%\cd{couldn't open socket: no buffer space available}
%\end{quote}
%\OOMMF\ uses network sockets for communications between its various
%components.  On \Windows~9X, the socket resources are limited, and
%most \OOMMF\ applications use several sockets.  The only workaround is
%to close unneeded \OOMMF\ applications, and any other applications
%that may be using network resources.  The system command line utility
%\cd{netstat} can be used to monitor network communications.  This
%problem does not arise on \Windows~NT or \Unix\ systems.

\item When starting \OOMMF\ in the Cygwin environment on \Windows, the
mmLaunch window appears briefly, then disappears without any error
messages.

Some old versions of \Tcl/\Tk\ included with the Cygwin environment
(i.e., \cd{/usr/bin/tclsh}) had bugs in the socket code
that caused \OOMMF\ to crash in this manner.  This problem is fixed in
current Cygwin releases.

\item I ran out of memory!

Are you using \hyperrefhtml{\app{mmGraph}}{\app{mmGraph}
(Ch.~}{)}{sec:mmgraph}\index{application!mmGraph} to monitor a
long-running simulation?  All data sent to \app{mmGraph} is kept in
memory by default.  See the \app{mmGraph} documentation for information
on how to manage this problem.

\item When running many \OOMMF\ applications, some of them become
  non-responsive.

Operating systems place an upper limit on the number of files a process
may have open at one time. The number varies, but the default count is
usually a few hundred to perhaps a few thousand. \OOMMF\ applications
communicate with one another across localhost sockets. Each socket
connection counts as an open file, so if you are running hundreds of
\OOMMF\ applications under one account you may hit this limit. You can
explore increasing the open file limit (e.g., on Linux and macOS see the
\cd{-n} option to \cd{ulimit}), but a more robust solution is to
divide your collection of \OOMMF\ jobs into several smaller independent
groups and run each group under separate host + account servers with
the \hyperrefhtml{\app{launchhost}}{\app{launchhost}
  (Sec.~}{)}{sec:launchhost} utility.

\end{enumerate}


% Apparently ``seealso'' needs to go after references, assuming you want
% the ``see also'' after the numeric listings.
\index{parallelization|seealso{NUMA}}
\index{NUMA|seealso{parallelization}}

%\include{biblio}
%\bibliographystyle{osa}
%\bibliographystyle{ieeetr}
%\bibliographystyle{acm}
\bibliographystyle{../common/oommf}
\bibliography{oommf}

\section{Credits}\label{sec:credits}

\newcommand{\myCredit}[3]{\html{\htmladdnormallink{#1}{#2}}\latex{#1 (#3)}}


\newcommand{\CreditMJD}{%
\myCredit{Michael J. Donahue}{mailto:michael.donahue@nist.gov}{michael.donahue@nist.gov}}

\newcommand{\CreditDGP}{%
\myCredit{Donald G. Porter}{mailto:donald.porter@nist.gov}{donald.porter@nist.gov}}

\newcommand{\CreditDPO}{%
Dianne P. O'Leary}
%\myCredit{Dianne P. O'Leary}{mailto:oleary@cs.umd.edu}{oleary@cs.umd.edu}}

\newcommand{\CreditRDM}{%
\myCredit{Robert D. McMichael}{mailto:rmcmichael@nist.gov}{rmcmichael@nist.gov}}

%\newcommand{\CreditJE}{%
%\myCredit{Jason Eicke}{mailto:jeicke@seas.gwu.edu}{jeicke@seas.gwu.edu}}
\newcommand{\CreditJE}{Jason Eicke}

The main contributors to this document are \CreditMJD\ and \CreditDGP,
both of
\htmladdnormallink{ITL}{http://www.nist.gov/itl/}/\htmladdnormallink{NIST}{http://www.nist.gov/}.
\html{The }\hyperrefhtml{Quick Start}{Section~}{}{sec:quickstart}\html{
section} is based on notes from \CreditDPO.

The \htmladdnormallinkfoot{\OOMMF}{http://math.nist.gov/oommf/} code is
being developed mainly by Michael Donahue and Donald Porter. \CreditRDM\
made contributions to the early development of the 2D micromagnetic
solver\index{simulation~2D}. \CreditJE\ wrote the first version of the
problem editor\index{application!mmProbEd}, and worked on the
self-magnetostatic\index{energy!demag} module of the 2D micromagnetic
solver.

Numerous users\index{contributors} have contributed to the development
of \OOMMF\ by submitting bug reports, small pieces of code, or
suggestions for improvements.  Many thanks to all these people,
including Gavin Abo, Atif Aziz, Loris Bennett, Richard Boardman,
Greg Brown, Dieter Buntinx, NgocNga Dao, Hans Fangohr, Colm Faulkner,
Xuanyao Fong, Marek Frankowski, Olivier G\'{e}rardin, Ping He,
Michael Ho, Mansoor B. A. Jalil, J\"{o}rg Jorzick, Pierre-Olivier Jubert,
Pavel Kabos, Michael Kleiber, Kristof Lebecki, Oliver Lemcke,
H. T. Leung, David Lewis, Sang Ho Lim, Yi Liu, Van Luu,
Andy P. Manners, Damien McGrouther, Johan Moulin, Wong Lai Mun, Edward
Myers, Andrew Newell, Valentine Novosad, Andrew Perrella, Angeline Phoa,
Anil Prabhakar, Robert Ravlic, Stanislas Rohart, Stephen E. Russek,
Renat Sabirianov, Zhupei Shi, Xiaobo Tan, Alexei Temiryazev,
Alexander Thieme, Stephen Thompson, Vassilios Tsiantos,
Antoine Vanhaverbeke, Pieter Visscher, Ruifang Wang, Scott
L. Whittenburg, Kong Xiangyang, Ming Yan, Tan Swee Yong, Chun-Yeol You,
Chengtao Yu, Steven A. Zielke, Juergen Zimmermann, and Pei Zou.

If you have bug reports\index{reporting~bugs}, contributed code, feature
requests, or other comments for the \OOMMF\ developers, please send them
in an e-mail\index{e-mail} message to {\htmladdnormallink{{\tt
<michael.donahue@nist.gov>}}{mailto:michael.donahue@nist.gov}}%
\index{contact~information}.

Acknowledgement is appreciated if the software is used.  We recommend
citing the following NIST technical report\index{citation~information}:
\begin{quote}
M. J. Donahue and D. G. Porter\\
OOMMF User's Guide, Version 1.0\\
Interagency Report NISTIR 6376\\
National Institute of Standards and Technology, Gaithersburg, MD (Sept 1999).
\end{quote}
and optionally include the URL of the \OOMMF\ home page,
\htmladdnormallink{\texttt{http://math.nist.gov/oommf/}}{http://math.nist.gov/oommf/}.
To help us keep our 
\htmladdnormallinkfoot{bibliography page}{http://math.nist.gov/oommf/bibliography.html}
current, please direct publication
information to {\htmladdnormallink{{\tt
<michael.donahue@nist.gov>}}{mailto:michael.donahue@nist.gov}}.

% Give Latex2Html version and reference, as specified by the
% Perl $INFO variable set in .latex2html-init.
\htmlinfo*


\printindex

\end{document}
