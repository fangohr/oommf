\chapter{Micromagnetic Problem Editor: mmProbEd}\label{sec:mmprobed}

\begin{center}
\includepic{mmprobed-ss}{mmProbEd Screen Shot}
\end{center}

\starsechead{Overview}
The application \app{mmProbEd}\index{application!mmProbEd} provides a
user interface for creating and editing micromagnetic problem
descriptions in the \hyperrefhtml{\MIF~1.1}{\MIF~1.1 (Sec.~}{,
page~\pageref{sec:mif1format})}{sec:mif1format}\index{file!MIF~1.1} and
\hyperrefhtml{\MIF~1.2}{\MIF~1.2
(Sec.~}{)}{sec:mif12format}\index{file!MIF~1.2} formats.  \app{mmProbEd}
also acts as a server, supplying problem descriptions to running
\app{mmSolve2D} micromagnetic solvers.

\starsechead{Launching}
\app{mmProbEd} may be started either by selecting the
\btn{mmProbEd} button on \htmlonlyref{\app{mmLaunch}}{sec:mmlaunch}, or
from the command line via
\begin{verbatim}
tclsh oommf.tcl mmProbEd [standard options] [-net <0|1>]
\end{verbatim}

\begin{description}
\item[\optkey{-net \boa 0\pipe 1\bca}]
  Disable or enable a server which provides problem descriptions
  to other applications.  By default, the server is enabled.  When
  the server is disabled, \app{mmProbEd} is only useful for editing
  problem descriptions and saving them to files.
\end{description}

\starsechead{Inputs}
The menu selection \btn{File\pipe Open...} displays a dialog box
for selecting a file from which to load a \MIF\ problem 
description.  Several example files are included in the 
\OOMMF\ release in the directory \fn{oommf/app/mmpe/examples}.
At startup, \app{mmProbEd} loads the problem contained in
\fn{oommf/app/mmpe/init.mif} as an initial problem.  Note: When loading
a file, \app{mmProbEd} discards comments and moves records it does not
understand to the bottom of its output file.  Use the
\hyperrefhtml{\app{FileSource} application}{{\bf FileSource} application
(Ch.}{)}{sec:filesource} to serve unmodified problem descriptions.

\starsechead{Outputs}
The menu selection \btn{File\pipe Save as...} displays a dialog box for
selecting/entering a file in which the problem description currently
held by \app{mmProbEd} is to be saved.  Because the internal data format
use by \app{mmProbEd} is an unordered array that does not include
comments (or unrecognized records), the simple operation of reading in a
\MIF\ file and then writing it back out may alter the file.

Each instance of \app{mmProbEd} contains exactly one problem description
at a time.  When the option \cd{-net 1} is active (the default), each also
services requests from client applications
(typically solvers) for the problem description it contains.

\starsechead{Controls}
The \btn{Options} menu allows selection of \MIF\ output format; either
\MIF~1.1 or \MIF~1.2 may be selected.  This affects both
\btn{File\pipe Save as...} file and \app{mmSolve2D} server
output.  See the \hyperrefhtml{\MIF~1.2}{\MIF~1.2 (Sec.~}{,
page~\pageref{sec:mif12format})}{sec:mif12format} format documentation
for details on the differences between these formats.

The main panel in the \app{mmProbEd} window contains buttons
corresponding to the sections in a \MIF~1.x problem description.
Selecting a button brings up another window through which the contents
of that section may be edited.  The \MIF\ sections and the elements they
contain are described in detail in the
\htmlonlyref{\MIF~1.1}{sec:mif1format} and
\htmlonlyref{\MIF~1.2}{sec:mif12format} documentation.
Only one editing window is displayed at a time.  The windows may be
navigated in order using their \btn{Next} or
\btn{Previous} buttons.

The \btn{Material Parameters} edit window includes a pull-down list of
pre-configured material settings.  \textbf{NOTE:} These values should
\textit{not} be taken as standard reference values for these materials.
The values are only approximate, and are included for convenience,
and as examples for users who wish to supply their own material types
with symbolic names.  To introduce additional material types, edit the
\texttt{Material} \texttt{Name}, \texttt{Ms}, \texttt{A}, \texttt{K1}, and
\texttt{Anisotropy} \texttt{Type} values as desired, and hit the \btn{Add}
button.  (The \texttt{Damp} \texttt{Coef} and \texttt{Anistropy}
\texttt{Init} settings are not affected by the Material Types
selection.)  The Material Name entry will appear in red if it does not
match any name in the Material Types list, or if the name matches but
one or more of the material values differs from the corresponding value
as stored in the Material Types list.  You can manage the Material Types
list with the \btn{Replace} and \btn{Delete} buttons, or by directly
editing the file \fn{oommf/app/mmpe/local/materials}; follow the format
of other entries in that file.  The format is the same as in the default
\fn{oommf/app/mmpe/materials} file included with the \OOMMF\ distribution.

The menu selection \btn{File\pipe Exit} terminates the
\app{mmProbEd} application.  The menu \btn{Help} provides
the usual help facilities.

