
\section{Command Line Launching}\label{sec:cll}
\index{launch!from~command~line}\index{launch!with~bootstrap~application}

Some of the \OOMMF\ applications are platform-independent Tcl
scripts.  Some of them are Tcl scripts that require special
platform-dependent interpreters.  Others are platform-dependent,
compiled C++ applications.  It is possible that some of them will 
change status in later releases of \OOMMF.  Each of these types
of application requires a different command line for launching.
Rather than require all \OOMMF\ users to manage this complexity,
we provide a pair of programs that provide simplified interfaces
for launching \OOMMF\ applications.

\index{application!bootstrap|(}
The first of these is used to launch \OOMMF\ applications from the
command line.  Because its function is only to start another
program, we refer to this program as the ``bootstrap application.''
The bootstrap application is the Tcl script \fn{oommf.tcl}.
In its simplest usage, it takes a single argument on the command line,
the name of the application to launch.  For example, to launch
\hyperrefhtml{\app{mmGraph}}{\app{mmGraph} (Sec.~}{)}{sec:mmgraph},
the command line is:
\begin{verbatim}
tclsh oommf.tcl mmGraph
\end{verbatim}
The search for an application matching the name is case-insensitive.
(Here, as elsewhere in this document, the current working
directory\index{working~directory} is assumed to be the \OOMMF\ root
directory.  For other cases, adjust the pathname to {\tt oommf.tcl} as
appropriate.)  As discussed \hyperrefhtml{earlier}{in
Sec.~}{}{sec:install.requirements}, the name of the \Tcl\ shell,
rendered here as \verb+tclsh+, may vary between systems.

If no command line arguments are passed to the bootstrap application,
by default it will launch the application
\hyperrefhtml{\app{mmLaunch}}{\app{mmLaunch} (Sec.~}{)}{sec:mmlaunch}.

\index{launch!command~line~arguments|(}
Any command line arguments to the bootstrap
application that begin with the character `\cd{+}' modify its
behavior.
For a summary of all command line options recognized by the bootstrap
application, run:
\begin{verbatim}
tclsh oommf.tcl +help
\end{verbatim}

\index{launch!foreground|(}
The command line arguments \cd{+bg} and \cd{+fg} control how the
bootstrap behaves after launching the requested application.  It
can exit immediately after launching the requested application 
in background mode (\cd{+bg}), or it can block until the
launched application exits (\cd{+fg}).  Each application
registers within the \OOMMF\ system whether it prefers to be launched
in foreground or background mode.  If neither option is requested on
the command line, the bootstrap launches the requested application
in its preferred mode.
\index{launch!foreground|)}

\index{launch!version~requirement|(}
The first command line argument that does not begin with the
character \cd{+} is interpreted as a specification of which
application should be launched.  As described above, this is 
usually the simple name of an application.
When a particular
version of an application is required, though, the bootstrap
allows the user to include that requirement as part of the
specification.  For example:
\begin{verbatim}
tclsh oommf.tcl "mmGraph 1.1"
\end{verbatim}
will guarantee that the instance of the application mmGraph it
launches is of at least version 1.1.  If no copy of mmGraph
satisfying the version requirement can be found, an error is
reported.
\index{launch!version~requirement|)}

The rest of the command line arguments that are not recognized by
the bootstrap are passed along as arguments to the application the
bootstrap launches.  Since the bootstrap recognizes command line
arguments that begin with \cd{+} and most other applications
recognize command line arguments that begin with \cd{-}, confusion
about which options are provided to which programs can be avoided.
For example,
\begin{verbatim}
tclsh oommf.tcl +help mmGraph
\end{verbatim}
prints out help information about the bootstrap and exits without
launching mmGraph.  However,
\begin{verbatim}
tclsh oommf.tcl mmGraph -help
\end{verbatim}
launches mmGraph with the command line argument \cd{-help}.
mmGraph then displays its own help message.

\index{launch!standard~options|(}
\index{standard~options|(}
Most \OOMMF\ applications accept the standard options listed below.
Some of the \OOMMF\ applications accept additional arguments when
launched from the command line, as documented in the corresponding
sections of this manual.  The \verb+-help+ command line option can
also be used to view the complete list of available options.  When an
option argument is specified as \verb+<0|1>+, \verb+0+ typically means
off, no or disable, and \verb+1+ means on, yes or enable.

\begin{description}
\item[\optkey{-console}]
Display a console widget in which Tcl
commands may be interactively typed into the application.
Useful for debugging.

\item[\optkey{-cwd directory}]
Set the current working directory of the application.

\item[\optkey{-help}]
Display a help message and exit.

\item[\optkey{-nickname \boa name\bca}]\index{nicknames}
Associates the specified \textit{name} as a nickname for the process.
The string \textit{name} should contain at least one non-numeric
character.  Nicknames can also be set at launch time via the
\htmlonlyref{\cd{Destination}}{html:destinationCmd} command\latex{
(Sec.~\ref{sec:mif2ExtensionCommands})} in \cd{MIF 2.x} files, or
after a process is running via the
\hyperrefhtml{\app{nickname}}{\app{nickname} (Sec.~}{)}{sec:nickname}
command line application.  Nicknames are used by the \MIF\ 2.x
\cd{Destination} command to associate \cd{Oxs} output streams with
particular application instances.  Multiple \cd{-nickname} options may
be used to set multiple nicknames.  (Technical detail: Nickname
functionality is only available to processes that connect to an
account server.)

\item[\optkey{-tk \boa 0\pipe 1\bca}]
Disable or enable Tk.  Tk must be enabled for an application to display
graphical widgets.  However, when Tk is enabled on Unix platforms
the application is dependent on an X Windows server.  If the 
X Windows server dies, it will kill the application.  Long-running
applications that do not inherently use display widgets support
disabling of Tk with \verb+-tk 0+.  
Other applications that must use display widgets are unable to run
with the option \verb+-tk 0+.  To run applications that require
\verb+-tk 1+ on a Unix system with no display, one might use
\htmladdnormallinkfoot{Xvfb}{https://www.x.org/archive/X11R7.6/doc/man/man1/Xvfb.1.xhtml}
\index{application!Xvfb}.

\item[\optkey{-version}]
Display the version of the application and exit.
\end{description}

In addition, those applications which enable Tk accept additional Tk
options, such as \verb+-display+.  See the Tk documentation for details.
\index{standard~options|)}
\index{launch!standard~options|)}
\index{launch!command~line~arguments|)}

The bootstrap application should be infrequently used by most users.
The application \hyperrefhtml{\app{mmLaunch}}{\app{mmLaunch}
(Sec.~}{)}{sec:mmlaunch} provides a more convenient graphical
interface for launching applications.  The main uses for the bootstrap
application are launching \app{mmLaunch}, launching \app{pimake},
launching programs which make up the \hyperrefhtml{OOMMF Batch
System}{OOMMF Batch System (Sec.~}{)}{sec:obs} and other programs
that are inherently command line driven, and in circumstances where
the user wishes to precisely control the command line arguments passed
to an \OOMMF\ application or the environment in which an \OOMMF\ application
runs.
\index{application!bootstrap|)}

\ssechead{Platform Issues}

\index{platform!Unix!executable~Tcl~scripts|(}
On most Unix platforms, if \fn{oommf.tcl} is marked executable,
it may be run directly, i.e., without specifying \cd{tclsh}.  This
works because the first few lines of the \fn{oommf.tcl} Tcl script are:
\begin{verbatim}
#!/bin/sh
# \
exec tclsh "$0" ${1+"$@"}
\end{verbatim}
When run, the first \fn{tclsh} on the execution path is invoked to
interpret the \fn{oommf.tcl} script.  If the Tcl shell program cannot be
invoked by the name \fn{tclsh} on your computer, edit the first lines of
\fn{oommf.tcl} to use the correct name.  Better still, use symbolic links
or some other means to make the Tcl shell program available by the name
\fn{tclsh}.  The latter solution will not be undone by file overwrites
from \OOMMF\ upgrades.

If in addition, the directory
\fn{.../path/to/oommf} is in the execution path, the command line can
be as simple as:
\begin{verbatim}
oommf.tcl appName
\end{verbatim}
from any working directory.
\index{platform!Unix!executable~Tcl~scripts|)}

\index{platform!Windows!file~extension~associations|(}
On Windows platforms, because \fn{oommf.tcl} has the file
extension \fn{.tcl}, it is normally associated by Windows with the
\fn{wish} interpreter.  The \fn{oommf.tcl} script has been
specially written so that either \fn{tclsh} or \fn{wish} is a suitable
interpreter.  This means that simply double-clicking on an icon
associated with the file \fn{oommf.tcl} 
(say, in Windows Explorer\index{application!Windows~Explorer})
will launch the bootstrap application with no arguments.  This will
result in the default behavior of launching the application
\app{mmLaunch}, which is suitable for launching other \OOMMF\
applications.  (If this doesn't work, refer back to the
\html{\htmlref{Windows Options}{sec:install.windows} section in the
installation instructions.)}  \latex{Windows Options section in the
installation instructions, Sec.~\ref{sec:install.windows}.)}
\index{platform!Windows!file~extension~associations|)}

