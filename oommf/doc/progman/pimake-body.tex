\chapter{Platform-Independent Make Operational Details}\label{sec:pimake}

The \OOMMF\ \app{pimake}\index{application!pimake} application compares
file timestamps to determine which libraries and executables are
out-of-date with respect to their source code, and then compiles and
links those files as necessary to make everything up to date. The design
and behavior of \app{pimake} is based on the
\Unix\ \app{make}\index{application!make} program, but \app{pimake} is
written in \Tcl\ and so can run on any platform where \Tcl\ is
installed. Analogous to the \fn{Makefile} or \fn{makefile} of
\app{make}, \app{pimake} uses \fn{makerules.tcl} files that specify
\textit{rules} (actions) for creating or updating \textit{targets} when
the targets are older than their corresponding \textit{dependencies}.
The \fn{makerules.tcl} files are \Tcl\ scripts augmented by a handful of
commands introduced by the \app{pimake} application.

The \fn{makerules.tcl} files in the \app{Oxs} application area include
rules to automatically compile and link all \Cplusplus\ code found under
the \fn{oommf/app/oxs/local/} directory, so programmers who are
developing \cd{Oxs\_Ext} extension modules generally do not need to be
concerned with the intricacies of \app{pimake} beyond the instructions
on running \app{pimake} presented in the
\htmladdnormallinkfoot{\textit{\OOMMF\ User's
    Guide}}{https://math.nist.gov/oommf/doc/}.

This chapter is intended instead for programmers who are debugging,
extending, or creating new \OOMMF\ modules outside of
\fn{oommf/app/oxs/local/}. The following sections provide an overview of
the structure of \fn{makerules.tcl} files and how they control the
behavior of \app{pimake}. Further details may be gleaned from the
\app{pimake} sources in \fn{oommf/app/pimake/}.

\section{Anatomy of \fn{makerules.tcl} files}\label{sec:anatomymakerules}
As may be deduced from the file extension, \fn{makerules.tcl} files are
\Tcl\ scripts and so can make use of the usual \Tcl\ commands. However,
\fn{makerules.tcl} files are run inside a \Tcl\ interpreter that has
been augmented by \app{pimake} with a number of additional commands. We
discuss both types of commands here, beginning with some of the standard
\Tcl\ commands commonly found in \fn{makerules.tcl} files:
\begin{description}
\item[list, llength, lappend, lsort, lindex, lsearch, concat]
  \Tcl\ list formation and access commands.
\item[file]
  Provides platform independent access to the file system, including
  subcommands to split and join file names by path component.
\item[glob] Returns a list of filenames matching a wildcard pattern.
\item[format, subst] Construct strings with variable substitutions.
\end{description}
Refer to the
\htmladdnormallinkfoot{\Tcl\ documentation}{https://www.tcl-lang.org/man}
for full details.

Notice that all the \Tcl\ command names are lowercase.  In contrast,
commands added by \app{pimake} have mixed-case names. The most common
\OOMMF\ commands you'll find in \fn{makerules.tcl} files are
\begin{description}
\item[MakeRule] Defines dependency rules, which is the principle goal
  of \fn{makerules.tcl} files. This command is documented in detail
  \hyperrefhtml{below}{below (Sec.~}{)}{sec:makerule}.
\item[Platform] Platform independent methods for common operations, with
  these subcommands:
\begin{description}
\item[Name] Identifier for current platform, e.g.,
  \cd{windows-x86\_64}, \cd{linux-x86\_64}, \cd{darwin}.
\item[Executables] Given a file stem returns the name for the
  corresponding executable on the current platform by prepending the
  platform directory and appending an execution suffix, if any. For
  example, \cd{Platform Executables varinfo} would return
  \fn{windows-x86\_64/varinfo.exe} on \Windows, and
  \fn{linux-x86\_64/varinfo} on \Linux.
\item[Objects] Similar to Platform Executables, but returns object file
  names; the object file suffix is \cd{.obj} on \Windows\ and \cd{.o} on
  \Linux\ and \MacOSX.
\item[Compile] Uses the compiler specified in the
 \fn{config/platform/<platform>.tcl} to compile the specified
 source code file (\cd{-src} option) into the named object file (\cd{-out}
 option).
\item[Link] Uses the linker specified in
 \fn{config/platform/<platform>.tcl} to link together the specified
 object files (\cd{-obj} option) into the named executable (\cd{-out}
 option).
\end{description}
\item[CSourceFile New] Creates an instance of the \cd{CSourceFile}
  class. The \cd{-inc} option to \cd{New} specifies directories to add
  to the search path for header files. \cd{CSourceFile} instances
  support these subcommands:
  \begin{description}
  \item[Dependencies] Dependency list for specified \Cplusplus\ source
    file consisting of the source file itself, header files included by
    \cd{\#include} statements in the source code files, and also any
    header files found by a recursive tracing of \cd{\#include}
    statements.  The header file search excludes system header files
    requested using angle-brackets, e.g., \cd{\#include <stdio.h>}. A
    source code file can speed the tracing process by placing a \cd{/*
      End includes */} comment following the last \cd{\#include}
    statement, as in this example from
    \fn{oommf/app/mmdisp/mmdispsh.cc}:
\begin{verbatim}
  /* FILE: mmdispsh.cc                 -*-Mode: c++-*-
   *
   * A shell program which includes Tcl commands needed to support a
   * vector display application.
   *
   */

  #include "oc.h"
  #include "vf.h"
  #include "mmdispcmds.h"

  /* End includes */
  ...
\end{verbatim}
   The \cd{/* End includes */} statement terminates the search for
   further \cd{\#include} statements in that file.
  \item[DepPath] List of directories containing files on which
    the specified \Cplusplus\ source file depends.
  \end{description}
\item[Recursive] Given a target, loads the \fn{makerules.tcl} file in
  each child directory of the current directory and executes the rule
  found there for the target. Primarily used with the default targets
  \cd{all}, \cd{configure}, \cd{clean}, \cd{mostlyclean}, \cd{objclean},
  \cd{maintainer-clean}, \cd{distclean}, and \cd{upgrade}. The default
  targets have an implicit rule to do nothing except recurse the action
  into the new child directories. If a \fn{makerules.tcl} file found in
  this manner has an explicit rule defined for the given target, then
  that rule is invoked instead of the implicit rule, and, unless the
  explicit rule makes a \cd{Recursive} call itself, the recursion on
  that directory branch will stop. As an example, the \fn{makerules.tcl}
  file in the \OOMMF\ root directory has the rule
\begin{verbatim}
  MakeRule Define {
    -targets   all
    -script    {Recursive all}
  }
\end{verbatim}
  All of \fn{makerules.tcl} files one level below \fn{oommf/pkg} and
  \fn{oommf/app} have ``\cd{all}'' targets that compile and link their
  corresponding libraries or executables. So
\begin{verbatim}
  tclsh oommf.tcl pimake all
\end{verbatim}
  run in the root \OOMMF\ directory will build all of those libraries
  and applications. In contrast, \fn{makerules.tcl} files under
  \fn{oommf/doc} do \textbf{not} have explicit \cd{all} targets, so the
  \cd{tclsh oommf.tcl pimake all} call has no effect in the \fn{oommf/doc/}
  subtree.

  On the other hand, the \fn{makerules.tcl} in directories under
  \fn{oommf/pkg/}, \fn{oommf/app/}, and \fn{oommf/doc/} \textbf{do} have
  explicit rules for the various \cd{clean} targets, so
\begin{verbatim}
  tclsh oommf.tcl pimake maintainer-clean
\end{verbatim}
  run from the \OOMMF\ root directory will be active throughout all
  three subtrees. The \cd{maintainer-clean} rules delete all files that
  can be regenerated from source, meaning object files, libraries,
  executables, and notably all the documentation files under
  \fn{oommf/doc/}. Building the \OOMMF\ documentation requires a working
  installation of
  \htmladdnormallinkfoot{\LaTeX}{https://www.latex-project.org} and
  either \htmladdnormallinkfoot{\LaTeX2HTML}{https://www.latex2html.org}
  or \htmladdnormallinkfoot{\LaTeXML}{http://dlmf.nist.gov/LaTeXML/}, so
  don't run the \cd{maintainer-clean} target unless you are prepared to
  rebuild the \OOMMF\ documentation!
\end{description}
The \Tcl\ source defining the \cd{MakeRule}, \cd{Platform},
\cd{CSourceFile}, and \cd{Recursive} commands can be found in the
\cd{oommf/app/pimake/} directory. Example use of these commands can be
found in the \htmlonlyref{following section}{sec:makerule}.

\section{The MakeRule command}\label{sec:makerule}
The \fn{makerules.tcl} files consist primarily of a collection of
\cd{MakeRule} commands surrounded by a sprinkling of \Tcl\ support
code. The order of the \cd{MakeRule} commands doesn't matter, except
that the first target in the file, usually \cd{all}, becomes the default
target. (The ``default'' target is the effective target if \app{pimake}
is run without specifying a target.)

The \cd{MakeRule} command supports a number of subcommands, but the
principle subcommand appearing in \fn{makerules.tcl} files is
\cd{Define}. This takes a single argument, which is a list of
option+value pairs, with valid options being \cd{-targets},
\cd{-dependencies}, and \cd{-script}. The value string for the
\cd{-targets} option is a list of one or more build targets. The targets
are usually files, in which case they must lie in the same directory or
a directory below the \fn{makerules.tcl} file. The \cd{-dependencies}
option is a list of one or more files or targets that the target depends
upon. The value to the \cd{-script} option is a \Tcl\ script that is run
if a target does not exist or if any of the file dependencies have a
newer timestamp than any of the targets. The dependency checking is done
recursively, that is, each dependency is checked to see if it up to date
with its own dependencies, and so on.  A target is out of date if it is
older than any of its dependencies, or the dependencies of the
dependencies, etc. If any of the dependencies is out of date with
respect to its own dependencies, then its script will be run during the
dependency resolution. The script associated with the original target is
only run after its dependency resolution is fully completed.

The following examples from \fn{oommf/app/omfsh/makerules.tcl}
should help flesh out the above description:
\begin{verbatim}
  MakeRule Define {
    -targets        [Platform Name]
    -dependencies   {}
    -script         {MakeDirectory [Platform Name]}
  }
\end{verbatim}
Here the target is the platform name, e.g., \fn{windows-x86\_64}, which
is a directory under the current working directory
\fn{oommf/app/omfsh/}. There are no dependencies to check, so the rule
script is run if and only if the directory \fn{windows-x86\_64} does not
exist. In that case the \OOMMF\ \cd{MakeDirectory} routine is called to
create it. This is an important rule because the compilation and linking
commands place their output into this directory, so it must exist before
those commands are run.

Next we look at a more complex rule that is really the bread and
butter of \fn{makerules.tcl}, a rule for compiling a \Cplusplus\ file:
\begin{verbatim}
  MakeRule Define {
    -targets        [Platform Objects omfsh]
    -dependencies   [concat [list [Platform Name]] \
                            [[CSourceFile New _ omfsh.cc] Dependencies]]
    -script         {Platform Compile C++ -opt 1 \
                             -inc [[CSourceFile New _ omfsh.cc] DepPath] \
                             -out omfsh -src omfsh.cc
                    }
  }
\end{verbatim}
In this example the target is the object file associated with the
stem \cd{omfsh}. On \Windows\ this would be
\fn{windows-x86\_64/omfsh.obj}. The dependencies are the platform
directory (e.g., \fn{windows-x86\_64/}), the file \fn{omfsh.cc}, and any
(non-system) files included by \fn{omfsh.cc}. Directory timestamps do
not affect the out-of-date computation, but directories will be
constructed by their \cd{MakeRule} if they don't exist.

Note that part of the \cd{-dependencies} list is
\begin{verbatim}
  [CSourceFile New _ omfsh.cc] Dependencies]
\end{verbatim}
As discussed \hyperrefhtml{earlier}{in Sec.~}{}{sec:anatomymakerules},
this command resolves to a list of all non-system \cd{\#include} header
files from \fn{omfsh.cc}, or header files found recursively from those
header files. The first part of \fn{omfsh.cc} is
\begin{verbatim}
  /* FILE: omfsh.cc                 -*-Mode: c++-*-
   *
   *      A Tcl shell extended by the OOMMF core (Oc) extension
   ...
   */

  /* Header files for system libraries */
  #include <cstring>

  /* Header files for the OOMMF extensions */
  #include "oc.h"
  #include "nb.h"
  #include "if.h"

  /* End includes */
  ...
\end{verbatim}
The header file \fn{cstring} is ignored by the dependency search because
it is specified inside angle brackets rather than double quotes. But the
\fn{oc.h}, \fn{nb.h}, and \fn{if.h} files are all considered. These
files are part of the \cd{Oc}, \cd{Nb}, and \cd{If} package libraries,
respectively, living in subdirectories under \fn{oommf/pkg/}. The file
\fn{oommf/pkg/oc/oc.h}, for example, will be checked for included files
in the same way, and so on. The full dependency tree can be quite
extensive. The \app{pimake} application supports a \cd{-d} option to
print out the dependency tree, e.g.,
\begin{verbatim}
  tclsh oommf.tcl pimake -cwd app/omfsh -d windows-x86_64/omfsh.obj
\end{verbatim}
This output can be helpful is diagnosing dependency issues.

The \verb+/* End includes */+ line terminates the \cd{\#include} file search
inside this file. It is optional but recommended as it will speed-up
dependency resolution.

If \fn{omfsh.obj} is older than any of its dependent files, then the
\Tcl\ script specified by the \cd{-script} option will be triggered. In
this case the script runs \cd{Platform Compile C++}, which is the
\Cplusplus\ compiler as specified by the
\fn{oommf/config/platforms/<platform>.tcl} file. In this command
\cd{-opt} enables compiler optimizations, \cd{-inc} supplements the
include search path for the compiler, \cd{-out omfsh} is the output
object file with name adjusted appropriately for the platform, and
\cd{-src omfsh.cc} specifies the \Cplusplus\ file to be compiled.

The rules for building executables and libraries from collections of
object modules are of a similar nature. See the various
\fn{makerules.tcl} files across the \OOMMF\ directory tree for examples.

In a normal rule, the target is a file and if the script is run it will
create or update the file. Thus, if \app{pimake} is run twice in
succession on the same target, the second run will not trigger the
script because the target will be up to date. In contrast, a
pseudo-target\index{pimake!pseudo-target} does not exist as a file on
the file system, and the associated script does not create the
pseudo-target. Since the pseudo-target never exists as a file, repeated
runs of \app{pimake} on the target will result in repeated runs of the
pseudo-target script.

Common pseudo-targets include \cd{all},
\cd{configure}, \cd{help}, and several \cd{clean} variants.  This last
example illustrates the chaining of \cd{clean} pseudo-targets to remove
constructed files.
\begin{verbatim}
  MakeRule Define {
    -targets         clean
    -dependencies    mostlyclean
  }

  MakeRule Define {
    -targets         mostlyclean
    -dependencies    objclean
    -script          {eval DeleteFiles [Platform Executables omfsh] \
                         [Platform Executables filtersh] \
                         [Platform Specific appindex.tcl]}
  }

  MakeRule Define {
    -targets         objclean
    -dependencies    {}
    -script          {
                      DeleteFiles [Platform Objects omfsh]
                      eval DeleteFiles \
                             [Platform Intermediate {omfsh filtersh}]
                     }
  }
\end{verbatim}
All three of these rules have targets that are non-existent files, so
all three are pseudo-targets. The first rule, for target \cd{clean}, has
no script so the script execution is a no-op. However, the dependencies
are still evaluated, which in this case means the rule for the target
\cd{mostlyclean} is checked. This rule has both a dependency and a
script. The dependencies are evaluated first, so the \cd{objclean}
script is called to delete the \cd{omfsh} object file and also any
intermediate files created as side effects of building the \app{omfsh}
and \app{filtersh} executables. Next the \cd{mostlyclean} script is run,
which deletes the \app{omfsh} and \app{filtersh} executables and also
the platform-specific \fn{appindex.tcl} file. Note that none of the
scripts create their target, so the targets will all remain
pseudo-targets.
