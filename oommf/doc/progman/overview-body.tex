\chapter{Programming Overview of \OOMMF}\label{sec:overview}
The
\htmladdnormallinkfoot{\OOMMF}{https://math.nist.gov/oommf/} (Object
Oriented Micromagnetic Framework) project in the
\htmladdnormallinkfoot{Information Technology Laboratory}{https://www.nist.gov/itl/}
(ITL) at the
\htmladdnormallinkfoot{National Institute of Standards and
Technology}{https://www.nist.gov/} (NIST) is intended to develop a
portable, extensible public domain micromagnetic program and associated
tools.  This manual aims to document the programming interfaces to
\OOMMF\ at the source code level.  The main developers of this code are
\psonly{\htmladdnormallinkfoot{Mike Donahue}{https://math.nist.gov/\%7EMDonahue}}
\notpsonly{\htmladdnormallink{Mike Donahue}{https://math.nist.gov/\%7EMDonahue}}
and
\psonly{\htmladdnormallinkfoot{Don Porter}{https://math.nist.gov/\%7EDPorter}.}
\notpsonly{\htmladdnormallink{Don Porter}{https://math.nist.gov/\%7EDPorter}.}

The underlying numerical engine for \OOMMF\ is written in \Cplusplus,
which provides a reasonable compromise with respect to efficiency,
functionality, availability and portability.  The interface and glue
code is written primarily in \Tcl/\Tk, which hides most platform
specific issues. \Tcl\ and \Tk\ are available for free
\htmladdnormallinkfoot{download}{http://purl.org/tcl/home/software/tcltk/choose.html}
from the
\htmladdnormallinkfoot{Tcl Developer Xchange}{http://purl.org/tcl/home/}.

The code may actually be modified at 3 distinct levels.  At the top
level, individual programs interact via well-defined protocols across
network sockets\index{network~socket}.  One may connect these modules
together in various ways from the user interface, and new modules
speaking the same protocol can be transparently added.  The second level
of modification is at the \Tcl/\Tk\ script level.  Some modules allow
\Tcl/\Tk\ scripts to be imported and executed at run time, and the top
level scripts are relatively easy to modify or replace.  The lowest
level is the \Cplusplus\ source code.  The OOMMF extensible solver, OXS,
is designed with modification at this level in mind.

If you want to receive e-mail\index{e-mail}
notification\index{announcements} of updates to this project, register
your e-mail address with the ``{\mumag} Announcement'' mailing list:
% Note: For some reason, the braces above about \mumag discourage line
% breaking between $\mu$ and Mag in the latexml/browser display.
\begin{center}
\htmladdnormallink{\url{https://www.ctcms.nist.gov/~rdm/email-list.html}}{https://www.ctcms.nist.gov/\%7Erdm/email-list.html}.
\end{center}

The \OOMMF\ developers are always interested in your comments about
\OOMMF.  See the \hyperrefhtml{Credits}{Credits (Ch.~}{) }{sec:credits}
for instructions on how to contact them.
