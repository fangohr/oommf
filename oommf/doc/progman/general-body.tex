\chapter{\OOMMF\ Variable Types and Macros}\label{sec:vartypes}
The following typedefs are defined in the
\fn{oommf/pkg/oc/{\it{platform}}/ocport.h} header file; this file is
created by the \app{pimake} build process (see
\fn{oommf/pkg/oc/procs.tcl}), and contains platform and machine
specific information.
\newcommand{\gbs}{\hspace{0.5em}}
\begin{itemize}
\item{\texttt{OC\_BOOL}} \gbs Boolean type, unspecified width.
\item{\texttt{OC\_BYTE}} \gbs Unsigned integer type exactly one byte wide.
\item{\texttt{OC\_CHAR}} \gbs Character type, may be signed or unsigned.
\item{\texttt{OC\_UCHAR}} \gbs Unsigned character type.
\item{\texttt{OC\_SCHAR}} \gbs Signed character type.  If \texttt{signed char}
  is not supported by a given compiler, then this falls back to a
  plain \texttt{char}, so use with caution.
\item{\texttt{OC\_INT2, OC\_INT4}} \gbs Signed integer with width of
  exactly 2, respectively 4, bytes.
\item{\texttt{OC\_INT2m, OC\_INT4m}} \gbs Signed integer with width of
  at least 2, respectively 4, bytes.  A type wider than the minimum
  may be specified if the wider type is handled faster by the
  particular machine.
\item{\texttt{OC\_UINT2, OC\_UINT4, OC\_UINT2m, OC\_UINT4m}} \gbs Unsigned
  integer versions of the preceding.
\item{\texttt{OC\_REAL4, OC\_REAL8}} \gbs Four byte, respectively eight
  byte, floating point variable.  Typically corresponds to \Cplusplus\
  ``float'' and ``double'' types.
\item{\texttt{OC\_REAL4m, OC\_REAL8m}} \gbs Floating point variable with
  width of at least 4, respectively 8, bytes.  A type wider than the minimum
  may be specified if the wider type is handled faster by the
  particular machine.
\item{\texttt{OC\_REALWIDE}} \gbs Widest type natively supported by the
  underlying hardware.  This is usually the \Cplusplus\ ``long
  double'' type, but may be overridden by the
\begin{center}
  \texttt{program\_compiler\_c++\_typedef\_realwide}
\end{center}
  option in the \fn{oommf/config/platform/{\it{platform}}.tcl} file.
\end{itemize}

The \fn{oommf/pkg/oc/{\it{platform}}/ocport.h} header file also
defines the following macros for use with the floating point variable
types:
\begin{itemize}
\item{\texttt{OC\_REAL8m\_IS\_DOUBLE}} \gbs True if \texttt{OC\_REAL8m} type
  corresponds to the \Cplusplus\ ``double'' type.
\item{\texttt{OC\_REAL8m\_IS\_REAL8}} \gbs True if \texttt{OC\_REAL8m} and
  \texttt{OC\_REAL8} refer to the same type.
\item{\texttt{OC\_REAL4\_EPSILON}} \gbs The smallest value that can be added to
  a \texttt{OC\_REAL4} value of ``1.0'' and yield a value different from
  ``1.0''.  For IEEE 754 compatible floating point, this should be
  \texttt{1.1920929e-007}.
\item{\texttt{OC\_SQRT\_REAL4\_EPSILON}}
    \gbs Square root of the preceding.
\item{\texttt{OC\_REAL8\_EPSILON}} \gbs The smallest value that can be added to
  a \texttt{OC\_REAL8} value of ``1.0'' and yield a value different from
  ``1.0''.  For IEEE 754 compatible floating point, this should be
  \texttt{2.2204460492503131e-016}.
\item{\texttt{OC\_SQRT\_REAL8\_EPSILON, OC\_CUBE\_ROOT\_REAL8\_EPSILON}}
    \gbs Square and cube roots of the preceding.
\item{\texttt{OC\_FP\_REGISTER\_EXTRA\_PRECISION}} \gbs True if
  intermediate floating point operations use a wider precision than
  the floating point variable type; notably, this occurs with some
  compilers on x86 hardware.
\end{itemize}

Note that all of the above macros have a leading ``\texttt{OC\_}''
prefix.  The prefix is intended to protect against possible name
collisions with system header files.  Versions of some of these macros
are also defined without the prefix; these definitions represent
backward support for existing \OOMMF\ extensions.  All new code
should use the versions with the ``\texttt{OC\_}'' prefix, and old code
should be updated where possible.  The complete list of deprecated
macros is:
\begin{quote}
\raggedright
\texttt{BOOL, UINT2m, INT4m, UINT4m,
    REAL4, REAL4m, REAL8, REAL8m, REALWIDE,
    REAL4\_EPSILON, REAL8\_EPSILON,
    SQRT\_REAL8\_EPSILON, CUBE\_ROOT\_REAL8\_EPSILON,
    FP\_REGISTER\_EXTRA\_PRECISION
}
\end{quote}

Macros for system identification:
\begin{itemize}
\item{\texttt{OC\_SYSTEM\_TYPE}} \gbs One of \texttt{OC\_UNIX} or
  \texttt{OC\_WINDOWS}.
\item{\texttt{OC\_SYSTEM\_SUBTYPE}} \gbs For unix systems, this is
    either \texttt{OC\_VANILLA} (general unix) or \texttt{OC\_DARWIN}
    (Mac OS X).  For Windows systems, this is generally
    \texttt{OC\_WINNT}, unless one is running out of a Cygwin shell,
    in which case the value is \texttt{OC\_CYGWIN}.
\end{itemize}

Additional macros and typedefs:
\begin{itemize}
\item{\texttt{OC\_POINTERWIDTH}} \gbs Width of pointer type, in bytes.
\item{\texttt{OC\_INDEX}} \gbs Typedef for signed array index type;
  typically the width of this (integer) type matches the width of the
  pointer type, but is in any event at least four bytes wide and not
  narrower than the pointer type.
\item{\texttt{OC\_UINDEX}} \gbs Typedef for unsigned version of
  OC\_INDEX.  It is intended for special-purpose use only.  In general,
  use OC\_INDEX where possible.
\item{\texttt{OC\_INDEX\_WIDTH}} \gbs Width of \texttt{OC\_INDEX} type.
\item{\texttt{OC\_BYTEORDER}} Either ``4321'' for little endian machines,
  or ``1234'' for big endian.
\item{\texttt{OC\_THROW(x)}} \gbs Throws a \Cplusplus\ exception with
  value ``x''.
\item{\texttt{OOMMF\_THREADS}} \gbs True for threaded (multi-processing) builds.
\item{\texttt{OC\_USE\_NUMA}} \gbs If true, then NUMA (non-uniform memory
  access) libraries are available.
\end{itemize}
